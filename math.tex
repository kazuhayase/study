$\pm$
$$
|a| =\begin{cases}
      a & (\text{if} a \geq 0 )\\
      -a & (\text{if}a < 0 \text{のとき})\\
     \end{cases}
$$

$$
\begin{cases}
\alpha + \beta = -\frac{b}{a}\\
\alpha  \beta = \frac{c}{a}\\
\end{cases}
$$

$\frac
$

のときは異なる実数解が2つ
のときは重解(実数解が1つ)
のときは実数解はない


[$]x=\frac{-b\pm\sqrt{b^2-4ac}}{2a}[/$]
bが偶数のとき ([$]b=2b'[/$])
[$]x=\frac{-b'\pm\sqrt{b'^2-4ac}}{a}[/$]

[$]D = b^2 - 4ac[/$] 
([$]D/4 = b'^2 - ac[/$] )


<図形と方程式>
① 2点間の距離
A( x1 , y1 ) , B ( x 2 , y 2 ) のとき

( x1 - x 2 ) 2 + ( y1 - y 2 ) 2
平方完成により、
② m : n に分ける点
A( x1 , y1 ) , B ( x 2 , y 2 ) のとき、線分 AB を m : n に分ける点は、
æ nx1 + mx 2 ny1 + my 2 ö
,
ç
÷
m+n ø
è m+n
点 A(a, b) に関して2点 P ( x1 , y1 ) , Q ( x 2 , y 2 ) が対称なとき、
直線の下部: y < ax + b
直線の方程式
y 2 - y1
( x - x1 )
x 2 - x1
注)分母、または、分子が0のときは座標軸と平行な直線
x = x1 , y = y1 となる。
2直線の位置関係
2直線の傾きが、 m1 , m2 のとき
平行: m1 = m2
垂直: m 2 = -
1
m1
(一致の場合も平行に含める)
(または、 m1 × m2 = -1 )
さらに、余裕があれば以下の公式も知っていると良い
一般形の場合は、 a1 x + b1 y + c1 = 0
a 2 x + b2 y + c 2 = 0
平行: a1b2 - a 2 b1 = 0
垂直: a1 a 2 + b1b2 = 0
⑦
交点の数に関しては、判別式の利用か、中心と直線までの距離
を利用して調べることが出来る。
⑪ 不等式と領域
直線の上部: y > ax + b
が成り立つ。
傾き m で、点 ( x1 , y1 ) を通る: y - y1 = m( x - x1 )
⑥
2
接線: ( x1 - a )( x - a ) + ( y1 - b)( y - b) = r 2
点に関して対称な点
2点 ( x1 , y1 ) ( x 2 , y 2 ) を通る: y - y1 =
接線: x1 x + y1 y = r
接点が点 ( x1 , y1 ) で中心 ( a, b) の円のとき
æ x + x 2 + x 3 y1 + y 2 + y 3 ö
の座標は、 ç 1
,
÷
3
3
è
ø
⑤
となり、中心 ( a, b) で半径 r の円を得る
接点が点 ( x1 , y1 ) で原点を中心とする円のとき
3点 A( x1 , y1 ) , B ( x 2 , y 2 ), C ( x3 , y 3 ) を頂点とする△ ABC の重心 G
x1 + x 2
y + y2
,b = 1
2
2
標準形: ( x - a ) 2 + ( y - b) 2 = r 2
⑩ 円と直線の関係
注) mn < 0 のとき外分点
③ 三角形の重心
a=
a2 + b2
⑨ 円の方程式
一般形: x 2 + y 2 + lx + my + n = 0
