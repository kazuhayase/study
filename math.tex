$\pm$
$$
|a| =\begin{cases}
      a & (\text{if} a \geq 0 )\\
      -a & (\text{if}a < 0 \text{のとき})\\
     \end{cases}
$$

$$
\begin{cases}
\alpha + \beta = -\frac{b}{a}\\
\alpha  \beta = \frac{c}{a}\\
\end{cases}
$$

$\frac
$

のときは異なる実数解が2つ
のときは重解(実数解が1つ)
のときは実数解はない


[$]x=\frac{-b\pm\sqrt{b^2-4ac}}{2a}[/$]
bが偶数のとき ([$]b=2b'[/$])
[$]x=\frac{-b'\pm\sqrt{b'^2-4ac}}{a}[/$]

[$]D = b^2 - 4ac[/$] 
([$]D/4 = b'^2 - ac[/$] )


<図形と方程式>
① 2点間の距離
A( x1 , y1 ) , B ( x 2 , y 2 ) のとき

( x1 - x 2 ) 2 + ( y1 - y 2 ) 2
平方完成により、
② m : n に分ける点
A( x1 , y1 ) , B ( x 2 , y 2 ) のとき、線分 AB を m : n に分ける点は、
æ nx1 + mx 2 ny1 + my 2 ö
,
ç
÷
m+n ø
è m+n
点 A(a, b) に関して2点 P ( x1 , y1 ) , Q ( x 2 , y 2 ) が対称なとき、
直線の下部: y < ax + b
直線の方程式
y 2 - y1
( x - x1 )
x 2 - x1
注)分母、または、分子が0のときは座標軸と平行な直線
x = x1 , y = y1 となる。
2直線の位置関係
2直線の傾きが、 m1 , m2 のとき
平行: m1 = m2
垂直: m 2 = -
1
m1
(一致の場合も平行に含める)
(または、 m1 × m2 = -1 )
さらに、余裕があれば以下の公式も知っていると良い
一般形の場合は、 a1 x + b1 y + c1 = 0
a 2 x + b2 y + c 2 = 0
平行: a1b2 - a 2 b1 = 0
垂直: a1 a 2 + b1b2 = 0
⑦
交点の数に関しては、判別式の利用か、中心と直線までの距離
を利用して調べることが出来る。
⑪ 不等式と領域
直線の上部: y > ax + b
が成り立つ。
傾き m で、点 ( x1 , y1 ) を通る: y - y1 = m( x - x1 )
⑥
2
接線: ( x1 - a )( x - a ) + ( y1 - b)( y - b) = r 2
点に関して対称な点
2点 ( x1 , y1 ) ( x 2 , y 2 ) を通る: y - y1 =
接線: x1 x + y1 y = r
接点が点 ( x1 , y1 ) で中心 ( a, b) の円のとき
æ x + x 2 + x 3 y1 + y 2 + y 3 ö
の座標は、 ç 1
,
÷
3
3
è
ø
⑤
となり、中心 ( a, b) で半径 r の円を得る
接点が点 ( x1 , y1 ) で原点を中心とする円のとき
3点 A( x1 , y1 ) , B ( x 2 , y 2 ), C ( x3 , y 3 ) を頂点とする△ ABC の重心 G
x1 + x 2
y + y2
,b = 1
2
2
標準形: ( x - a ) 2 + ( y - b) 2 = r 2
⑩ 円と直線の関係
注) mn < 0 のとき外分点
③ 三角形の重心
a=
a2 + b2
⑨ 円の方程式
一般形: x 2 + y 2 + lx + my + n = 0


(1) 両辺をxで微分すると 
$
8x+18yy'=0
$
よって
$
y'=\frac{dy}{dx}=-\frac{8x}{18y}=-\frac{4x}{9y}
$

$
\frac{d^2y}{dx^2}=-\frac{4}{9}\cdot\frac{(x)'y-xy'}{y^2}=-\frac{4}{9}\cdot\frac{y-x(-\frac{4x}{9y})}{y^2}
$
分母分子に$ 9y$を掛けて, 
$
= -\frac{4}{9}\cdot\frac{9y^2+4x^2)}{9y^3}
$
元のx,yが満たす陰関数の式を入れて
$
= -\frac{4}{9}\cdot\frac{36}{9y^3}
= -\frac{16}{9y^3}
$

(2) 両辺をxで微分する と$ 2x -\{(x)'y+xy'\}+4yy'=0$
よって $(x-4y)y'=2x-y$ より $\frac{dy}{dx}=\frac{2x-y}{x-4y}$

(3) 両辺をxで微分すると $\cos{x}-(-\sin{y})y'=0$
$\therefore \frac{dy}{dx}=-\frac{\cos{x}}{\sin{y}}$

x>0なので真数(log の大きい方の数字)が正であることを確認。

両辺正なので,自然対数をとると

$
\log{y}=\log{x^x} \Leftrightarrow 
\log{y}=x\log{x} 
$

両辺xで微分

$
\frac{1}{y}\cdot y'=\log{x}+x\cdot\frac{1}{x}
$

両辺yをかける
$
y'=(\log{x}+1)y=(\log{x}+1)x^x
$

両辺正なので,自然対数をとると
$
\log{y}=\log{x^{x^x}} \Leftrightarrow \log{y}=x^x\log{x}
$
両辺xで微分 
$
\frac{1}{y}\cdot y'=(x^x)'\log{x}+x^x\cdot\frac{1}{x}
$
両辺に y をかけて整理すると
$
y'=\{(\log{x}+1)x^x\cdot\log{x}+x^{x-1}\}y
$
\therefore y'=\{(\log{x}+1)x\log{x}+1\}x^{x^x+x-1}

(4)

両辺の絶対値の自然対数をとると
$
\log{|y|}=\log{|(x-a_1)(x-a_2)\cdots(x-a_n)|}=\sum^{n}_{k=1}\log{|x-a_k|}
$
両辺 x で微分すると
$
\frac{1}{y}\cdoty'=\sum^{n}{k=1}\frac{1}{x-a_k}
$
両辺に y をかけて整理してすると
$
y'=y\sum^{n}{k=1}\frac{1}{x-a_k}
$
$
\therefore y'=(x-a_1)(x-a_2)\cdots(x-a_n)\sum^{n}{k=1}\frac{1}{x-a_k}
$

$
\frac{dx}{dy}=\frac{1}{\frac{dy}{dx}}=\frac{1}{f'(x)}
$

(1)  
$
\frac{dx}{dy}=2y+4
$
より
$
\frac{dy}{dx}=\frac{1}{\frac{dx}{dy}}=\frac{1}{2y+4}
$
$
y^2+4y-1-x=0  
$
より
$
y=-2\pm\sqrt{x+5}
$
$
\therefore \frac{dy}{dx}=\frac{1}{2y+4}==\frac{1}{2(-2\pm\sqrt{x+5})+4} = \pm\frac{1}{2\sqrt{x+5}}
$

(-\frac{\pi}{2}<y<\frac{\pi}{2})

(2)
$
\frac{dx}{dy}=\cos{y}
$
より
$
\frac{dy}{dx}=\frac{1}{\frac{dx}{dy}}=\frac{1}{\cos{y}}=\frac{1}{\sqrt{1-\sin^2{y}}}=\frac{1}{\sqrt{1-x^2}} (\therefore \cos{y}>0)
$
(3) $
\frac{dx}{dy}=\frac{1}{\cos^2{y}}
$ より$
\frac{dy}{dx}=\frac{1}{\frac{dx}{dy}}=\frac{1}{\frac{1}{\cos^2{y}}}=\cos^2{y}=\frac{1}{\tan^2{y}}=\frac{1}{1+x^2}
$

$
x\geq 0$ で定義される関数 $f(x)=xe^x$ とその逆関数 $g(x)$ について、次を求めよ

(1) $f'(1), f''(1))$
(2) $g'(e), g''(e))$
--
(1 )
$
f'(x)=(x)'e^x+x(e^x)'=(1+x)e^x
f''(x)=\{(1+x)e^x\}' = (1+x)'e^x + (1+x)(e^x)' = (2+x)e^x
$
$
\therefore f'(1)=2e, f''(1)=3e
$

(2)
$
y=xe^x
$
の逆関数は
$
x=ye^y
$
$
\frac{dx}{dy}=(1+y)e^y
$
より
$
g'(x)=\frac{dy}{dx}=\frac{1}{(1+y)e^y}
$
$
g''(x)=\frac{d}{dx}\{\frac{1}{(1+y)e^y}\}=-\frac{\{(1+y)e^y\}'}{\{(1+y)e^y\}^2}\cdot\frac{dy}{dx}=-\frac{(2+y)e^y}{\{(1+y)e^y\}^3}
$
$x=ye^y$において, $x=e$ のとき$y=1$ である.
$
\therefore g'(e)=\frac{1}{(1+1)e^1}=\frac{1}{2e}, g''(e)=-\frac{(2+1)e^1}{\{(1+1)e^1\}^3}=-\frac{3}{8e^2}
$

$
\frac{dy}{dx}, 
\frac{d^2y}{dx^2}
$

(1)
$
\begin{cases}
 x=t-\sin{t}\\
 y=1-\cos{t}
\end{cases}
$
(2)
$
\begin{cases}
 x=\cos^3{t}\\
 y=\sin^3{t}
\end{cases}
$

(3)
$
\begin{cases}
 x=\frac{t^2-1}{t^2+1}\\
 y=\frac{2t}{t^2+1}
\end{cases}
$

(4)
$
\begin{cases}
 x=2\cos{t}+\cos{2t}\\
 y=2\sin{t}-\sin{2t}
\end{cases}
$

---
(1) 
$
\frac{dx}{dt}=1-\cos{t}, \frac{dy}{dt}=\sin{t}
$ より $
\frac{dy}{dx}=\frac{\frac{dy}{dt}}{\frac{dx}{dt}} = \frac{\sin t}{1-\cos t}
\frac{d^2y}{dx^2}= \frac{d}{dt}(\frac{dy}{dx})\cdot\frac{dt}{dx}=\frac{d}{dt}(\frac{\sin t}{1-\cos t})\cdot\frac{dt}{dx}
=\frac{(\sin t)'(1-\cos t)-\sin t(1-\cos t)'}{(1-\cos t)^2} \cdot \frac{1}{1-\cos t}
=\frac{\cos t(1-\cos t)-\sin^2 t}{(1-\cos t)^3}
=\frac{\cos t-(\cos^2 t + \sin^2 t)}{(1-\cos t)^3}
= \frac{\cos t -1}{(1-\cos t)^3}=-\frac{1-\cos t}{(1-\cos t)^3}
=-\frac{1}{(1-\cos t)^2}
$
( 本問の関数は、大学入試2大頻出有名曲線の1つのサイクロイド)
(2)
$
\frac{dx}{dt}=3\cos^2 t(\cos t)'=-3\cos^2 t \sin t, 
\frac{dy}{dt}=3\sin^2 t(\sin t)'=-3\sin^2 t \cos t
$
$
\frac{dy}{dx}=\frac{\frac{dy}{dt}}{\frac{dx}{dt}}=\frac{3\sin^2 t \cos t}{-3\cos^2 t \sin t}=-\frac{\sin t}{\cos t}=-\tan t
$
$
\frac{d^2y}{dx^2}=\frac{d}{dt}(\frac{dy}{dx})\cdot\frac{dt}{dx}=\frac{d}{dt}(-\tan t)\cdot\frac{dt}{dx}
= -\frac{1}{\cos^2 t}\cdot\frac{1}{-3\cos^2 t\sin t}=\frac{1}{3\cos^4 t\sin t}
$
( 本問の関数は、大学入試2大頻出有名曲線のもう1つのアステロイド. アステロイドは陰関数表示できるので別解がある)
$
\cos t = x^{\frac{1}{3}}, \sin t = y^{\frac{1}{3}}
$
とし、tを消去すると $
x^{\frac{2}{3}}+y^{\frac{2}{3}}=1
$
両辺をxで微分して$
\frac{2}{3}x^{-\frac{1}{3}}+\frac{2}{3}y^{-\frac{1}{3}}\frac{dy}{dx}=0
$ よって $
\frac{dy}{dx}=-(\frac{x}{y}^{-\frac{1}{3}})=-(\frac{y}{x}^{\frac{1}{3}})=-\sqrt[3]{\frac{y}{x}}
\frac{d^2y}{dx^2}=-\frac{1}{3}(\frac{y}{x})^{-\frac{2}{3}}(\frac{y}{x})'=-\frac{1}{3}(\frac{y}{x})^{-\frac{2}{3}}\frac{y'x-y(x)'}{x^2}
=-\frac{1}{3}(\frac{y}{x})^{-\frac{2}{3}}\frac{-(\frac{y}{x})^{-\frac{1}{3}}x-y}{x^2}
=-\frac{1}{3}(\frac{y}{x})^{-\frac{2}{3}}\frac{-x^{\frac{2}{3}}y^{\frac{1}{3}}-y}{x^2}
=-\frac{y^{-\frac{2}{3}}\cdot y^{\frac{1}{3}}(x^{\frac{2}{3}}+y^{\frac{2}{3}})}{3\cdot x^{-\frac{2}{3}}\cdot x^2}
=-\frac{y^{-\frac{2}{3}}\cdot y^{\frac{1}{3}}(x^{\frac{2}{3}}+y^{\frac{2}{3}})}{3\cdot x^{-\frac{2}{3}}\cdot x^2}
=-\frac{y^{-\frac{1}{3}}(x^{\frac{2}{3}}+y^{\frac{2}{3}})}{3 x^{\frac{4}{3}}}
=-\frac{x^{\frac{2}{3}}+y^{\frac{2}{3}}}{3 x^{\frac{4}{3}}y^{\frac{1}{3}}}=\frac{1}{3x\sqrt[3]{xy}}
$
(3)
$
\frac{dx}{dt}
=\frac{(t^2-1)'(t^2+1)-(t^2-1)(t^2+1)'}{(t^2+1)^2}
=\frac{2t(t^2+1)-2t(t^2-1)}{(t^2+1)^2}
=\frac{4t}{(t^2+1)^2}
\frac{dy}{dt}
=\frac{(2t)'(t^2+1)-2t(t^2+1)'}{(t^2+1)^2}
=\frac{2(t^2+1)-2t\cdot 2t}{(t^2+1)^2}
=\frac{2(1-t^2)}{(t^2+1)^2}
\frac{dy}{dx} = \frac{\frac{dy}{dt}}{\frac{dx}{dt}}
=\frac{\frac{2(1-t^2)}{(t^2+1)^2}}{\frac{4t}{(t^2+1)^2}}
=\frac{1-t^2}{2t}
\frac{d^2y}{dx^2} = \frac{d}{dt}(\frac{dy}{dx})\cdot\frac{dt}{dx}
= \frac{d}{dt}(\frac{1-t^2}{2t})\cdot\frac{dt}{dx}
= \frac{(1-t^2)'2t-(1-t^2)(2t)')}{4t^2}\cdot\frac{(t^2+1)^2}{4t}=-\frac{(t^2+1)^3}{8t^3}
$
本文は円の媒介変数表示の1つで、$
x^2+y^2
$を計算すると$
x^2+y^2=1
$となってtが消去でき、陰関数の微分法で求められる

(4)
$
\frac{dx}{dt}
=-2\sin t-\sin{2t}(2t)'
=-2\sin t-2\sin{2t}
=-2(\sin t+\sin{2t})
=-4\sin{\frac{3}{2}t}\cos{\frac{t}{2}}
\frac{dy}{dt}
=2\cos t-\cos{2t}(2t)'
=2\cos t-2\cos{2t}
=-2(\cos{2t}-\cos t)
=4\sin{\frac{3}{2}t}\sin{\frac{t}{2}}
\frac{dy}{dx} = \frac{\frac{dy}{dt}}{\frac{dx}{dt}}
=\frac{4\sin{\frac{3}{2}t}\sin{\frac{t}{2}}}{-4\sin{\frac{3}{2}t}\cos{\frac{t}{2}}}
=-\frac{\sin{\frac{t}{2}}}{\cos{\frac{t}{2}}} = -\tan{\frac{t}{2}}
\frac{d^2y}{dx^2} = \frac{d}{dt}(\frac{dy}{dx})\cdot\frac{dt}{dx}
=\frac{d}{dt}(-\tan{\frac{t}{2}})\cdot\frac{dt}{dx}
=-\frac{(\frac{t}{2})'}{\cos^2{\frac{t}{2}}}\cdot\frac{1}{-4\sin{\frac{3}{2}t}\cos{\frac{t}{2}}}
=\frac{1}{8\cos^3{\frac{t}{2}}\sin{\frac{3}{2}t}}
$

$
f''(x)=\frac{d}{dx}f'(x)
f'''(x)=\frac{d}{dx}f''(x)
\frac{d^n}{dx^n} f(x)=f^(n)(x)
$
(n 階微分)

$
y=xe^x
$ の第 n 次導関数を求めなさい。

<基本的な関数の微分>
$
y=c \Rightarrow y'=0    (c は定数)
y=x^n \Rightarrow y'=nx^{n-1}   (n は実数(整数でなくても良い))
y=\sin x \Rightarrow y' = \cos x
y=\cos x \Rightarrow y' = -\sin x
y=\tan x \Rightarrow y' = \frac{1}{\cos^2 x}
y=\log{|x|} \Rightarrow y' = \frac{1}{x}
y=\log[a]{|x|} \Rightarrow y' = \frac{1}{x\log{a}}
y=e^x \Rightarrow y' = e^x
y=a^x \Rightarrow y' = a^x\log{a}   (a>0, a\neq 1)
$

①関数f(x)が区間[a,b]でf'(x)を持てば、$
\frac{f(b)-f(a)}{b-a}=f'(c)
$となるcが、区間(a,b)に少なくとも1つ存在する
②表現の仕方を変えると以下の式を満たす $\theta$ が存在する。
$
f(a+h)=f(a)+hf'(a+\theta h)  (0<\theta<1)
$

極限値を求める問題にも応用される

次の不等式が成り立つことを示せ.
(1) $
e^a(b-a) <  e^b-e^a < e^b(b-a)   (a-b)
$

(2) $
0<\sin b - \sin a < b-a  (0<a<b<\frac{\pi}{2})
$

(3)$
a<\frac{ab}{b-a}\log{\frac{b}{a}}<b (0<a<b)
$

(4) $
\log{(\log{q})}- \log{(\log{p})} < \frac{q-p}{e} (e\leq p<q)
$ [ 名古屋大]


いずれも、「連続で微分可能な関数」 f(x)を見つけて,平均値の定理を利用する。
(1) $ f(x)=e^x $
(2) $ f(x)=\sin x $
(3) $ f(x)=\log x $
(4) $ f(x)=\log(\log x) $

(1) $ f(x)=e^x $とすると, f(x) は連続で微分可能な関数である.  $f'(x)=e^x$
(注)ここまでの「f(x)... 連続で微分可能な関数である.  」は常套文句なので(2)以降は省略

平均値の定理より $\frac{e^b-e^a}{b-a}=e^c$  (a<c<b)を満たす実数cが存在する
a<c<b より $e^a<e^c<e^b$
よって $e^a<\frac{e^b-e^a}{b-a}<e^b$.   b-a>0より $e^a(b-a)<\frac{e^b-e^a}{b-a}<e^b(b-a)$.   

(2) 平均値の定理より $\frac{\sin b-\sin a}{b-a}=\cos c$  (a<c<b)を満たす実数cが存在する
$ (0<a<b<\frac{\pi}{2}) $ より $0<\cos c<1$ よって $0<\frac{\sin b-\sin a}{b-a}<1$
$b-a>0  $より $0<\sin b - \sin a < b-a$

(3) 平均値の定理より $\frac{\log b-\log a}{b-a}=\frac{1}{c} $  (a<c<b)を満たす実数cが存在する
a<c<b より $\frac{1}{b}<\frac{1}{c}<\frac{1}{a}$ よって $\frac{1}{b}<\frac{\log b - \log a}{b-a}<\frac{1}{a}$
各辺にab(>0) をかける
a<\frac{ab}{b-a}\log{\frac{b}{a}}<b (0<a<b)

(4) 平均値の定理より $\frac{\log(\log q) -\log(\log p)}{q-p}=\frac{1}{c\log c} $  (p<c<q)を満たす実数cが存在する
$e\leq p<c<q$より $c<c\log c$よって $e<c\log c$ 両辺の逆数をとって $\frac{1}{c\log c}<\frac{1}{e}$
よって$\frac{\log(\log q) -\log(\log p)}{q-p} < \frac{1}{e}$
$\therefore \log(\log q)-\log(\log p)<\frac{q-p}{e}$

$
f(a+h) \cong f(a)+f'(a)h
x=a+h
f(x) \cong f(a)+f'(a)(x-a)
f(x) \cong f(0)+f'(0)x
(1+x)\p \cong 1+px
$

$
\int x^n dx = \frac{x^{n+1}}{n+1} +C
\int \frac{1}{x} dx = \log |x| +C
\int \sin x dx  = -\cos x +C
\int \cos x dx  = \sin x +C
\int e^x dx  = e^x +C
\int a^x dx  = \frac{a^x}{\log a}+C
$


$
x=g(t)
$と置く.両辺それぞれで微分$
dx = g'(t)dt
$ より$
\inx f(x) dx = \inx f(g(t))g'(t)dt = \int f(g(t)) \frac{dx}{dt}dt
$
「非積分関数を新しい変数tで置き換える」
「dx/dtを掛ける」

(1) $
\int x(2-x)^4 dx
$

t=2-x と置換。\frac{dt}{dx} = -1 より$
\int x(2-x)^4dx = \int (2-t)t^4\cdot(-1) dt = \int (t^5-2t^4)  dt 
= \frac{1}{6} t^6 + \frac{2}{5} t^5 +C = \frac{1}{6} (2-x)^6 + \frac{2}{5} (2-x)^5 +C 
$

(2) $
\int \sqrt{x+1}(x+2)dx
$

$\sqrt{x+1}=t$と置換。$x=t^2-1$で両辺微分し、$\frac{dx}{dt}=2t$ よって$
\int \sqrt{x+1}(x+2) dx = \int (t^3+t)2t dt = \int 2t^4+2t^2 dt 
= \frac{2}{5}t^5+\frac{2}{3}t^3+C = \frac{2}{5}(x+1)^{\frac{5}{2}}+\frac{2}{3}(x+1)^{\frac{3}{2}}+C

と単調に変化するとき、xはa→bと単調に変化するものとします。

(1)$
\int_{0}^{1}\frac{x}{(1-x^2)^{1/3}}dx
$

$1-x^2=t$ と置換. $-2x dx = dt \leftrightarrow -2x \frac{dx}{dt}=1$かつ$x: 0\rightarrow 1$ のとき、$t: 1\rightarrow 0$となるから$
\int_{0}^{1}\frac{x}{(1-x^2)^{1/3}}dx 
= - \int_{1}^{0}\frac{1}{2}t^{-1/3}dt  
= [\frac{3}{4}t^{2/3}]^1_0
$

(2)$
\int_{0}^{1} \frac{dx}{1+x^2}
$
置換がポイント.  $x=\tan t (-\pi/2 < t < \pi/2)$
$\frac{dx}{dt}=\frac{1}{\cos^2 t}$,  $x: 0\rightarrow 1$,  $t: 0\rightarrow \pi/4$だから
$
\int_{0}^{1}\frac{dx}{1+x^2}=\int_{0}^{\pi/4}\frac{1}{1+\tan^2 t}\frac{dt}{cos^2 t}
=\int_{0}^{\pi/4} \cos^2 t \frac{dt}{cos^2 t}
=\int_{0}^{\pi/4} dt = \frac{\pi}{4}
$

\begin{eqnarray*}
 \int_{0}^{1}\frac{dx}{1+x^2} &=&\int_{0}^{\pi/4}\frac{1}{1+\tan^2 t}\frac{dt}{cos^2 t}\\
&=&\int_{0}^{\pi/4} \cos^2 t \frac{dt}{cos^2 t}\\
&=&\int_{0}^{\pi/4} dt = \frac{\pi}{4}
\end{eqnarray*}


$a>0$ のとき$\int_{0}^{a} \frac{1}{x^2+z^2}dx$を計算.  
$x=a\tan t$と置換. 被積分関数は $\frac{1}{x^2+a^2}=\frac{1}{a^2(1+\tan^2 t)}=\frac{\cos^2 t}{a^2}$. 
$\frac{dx}{dt}=\frac{a}{\cos^2 t}$, $x:0\rightarrow a$, $t:0\rightarrow \pi/4$

\begin{eqnarray*}
\int_{0}^{a}\frac{1}{x^2+a^2}dx&=&\int_{0}^{\pi/4} \frac{\cos^2 t}{a^2}\frac{a}{\cos^2 t}dt \\
&=& \int_{0}^{\pi/4}\frac{1}{a}dt\\
&=&\frac{\pi}{4a}
\end{eqnarray*}

$a>0$ のとき$\int_{0}^{a} \sqrt{a^2-x^2}dx$を計算.  
$x=a\sin t$と置換. 被積分関数は $\sqrt{a^2-x^2}=\sqrt{a^2}(1-sin^2 t)=\sqrt{a^2}\cos^2 t=a|\cos t|$
$\frac{dx}{dt}=a\cos t$, 
$x:0\rightarrow a$, $t:0\rightarrow \pi/2$

\begin{eqnarray*}
\int_{0}^{a} \sqrt{a^2-x^2}dx &=& \int_{0}^{\pi/2} a\cos t\cdot a\cos t dt \\
&=& a^2\int_{0}^{\pi/2} \cos^2 t dt\\
&=& a^2\int_{0}^{\pi/2} \frac{1+\cos 2t}{2}dt\\
&=&\frac{a^2\pi}{4}
\end{eqnarray*}

(1) $\int x \sin x dx =?$

xを微分, sin x を積分. 

\begin{eqnarray*}
\int x \sin x dx &=& x\cdot(-\cos x)-\int 1\cdot(-\cos x)dx\\
&=& -x\cos x+\int \cos x dx\\
&=& -x\cos x+\sin x +C
\end{eqnarray*}

(2) $\int x e^x dx =?$

xを微分, e を積分. 

\begin{eqnarray*}
\int x e^x dx &=& x\cdot e^x - \int 1\cdot e^x dx\\
&=& x e^x - \int e^x dx\\
 &= & x e^x - e^x +C
\end{eqnarray*}

(3) $\int x\cos 2x dx=?$

xを微分, cos x を積分. 
\begin{eqnarray*}
\int x\cos 2x dx &=& x\cdot\frac{\sin 2x}{2}-\int 1\cdot\frac{\sin 2x}{2}dx\\ 
 &= & \frac{1}{2}x\sin 2x - \frac{1}{2}\int\sin 2x dx\\
&=& \frac{1}{2}x \sin 2x + \frac{1}{4}\cos 2x +C
\end{eqnarray*}

(4) $\int \log_e x dx=?$
(hint; $\log_e x = 1\times\log_e x$)
暗記でも良いが,部分積分もできるように

logを微分, 1 を積分. 
\begin{eqnarray*}
 \int log_e x dx = \int(\log_e x)(x)' dx &=& x\log_e x - \int\frac{1}{x}\cdot x dx\\
&=& x\log_e x - \int 1 dx
&=& x\log_e x - x +C
\end{eqnarray*}

\begin{eqnarray*}
 \int x^2\cos 2x dx &=& \frac{1}{2}x^2\sin 2x - \frac{1}{2}\cdot 2\int x \sin 2x dx\\
&=& \frac{1}{2}x^2\sin 2x - \{x(-\frac{\cos 2x}{2})-\int 1(-\frac{\cos 2x}{2}) dx\}\\
&=& \frac{1}{2}x^2\sin 2x + \frac{1}{2}x \cos 2x-\frac{1}{4}\sin 2x +C\\
&=& \frac{2x^2-1}{4}\sin 2x+\frac{1}{2}x\cos 2x +C
\end{eqnarray*}

$\int e^x\sin x dx=?$
\begin{eqnarray*}
 I=\int e^x\sin x dx &=& e^x \sin x - \int e^x \cos x dx\\
&=& e^x \sin x - \{e^x\cos x - \int e^x(-\sin x)dx\}\\
&=& e^x \sin x - e^x\cos x - \int e^x(\sin x)dx \\
\Leftrightarrow 2I &=& e^x\sin x - e^x\cos x\\
I=\int e^x \sin x dx &=& \frac{1}{2}(e^x\sin x - e^x\cos x)+C
\end{eqnarray*}

$
\int \frac{f'(x)}{f(x)}dx = \log|f(x)|
$

$
\int_{a}^{b}f(x)dx = \left[ F(x)\right]^b_a=S
\int_{-a}^{a}\sqrt{a^2-x^2}dx = \frac{\pi a^2}{2}
\int_{a}^{b}cf(x)dx = c\int_{a}^{b}f(x)dx 
\int_{a}^{a}f(x)dx = 0
\int_{a}^{b}f(x)dx = -\int_{b}^{a}f(x)dx 
\int_{a}^{b}f(x)dx + \int_{b}^{c}f(x)dx = \int_{a}^{c}f(x)dx 
\int_{a}^{b}f(x)dx \pm \int_{a}^{b}g(x)dx  = \int_{a}^{b} \{f(x)\pm g(x)\}dx 
\int_{-a}^{a}f(x)dx = 
\begin{cases}
2 \int_{0}^{a}f(x)dx & f(x): \text{偶関数}\\
0 & f(x): \text{奇関数}
\end{cases}
f(x)\geq g(x) \Rightarrow \int_{a}^{b}f(x)dx \geq \int_{a}^{b} g(x)dx
\left\{\int_{a}^{b}f(x)g(x)dx\right\}^2 \leq \left( \int_{a}^{b}\{f(x)\}^2dx\right)\left( \int_{a}^{b}\{g(x)\}^2dx\right)
$

$
\int_{a}^{b}f(x)dx = \int_{a}^{b}f(a+b-x)dx
\int_{a}^{b}(x-a)(x-b)dx = -\frac{(b-a)^3}{6}
\frac{d}{dx}\int_{a}^{x}f(t)dt=f(x)
$

$
\lim\limits_{n\to\infty}\left(\frac{1}{n+1}+\frac{1}{n+2}+\cdots+\frac{1}{n+an}\right)
\lim\limits_{n\to\infty}\frac{1}{n^2}\left(\sqrt{n^2-1}+\sqrt{n^2-4}+\cdots+\sqrt{n^2-(n-1)^2}\right)
\lim\limits_{n\to\infty}\left(\frac{(1+2+\cdots+n)(1^4+2^4+\cdots+n^4)}{(1^2+2^2+\cdot+n^2)(1^3+2^3+\cdot+n^3)}\right)
$

\begin{eqnarray*}
&&\lim\limits_{n\to\infty}\left(\frac{1}{n+1}+\frac{1}{n+2}+\cdots+\frac{1}{n+an}\right)\\
&=&\lim\limits_{n\to\infty}\sum\limits_{k=1}^{an}\frac{1}{n}\cdot\frac{1}{1+\frac{k}{n}}\\
&=&\int_{0}^{a}\frac{1}{1+x} dx\\
&=&\left[\log(1+x)\right]^a_0\\
&=&\log(1+a)
\end{eqnarray*}

\begin{eqnarray*}
&&\lim\limits_{n\to\infty}\frac{1}{n^2}\left(\sqrt{n^2-1}+\sqrt{n^2-4}+\cdots+\sqrt{n^2-(n-1)^2}\right)\\
&=&\lim\limits_{n\to\infty}\sum\limits_{k=1}^{n-1}\frac{1}{n}\sqrt{1-\left(\frac{k}{n}\right)^2}\\
&= & \int_{0}^{1}\sqrt{1-x^2}dx\\
&=&\frac{\pi}{4}
\end{eqnarray*}

\begin{eqnarray*}
&&\lim\limits_{n\to\infty}\frac{1}{n^2}\left(\frac{(1+2+\cdots+n)(1^4+2^4+\cdots+n^4)}{(1^2+2^2+\cdot+n^2)(1^3+2^3+\cdot+n^3)}\right)\\
 &= & \lim\limits_{n\to\infty}\left(\frac{(1/n+2/n+\cdots+n/n)((1/n)^4+(2/n)^4+\cdots+(n/n)^4)}{((1/n)^2+(2/n)^2+\cdot+(n/n)^2)((1/n)^3+(2/n)^3+\cdot+(n/n)^3)}\right)\\ 
 &= & \lim\limits_{n\to\infty}\frac{\frac{1}{n}\sum\limits^n_{k=1}\left(\frac{k}{n}\right)\cdot\frac{1}{n}\sum\limits^n_{k=1}\left(\frac{k}{n}\right)^4}{\frac{1}{n}\sum\limits^n_{k=1}\left(\frac{k}{n}\right)^2+\frac{1}{n}\sum\limits^n_{k=1}\left(\frac{k}{n}\right)^3}\\
&=&\frac{\int_{0}^{1}x dx\cdot\int_{0}^{1}x^4 dx}{\int_{0}^{1}x^2 dx\cdot\int_{0}^{1}x^3 dx}\\
&=& \frac{\frac{1}{2}\cdot\frac{1}{5}}{\frac{1}{3}\cdot\frac{1}{4}}\\
&=&\frac{6}{5}
\end{eqnarray*}

$
S=\int_{a}^{b}|f(x)|dx
S=\int_{a}^{b}|f(x)-g(x)|dx
V=\int_{a}^{b}S(x)dx
V=\pi\int_{a}^{b}\{f(x)\}^2dx
s=\int_{a}^{b}\sqrt{1+\{f'(x)\}^2}dx
s=\int_{a}^{b}\sqrt{\{f'(t)\}^2+\{g'(t)\}^2}dx
$

$
x_1 x+y_1 y=r^2
\frac{x^2}{a^2}+\frac{y^2}{b^2}=1
\frac{x_1 x}{a^2}+\frac{y_1 y}{b^2}=1
\frac{x^2}{a^2}-\frac{y^2}{b^2}=1
y^2=4px
$

複素数平面
共役複素数

共役複素数

z, w を複素数とすると、以下が成り立つ。

共役複素数の和と差
z+z¯¯¯ は実数
z−z¯¯¯ は純虚数または 0
実数、純虚数の共役
z が実数 ⟺ z=z¯¯¯
z が純虚数 ⟺ z=−z¯¯¯ かつ z≠0
和・差・積・商の共
$
\overline{z+w}=\overline{z}+\overline{w}
\overline{z-w}=\overline{z}-\overline{w}
\overline{zw}=\overline{z}\cdot\overline{w}
\overline{\left(\frac{z}{w}\right)}=\frac{\overline{z}}{\overline{w}}  \quad (w\neq 0)
\overline{\overline{z}}=z
$


z+w¯¯¯¯¯¯¯¯¯¯¯¯=z¯¯¯+w¯¯¯¯
z−w¯¯¯¯¯¯¯¯¯¯¯¯=z¯¯¯−w¯¯¯¯
zw¯¯¯¯¯¯=z¯¯¯⋅w¯¯¯¯
(zw)¯¯¯¯¯¯¯¯¯¯¯=z¯¯¯w¯¯¯¯ (w≠0)
共役の共役
z¯¯¯¯¯¯=z

複素数の絶対値
$
|z| = |x+yi| = \sqrt{x^2+y^2}
$
複素数平面における二点間の距離
$z=a+bi, w=c+di$
に対応する点 (a,b), (c,d) の距離は
$
|w-z| = \sqrt{(c-a)^2+(d-b)^2}
$
複素数の絶対値の性質
$
|z|=0 \Leftrightarrow z=0
|z|=|-z|=|\overline{z}|
z\overline{z}=|z|^2
|zw|=|z||w|
|\frac{z}{w}|=\frac{|z|}{|w|} \quad (w\neq 0)
$

複素数の極形式
複素数 zの偏角がθのとき、zを極形式で表すと
$
z=|z|(\cos\theta+i\sin\theta)
$

複素数の加法・減法
$
z_1=x_1+y_1i, z_2=x_2+y_2i
w=z_1+z_2
w=(x_1+x_2)+(y_1+y_2)i
w=z_1-z_2
w=(x_1-x_2)+(y_1-y_2)i
$
複素数の乗法
$
z_1=|z_1|(\cos\theta_1+i\sin\theta_1), z_2=|z_2|(\cos\theta_2+i\sin\theta_2)
z_1z_2=|z_1||z_2|(\cos(\theta_1+\theta_2)+i\sin(\theta_1+\theta_2))
$
複素数の除法
$
z_1=|z_1|(\cos\theta_1+i\sin\theta_1), z_2=|z_2|(\cos\theta_2+i\sin\theta_2)
\frac{z_1}{z_2}=\frac{|z_1|}{|z_2|}(\cos(\theta_1-\theta_2)+i\sin(\theta_1-\theta_2))
$

(複素数平面上)原点中心の回転
$
z=|z|(\cos\theta_z+i\sin\theta_z)
z'=z(\cos\theta+i\sin\theta)=|z|\{\cos(\theta_z+\theta)+i\sin(\theta_z+\theta)\}
z'=(z-\alpha)(\cos\theta+i\sin\theta)+\alpha
$
複素数zを複素数平面上で原点を中心に θ 回転させた複素数 z′ は

(複素数平面上)原点中心でない回転
1.平行移動, 2.回転, 3.平行移動

複素数zを複素数平面上で点αを中心に θ 回転させた複素数 z′ は

ドモアブルの定理
任意の実数 θ、任意の整数 n に対し
$
(\cos\theta+i\sin\theta)^n=\cos n\theta+i\sin n\theta
$
が成り立つ

1 の n 乗根
1 の n 乗根は、次の n 個の複素数 zk (k=0,1,⋯,n−1) である。

$z_k = \cos\frac{2k\pi}{n} + i\sin\frac{2k\pi}{n}$