\documentclass[report,gutter=10mm,fore-edge=10mm,uplatex,dvipdfmx]{jlreq}

\usepackage{lmodern}
\usepackage{amssymb,amsmath}
\usepackage{ifxetex,ifluatex}
\usepackage{actuarialsymbol}
\usepackage[]{natbib}
\RequirePackage{plautopatch}

% maru suji ① etc.
\usepackage{tikz}
\newcommand{\cir}[1]{\tikz[baseline]{%
\node[anchor=base, draw, circle, inner sep=0, minimum width=1.2em]{#1};}}

\usepackage{comment}

\begin{comment}

\ifnum0\ifxetex1\fi\ifluatex1\fi=0 % if pdftex
  \usepackage[T1]{fontenc}
  \usepackage[utf8]{inputenc}
  \usepackage{textcomp} % provide euro and other symbols
\else % if luatex or xetex
  \usepackage{unicode-math}
  \defaultfontfeatures{Scale=MatchLowercase}
  \defaultfontfeatures[\rmfamily]{Ligatures=TeX,Scale=1}
\fi
% Use upquote if available, for straight quotes in verbatim environments
\IfFileExists{upquote.sty}{\usepackage{upquote}}{}
\IfFileExists{microtype.sty}{% use microtype if available
  \usepackage[]{microtype}
  \UseMicrotypeSet[protrusion]{basicmath} % disable protrusion for tt fonts
}{}
\makeatletter
\@ifundefined{KOMAClassName}{% if non-KOMA class
  \IfFileExists{parskip.sty}{%
    \usepackage{parskip}
  }{% else
    \setlength{\parindent}{0pt}
    \setlength{\parskip}{6pt plus 2pt minus 1pt}}
}{% if KOMA class
  \KOMAoptions{parskip=half}}
\makeatother
\usepackage{xcolor}
\IfFileExists{xurl.sty}{\usepackage{xurl}}{} % add URL line breaks if available
\IfFileExists{bookmark.sty}{\usepackage{bookmark}}{\usepackage{hyperref}}
\hypersetup{
  hidelinks,
  pdfcreator={LaTeX via pandoc}}
\urlstyle{same} % disable monospaced font for URLs
\usepackage{longtable,booktabs}
% Correct order of tables after \paragraph or \subparagraph
\usepackage{etoolbox}
\makeatletter
\patchcmd\longtable{\par}{\if@noskipsec\mbox{}\fi\par}{}{}
\makeatother
% Allow footnotes in longtable head/foot
\IfFileExists{footnotehyper.sty}{\usepackage{footnotehyper}}{\usepackage{footnote}}

\end{comment}
%\makesavenoteenv{longtable}
\setlength{\emergencystretch}{3em} % prevent overfull lines
\providecommand{\tightlist}{%
  \setlength{\itemsep}{0pt}\setlength{\parskip}{0pt}}
\setcounter{secnumdepth}{-\maxdimen} % remove section numbering

\author{kazuyoshi}
\date{}

\newcommand{\problem}[1]{\subsubsection{#1}\setcounter{equation}{0}}
%\newcommand{\answer}[1]{\subsubsection{#1}}
\newcommand{\answer}[1]{\subsubsection{解答}}

%Pdf%\newcommand{\wakumaru}[1]{\framebox[3zw]{#1}}
\newcommand{\wakumaru}[1]{#1}





\begin{document}
\section{6.1 生命保険会社のリスクとソルベンシーの確保}
\problem{H27 生保2問題 2(2)、H21 生保2問題 3(1)}
ソルベンシー評価の意義について、簡潔に説明しなさい。また、現在の日本の法令等に基づく、静
的なソルベンシーの検証および動的なソルベンシーの検証について、それぞれのメリット・デメリット
を含め、簡潔に説明しなさい。
\answer{}
(ソルベンシー評価の意義)

 生命保険会社の使命は、保険事故発生に対して保険金の支払を全うすることであり、契約時に
約定された保険給付は、予定外の突発的な事態が起ころうとも、よほどのことがない限り保証
されるべきである。

 ソルベンシーとは、こうした保険契約上の債務を将来にわたり履行するための財政的基盤であ
る。

 債務履行にあたって、保険料の設定に十分な配慮がなされるのは当然だが、契約締結後におい
ても決算等機会があるごとにソルベンシーが確保されているかの検証を行い、必要に応じて対
策を講じていくことが求められる。

 このことからソルベンシー評価は、将来の債務履行の確度向上を図るうえでの重要な役割を担
うものと意義付けられる。

 生命保険会社の事業継続を前提とし、当該事業をとりまく様々なリスクを計測すること、およ
びそのリスクに対応するソルベンシーが十分であるかを適切に評価することが重要である。

 通常の予測可能なリスクへの対応として責任準備金を健全な保険数理・法令等に則り適正に積
み立て、通常の予測を超えるリスクに対応するために、狭義の責任準備金を超えて保有する支
払余力として広義の自己資本を確保することが求められる。


(日本の法令に基づく静的なソルベンシーの検証および動的なソルベンシーの検証)

○静的なソルベンシーの検証

 フォーミュラ方式によるソルベンシー・チェックであり、日本ではソルベンシー・マージン比
率や実質資産負債差額による検証が行なわれている

 フォーミュラ方式による検証は、実行可能性や検証可能性に優れており、全ての保険会社を統
一的に取り扱うことが可能なことから、客観的な指標として監督行政に活用されている

 一方、各保険会社固有のリスクが必ずしも反映されないことや、あくまで一時点の検証に過ぎ
ない、といったデメリットがあるため、動的なソルベンシーの検証と併せた検証が必要である

○動的なソルベンシーの検証

 将来のキャッシュフロー分析に基づくシミュレーションによるソルベンシー検証の方法であり、
日本では保険計理人の実務基準に基づく将来収支分析が規定されているほか、監督指針におい
てストレステストの自主的な実施が求められている

 会社の業務政策・投資戦略・ALM・市場戦略・配当(社員・契約者)・株主配当等を反映させる
ことで、会社固有のリスクや将来の変動に対するソルベンシー確保の検証を行うことが出来る。

 一方、計算実務が繁雑であること、計算結果の説明が必ずしも容易でないこと、恣意的なシナ
リオ設定の排除が難しい側面があること等のデメリットがある。
\problem{H9 生保1問題 1(6)}
次の①~③を適当な語句もしくは算式で埋めよ。
レディントンのイミュナイゼーション
$$
\frac{d}{dt}(\sum v^tA_t-\sum v^tB_t-\sum v^tL_t)=0
$$

および
①を満たすように、資産と負債の②
を調整すれば、そのポートフォリオは利率$i$の変動に対し③を持つということである。
($A_t, L_t$ はそれぞれ時間 $t$ における資産側、負債側のキャッシュフロー、また$v=\frac{1}{1+i}$ )

\answer{}
① $\frac{d^2(\sum v^tA_t - \sum v^tL_t )}{di^2}>0$
(レディントン条件)

② デュレーション
③ 免疫性


(補足) 生保2 6-8では、$i$ではなく $\delta$を用いて、以下の通り記載されている。

レディントンのイミュナイゼーション

「A=Lの場合に $D_A=D_L$, $\frac{d^2(A-L)}{d\delta^2}$としておくと、$\Delta\delta$ の正負に関わらず、$A'-L'>0$が成立する」

レディントン条件

$$
\frac{d^2}{d\delta^2}=\sum t^2\cdot A_t \cdot \exp(-\delta\cdot t)-\sum t^2\cdot L_t \cdot\exp(-\delta\cdot t)>0
$$

\section{6.2 静的なソルベンシーの検証(フォーミュラ方式のソルベンシー・チェック)}
\problem{H11 生保2問題 1(10)}
企業会計原則注解における負債の部に計上できる引当金の設定要件を 4 つ挙げよ。
\answer{}

将来の特定の費用又は損失であること

その発生が当期以前の事象に起因していること

当該事象の発生の可能性が高いこと

その金額を合理的に見積もることができること

\problem{H28 生保2問題 1(6)、H18 生保2問題 3(2)①、H10 生保2問題 2(3)①}
生命保険会社の自己資本が有していると考えられる機能を 4 つ列挙しなさい。

\answer{}
経営上の諸リスクの顕在化に対する緩衝

支払能力に対する信頼性の確保

経営に必要な固定資産等の取得資金

無コスト資金としての収益性向上への寄与

\problem{H9 生保2問題 1(5)}
次の①~⑤の相互会社の決算処理のうち、保険業法および関連規定に照らして、正しいものには○、
誤りのあるものには×をつけた上で、その誤りの理由を述べよ。なお、大蔵大臣の認可等による特別
取扱いはないものとする。

①剰余金として処分する額 900 億円のうち、600 億円を社員配当準備金に、50 億円を社員配当平衡積
立金に積み立てた(ただし、保険業法施行規則第 27 条第 1 号から第 6 号に定める控除すべき金額は
0 であった)。

②差益はプラスになったが、ソルベンシー・マージン基準における予定利率リスク相当額が前年度と
変わらなかったので、危険準備金Ⅱへの積立を行なわなかった。

③剰余金の処分として支出する金額が 800 億円だったので、
損失てん補準備金に 2 億円を積み立てた。

④社員に対する剰余金の分配を安定させる目的で、任意積立金に 50 億円を積み立てることとし、貸借
対照表上の負債の部に計上した。

⑤当年度募集した基金 500 億円の償却方法は 5 年後の一括償却だったので、剰余金として処分する額
700 億円のうち、100 億円を基金償却積立金に積み立てた。

\answer{}

①×(理由)社員配当準備金繰入は、剰余金の80%以上でなくてはならない。
(現行は20\%以上)

②×(理由)危険準備金Ⅱの積立を行わなくてはならない。

③×(理由)損失てん補準備金の積立は、剰余金の3/1000以上でなくてはならない。
業法58条

④×(理由)資本の部に計上しなくてはならない。
社員配当平衡積立金は任意積立金で資本の部。社員配当準備金は負債の部。

⑤×(理由)基金償却準備金(基金償却積立金ではない)を積み立てなくてはならない。
基金を償却するとき;   
  基金償却準備金の範囲内で取締役会決議.   
  償却する金額と同額を基金償却準備金から基金償却積立金に振替える.

  総代会決議. 剰余金の処分において、基金償却積立金を積立て、これと同額の基金の償却を行うことができる

\problem{H9 生保2問題 1(2)(改)}
予定利率 2.15%および 1.25%の場合について、保険料積立金残高 100 億円あたりの予定利率リスク
相当額を計算せよ。なお、予定利率リスク相当額の計算は、告示第 50 号(平成 8 年 2 月 29 日付)に
基づくものとする(係数は下表の通り)
( https://www.fsa.go.jp/singi/solvency/siryou/20070209/06-4.pdf)

\begin{tabular}{|c|c|}
\hline 予定利率の区分& リスク係数\\ \hline
 0.0\%以下の部分&0.0 \\
 0.0\%超 2\%以下の部分&0.01 \\
 2\%超 3\%以下の部分&0.2 \\
 3\%超 4\%以下の部分&0.4 \\
 4\%超 5\%以下の部分&0.6 \\
 5\%超 6\%以下の部分&0.8 \\
 6\%超 &1.0 \\
\hline
\end{tabular}

(元の問題では、問題で設定された予定利率が5\%, 2.5\%の設定で以下の表)

\begin{tabular}{|c|c|}
\hline 予定利率の区分& リスク係数\\ \hline
 0.0\%以下の部分&0.0 \\
 0.0\%超 3\%以下の部分&0.01 \\
 3\%超 4\%以下の部分&0.1 \\
 4\%超 5\%以下の部分&0.4 \\
 5\%超 6\%以下の部分&0.8 \\
 6\%超 &1.0 \\ \hline
\end{tabular}

\answer{}
予定利率が2.15\%の場合
100億x (0.01x2\% + 0.2x0.15\%) = 100億x (0.0002 + 0.003 ) =3,200万円
0.0002+0.003=0.0032

予定利率が1.25\%の場合
100億x (0.01x1.25\%) = 100億x (0.000125) = 125万円

(元の問題)

予定利率が5%の場合: 5,300万円
100億x (0.01x3\% + 0.1x1\% + 0.4x1\%) = 100億x (0.0003 + 0.001+ 0.004) =5,300万円
(0.0003+0.001+0.004=0.0053)

予定利率が2.5%の場合: 250万円
100億x (0.01x2.5\%) = 100億x (0.00025) = 250万円

\problem{H27 生保2問題 1(5)}
平成 8 年・大蔵省告示第 50 号別表第 6 の 2 に規定されている、変額年金保険等の最低保証リスク相
当額の算出について、次の A~E に適切な語句を記入しなさい。

Ⅱ.最低保証リスク相当額の算出

1.標準的方式

(1) 最低保証リスク相当額は、次のイに掲げる額からロに掲げる額を控除した額とする。

イ [A] 責任準備金の額(原則として法第 4 条第 2 項第 4 号に掲げる書類に記載された商品区分
ごとに、次の①から④までに定める手順に基づき算出した額をいう。)

① 次に掲げる区分に応じたリスク対象資産の額から、別表第 7 の 2 の区分によるそれぞれの対
象取引残高の欄に掲げる額(別表第 7 の 2 によりリスクヘッジの有効性が確認できたものに
限る。)を控除した残高に、次の表に掲げる区分に応じた下落率をそれぞれ乗じた額の合計額
を算出する。(省略)

② 上記①に掲げる額から、その額に次に掲げる算式により計算した[B]係数を乗じた額を控除
する。(省略)

③ 上記②により算出した額を特別勘定資産の額の合計額で除した率を算出する。

④ 上記③により算出した率に基づき資産下落が生じたとした場合の、一般勘定における[C]の
額を算出する。

口 法第 4 条第 2 項第 4 号に掲げる書類に記載された方法に基づき算出された一般勘定における[C]の額

(2) (省略)

(3) (省略)

2.代替的方式

次の①から⑬に定める基準を満たす保険会社、外国保険会社等又は免許特定法人(以下「保険会
社等」という。)は代替的方式を用いることができる。ただし、代替的方式を用いた場合は、 [D]
の結果、代替的方式の使用を継続することが不適当と認められ、代替的方式の使用を中断する旨又は
[E]に重大な変更を加える旨をあらかじめ金融庁長官に届け出た場合を除き、これを継続して使用
しなければならない。
(以下、省略)

\answer{}
A資産価格下落後
B分散投資効果
C最低保証に係る責任準備金
Dバック・テスティング
Eリスク計測モデル

\problem{H10 生保2問題 1(8)(改)}
我が国のソルベンシー・マージン基準に関する以下の①~⑤について、正しいものに○、誤ってい
るものに×をつけよ。

①「貸借対照表の純資産の部合計」はソルベンシー・マージンの構成項目である。

②保険リスク相当額の算出においては、
「普通死亡リスク」と「生存保障リスク」について、相関係数 0 として合計する。

(注)元の問題では、「普通死亡リスク」のところが「普通死亡リスク + 災害死亡リスク」となっていた
おそらく、H8 大蔵省告示第50号のあたりで変更された?
相関係数0はH8 大蔵省告示第50号(別表第2)に $\sqrt{A^2+B^2}+C$とある。A; 普通死亡リスク相当額, B; 生存保障リスク相当額.

③価格変動リスクにおいて、リスク係数が最も高いのは外国株式に該当する資産である。

④予定利率リスクにおいて、予定利率が 6%を超える部分のリスク係数は 1.0 である。

⑤子会社等リスクにおいて、金融関連会社よりも非金融関連会社の方がリスク係数は高い。

(注)元の問題では、「子会社等リスク」のところが「関連会社リスク」となっていたが、これは H8 大蔵省告示第50号第2条7として記載あり。


\answer{}

①×

(資本の部は) 現行では「純資産の部」

ソルベンシー・マージン総額(6-58)において、90\%や85\%にしている項目がある。

その他有価証券評価差額(税効果控除前) 90\%又は100\% (正負で異なる?)

土地の含み損益 85\%又は100\% (正負で異なる?)

ソルベンシー・マージン総額(6-58)において、不算入額・控除項目がある

繰延税金資産の不算入額
税効果相当額の不算入額
負債性資本調達手段等、保険料積立金等余剰部分の不算入額
控除項目

②◯

③×

価格変動等リスク(6-65)では「告示第50号第2条第5項...リスク係数が定められている」
-> 別表第7 (下表) http://www.nn.em-net.ne.jp/~s-iwk/current/H08-050/t007.html
によれば, 国内株式が最も高い20\%.

\begin{tabular}{|c|c|}
 \hline
リスク対象資産&リスク係数 \\ 
 \hline
国内株式	& 20%\\
外国株式	& 10%\\
邦貨建債券	& 2%\\
外貨建債券、外貨建貸付金等	& 1%\\
不動産(土地(海外の土地を含む。))  &	10%\\
金地金	& 25%\\
商品有価証券	& 1%\\
為替リスクを含むもの	& 10%\\
\hline
\end{tabular}

④◯

⑤×

リスク係数は別表第10

\begin{tabular}{|c|c|c|c|}
\hline
&事業形態 &リスク対象資産 &リスク係数\\
\hline
 \multirow{4}{*}{国内会社}&\multirow{2}{*}{金融業務} &株式 & 30\%\\
& &貸付金 & 1.5\%\\
\cline{2-4}
&\multirow{2}{*}{非金融業務} &株式 & 20\%\\
& &貸付金 & 1\%\\
\hline
 \multirow{4}{*}{海外法人}&\multirow{2}{*}{金融業務} &株式 & 25\%\\
& &貸付金 & 9.5\%\\
\cline{2-4}
&\multirow{2}{*}{非金融業務} &株式 & 15\%\\
& &貸付金 & 9\%\\
\hline
 \multirow{2}{*}{ランク4子会社等}& &株式 & 100\%\\
&&貸付金& 30\%\\
\hline
\end{tabular}

\problem{H13 生保2問題 1(6)}
ソルベンシー・マージン基準について、次の①~⑤を適当な語句で埋めよ。

(a)保険料積立金等余剰部分の算出にあたって、解約返戻金相当額が①を下回る場合は、解約
返戻金相当額として①を使用して算出する

(b)将来利益は、 ②の直近の 5 事業年度の平均値に相当する額または直近の②の額のいず
れか小さい額に 50%を乗じた額である。

(c)税効果相当額は、③(算式)により得られる額である。ここに、

A は、資本の部の剰余金の額から利益または剰余金の処分として支出する額、法定準備金に積
み立てる類およびこれに準じたものの額の合計額を控除した額

t は、繰延税金資産および繰延税金負債の計算に用いた法定実効税率

とする。

(d)満期保有目的の債券として分類している邦貨建債券の価格変動等リスク相当額を算出する際に
使用するリスク係数は④である。

(e)その他有価証券の価格変動等リスク相当額は、リスク係数×⑤で計算される。
\answer{}
①全期チルメル式責任準備金

②配当準備金繰入額

③A×t/(1−t)

④0%

⑤貸借対照表計上額

\end{document}