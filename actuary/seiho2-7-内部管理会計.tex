2\documentclass[report,gutter=10mm,fore-edge=10mm,uplatex,dvipdfmx]{jlreq}
\usepackage{lmodern}
\usepackage{amssymb,amsmath}
\usepackage{ifxetex,ifluatex}
\usepackage{actuarialsymbol}
\usepackage[]{natbib}
\RequirePackage{plautopatch}

% maru suji ① etc.
\usepackage{tikz}
\newcommand{\cir}[1]{\tikz[baseline]{%
\node[anchor=base, draw, circle, inner sep=0, minimum width=1.2em]{#1};}}

\usepackage{comment}

\begin{comment}

\ifnum0\ifxetex1\fi\ifluatex1\fi=0 % if pdftex
  \usepackage[T1]{fontenc}
  \usepackage[utf8]{inputenc}
  \usepackage{textcomp} % provide euro and other symbols
\else % if luatex or xetex
  \usepackage{unicode-math}
  \defaultfontfeatures{Scale=MatchLowercase}
  \defaultfontfeatures[\rmfamily]{Ligatures=TeX,Scale=1}
\fi
% Use upquote if available, for straight quotes in verbatim environments
\IfFileExists{upquote.sty}{\usepackage{upquote}}{}
\IfFileExists{microtype.sty}{% use microtype if available
  \usepackage[]{microtype}
  \UseMicrotypeSet[protrusion]{basicmath} % disable protrusion for tt fonts
}{}
\makeatletter
\@ifundefined{KOMAClassName}{% if non-KOMA class
  \IfFileExists{parskip.sty}{%
    \usepackage{parskip}
  }{% else
    \setlength{\parindent}{0pt}
    \setlength{\parskip}{6pt plus 2pt minus 1pt}}
}{% if KOMA class
  \KOMAoptions{parskip=half}}
\makeatother
\usepackage{xcolor}
\IfFileExists{xurl.sty}{\usepackage{xurl}}{} % add URL line breaks if available
\IfFileExists{bookmark.sty}{\usepackage{bookmark}}{\usepackage{hyperref}}
\hypersetup{
  hidelinks,
  pdfcreator={LaTeX via pandoc}}
\urlstyle{same} % disable monospaced font for URLs
\usepackage{longtable,booktabs}
% Correct order of tables after \paragraph or \subparagraph
\usepackage{etoolbox}
\makeatletter
\patchcmd\longtable{\par}{\if@noskipsec\mbox{}\fi\par}{}{}
\makeatother
% Allow footnotes in longtable head/foot
\IfFileExists{footnotehyper.sty}{\usepackage{footnotehyper}}{\usepackage{footnote}}

\end{comment}
%\makesavenoteenv{longtable}
\setlength{\emergencystretch}{3em} % prevent overfull lines
\providecommand{\tightlist}{%
  \setlength{\itemsep}{0pt}\setlength{\parskip}{0pt}}
\setcounter{secnumdepth}{-\maxdimen} % remove section numbering

\author{kazuyoshi}
\date{}

\newcommand{\problem}[1]{\subsubsection{#1}\setcounter{equation}{0}}
%\newcommand{\answer}[1]{\subsubsection{#1}}
\newcommand{\answer}[1]{\subsubsection{解答}}

%Pdf%\newcommand{\wakumaru}[1]{\framebox[3zw]{#1}}
\newcommand{\wakumaru}[1]{#1}





\begin{document}
\chapter{保険2第7章 内部管理会計}
\section{7.1 内部管理会計の意義}
\problem{H12 生保2問題 1(6)}
次の①~⑥を適切な語句で埋めよ。

生命保険会計において「内部管理会計」という場合、主に、経営管理や経営上の意思決定に役立つ
ような会計情報を\wakumaru{①}向けに提供することを目的とする会計システムを意味する。従って、
(A)「\wakumaru{②}や\wakumaru{③}の的確な把握」や、
(B)「\wakumaru{④}の詳細な把握」に関して
\wakumaru{⑤}等を補足することが内部管理会計の重要なテーマとなる。
(A)を目的とする内部管理会計手法としては、例えば米国
GAAP 会計などを候補とすることができる。一方(B)を目的とする内部管理会計手法としては、例え
ば区分経理を候補とすることができる。
\answer{}
\begin{itemize}
\item[ ①: ] 経営者(経営陣等も可)
\item[ ②: ] 経営成績、(業績等も可)[期間損益]
\item[ ③: ] 期間損益(期間収益、期間収支等も可)[経営成績]
\item[ ④: ] 収支構造(収益構造等も可)
\item[ ⑤: ] 法定会計(SAP,SAP会計も可)
\end{itemize}

\section{7.2 内部管理会計の必要性}
\problem{H26 生保2問題 2(3)}
内部管理会計の意義および必要性について、現行法定会計の特徴と限界に触れつつ、簡潔に説
明しなさい。
\answer{}
<内部管理会計の意義>
\begin{itemize}
 \item[] 法定会計は一般に保険会社のソルベンシー確保を目的として保険監督当局が提出を要求するもの
 であり、GAAP 会計は一般投資家等への会社の会計情報の提供を目的として作成されるものであ
 って、これらの提供する会計情報は、必ずしも経営者の経営判断に役立つものとは限らない。
 \item[] そこで、経営者の経営判断に役立つ会計情報の提供を目的とした会計システムたる内部管理会計
 が必要となる。その内容や属性は、企業のおかれた環境や時代により又企業自体の規模や性格に
 より相違するものと考えるべきであろうが、現行の法定会計では必ずしも十分捉えきれない。内
 部管理会計が経営判断に役立つ会計情報の提供を目的とする会計である以上、「経営成績や期間
 損益の的確な把握」や「収支構造の詳細な把握」などに関して、法定会計等を補足することが内
 部管理会計の存在意義と言えよう。
 \item[] 内部管理会計に望まれる属性としては、以下が挙げられる。
\begin{itemize}
  \item[] 会社の事業の基礎となる経済的な基盤(資金調達方法など)を反映するものであること
 \item[] 経営者が、区分が必要と考えるプロフィット・センター毎(プロダクト・ライン別、チャネ
 ル別、戦略事業単位別など)に結果が得られること
 \item[] 結果が上層部の経営者にとって理解しやすいものであること
\end{itemize}
\end{itemize}
<内部管理会計の必要性>
\begin{itemize}
 \item[] 現行法定会計の限界とそれを踏まえた内部管理会計の必要性は以下のとおりである。\\
\item[] (経営成績や期間損益を的確に把握する内部管理会計の必要性)
\begin{itemize}
  \item[] 長期の評価性債務を抱える生命保険会社の法定会計は、標準責任準備金制度に基づく保守的な責
 任準備金の積立等による支払能力の確保を重視する。このため、一般的に、コミッション等の関
 係で新契約の獲得が単年度利益にマイナスの影響を与える一方、解約控除の存在により解約契約
 の増加が単年度利益にプラスの影響を及ぼすといった課題がある。
 \item[] 高度の経営判断に用いる会計としては、現在の経営成績の状況を適切に表示する、期間損益を的
 確に把握し得る会計制度が必要となる。\\
\end{itemize} 
\item[](保険種類毎の収支構造を把握する内部管理会計の必要性)
\begin{itemize}
  \item[] 生命保険会社を巡る事業環境の急激な変化の中で、過去の「単一の価格設定」
 「均等な資産運用」
 といった従来の一般勘定における一括管理の手法が限界を迎えたことから、リスク管理の高度化、
 利用者ニーズへの対応の観点から、保険種類毎の収支構造の把握に向けた区分経理が不可欠とな
 り、平成8年の保険業法改正時に導入された。
 \item[] その後の商品内容・給付およびチャネルの多様化等を踏まえると、区分経理をさらに細分化した
 保険種類毎の収支構造等を把握する内部管理会計等の必要性も高まってきている。
\end{itemize}
\end{itemize}
\section{7.3 経営成績や期間損益を的確に把握する内部管理会計}

\problem{2021 生保2問題 2(2)}
市場整合的EVと伝統的EVとの主な相違点について簡潔に説明しなさい。ただし、EVとはエ
ンベディッド・バリュー、市場整合的EVとはCFOフォーラムが2008年6月に発表した
MCEV原則に基づく市場整合的EVを指すものとする。
\answer{}
\begin{itemize}
\item[] 伝統的 EV では、ハードル・レート(およびそれに付随する資本コスト)によりリスクの反映を行
っているため、負債対応資産について株式等のリスク資産の構成比を高めた場合、期待投資収益の
増加は、資産運用リスクに対応した資本コストの増加や割引率の増加によってある程度の調整はさ
れるが、市場リスクに係る完全なリスク調整は期待できないため、リスクの大きい投資行動を過大
評価する傾向があると言われている。この伝統的 EV の欠点を解消するため、市場整合的 EV(以
下、MCEV)では、リスクを明示的に反映することが求められている。
\item[] MCEV では、保険負債のキャッシュ・フローの現在価値は、理論的には対応する複製ポートフォ
リオの価値により評価される。保険契約に内在する「金融オプションと保証の時間価値
(以下、TVFOG; Time Value of Financial Options and Guarantees)」も、それを複製するため
のヘッジコストとして評価される。伝統的 EV では、TVFOG は明示的には評価されていない。
TVFOG の評価対象としては、契約者配当や解約に係る権利、変額年金の各種最低保証などが挙げ
られる。
\item[] 伝統的 EV では、ハードル・レートで割引を行うことによりリスクの反映を行う。そのため、保有
する必要資本に対して資本コストが発生し、資本コストの控除が行われることになる。一方、市場
整合的評価を行うことでリスクの反映を明示的に行う MCEV では、市場リスクに係る調整を反映
するため、負債対応資産の構成内容にかかわらず投資収益率及び割引率としてリスクフリー・レー
トが適用されることから、伝統的 EV の意味での資本コストの控除を別途行う必要はない。
\item[] ただし、MCEV においても、リスクに対する資本は必要であり、その必要資本に対するコストと
して、
「必要資本に対するフリクショナル・コスト」の反映を行う必要がある。
「必要資本に対する
フリクショナル・コスト」とは、例えば、実際の市場には投資収益に係る二重課税のコスト等の摩
擦が存在するため、監督規制・格付・リスク管理等のために保険会社が維持する必要資本に対して
発生する摩擦的な資本コストである。
\item[] また、仮に複製ポートフォリオに投資したとしても、市場リスク以外のヘッジ不能リスクがあるた
め、「残余ヘッジ不能リスクに対するコスト」が発生する。例えば、保険リスク等がヘッジ不能リ
スクに分類される。
\item[] 企業価値評価としては、財務的困難のコストやエージェンシー・コストも評価すべきリスクである
が、MCEV 原則の結論の背景において、CFO フォーラムは、これらは、企業の経営が評価すべき
一般的な事業リスクではなく、個々の投資家が評価すべき一般的な企業リスクであるとして、
MCEV 算出の際には考慮せず、個々の投資家が必要に応じて考慮するものとしている。
\item[] MCEV は、MCEV 原則およびそのガイダンスに準拠する形での外部レビューおよび開示が行われ
ていることから、透明性の向上と比較可能性の確保が図られている。
\item[] MCEV は修正純資産と保有契約価値の合計として表され、MCEV における保有契約価値は、確実
性等価利益現価、金融オプションと保証の時間価値、必要資本に対するフリクショナル・コスト、
そして残余ヘッジ不能リスクに対するコストから構成される。
\end{itemize}

\problem{H13 生保2問題 1(10)}
エンベディッド・バリュー(Embedded Value)とアプレイザル・バリュー(Appraisal Value)に
ついて簡潔に説明せよ。
\answer{}
「エンベディッド・バリュー」は、価値基準会計における生命保険会社の
経済的価値であり、純資産の額に保有契約の経済的価値(将来その契約から
期待される法定会計上の利益からソルベンシー・マージン等の必要サープラ
スの増加分を差し引いた利益(使用可能利益)をリスク割引率(ハードル・
レート)で割り引いた現価)を加えたものである。

「アプレイザル・バリュー」は、生命保険会社が保有すると考えられる新契
約獲得のポテンシャリティを、将来見込まれる新契約から得られるであろう
将来の年々の法定会計上の損益の割引現価により評価し、「エンベディッド・
バリュー」に加算したものである。

\problem{2019 生保2問題 3(2)②}
潜在価値会計(エンベディッド・バリュー(EV))の概要(意義、考え方、特徴など)について簡潔
に説明しなさい。

\answer{}
<意義・目的>
\begin{itemize}
 \item[] 経営者や投資家等にとって、法定会計による財務情報のみでは、生命保険会社の経営成績や企業価 値を読み取るのが困難と考えられる。
 \item[] 潜在価値会計は、生命保険会社の経営成績の実態に則した会計であり、生命保険会社の経済的価値 を表すことから、法定会計による財務情報を補うことができる。
\end{itemize}
<考え方>
\begin{itemize}
 \item[] 潜在価値会計は、保険契約から生じる将来のキャッシュ・フローを予測し、それから計算される将来の期待利益の割引現価と、純資産等に基づき定義される「生命保険会社の経済的価値」を計算す
 るとともに、その額の年間の変化量により当期純利益を計算しようとするもので、資産負債法によ
 る会計の1つである。
\end{itemize}
<特徴>
\begin{itemize}
 \item[] 新契約時に将来利益が認識され、解約時に将来利益の喪失が認識されるため、経営成績の実態に即 している。
 \item[] 経済的価値の水準は、計算前提により大きく変化するため、期間損益の把握は容易ではない。
また、計算前提の設定には共通原則がないため、客観性、比較可能性が十分でない。
 \item[] 経済的価値の把握のため、ロックイン方式は適用されない。
\end{itemize}

\problem{H12 生保2問題 1(7)}
次の①~⑤に入る適当な語句を、以下の A~O から記号を選択せよ。

ハードル・レートは\wakumaru{①}という指標との比較を通じて商品の\wakumaru{②}と関係を持つ。
\wakumaru{①}は商品から得られる
\wakumaru{③}の現在価値とその商品の販売のために使った\wakumaru{④}が等しくなる
\wakumaru{⑤}であり、IRR とも呼ばれる。

語群
\begin{tabular}{lll}
A.資本 & B.キャッシュフロー& C.平均利回り\\
D.負債 &E.経済的価値 & F.平準 ROE\\
G.繰延べ資産 & H.プロフィットマージン&  I.運用利回り\\
J.プライシング & K.必要サープラス&L.将来利益 \\
M.当期利益 &N.ROI &O.アクルーアル方式 \\
\end{tabular}

\answer{}
\begin{itemize}
\item[①] N; ROI
\item[②] J; プライシング
\item[③] L; 将来利益 
\item[④] A; 資本
\item[⑤] C; 平均利回り
\end{itemize}

\problem{H9 生保2問題 2(1)}
「ハードル・レート」について、簡潔に説明せよ。
\answer{}
内部管理会計の手法である価値基準会計等において、将来のキャッシュフローの予
測から期待される法定会計上の利益の合計額を契約時点で利益計上するために用いる
割引率。利益が実現するまでの時間的な遅れと利益実現に関する不確実性のリスクを
考慮した上で株主等が投下資本に対して期待する収益率であり、資本の調達コストに
対応するもの。

\problem{H30 生保2問題 1(6)}
価値基準会計(潜在価値会計)等におけるハードル・レート(リスク割引率)の設定方法である「ト
ップ・ダウンアプローチ」と「ボトム・アップアプローチ」についてそれぞれ簡潔に説明しなさい。
\answer{}
トップ・ダウンアプローチは、会社のリスク特性に基づき、すべての商品について単一の
割引率を適用する方法である。通常、リスク割引率を、加重平均資本コスト(WACC)
を計算して求める。

ボトム・アップアプローチは、各キャッシュ・フローのリスク特性に基づきリスク割引
率を設定する方法である。各契約ラインに付随するリスクを反映して割引率を設定するも
のであり、トップ・ダウンアプローチよりも透明性の高い、各事業リスクに適した割引率
の設定が可能となるが、実務上の取り扱いは煩雑となる。市場整合的EVは代表的なボト
ム・アップアプローチに分類される。

\problem{H20 生保2問題 2(4)}
潜在価値会計に関し、次の①、②の各問に答えなさい。

\begin{itemize}
\item[①] 潜在価値会計における資本コスト(Cost of Capital)について簡潔に説明しなさい。
\item[②] 下記の条件において、潜在価値会計のトップダウンアプローチによるハードル・レート(リスク割 引率)を、WACC を用いて計算しなさい。解答にあたっては計算過程も記載しなさい。
\begin{tabular}{ll}
リスクフリー・レート(リスクフリーの投資収益率)&2.00%  \\
 債務コスト(税引後)&2.50%\\
 株式投資の平均的収益率(市場ポートフォリオの期待収益率) &4.00%\\
 (当該保険会社の)ベータ&1.50\\
 株主資本(株式時価総額)の割合&90%\\
 債務(社債・借入金の時価総額)の割合&10%\\
\end{tabular} 
\end{itemize}
\answer{}
①

資本コストとは、計算基準日に留保するソルベンシー・キャピタル(必要資本)から、将来的に
リリースされるソルベンシー・キャピタル及びソルベンシー・キャピタルに対する税引後運用収
益の現価(リスク割引率による)を差し引くことによって計算されるコストのことである。

(別解1)将来のソルベンシー・キャピタルのリリースの現価\\
+ 将来のソルベンシー・キャピタルに対する税引後運用収益の現価\\
- 計算基準日に留保するソルベンシー・キャピタル

(別解2)将来におけるソルベンシー・キャピタル\\
×(ハードル・レート - 資産運用利回り(税引後))の現価\\

つまり、目標ソルベンシー・マージン比率等を達成するために拘束されたソルベンシー・キャピ
タルは、リリースされるまでは資産運用収益しか生み出さないが、資本の提供者はハードル・レ
ートによる投資の機会を求めているため、資産運用利回りがハードル・レートに満たない場合、
資本の提供者は投資機会を喪失したことによるコストを負担していると考えることができる。

(※1)市場整合的EVにおける摩擦的資本コスト等に触れている場合、加点した。

(※2)別解のように、資本コストを表現する式については文献により正負の定義等が異なる場合がある。

②

株主資本コスト=2.00% + 1.5×(4.00% - 2.00%) = 5.00%\\
WACC=90%×5.00% + 10%×2.50% = 4.75%

\problem{H21 生保2問題 1(4)}
以下の商品の解約の実績が増加した場合、当期の「当期純利益(純剰余)」及び Embedded Value の
うちの「保有契約価値」に与える主な影響について、商品ごとにそれぞれ簡潔に説明しなさい。
\begin{itemize}
\item[] 逆ざや契約である養老保険(年払)
\item[] 保険期間の短い定期保険(年払)
\end{itemize}
なお、両商品とも営業職員の比例給は契約時のみに支払われ、解約による戻入はないものとし、比
例給支払後は費差益であるものとする。
\answer{}

 \begin{tabular}{|c|c|p{0.4\textwidth}|}
 \hline{}
商品&当期純利益(純剰余)& EmbeddedVa1ueの保有契約価値\\  \hline{}
養老保険&解約控除によりプラスの影響。&
\begin{itemize}
\item[] 将来の逆ざやの減少によりプラスの影響。
\item[] 将来の費差益の減少によりマイナスの影響。
\end{itemize}\\ \hline{}
定期保険& 
単年度の利益への影響は小さい。& 
\begin{itemize}
\item[]  将来の死差益の減少によりマイナスの影響。
\item[]  将来の費差益の減少によりマイナスの影響。
\end{itemize} \\
\hline
\end{tabular}

※上記は典型的な場合を想定した解答例。条件を付して上記以外のケースについて解答してもよい。

\problem{H9 生保2問題 1(3)}
以下の方式①から方式④は米国における、法定会計、GAAP 会計、価値基準会計、平準 ROE 会計に
より計上された同一保険会社の当初 5 年間の当期利益の推移である。それぞれが、どの方式により計
算されたものであるかを解答せよ。なお、初年度始に保険契約を販売し、それ以後は販売を停止し、
保険契約の維持保全のみを行なうものとし、プライシングに用いた予測値と実績値は一致するものと
する。

\begin{tabular}{lllll}
年度&方式①&方式②&方式③&方式④\\ \hline
1&14.14&-15.83&0.00&-98.87\\
2&16.80&15.11&17.88&14.51\\
3&17.15&15.96&18.49&17.04\\
4&17.16&16.71&18.76&19.73\\
5&16.78&17.33&18.58&22.61\\
\end{tabular}
\answer{}

\begin{itemize}
\item[ 方式①: ] 価値基準会計
\item[ 方式②: ] 米国GAAP会計
\item[ 方式③: ] 平準ROE
\item[ 方式④: ] 米国法定会計
\end{itemize}

\problem{H14 生保2問題 1(4)}
次の表は、ある生命保険会社のある商品に対する価値基準会計に関する諸数値である。
\begin{itemize}
\item[① ] 第 1 および第 5 保険年度の価値基準会計上の税引前当期利益を計算せよ。
\item[② ] 第 1 および第 5 保険年度末の価値基準会計上の広義責任準備金を計算せよ。
\end{itemize}
なお、いずれも、解答は小数点以下第 3 位を四捨五入して小数点以下第 2 位まで求め、その計算過程
についても記載すること。

 \begin{tabular}{|p{0.2\textwidth}|c|c|c|c|c|c|c|c|c|c|}
  \hline
保険年度&1&2&3&4&5&6&7&8&9&10\\ \hline
法定責任準備金&0.00&75.87&156.29&241.53&331.89&427.67&529.20&636.82&750.90&871.82\\ \hline
\begin{itemize}
\item[]  税引前当期純利益
\item[]  (法定会計ベース)
\end{itemize}&
-97.87&14.51&17.04&19.73&22.61&25.66&28.92&32.39&36.09&40.02\\ \hline
将来利益の現価&12.30&112.01&114.31&114.41&111.84&106.01&96.25&81.77&61.64&34.80\\ \hline
 \end{tabular}

(注)
「将来利益の現価」は「税引前当期利益(法定会計ベース)
」をハードル・レートにより割り引
いた現価である。また、
「法定責任準備金」と「将来利益の現価」は年始状態の値とし、責任準
備金評価利率は 6.00\%、実際利回りは 10.00\%とする。
\answer{}
①第1保険年度:14.15、第5保険年度:16.78\\
[計算過程]

表におけるハードル・レートは、例えば、第10保険年度の税引前当期利益と将来利益
の現価より40.02÷34.80=1.15であることから、15.00%であることがわかる。

したがって、価値基準会計上の税引前当期利益は、次のとおりとなる。\footnote{教科書7-8にある通り1年目は期待利益現価+ unwinding, 2年目以降は unwindingのみとなる。}
\begin{itemize}
\item[]  第1保険年度:12.30(第1保険年度の将来利益の現価)×1.15=14.15
\item[]  第5保険年度:111.84(第5保険年度の将来利益の現価)×0.15=16.78
\end{itemize}
(別解)

「価値基準会計上の当期利益 = 税引前当期利益(法定会計べ一ス)+ 将来利益の現価
の増分」であることから、
\begin{itemize}
\item[]  第1保険年度: ▲97.87 + 112.01 = 14.14(注: 上記と端数誤差が生じる。)
\item[]  第5保険年度: 22.61  + (106.01 − 111.84)=16.78
\end{itemize}

②第1保険年度末:▲36.14、第5保険年度末:321.66\\
[計算過程]

「価値基準会計上の広義責任準備金 = 法定責任準備金 - 将来利益の現価」であること
から、価値基準会計上の広義責任準備金は、次のとおりとなる。
\begin{itemize}
\item[] 第1保険年度末:75.87  −  112.01 = ▲36.14
\item[] 第5保険年度末:427.67 − 106.01 = 321.66
\end{itemize}
\problem{H9 生保2問題 1(1)}
以下の前提で、ROE を計算せよ。なお、運用収益に係るキャッシュフローは年度末に、それ以外の
キャッシュフローは年度始に発生するものとし、
「総資産=責任準備金+自己資本」とする。

\begin{tabular}{ll}
 収入保険料&1,000\\
 事業費&200\\
 死亡給付&300\\
 前年度末総資産&10,000\\
 前年度未責任準備金 &9,000\\
 当年度末責任準備金 &9,900\\
 運用利回り&5\%
\end{tabular}

\answer{}
\noindent 12.5\%\\
分母 = 年始自己資本 = 10,000-9,000 = 1,000\\
分子 =年末自己資本 -年始自己資本 = ((10,000+1,000-200-300)*1.05 - 9,900) - (10,000-9,000) = 1,125 - 1,000 = 125\\
ROE = 125 / 1000 = 12.5\%

\problem{(参考)H4 生保2問題 1(5)}
キャッシュフロー予測について簡潔に説明せよ。
\answer{}
保険金杜の経営実績が、金利などの変化によりどのような影響を受けるかを測定する
ために、保険関係取引と投資関係取引の両方を含むキャッシュ・フロー予測が行われる。
キャッシュ・フロー予測の目的としては、①流動性の確保、②ソルベンシーの測定、の二
つが挙げられる。

キャッシュ・フローは金利の変動により影響を受けるから、異なったシナリオに基づいて
それぞれのキャッシュ・フロー予測を行うことが必要である。

\section{7.4 区分経理}
\problem{H20 生保2問題 1(2)}
区分経理に関し、次の①~⑤の空欄にあてはまる最も適切な語句を記入しなさい。

\begin{itemize}
 \item[] 区分経理は、内部管理会計として行っている状況であるが、
 「保険会社向けの総合的な監督指針」
 (金融庁)には、区分経理の明確化として内容が規定されている。
 \item[] 会社の損益等を区分する単位として、\wakumaru{①}及び\wakumaru{②}
 を設定する。\wakumaru{①}については、損益を把握する単位として
 適切なものとなっている必要があり、保険の性質の相違等により理論的・合理的
 な区分とする必要がある。\wakumaru{②}には例えば次のイからニの機能がある。
\begin{itemize}
\item[] 死亡保障リスク等の\wakumaru{③}機能
\item[] 新商品開発に係る事業運営資金提供機能
\item[] 会社全体で共有する資産・共通する経費等の管理機能
\item[] 現預金等の管理機能
\end{itemize} 
\item[] 運用資産は、資産区分ごとに、資産分別管理方式・\wakumaru{④}方式・
 \wakumaru{⑤}方式の中から適切な管理方式を選択し管理する。
\end{itemize}

\answer{}
\begin{itemize}
\item[ ①: ] 商品区分
\item[ ②: ] 全社区分
\item[ ③: ] リスクバッファー
\item[ ④: ] 資産単位別持分管理
\item[ ⑤: ] 資産持分管理(またはマザーファンド)
\end{itemize}

※④と⑤は順不同

\problem{H30 生保2問題 1(3)}
「保険会社向けの総合的な監督指針」
【Ⅱ-2-4 生命保険会社の区分経理の明確化】について、
以下の A~E の空欄に当てはまる適切な語句を記入しなさい。\\
\vspace{1zh}
\noindent Ⅱ-2-4-1 意義(省略)\\
Ⅱ-2-4-2 主な着眼点

各生命保険会社においては、適切な区分経理を行うため、例えば、以下のような考えに基づく区
分経理に関する管理方針を策定しているか。また、区分経理の状況が、取締役会その他これに準ず
る機関に対して報告されているか。\\
(1)~(4) (省略)\\
(5) 資産の配賦方法及び管理基準
\begin{itemize}
\item[①] 運用資産の配賦方法\\
 運用資産は、原則として、資産の購入時に配賦する資産区分を決める。
\item[②] 運用資産の管理\\
 運用資産は、資産区分ごとに、次に掲げる方式の中から適切な方式を選択し管理する。

\begin{itemize}
\item[ ア.] \wakumaru{A}: 個々の資産を銘柄ごとに、資産区分に直接帰属させる方式
\item[ イ.] \wakumaru{B}: 取引単位(例えば、不動産では物件ごと)ごとに、資産区分の持分で管理する方式
\item[ ウ.] 資産持分管理方式: 投資対象資産ごとのマザーファンドを設定し、各資産のマザーファン
 ドに対する持分を管理する方式
\end{itemize} 
(注)資産持分管理方式を用いる場合は、一般勘定資産(\wakumaru{C} 保険に対応する資産を除く。)全体を 一個のマザーファンドとして扱わない。
\item[③] 運用資産以外の配賦方法\\
 再保険貸等、各資産区分に直課できるものは直課し、直課できないものは、区分経理に関する
 管理方針に基づいて配賦する。
\item[④] 全社区分の資産\\
 \wakumaru{D}、子会社・関連会社株式、\wakumaru{E}(\wakumaru{E}等の管理機能を持つ場合)
 、その他全社区分に配賦することが相応しい資産の全部又は一部を配賦するものとする。
\end{itemize}
\noindent (6)、(7) (省略)\\
Ⅱ-2-4-3 監督手法・対応 (省略)

\answer{}
\begin{itemize}
\item[ A: ] 資産分別管理方式
\item[ B: ] 資産単位別持分管理方式
\item[ C: ] 無配当
\item[ D: ] 営業用不動産
\item[ E: ] 現預金
\end{itemize}

\problem{2020 生保2問題 3(1)①}
区分経理の意義について簡潔に説明しなさい。(3点)
\problem{H23 生保2問題 2(1)、H3 生保2問題 3(2)}
区分経理の意義および商品区分の設定について、
「保険会社向けの総合的な監督指針」および「保険
検査マニュアル」の内容を踏まえ、簡潔に説明しなさい。
\problem{H28 生保2問題 2(1)、H18 生保2問題 3(1)①}
区分経理の意義および、活用方法について、活用時の留意点に触れながら簡潔に説明しなさい。
\answer{}
\noindent ○区分経理の意義.
\begin{itemize}
\item[] 生命保険会社においては、利益還元の公平性・透明性の確保、保険種類相互間の内部補助の遮
 断、事業運営の効率化、商品設計や価格設定面での創意工夫などを図る観点から、一般勘定に
 ついて保険商品の特性に応じた区分経理を行うことが重要である。
\item[] 各生命保険会社において自己責任原則のもと、保険経理の透明性、保険契約者間の公平性確保
 等の観点から、適切な区分経理が行われる必要がある。
\item[] また、区分経理を導入するにあたっては、資産の配分方法、含み損益の配賦方法等について、
 アセットシェア等に基づき適切に配分方法が定められていることが重要である。
\item[] なお、区分経理は保険計理人の確認業務(責任準備金に関する事項、剰余金の分配または契約
 者配当に関する事項)にも関連している。
\end{itemize}
\noindent ○商品区分の設定
\begin{itemize}
\item[] 区分経理を活用する上では、その目的に応じた適切かつ有効な区分を設定することが重要であ
 り、保険の性質の相違等により理論的かつ合理的な区分とする必要がある。したがって、会社
 収支に重大な影響を与える場合等は、商品区分の新設や細分化をして管理することが望ましい。
 ただし、期間損益の安定性や事務負荷等を踏まえて重要性を検討した上で導入することが重要
 である。また、設定した商品区分については、商品ポートフォリオが大きく変化する場合等に
 は設定を見直す必要もある。
\item[] 保険検査マニュアルには、次のような留意点が記載されている。
\begin{itemize}
\item[①] 商品区分は、損益及び負債の管理を行うためのものであるが、商品の特性や契約の保有状況
に照らして、損益を把握する単位として適切なものとなっているか。\\
 例えば、「掛捨型の短期保険と貯蓄型の長期保険」、「無配当保険と有配当保険」、「予
 定利率固定型保険と予定利率変動型保険」、「個人保険と企業保険」などが、原則として、
 別区分で管理されているか。\\
 なお、主契約に付加された特約等は、原則として、主契約と同じ商品区分に帰属させてい
 るか。
\item[②] 新規商品の発売による当該保有契約の増大や、ある商品区分の中の一部の保険種類の契約
 の増大などにより、保険会社全体や商品区分の収支に重大な影響を与えるような場合に、
 新たな商品区分又は同種の細分化した商品区分を設定する際に、契約者間の公平性等に留意し、
合理的な方法で行っているか。
\item[③] 設定した商品区分について、合理的な理由(保有契約が減少し、商品区分の存在意義がな
 くなった場合等)がないにもかかわらず、その変更(他の商品区分に統合することを含
 む。)を行っていないか。
\end{itemize}
\end{itemize}
\noindent ○活用方法および留意点

\begin{itemize}
 \item[] 利益還元の公平性・透明性の確保
\begin{itemize}
\item[] 区分経理を行うことで、区分毎の損益の状況を明確にすることが可能となるため、利益還元の
 公平性・透明性を確保することができる。特に、有配当区分における契約者への配当の公平性・
 透明性を確保するために、有配当契約・無配当契約を区分するといった適切な区分経理の実施
 は重要である。
\item[] 法令では、剰余金の分配または契約者配当の計算は、
 「保険契約の特性に応じて設定した区分ご
 とに」計算することが規定されており、また、生命保険会社の保険計理人の実務基準では、公
 正・衡平な配当の確認における商品区分単位の配当可能財源の確認は「区分経理の商品区分毎
 に」行うことと規定されている。
\end{itemize}
 \item[] 保険種類相互間の内部補助の遮断
\begin{itemize}
\item[] 商品区分はセルフサポートが基本であり、その中で保険料および責任準備金の十分性を満たす
 必要がある。言い換えれば、区分毎の十分性の確保が、契約者間の公平性の確保および会社の
 健全性の確保につながる。特に、生命保険会社の保険計理人の実務基準では、責任準備金の十
 分性確認において区分経理の商品区分毎に将来収支分析を行うことと規定されている。
\item[] ただし、区分経理は現状ではあくまで内部管理会計であることもあり、最終的には全体で支払
 能力を裏付けていることにも留意する。
\item[] 商品区分の規模が小さくなると、分散効果の低下により毎年の保険収支が不安定化したり資産
 運用効率が低下したりすることから、安定的・効率的な保険制度の運営が難しくなる。このこ
 とから、無闇・頻繁な区分の変更は当然避けられるべきではあるが、一方で、発売間もない新
 商品の区分や販売停止して長期間経つなどして保有が減少した商品の区分など、小さすぎる区
 分は、他の商品区分への統合も検討するなど、区分の更新を検討する必要がある。また、この
 ような目的のために厳格な貸借・出資を前提として全社区分を活用することも考えられる。
\end{itemize}
 \item[] 事業運営の効率化
\begin{itemize}
\item[] 区分経理を行うことで、区分毎の効率性を把握することが可能となり、不採算区分の事業規模
 縮小・撤退などを検討する上での有効な判断材料となる。さらに、その区分の特性を把握でき、
 手数料などの販売政策、経営資源投入などの経営戦略の策定が可能となる。
\item[] 商品に対応する資産の運用特性に沿った区分とすることで、資産運用の効率性・資産負債マッ
 チングの向上や責任準備金対応債券の効果的な運用など、ALM を効果的に行うことができる。
\item[] 区分経理を行う上で算定・利用される保険関係収支などの各種の情報やインフラはリスク管理
 にも利用することができる。
\item[] 分析を行う上でも経営上の諸作を行う上でも区分経理が有効に働くように、商品特性・資産運
 用特性などに沿った商品区分とすることが必要である。例えば、配当の有無・保証性または貯
 蓄性・外貨建かどうかなどによって区分することは必要であろう。
\item[] ただし、区分経理の商品区分に基づく分析だけでなく、保険種類毎や販売チャネル毎など、更
 に細分化した分析や、単年度損益に加え、エンベディッドバリューや新契約価値などの評価手
 法を併用するなど、多面的な分析を行った上で経営判断に役立てていくことが重要である。
\item[] 区分毎の効率性を把握するためには事業費の配賦が不可欠ではあるが、間接経費の詳細な配賦
 は一般的に困難である。これらはあくまでも配賦によって得られた数字であり、常に精度改善
 の余地を持つことに留意が必要である。
\item[] また、一般に、細分化には情報収集コスト・インフラ整備が必要であり、費用対効果に留意が
 必要である。
\item[] 区分経理を効果的に経営に反映させるためにも、経営陣の区分経理に対する理解促進を図るこ
 と・アクチュアリー自身の説明能力の向上を図ることが必要である。
\end{itemize}
 \item[] 商品設計や価格設定面での創意工夫などを図る
\begin{itemize}
\item[] 例えば、独立した商品区分及び資産区分を設定することにより、利率変動型商品や外貨建商品
 などのような資産運用結果を契約者価額に反映させた商品の開発が可能となる。
\item[] また、他の金融商品に競合する商品を開発する場合には、資産区分の資産運用方針に基づき予
 定利率を定め、必要ならば解約返戻金を市場価格調整型とすることで、リスクコントロールを
 しつつ、魅力ある商品設計が可能になる。
\item[] 利源分析を区分毎に行うことで、計算基礎率の妥当性のチェックなどのより詳細な分析を行う
 こともでき、これを新商品開発時の計算基礎率に反映することができる。特に、商品区分別の
 事業費を把握し、それを保険種類別・販売チャネル別に按分することでそれぞれの保険種類・
 販売チャネルに必要な予定事業費率を把握することができるようになる。
\end{itemize}
\end{itemize}

\problem{2019 生保2問題 1(5)、H27 生保2問題 1(4)、H17 生保2問題 2(4)、H9 生保2問題 2(2)}
区分経理における全社区分の機能およびその財源の代表的な例について、簡潔に説明せよ。
\answer{}

\begin{itemize}
\item[]  死亡保障リスク等のリスクバッファー機能\par
  死亡保障リスク、予定利率リスク、価格変動等リスク、経営管理リスク等に対応するためのリスクバッファー機能
\item[]  新商品開発に係る事業運営資金提供機能\par
  保険業法第106条または第108条(現保険業法では削除済)の規定に基づく子会社への出資を含む。
\item[]  会社全体で共有する資産・共通する経費等の管理機能
\item[]  現預金等の管理機能
\end{itemize}

[全社区分の財源]
全社区分には、基金または資本金、法定準備金、任意積立金(配当平衡積立金を含む)
等の資本、(ただし、未処分利益または未処分剰余金を除く。)および危険準備金、価格変
動準備金、退職給与引当金等の負債、その他いずれの商品区分にも帰属していない負債・
資本の全部または一部を配賦する。

また、資産には、全社区分の機能を果たすため、営業用不動産、動産、現預金等を配賦
する。これらの資産が全社区分に帰属する負債または資本を超過する場合には、その超過
分は、商品区分から借入れをするか出資を受けることとなる。

\problem{H22 生保2問題 2(2)}
区分経理における商品区分と全社区分との取引について分類し、簡潔に説明しなさい。
\answer{}
商品区分と全社区分との間の取引は、資金の流動一性の確保又は保険金等の円滑な支払いのため
に行われる貸借、出資、その他の取引に分類される。

\begin{itemize}
\item[] 貸借
\begin{itemize}
\item[] 現預金等の貸借\par
貸借ごとに他の商品区分又は全社区分と区別して管理する。また借越しが継続しないように
限度額等を設ける必要がある。
\item[] 現預金等以外の貸借(貸付)\par
全社区分から商品区分への貸付は、異常な保険金の支払い、新商品の販売に伴う事業運営資
金、その他やむを得ない事情がある場合に限定される。また、商品区分から全社区分への貸付
は、全社区分の規模が小さいためにその機能を十分に果たすことができない場合に限定される。
貸付を行う場合、金額、利率、期限その他の返済条件をあらかじめ定めておく必要がある。
\end{itemize}
\item[] 出資
全社区分から商品区分への出資および商品区分から全社区分への出資も、貸付と同様の場合
に限定される。出資を受けた商品区分又は全社区分において、剰余金が発生した場合、出資に
対応する金額を出資した商品区分又は全社区分に配分される。
\item[] その他の取引
\begin{itemize}
\item[] 資本又は危険準備金等の積み増し、取り崩しに係る取引
\item[] 転換等により、責任準備金等を転換後等の商品区分に支払う取引
\item[] 新契約費を全社区分から支払う場合に、商品区分から全社区分に新契約費相当分を支払う取引
\item[] 全社区分における共有資産等に対する対価として、各商品区分が使用料等を支払う取引
\item[] 商品区分における特定のリスク発生による損失実現時に、全社区分から当該商品区分に当該
損失実現額を支払う取引(あらかじめ保険数理的に定められた対価を支払ったものに限る。)
\item[] 商品区分または全社区分において、将来回復が見込めない重大な損害が発生し、全社区分ま
たは商品区分からその損害のてん補を受ける取引
\end{itemize}
\end{itemize}



\end{document}