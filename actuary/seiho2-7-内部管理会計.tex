\documentclass[report,gutter=10mm,fore-edge=10mm,uplatex,dvipdfmx]{jlreq}
\usepackage{lmodern}
\usepackage{amssymb,amsmath}
\usepackage{mathtools}
\usepackage{ifxetex,ifluatex}
\usepackage{actuarialsymbol}
\usepackage[]{natbib}

%strike through 
%https://tex.stackexchange.com/questions/23711/strikethrough-text
%\usepackage[]{ulem}

\usepackage[normalem]{ulem}
\usepackage{enumerate}

% Tables
\usepackage{multirow}
\usepackage{tabularx}
%\usepackage{booktabs} % http://www.yamamo10.jp/yamamoto/comp/latex/make_doc/table/table.php

%Framedbox
%https://hakuoku.github.io/agakuTeX/tutorial/5_6framed/
\usepackage{framed}

%https://tgnx8810.wordpress.com/2014/11/29/latex%E3%81%A7%E8%A1%A8%E3%81%AE%E3%82%BB%E3%83%AB%E5%86%85%E6%94%B9%E8%A1%8C%E3%81%AFtabularx%E7%92%B0%E5%A2%83%E3%82%92%E4%BD%BF%E3%81%86%E3%81%A8%E6%A5%BD/
\usepackage{longtable}
\usepackage{booktabs}
\RequirePackage{plautopatch}

% maru suji ① etc.
\usepackage{tikz}
\newcommand{\cir}[1]{\tikz[baseline]{%
\node[anchor=base, draw, circle, inner sep=0, minimum width=1.2em]{#1};}}

%http://yamamo10.jp/yamamoto/comp/latex/make_doc/box/box.php
%枠付き文章
\usepackage{ascmac}
\usepackage{fancybox}

\usepackage{comment}

\begin{comment}

\ifnum0\ifxetex1\fi\ifluatex1\fi=0 % if pdftex
  \usepackage[T1]{fontenc}
  \usepackage[utf8]{inputenc}
  \usepackage{textcomp} % provide euro and other symbols
\else % if luatex or xetex
  \usepackage{unicode-math}
  \defaultfontfeatures{Scale=MatchLowercase}
  \defaultfontfeatures[\rmfamily]{Ligatures=TeX,Scale=1}
\fi
% Use upquote if available, for straight quotes in verbatim environments
\IfFileExists{upquote.sty}{\usepackage{upquote}}{}
\IfFileExists{microtype.sty}{% use microtype if available
  \usepackage[]{microtype}
  \UseMicrotypeSet[protrusion]{basicmath} % disable protrusion for tt fonts
}{}
\makeatletter
\@ifundefined{KOMAClassName}{% if non-KOMA class
  \IfFileExists{parskip.sty}{%
    \usepackage{parskip}
  }{% else
    \setlength{\parindent}{0pt}
    \setlength{\parskip}{6pt plus 2pt minus 1pt}}
}{% if KOMA class
  \KOMAoptions{parskip=half}}
\makeatother
\usepackage{xcolor}
\IfFileExists{xurl.sty}{\usepackage{xurl}}{} % add URL line breaks if available
\IfFileExists{bookmark.sty}{\usepackage{bookmark}}{\usepackage{hyperref}}
\hypersetup{
  hidelinks,
  pdfcreator={LaTeX via pandoc}}
\urlstyle{same} % disable monospaced font for URLs
\usepackage{longtable,booktabs}
% Correct order of tables after \paragraph or \subparagraph
\usepackage{etoolbox}
\makeatletter
\patchcmd\longtable{\par}{\if@noskipsec\mbox{}\fi\par}{}{}
\makeatother
% Allow footnotes in longtable head/foot
\IfFileExists{footnotehyper.sty}{\usepackage{footnotehyper}}{\usepackage{footnote}}

\end{comment}
%\makesavenoteenv{longtable}
\setlength{\emergencystretch}{3em} % prevent overfull lines
\providecommand{\tightlist}{%
  \setlength{\itemsep}{0pt}\setlength{\parskip}{0pt}}
\setcounter{secnumdepth}{-\maxdimen} % remove section numbering

\author{kazuyoshi}
\date{}





\begin{document}
\chapter{保険2第7章 内部管理会計}
\section{7.1 内部管理会計の意義}
\problem{H12 生保2問題 1(6)}
次の①~⑥を適切な語句で埋めよ。

生命保険会計において「内部管理会計」という場合、主に、経営管理や経営上の意思決定に役立つ
ような会計情報を\wakumaru{①}向けに提供することを目的とする会計システムを意味する。従って、
(A)「\wakumaru{②}や\wakumaru{③}の的確な把握」や、
(B)「\wakumaru{④}の詳細な把握」に関して
\wakumaru{⑤}等を補足することが内部管理会計の重要なテーマとなる。
(A)を目的とする内部管理会計手法としては、例えば米国
GAAP 会計などを候補とすることができる。一方(B)を目的とする内部管理会計手法としては、例え
ば区分経理を候補とすることができる。
\answer{}
\begin{itemize}
\item[ ①: ] 経営者(経営陣等も可)
\item[ ②: ] 経営成績、(業績等も可)[期間損益]
\item[ ③: ] 期間損益(期間収益、期間収支等も可)[経営成績]
\item[ ④: ] 収支構造(収益構造等も可)
\item[ ⑤: ] 法定会計(SAP,SAP会計も可)
\end{itemize}

\section{7.2 内部管理会計の必要性}
\problem{H26 生保2問題 2(3)}
内部管理会計の意義および必要性について、現行法定会計の特徴と限界に触れつつ、簡潔に説
明しなさい。
\answer{}
<内部管理会計の意義>
\begin{itemize}
 \item[] 法定会計は一般に保険会社のソルベンシー確保を目的として保険監督当局が提出を要求するもの
 であり、GAAP 会計は一般投資家等への会社の会計情報の提供を目的として作成されるものであ
 って、これらの提供する会計情報は、必ずしも経営者の経営判断に役立つものとは限らない。
 \item[] そこで、経営者の経営判断に役立つ会計情報の提供を目的とした会計システムたる内部管理会計
 が必要となる。その内容や属性は、企業のおかれた環境や時代により又企業自体の規模や性格に
 より相違するものと考えるべきであろうが、現行の法定会計では必ずしも十分捉えきれない。内
 部管理会計が経営判断に役立つ会計情報の提供を目的とする会計である以上、「経営成績や期間
 損益の的確な把握」や「収支構造の詳細な把握」などに関して、法定会計等を補足することが内
 部管理会計の存在意義と言えよう。
 \item[] 内部管理会計に望まれる属性としては、以下が挙げられる。
\begin{itemize}
  \item[] 会社の事業の基礎となる経済的な基盤(資金調達方法など)を反映するものであること
 \item[] 経営者が、区分が必要と考えるプロフィット・センター毎(プロダクト・ライン別、チャネ
 ル別、戦略事業単位別など)に結果が得られること
 \item[] 結果が上層部の経営者にとって理解しやすいものであること
\end{itemize}
\end{itemize}
<内部管理会計の必要性>
\begin{itemize}
 \item[] 現行法定会計の限界とそれを踏まえた内部管理会計の必要性は以下のとおりである。\\
\item[] (経営成績や期間損益を的確に把握する内部管理会計の必要性)
\begin{itemize}
  \item[] 長期の評価性債務を抱える生命保険会社の法定会計は、標準責任準備金制度に基づく保守的な責
 任準備金の積立等による支払能力の確保を重視する。このため、一般的に、コミッション等の関
 係で新契約の獲得が単年度利益にマイナスの影響を与える一方、解約控除の存在により解約契約
 の増加が単年度利益にプラスの影響を及ぼすといった課題がある。
 \item[] 高度の経営判断に用いる会計としては、現在の経営成績の状況を適切に表示する、期間損益を的
 確に把握し得る会計制度が必要となる。\\
\end{itemize} 
\item[](保険種類毎の収支構造を把握する内部管理会計の必要性)
\begin{itemize}
  \item[] 生命保険会社を巡る事業環境の急激な変化の中で、過去の「単一の価格設定」
 「均等な資産運用」
 といった従来の一般勘定における一括管理の手法が限界を迎えたことから、リスク管理の高度化、
 利用者ニーズへの対応の観点から、保険種類毎の収支構造の把握に向けた区分経理が不可欠とな
 り、平成8年の保険業法改正時に導入された。
 \item[] その後の商品内容・給付およびチャネルの多様化等を踏まえると、区分経理をさらに細分化した
 保険種類毎の収支構造等を把握する内部管理会計等の必要性も高まってきている。
\end{itemize}
\end{itemize}
\section{7.3 経営成績や期間損益を的確に把握する内部管理会計}

\problem{H13 生保2問題 1(10)}
エンベディッド・バリュー(Embedded Value)とアプレイザル・バリュー(Appraisal Value)に
ついて簡潔に説明せよ。
\answer{}
「エンベディッド・バリュー」は、価値基準会計における生命保険会社の
経済的価値であり、純資産の額に保有契約の経済的価値(将来その契約から
期待される法定会計上の利益からソルベンシー・マージン等の必要サープラ
スの増加分を差し引いた利益(使用可能利益)をリスク割引率(ハードル・
レート)で割り引いた現価)を加えたものである。

「アプレイザル・バリュー」は、生命保険会社が保有すると考えられる新契
約獲得のポテンシャリティを、将来見込まれる新契約から得られるであろう
将来の年々の法定会計上の損益の割引現価により評価し、「エンベディッド・
バリュー」に加算したものである。

\problem{2019 生保2問題 3(2)②}
潜在価値会計(エンベディッド・バリュー(EV))の概要(意義、考え方、特徴など)について簡潔
に説明しなさい。

\answer{}
<意義・目的>
\begin{itemize}
 \item[] 経営者や投資家等にとって、法定会計による財務情報のみでは、生命保険会社の経営成績や企業価 値を読み取るのが困難と考えられる。
 \item[] 潜在価値会計は、生命保険会社の経営成績の実態に則した会計であり、生命保険会社の経済的価値 を表すことから、法定会計による財務情報を補うことができる。
\end{itemize}
<考え方>
\begin{itemize}
 \item[] 潜在価値会計は、保険契約から生じる将来のキャッシュ・フローを予測し、それから計算される将来の期待利益の割引現価と、純資産等に基づき定義される「生命保険会社の経済的価値」を計算す
 るとともに、その額の年間の変化量により当期純利益を計算しようとするもので、資産負債法によ
 る会計の1つである。
\end{itemize}
<特徴>
\begin{itemize}
 \item[] 新契約時に将来利益が認識され、解約時に将来利益の喪失が認識されるため、経営成績の実態に即 している。
 \item[] 経済的価値の水準は、計算前提により大きく変化するため、期間損益の把握は容易ではない。
また、計算前提の設定には共通原則がないため、客観性、比較可能性が十分でない。
 \item[] 経済的価値の把握のため、ロックイン方式は適用されない。
\end{itemize}

\problem{H12 生保2問題 1(7)}
次の①~⑤に入る適当な語句を、以下の A~O から記号を選択せよ。

ハードル・レートは\wakumaru{①}という指標との比較を通じて商品の\wakumaru{②}と関係を持つ。
\wakumaru{①}は商品から得られる
\wakumaru{③}の現在価値とその商品の販売のために使った\wakumaru{④}が等しくなる
\wakumaru{⑤}であり、IRR とも呼ばれる。

語群
\begin{tabular}{lll}
A.資本 & B.キャッシュフロー& C.平均利回り\\
D.負債 &E.経済的価値 & F.平準 ROE\\
G.繰延べ資産 & H.プロフィットマージン&  I.運用利回り\\
J.プライシング & K.必要サープラス&L.将来利益 \\
M.当期利益 &N.ROI &O.アクルーアル方式 \\
\end{tabular}

\answer{}
\begin{itemize}
\item[①] N; ROI
\item[②] J; プライシング
\item[③] L; 将来利益 
\item[④] A; 資本
\item[⑤] C; 平均利回り
\end{itemize}

\problem{H9 生保2問題 2(1)}
「ハードル・レート」について、簡潔に説明せよ。
\answer{}
内部管理会計の手法である価値基準会計等において、将来のキャッシュフローの予
測から期待される法定会計上の利益の合計額を契約時点で利益計上するために用いる
割引率。利益が実現するまでの時間的な遅れと利益実現に関する不確実性のリスクを
考慮した上で株主等が投下資本に対して期待する収益率であり、資本の調達コストに
対応するもの。

\problem{H30 生保2問題 1(6)}
価値基準会計(潜在価値会計)等におけるハードル・レート(リスク割引率)の設定方法である「ト
ップ・ダウンアプローチ」と「ボトム・アップアプローチ」についてそれぞれ簡潔に説明しなさい。
\answer{}
トップ・ダウンアプローチは、会社のリスク特性に基づき、すべての商品について単一の
割引率を適用する方法である。通常、リスク割引率を、加重平均資本コスト(WACC)
を計算して求める。

ボトム・アップアプローチは、各キャッシュ・フローのリスク特性に基づきリスク割引
率を設定する方法である。各契約ラインに付随するリスクを反映して割引率を設定するも
のであり、トップ・ダウンアプローチよりも透明性の高い、各事業リスクに適した割引率
の設定が可能となるが、実務上の取り扱いは煩雑となる。市場整合的EVは代表的なボト
ム・アップアプローチに分類される。

\problem{H20 生保2問題 2(4)}
潜在価値会計に関し、次の①、②の各問に答えなさい。

\begin{itemize}
\item[①] 潜在価値会計における資本コスト(Cost of Capital)について簡潔に説明しなさい。
\item[②] 下記の条件において、潜在価値会計のトップダウンアプローチによるハードル・レート(リスク割 引率)を、WACC を用いて計算しなさい。解答にあたっては計算過程も記載しなさい。
\begin{tabular}{ll}
リスクフリー・レート(リスクフリーの投資収益率)&2.00%  \\
 債務コスト(税引後)&2.50%\\
 株式投資の平均的収益率(市場ポートフォリオの期待収益率) &4.00%\\
 (当該保険会社の)ベータ&1.50\\
 株主資本(株式時価総額)の割合&90%\\
 債務(社債・借入金の時価総額)の割合&10%\\
\end{tabular} 
\end{itemize}
\answer{}
①

資本コストとは、計算基準日に留保するソルベンシー・キャピタル(必要資本)から、将来的に
リリースされるソルベンシー・キャピタル及びソルベンシー・キャピタルに対する税引後運用収
益の現価(リスク割引率による)を差し引くことによって計算されるコストのことである。

(別解1)将来のソルベンシー・キャピタルのリリースの現価\\
+ 将来のソルベンシー・キャピタルに対する税引後運用収益の現価\\
- 計算基準日に留保するソルベンシー・キャピタル

(別解2)将来におけるソルベンシー・キャピタル\\
×(ハードル・レート - 資産運用利回り(税引後))の現価\\

つまり、目標ソルベンシー・マージン比率等を達成するために拘束されたソルベンシー・キャピ
タルは、リリースされるまでは資産運用収益しか生み出さないが、資本の提供者はハードル・レ
ートによる投資の機会を求めているため、資産運用利回りがハードル・レートに満たない場合、
資本の提供者は投資機会を喪失したことによるコストを負担していると考えることができる。

(※1)市場整合的EVにおける摩擦的資本コスト等に触れている場合、加点した。

(※2)別解のように、資本コストを表現する式については文献により正負の定義等が異なる場合がある。

②

株主資本コスト=2.00% + 1.5×(4.00% - 2.00%) = 5.00%\\
WACC=90%×5.00% + 10%×2.50% = 4.75%

\end{document}