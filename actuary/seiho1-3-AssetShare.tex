\documentclass[report,gutter=10mm,fore-edge=10mm,uplatex,dvipdfmx]{jlreq}

\usepackage{lmodern}
\usepackage{amssymb,amsmath}
\usepackage{ifxetex,ifluatex}
\usepackage{actuarialsymbol}
\usepackage[]{natbib}
\RequirePackage{plautopatch}

% maru suji ① etc.
\usepackage{tikz}
\newcommand{\cir}[1]{\tikz[baseline]{%
\node[anchor=base, draw, circle, inner sep=0, minimum width=1.2em]{#1};}}

\usepackage{comment}

\begin{comment}

\ifnum0\ifxetex1\fi\ifluatex1\fi=0 % if pdftex
  \usepackage[T1]{fontenc}
  \usepackage[utf8]{inputenc}
  \usepackage{textcomp} % provide euro and other symbols
\else % if luatex or xetex
  \usepackage{unicode-math}
  \defaultfontfeatures{Scale=MatchLowercase}
  \defaultfontfeatures[\rmfamily]{Ligatures=TeX,Scale=1}
\fi
% Use upquote if available, for straight quotes in verbatim environments
\IfFileExists{upquote.sty}{\usepackage{upquote}}{}
\IfFileExists{microtype.sty}{% use microtype if available
  \usepackage[]{microtype}
  \UseMicrotypeSet[protrusion]{basicmath} % disable protrusion for tt fonts
}{}
\makeatletter
\@ifundefined{KOMAClassName}{% if non-KOMA class
  \IfFileExists{parskip.sty}{%
    \usepackage{parskip}
  }{% else
    \setlength{\parindent}{0pt}
    \setlength{\parskip}{6pt plus 2pt minus 1pt}}
}{% if KOMA class
  \KOMAoptions{parskip=half}}
\makeatother
\usepackage{xcolor}
\IfFileExists{xurl.sty}{\usepackage{xurl}}{} % add URL line breaks if available
\IfFileExists{bookmark.sty}{\usepackage{bookmark}}{\usepackage{hyperref}}
\hypersetup{
  hidelinks,
  pdfcreator={LaTeX via pandoc}}
\urlstyle{same} % disable monospaced font for URLs
\usepackage{longtable,booktabs}
% Correct order of tables after \paragraph or \subparagraph
\usepackage{etoolbox}
\makeatletter
\patchcmd\longtable{\par}{\if@noskipsec\mbox{}\fi\par}{}{}
\makeatother
% Allow footnotes in longtable head/foot
\IfFileExists{footnotehyper.sty}{\usepackage{footnotehyper}}{\usepackage{footnote}}

\end{comment}
%\makesavenoteenv{longtable}
\setlength{\emergencystretch}{3em} % prevent overfull lines
\providecommand{\tightlist}{%
  \setlength{\itemsep}{0pt}\setlength{\parskip}{0pt}}
\setcounter{secnumdepth}{-\maxdimen} % remove section numbering

\author{kazuyoshi}
\date{}

\newcommand{\problem}[1]{\subsubsection{#1}\setcounter{equation}{0}}
%\newcommand{\answer}[1]{\subsubsection{#1}}
\newcommand{\answer}[1]{\subsubsection{解答}}

%Pdf%\newcommand{\wakumaru}[1]{\framebox[3zw]{#1}}
\newcommand{\wakumaru}[1]{#1}







\begin{document}
\chapter{生保1第3章アセットシェア}
\section{3.1 序論}
\subsection{アセット・シェアの定義}
\problem{H12 生保1問題 1(4)}
次の①~⑤を適当な語句で埋めよ。
アセット・シェアは、一般的には、
「①を保険数理上同質と認められる②に区分し、
これから生じる③を実績に基づく
運用利回り、死亡率、事業費、解約失効率等を用いて計算して得られる④を
ある時点において各契約に割り当てた個々契約の持分もしくは⑤」として定義される。
\answer{}
\noindent
① 保有契約\\
② 群団\\
③ キャッシュ・フロー\\
④ 正味資産\\
⑤ 貢献度

\problem{H13 生保1問題 1(2)}
次について、正しいものには○、誤りのあるものには×をつけよ。\\
① 契約のアセット・シェアと対応責任準備金の差をネット・アセット・シェアという。\\
②保険計理人の実務基準第23条では、配当の確認におけるアセット・シエア方式の利用について記載しているが、ここで代表契約選定単位の最低限の区分として、a.区分経理の商品区分、b.保険事故の種類、c.契約経過年度、の3つがあげられている。\\
③保険計理人は、実務基準における配当の確認において、代表契約について翌年度に支払われる通常配当と、当該契約が翌年度に消滅した場合に支払われる消滅時配当の合計が、当年度末アセット・シェアを決して超えないことを確認しなくてはならない。\\
④保険業法施行規則第25条によれば、生命保険相互会社における剰余金の分配はアセット・シェア方式によらなくてはならない。\\
⑤アセット・シェアの具体的な計算においては、「契約群団方式」と「代表契約方式」の2とおりがある。
\answer{}
\noindent
① ○\\
② ○\\
③ ×

原則として超えないことを確認するにとどまり「決して超えない」ことの確認ではない\\
④ ×

3利源方式なども認められている。\\
⑤ ○

\problem{H9 生保1問題 1(4)、H1 生保1問題 1(3)}
アセット・シェアについて簡潔に説明せよ。

次の①〜⑤を適当な語句で埋めよ。
アセットシェアとは、①、②、同じ契約応当日、同
料率、同保険価格である同一とみなされる多数の
③の集積された④が、ある時点で群団の全ての契約に分配された場合に、個
々の契約の⑤を保険金額に対して表したものである。

\answer{}
\noindent
①同じ種類(同じ契約年齢)\\
②同じ契約年齢(同じ種類)\\
③契約群団\\
④正味資産\\
⑤持分

\problem{H5 生保1問題 2(1)}
ヒストリカル・アセットシェアとプロジェクティド・アセットシェアについて説明せよ。
\answer{}
アセットシェアとは、同じ種類、同じ契約年齢、同じ契約応当日、同料率、同保
険価格である同一とみなされる多数の契約群団の集積された正味資産が、ある時点
で群団の全ての契約に分配された場合に、個々の契約の持分を保険金額に対して
表わしたものである。
アセットシェアの計算は、継続中の契約に対する「過去の経験」に基づいて行わ
れる場合(ヒストリカル・アセットシェア)と将来のある時点に対して「想定される経験」
に基づいて行われる場合(プロジェクティド・アセットシェア)とがあり、
前者は据え置き期間付配当支払の契約の配当率を決定する場合とか、積立支払方式
の配当について会社の支払義務額が現在との位であるか調べる場合などに用いられ、
後者は、新しい保険を設計する場合や保険料率、不没収価格、配当率の適否を判断
する材料として重要な手法となる。

\subsection{アセット・シェア計算の目的}
\problem{H14 生保1問題 1(1)}

アセット・シェア計算の代表的な活用目的について、次の①~⑤を適当な語句で埋めよ。

(a)将来の利益目標、例えば毎年、保険金額の一定割合のネット・アセット・シェアの確保等
を定めて①を算出する。

(b)責任準備金の充分性の確認として、責任準備金が②を維持しえるかどうかを
アセット・シェアの手法を援用した③を使って判断する。

(c)配当を分配するにあたり・各契約群団での分配可能額の算定や各群団間で④な取扱と
なっているか否かの検証に活用する。

(d)保険相互会社における株式会社化、清算等の会社組織変更にあたっては、社員毎の⑤の
算定が必要となるが、この場合にもアセット・シェアが活用される。
\answer{}
\noindent ①保険料率\\②保険金支払能力\\
③将来収支分析\\④公正、衡平\\⑤持分資産
\problem{H24 生保1問題 1(4)、H17 生保1問題 2(2)}
アセット・シェア計算の代表的な活用目的を 4 つ挙げ、それぞれについて簡潔に説明せよ。
\answer{}
\noindent
① 保険料率の計算\\
・ 将来の利益目標(例えば、毎年保険金額の一定割合のネット・アセット・シェアを確保する、
あるいは、一定年数後にその時点での解約返戻金の一定倍率のアセット・シェアを確保する
等)を定めて保険料率を算出する。\\
・ 無配当保険の場合に比較的良く用いられる。有配当保険においても前提となる要素が増加し
て計算が複雑になるが、考え方は同様である。\\
② 責任準備金の充分性の確認\\
・ 保険会社の負債としての責任準備金が、保険金支払い能力を維持し得るかどうか、アセッ
ト・シェアの手法を利用した将来収支分析を行って判断する。\\
③ 配当率決定と財源確認\\
・ 配当を分配するにあたり、各契約群団での分配可能額の算定や各群団間で公正衡平な取扱い
となっているか否かの検証に活用する。\\
④ 会社組織変更における社員持分資産の確定\\
・ 保険相互会社についての、株式会社化、清算等の会社組織変更にあたっては、社員毎の持分
資産の算定が必要となるが、この場合にもアセット・シェアが活用される。\\
上記の他にも、解約返戻金の水準検証、商品販売政策の立案、営業職員給与規定の検証等に活用
される。

\section{3.2 アセット・シェア計算の原理}
\subsection{アセット・シェア計算の考え方}
\problem{2019 生保1問題 1(5)、H19 生保1問題 1(4)}
アセット・シェアの計算における「契約群団方式」と「代表契約方式」について簡潔に説明せよ。
\answer{}
\noindent
く契約群団方式>
契約群団として包括的にアセット・シェア計算を行う方式。契約群団の設定にあたっ
ては、その群団を構成する保険契約が損益の発生状況のうえで同等とみなせる範囲で
設定することが求められる。区分経理上の商品区分、保険事故の種類、契約経過年度
別が最低限の群団化となるが、実際にはさらなる細分化が必要と想定される。ただし、
精度の向上と実務負荷はトレード・オフの関係にあるため、アセット・シェアの活用
目的や重要性に応じて判断することになる。\\
〈代表契約方式>
各契約群団から代表契約を選定して、この契約のアセット・シェアが当該契約群団を
構成する保険契約のアセット・シェアを代表するとみなす方式。この方式は実務負荷
に配慮しつつ精度向上を図る際に有効な手法といえる。
さらに、代表契約方式には、

・契約群団を代表契約1件で代表できるまで細分化(セル細分)する方式

・契約群団の細分化は一定レベルにとどめ、契約1件で代表できないときは複数件
の代表契約を選定する方式\\
の両者がある。後者の方式は、計算は簡易にできる長所はあるが、契約群団を大括り
に設定する場合には、代表契約の選定にあたり、その妥当性を充分に検証する必要が
ある。
なお、日本アクチェアリー会の「生命保険会社の保険計理人の実務基準」に規定され
る配当財源の確認に際してのアセット・シェアの活用にあたっては、代表契約方式の
採用が明記されている。

\subsection{アセット・シェアの計算問題}

[教科書 3-6より]

$
AS_t=\{(AS_{t-1}+\pi - E_t)(1+i_t) - (q_t^d+W_tq_t^w)(1+i_t)^{1/2}-(1-q_t^w)D_t\}/(1-q_t^d-q_t^w)
$

\noindent
$AS_t$: 保険金額1あたり第ι年度末有効契約に対するアセット・シェア $(AS_0=0)$\\
$\pi$: 営業保険料率\\
$E_t$: 第$t$年度事業比率\\
$W_t$: 第$t$年度解約返戻金率\\
$D_t$: 第$t$年度配当率 (3年目配当の場合には$D_1=0$)\\
$i_t$: 第$t$年度資産運用利回り\\
$q_t^d$: 第$t$年度死亡率\\
$q_t^d$: 第$t$年度解約・失効率 (対年始率)

\problem{H9 生保1問題 1(5)}
以下について五つの選択肢の中から正しい答を一つ選べ。

下記の無配当終身保険の第 4 保険年度末のアセットシェアが 48,544 円(対保険金額 100 万円)であ
るとき、第 5 保険年度末のアセットシェアの値は次のうちどれに最も近いか。

(A)64,430 円
(B)66,220 円
(C)67,770 円
(D)67,880 円
(E)67,920 円

\noindent
計算前提\\
・無配当終身保険 男性
30 歳契約 60 歳払込完了
年払営業保険料 18,000 円(対保険金額 100 万円)\\
・死亡率 qx; q30=0.67‰,q31=0.68‰,q32=0.70‰ q33=0.74‰,q34=0.78‰,q35=0.82‰\\
・解約率初年度 10%,次年度以降 5%\\
・事業費率初年度 20‰,次年度以降 2‰(対保険金額)\\
・解約返戻金率 初年度 0‰,次年度 10‰,3 年度 20‰,4 年度 35‰,5 年度 50‰(対保険金額)\\
・運用利回り5%\\
・死亡は保険年度の中央で発生し、保険金は保険年度の中央で支払うものとする。解約は保険年度末
で発生し、解約返戻金は保険年度末で支払うものとする。また、事業費は保険年度始に支出される
ものとする。
\answer{}
E\\
第5保険年度末アセットシェア=\\
{(48,544+18,000-1,000,000×0.002)×\\
1.05−1,000,000×0.00078×1.025 −\\
1,000,000X0.05×0.05}/\\
(1-0.00078-0.05)\\
≒67,920


\end{document} 