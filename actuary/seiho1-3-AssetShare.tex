\documentclass[report,gutter=10mm,fore-edge=10mm,uplatex,dvipdfmx]{jlreq}

\usepackage{lmodern}
\usepackage{amssymb,amsmath}
\usepackage{ifxetex,ifluatex}
\usepackage{actuarialsymbol}
\usepackage[]{natbib}
\RequirePackage{plautopatch}

% maru suji ① etc.
\usepackage{tikz}
\newcommand{\cir}[1]{\tikz[baseline]{%
\node[anchor=base, draw, circle, inner sep=0, minimum width=1.2em]{#1};}}

\usepackage{comment}

\begin{comment}

\ifnum0\ifxetex1\fi\ifluatex1\fi=0 % if pdftex
  \usepackage[T1]{fontenc}
  \usepackage[utf8]{inputenc}
  \usepackage{textcomp} % provide euro and other symbols
\else % if luatex or xetex
  \usepackage{unicode-math}
  \defaultfontfeatures{Scale=MatchLowercase}
  \defaultfontfeatures[\rmfamily]{Ligatures=TeX,Scale=1}
\fi
% Use upquote if available, for straight quotes in verbatim environments
\IfFileExists{upquote.sty}{\usepackage{upquote}}{}
\IfFileExists{microtype.sty}{% use microtype if available
  \usepackage[]{microtype}
  \UseMicrotypeSet[protrusion]{basicmath} % disable protrusion for tt fonts
}{}
\makeatletter
\@ifundefined{KOMAClassName}{% if non-KOMA class
  \IfFileExists{parskip.sty}{%
    \usepackage{parskip}
  }{% else
    \setlength{\parindent}{0pt}
    \setlength{\parskip}{6pt plus 2pt minus 1pt}}
}{% if KOMA class
  \KOMAoptions{parskip=half}}
\makeatother
\usepackage{xcolor}
\IfFileExists{xurl.sty}{\usepackage{xurl}}{} % add URL line breaks if available
\IfFileExists{bookmark.sty}{\usepackage{bookmark}}{\usepackage{hyperref}}
\hypersetup{
  hidelinks,
  pdfcreator={LaTeX via pandoc}}
\urlstyle{same} % disable monospaced font for URLs
\usepackage{longtable,booktabs}
% Correct order of tables after \paragraph or \subparagraph
\usepackage{etoolbox}
\makeatletter
\patchcmd\longtable{\par}{\if@noskipsec\mbox{}\fi\par}{}{}
\makeatother
% Allow footnotes in longtable head/foot
\IfFileExists{footnotehyper.sty}{\usepackage{footnotehyper}}{\usepackage{footnote}}

\end{comment}
%\makesavenoteenv{longtable}
\setlength{\emergencystretch}{3em} % prevent overfull lines
\providecommand{\tightlist}{%
  \setlength{\itemsep}{0pt}\setlength{\parskip}{0pt}}
\setcounter{secnumdepth}{-\maxdimen} % remove section numbering

\author{kazuyoshi}
\date{}

\newcommand{\problem}[1]{\subsubsection{#1}\setcounter{equation}{0}}
%\newcommand{\answer}[1]{\subsubsection{#1}}
\newcommand{\answer}[1]{\subsubsection{解答}}

%Pdf%\newcommand{\wakumaru}[1]{\framebox[3zw]{#1}}
\newcommand{\wakumaru}[1]{#1}







\begin{document}
\chapter{生保1第3章アセットシェア}
\section{3.1 序論}
\problem{H12 生保1問題 1(4)}
次の①~⑤を適当な語句で埋めよ。
アセット・シェアは、一般的には、
「①を保険数理上同質と認められる②に区分し、
これから生じる③を実績に基づく
運用利回り、死亡率、事業費、解約失効率等を用いて計算して得られる④を
ある時点において各契約に割り当てた個々契約の持分もしくは⑤」として定義される。
\answer{}
① 保有契約
② 群団
③ キャッシュ・フロー
④ 正味資産
⑤ 貢献度

\problem{H13 生保1問題 1(2)}
次について、正しいものには○、誤りのあるものには×をつけよ。
① 契約のアセット・シェアと対応責任準備金の差をネット・アセット・シェアという。
②保険計理人の実務基準第23条では、配当の確認におけるアセット・シエア方式の利用について記載しているが、ここで代表契約選定単位の最低限の区分として、a.区分経理の商品区分、b.保険事故の種類、c.契約経過年度、の3つがあげられている。
③保険計理人は、実務基準における配当の確認において、代表契約について翌年度に支払われる通常配当と、当該契約が翌年度に消滅した場合に支払われる消滅時配当の合計が、当年度末アセット・シェアを決して超えないことを確認しなくてはならない。
④保険業法施行規則第25条によれば、生命保険相互会社における剰余金の分配はアセット・シェア方式によらなくてはならない。
⑤アセット・シェアの具体的な計算においては、喫約群団方式」と「代表契約方式」の2とおりがある。
\answer{}
① ○
② ○

③ ×
原則として超えないことを確認するにとどまり「決して超えない」ことの確認ではない

④ ×
3利源方式なども認められている。

⑤ ○

\problem{H9 生保1問題 1(4)、H1 生保1問題 1(3)}
アセット・シェアについて簡潔に説明せよ。

次の①〜⑤を適当な語句で埋めよ。
アセットシェアとは、①、②、同じ契約応当日、同
料率、同保険価格である同一とみなされる多数の
③の集積された④が、ある時点で群団の全ての契約に分配された場合に、個
々の契約の⑤を保険金額に対して表したものである。

\answer{}
①同じ種類(同じ契約年齢)、②同じ契約年齢(同じ種類)、
③契約群団、④正味資産、⑤持分

\problem{H5 生保1問題 2(1)}
ヒストリカル・アセットシェアとプロジェクティド・アセットシェアについて説明せよ。
\answer{}
アセットシェアとは、同じ種類、同じ契約年齢、同じ契約応当日、同料率、同保
険価格である同一とみなされる多数の契約群団の集積された正味資産が、ある時点
で群団の全ての契約に分配された場合に、個々の契約の持分を保険金額に対して表わしたものである。
アセットシェアの計算は、継続中の契約に対する「過去の経験」に基づいて行わ
れる場合(ヒストリカル・アセットシェア)と将来のある時点に対して「想定される経験」に基づいて行われる場合(プロジェクティド・アセットシェア)とがあり、
前者は据え置き期間付配当支払の契約の配当率を決定する場合とか、積立支払方式
の配当について会社の支払義務額が現在との位であるか調べる場合などに用いられ、
後者は、新しい保険を設計する場合や保険料率、不没収価格、配当率の適否を判断
する材料として重要な手法となる。
\end{document} 