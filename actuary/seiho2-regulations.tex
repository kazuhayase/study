\documentclass[report,gutter=10mm,fore-edge=10mm,uplatex,dvipdfmx]{jlreq}

\usepackage{lmodern}
\usepackage{amssymb,amsmath}
\usepackage{mathtools}
\usepackage{ifxetex,ifluatex}
\usepackage{actuarialsymbol}
\usepackage[]{natbib}

%strike through 
%https://tex.stackexchange.com/questions/23711/strikethrough-text
%\usepackage[]{ulem}

\usepackage[normalem]{ulem}
\usepackage{enumerate}

% Tables
\usepackage{multirow}
\usepackage{tabularx}
%\usepackage{booktabs} % http://www.yamamo10.jp/yamamoto/comp/latex/make_doc/table/table.php

%Framedbox
%https://hakuoku.github.io/agakuTeX/tutorial/5_6framed/
\usepackage{framed}

%https://tgnx8810.wordpress.com/2014/11/29/latex%E3%81%A7%E8%A1%A8%E3%81%AE%E3%82%BB%E3%83%AB%E5%86%85%E6%94%B9%E8%A1%8C%E3%81%AFtabularx%E7%92%B0%E5%A2%83%E3%82%92%E4%BD%BF%E3%81%86%E3%81%A8%E6%A5%BD/
\usepackage{longtable}
\usepackage{booktabs}
\RequirePackage{plautopatch}

% maru suji ① etc.
\usepackage{tikz}
\newcommand{\cir}[1]{\tikz[baseline]{%
\node[anchor=base, draw, circle, inner sep=0, minimum width=1.2em]{#1};}}

%http://yamamo10.jp/yamamoto/comp/latex/make_doc/box/box.php
%枠付き文章
\usepackage{ascmac}
\usepackage{fancybox}

\usepackage{comment}

\begin{comment}

\ifnum0\ifxetex1\fi\ifluatex1\fi=0 % if pdftex
  \usepackage[T1]{fontenc}
  \usepackage[utf8]{inputenc}
  \usepackage{textcomp} % provide euro and other symbols
\else % if luatex or xetex
  \usepackage{unicode-math}
  \defaultfontfeatures{Scale=MatchLowercase}
  \defaultfontfeatures[\rmfamily]{Ligatures=TeX,Scale=1}
\fi
% Use upquote if available, for straight quotes in verbatim environments
\IfFileExists{upquote.sty}{\usepackage{upquote}}{}
\IfFileExists{microtype.sty}{% use microtype if available
  \usepackage[]{microtype}
  \UseMicrotypeSet[protrusion]{basicmath} % disable protrusion for tt fonts
}{}
\makeatletter
\@ifundefined{KOMAClassName}{% if non-KOMA class
  \IfFileExists{parskip.sty}{%
    \usepackage{parskip}
  }{% else
    \setlength{\parindent}{0pt}
    \setlength{\parskip}{6pt plus 2pt minus 1pt}}
}{% if KOMA class
  \KOMAoptions{parskip=half}}
\makeatother
\usepackage{xcolor}
\IfFileExists{xurl.sty}{\usepackage{xurl}}{} % add URL line breaks if available
\IfFileExists{bookmark.sty}{\usepackage{bookmark}}{\usepackage{hyperref}}
\hypersetup{
  hidelinks,
  pdfcreator={LaTeX via pandoc}}
\urlstyle{same} % disable monospaced font for URLs
\usepackage{longtable,booktabs}
% Correct order of tables after \paragraph or \subparagraph
\usepackage{etoolbox}
\makeatletter
\patchcmd\longtable{\par}{\if@noskipsec\mbox{}\fi\par}{}{}
\makeatother
% Allow footnotes in longtable head/foot
\IfFileExists{footnotehyper.sty}{\usepackage{footnotehyper}}{\usepackage{footnote}}

\end{comment}
%\makesavenoteenv{longtable}
\setlength{\emergencystretch}{3em} % prevent overfull lines
\providecommand{\tightlist}{%
  \setlength{\itemsep}{0pt}\setlength{\parskip}{0pt}}
\setcounter{secnumdepth}{-\maxdimen} % remove section numbering

\author{kazuyoshi}
\date{}






\begin{document}
\chapter{保険2 その他}
\section{1. 監督指針, 施行規則, 業法}

\problem{2022 生保2問題 1(1)【監督指針】}

「保険会社向けの総合的な監督指針」
【Ⅱ-2-1 責任準備金等の積立の適切性】について、以下
の(a)~(g)の空欄に当てはまる適切な語句または数値を記入しなさい。(7点)

\begin{itemize}
 \item [(ア)]「Ⅱ-2-1-2 積立方式(2)」においては以下が規定されている。\\
「第一分野及び第三分野において、保険会社の業務又は財産の状況及び保険契約の特性等
に照らし特別な事情がある場合に、保険数理に基づき、合理的かつ妥当なものとして、いわゆるチルメル式責任準備金の積立てを行っている場合には、 (a) に照らしチルメル歩合が妥当なものとなっているか。」
 \item [(イ)]「Ⅱ-2-1-2 積立方式(4)」においては以下が規定されている。\\
「特定の疾病による所定の状態、所定の身体障害の状態、所定の要介護状態その他の保険料
払込の免除事由に該当し、以後の保険料払込が免除されることとなった保険契約のうち、(b)の
可能な保険契約に係る責任準備金については、最終の保険期間満了日まで全て(b)が行われるものとして計算した金額を積み立てることとなっているか。」
\item[(ウ)]「Ⅱ-2-1-2 積立方式(5)」においては以下が規定されている。\\
「 (c) Ⅰ及びⅣにおける「 (d) 」に係る積立基準並びに積立限度の設定につい
ては、手術給付、介護給付その他の保険給付のリスクに応じたものとなっているか。
」
\item[(エ)]「Ⅱ-2-1-2 積立方式(7)④」においては以下が規定されている。\\
「ストレステスト及び (e) の基礎率を同じくする契約区分は同一のものを使用するこ
ととする。」
\item[(オ)]「Ⅱ-2-1-3-1 保険料積立金の積立(1)標準的方式①」においては以下が規定さ
れている。\\
「通常予測されるリスクに対応するものとして、標準的な計算式(
「一般勘定における最低
保証に係る保険金等の支出現価」から「一般勘定における最低保証に係る純保険料の収入現
価」を控除する形式の計算式)によって、概ね (f) %の事象をカバーできる水準に対
応する額を算出するものとなっているか。」
\item[(カ)]「Ⅱ-2-1-3-1 保険料積立金の積立(2)代替的方式④」において、平成8年2月
29日大蔵省告示第48号に列記する国内株式等の期待収益率及び (g) について、当
該告示に定めるものを使用する場合を除き、過去の実績や将来の資産運用環境の見通し、リ
スク中立の観点等から、合理的かつ客観的根拠に基づき定められる必要があることが規定
されている。
\end{itemize}
\answer{}
\begin{itemize}
\item[ (a): ] 新契約費水準
\item[ (b): ] 自動更新
\item[ (c): ] 危険準備金
\item[ (d): ] その他のリスク
\item[ (e): ] 負債十分性テスト
\item[ (f): ] 50
\item[ (g): ] ボラティリティ
\end{itemize}

\problem{2022 生保2問題 1(2)【業法】}
生命保険会社の保険計理人の職務(保険業法第121条)について、以下の(a)~(e)の空
欄に当てはまる適切な語句を記入しなさい。
(5点)

\begin{itemize}
\item[(ア)] 保険計理人は、毎決算期において、次に掲げる事項について、内閣府令で定めるところにより確認し、その結果を記載した(a)を(b)に提出しなければならない。
\begin{itemize}
\item[・]  内閣府令で定める保険契約に係る責任準備金が (c) に基づいて積み立てられているかどうか。
\item[・]  契約者配当又は社員に対する剰余金の分配が(d)に行われているかどうか。
\item[・]  その他内閣府令で定める事項
\end{itemize}
\item[(イ)] 保険計理人は、(ア)の(a)を(b)に提出した後、遅滞なく、その写しを(e)に提出しなければならない。
\item[(ウ)] (e)は、保険計理人に対し、(イ)の(a)の写しについてその説明を求め、その他その職務に属する事項について意見を求めることができる。
\item[(エ)] 上記に定めるもののほか、(ア)の(a)に関し必要な事項は、内閣府令で定める。
\end{itemize}

\answer{}
\begin{itemize}
\item[ (a): ] 意見書
\item[ (b): ] 取締役会
\item[ (c): ] 健全な保険数理
\item[ (d): ] 公正かつ衡平
\item[ (e): ] 内閣総理大臣
\end{itemize}
※(e)は「金融庁(長官)」も正答とした。

\problem{2022 生保2問題 1(4)【監督指針】}
「保険会社向けの総合的な監督指針」
【Ⅱ-2-4 生命保険会社の区分経理の明確化】について、
以下の(a)~(e)の空欄に当てはまる適切な語句を記入しなさい。
(5点)

Ⅱ-2-4-2 主な着眼点

各生命保険会社においては、適切な区分経理を行うため、例えば、以下のような考えに基づく
区分経理に関する管理方針を策定しているか。また、区分経理の状況が、取締役会その他これ
に準ずる機関に対して報告されているか。

(1)〜(6) (省略)

(7)各区分間の取引等
\begin{itemize}
\item[①] 資産区分間の取引\\
資金移動(流入・流出)管理、 (a) 確保、ポートフォリオの改善等、必要な取引とし、市場価格等の適正な価格をもって適切に管理する。
\item[②]商品区分と全社区分との取引\\
\begin{itemize}
\item[ア.] 現預金等の貸借
\begin{itemize}
\item[(ア)] 商品区分又は全社区分毎に区別して管理する。
\item[(イ)] (b)が継続しないよう限度額等を設ける。
\end{itemize}
\item[イ.] 現預金等以外の貸借
\begin{itemize}
\item[(ア)]  (c) から (d) への貸付は、異常な保険金の支払い、新商品の販売に伴う事業運営資金、その他やむを得ない事情がある場合に限る。
\item[(イ)]  (d) から (c) への貸付は、 (c) の規模が小さいために、その機能を十分に果たすことができない場合に限る。
\item[(ウ)] 上記の貸借は、金額、利率(貸付期間に応じた市中金利等を基に設定すること)、期限その他の返済条件をあらかじめ定める。
\item[(エ)] 貸付条件の緩和や債務免除は、回収が不可能な損失が発生している場合等、やむを得ない事情がある場合を除き、 (e) 。なお、貸付条件の緩和等を行った後に利益が生じた場合は、当該利益を返済に充てるものとする。
\end{itemize}
\item[ウ.] 出資 (省略)
\item[エ.] その他の取引 (省略)
\end{itemize}
\end{itemize}

\answer{}
\begin{itemize}
\item[ (a): ]  流動性
\item[ (b): ]  借越し
\item[ (c): ]  全社区分
\item[ (d): ]  商品区分
\item[ (e): ]  行わない
\end{itemize}

\problem{2021 生保2問題 1(1)【実務基準】}

生命保険会社の保険計理人に関する以下の①~④の文章について、下線部分が正しい場合は
○を、誤っている場合は×を記入するとともに、下線部分を正しい内容に改めなさい。

\begin{itemize}
\item[①] 「生命保険会社の保険計理人の実務基準」
(以下、実務基準)第19条によれば、会社全体
の\underline{翌期配当所要額}が、相互会社においては社員配当準備金繰入額(当年度末の未割当額を含
む)以下であること、株式会社においては当年度末の契約者配当準備金(分配済未払、積立
配当金を除く)以下であることを確認しなければならない。
\item[②] 実務基準第20条および実務基準第22条によれば、翌期の全件消滅ベースの配当所要額
が、配当可能財源の範囲内であることを確認しなければならないのは\underline{会社全体のみ}である。
\item[③] 実務基準第21条によれば、会社全体の翌期配当所要額が、会社の配当可能財源から、
\underline{危険準備金積立限度額}を維持するために必要な額を控除した額の範囲内であることを確認しな
ければならない。
\item[④] 保険業法施行規則第77条に定める保険計理人の関与事項には、第7号として「\underline{将来収支}に
関する計画」(解答欄④-1)および第8号として「生命保険募集人の\underline{給与}等に関する規程の作成」
(解答欄④-2)が規定されている。
\end{itemize}
 
\answer{}
\begin{itemize}
\item[ ①: ] ○
\item[ ②: ] ×会社全体および商品区分毎
\item[ ③: ] ×会社の健全性の基準
\item[ ④―1: ] ×保険募集
\item[ ④―2: ] ○
\end{itemize}


\problem{2021 生保2問題 1(4)【税制】}
生命保険会社の税制について、以下の①~⑤の空欄に当てはまる適切な語句または数値を記入し
なさい。

\begin{itemize}
\item[・] 責任準備金繰入額については、保険料積立金及び未経過保険料の部分に限り、保険料及び責
任準備金の算出方法書に定められている\wakumaru{①}の計算基礎を基として計算した額を限度
として損金算入できる。ただし、
\wakumaru{②}の対象契約については、平成8年の大蔵省告示
第48号に定められた計算基礎を基として計算した額を損金算入限度額とすることができる。
\item[・] 法人事業税(地方税)の課税標準は、生命保険業にあっては各事業年度の収入金額とされて
おり、生命保険業の各事業年度の収入金額は、収入保険料中の\wakumaru{③}相当額とするとの
考え方から、収入保険料に一定割合を乗じた金額と定められている。なお、法人事業税の一
部を分離した地方法人特別税は令和元年9月30日までに開始する事業年度をもって廃止さ
れた一方で、令和元年10月1日以後に開始する事業年度から\wakumaru{④}(国税)が創設さ
れている。
\item[・] 課税所得が当該事業年度の剰余金の額の\wakumaru{⑤}相当額に満たない場合は、契約者(社員)
配当準備金繰入額の損金算入を制限し、この剰余金の\wakumaru{⑤}相当額を課税標準とする制
度が設けられている。
\end{itemize}
\answer{}
\begin{itemize}
\item[ ①: ] 保険料
\item[ ②: ] 標準責任準備金
\item[ ③: ] 付加保険料
\item[ ④: ] 特別法人事業税
\item[ ⑤: ] 7%
\end{itemize}

\problem{2020 生保2問題 1(4)【監督指針】]}
金融庁による事業費モニタリングについて、以下の①~⑤の空欄に当てはまる適切な語句を記入
しなさい。

\noindent ○「5-7\wakumaru{①}の充足状況」

保険種類・\wakumaru{②}
の区分ごとの新契約に係る事業費の効率等を見る資料で、定期的に
金融庁宛報告を要する。報告対象は、原則として、当該期における新契約の全て。

\wakumaru{①}に関して、保険種類および\wakumaru{②}の区分ごとに、
「\wakumaru{③}」、「事業費」、「\wakumaru{④}」を算出し、
「効率(事業費÷\wakumaru{③})」および「回収予定平均年数(事業
費÷\wakumaru{④})」を報告する。

\noindent ○「5-9 \wakumaru{⑤}の充足状況」

保険種類・\wakumaru{②}
の区分ごとの契約維持・管理のために支出する事業費の回収状況を
見る資料で、定期的に金融庁宛報告を要する。報告対象は、当該期における全保有契約。
\answer{}
\begin{itemize}
\item[ ①: ] イニシャルコスト
\item[ ②: ] 販売経路
\item[ ③: ] 予定事業費現価
\item[ ④: ] 年換算予定事業費
\item[ ⑤: ] ランニングコスト
\end{itemize}

\problem{2020 生保2問題 1(6)【監督指針】}
変額年金保険等の最低保証に係る保険料積立金の積立てに際して予定解約率を使用する場合の留
意点について、保険会社向けの総合的な監督指針の記載を踏まえ、簡潔に説明しなさい。

\answer{}
\begin{itemize}
\item[]  予定解約率が過去の実績や商品性等から、合理的に定められたものとなっているか。
\item[]  例えば、以下の事例等に留意しているか。
\begin{itemize}
\item[]  特別勘定の残高が最低保証額を下回る状態にあるときの解約率が、特別勘定の残高が最低保証額を超える状態にあるときの解約率より低い率となっているか。
\item[]  解約控除期間における解約率が、解約控除期間終了後の解約率と比べ、低い率となっているか。
\item[]  最低年金原資保証が付された保険契約で、年金開始前における特別勘定の残高が最低保証額を下回る状態にある場合において解約率を保守的に設定しているか。
\item[]  設定された予定解約率について、解約実績との比較などにより、検証を行うこととなっているか。
\end{itemize}
\end{itemize}

\problem{2020 生保2問題 2(1)【業法, 施行規則】}
保険計理人の確認事項のうち、保険業法第121条第1項第3号および保険業法施行規則第79
条の2第1号に規定されている財産の状況の確認について、
「生命保険会社の保険計理人の実務基
準」を踏まえて、簡潔に説明しなさい。

\answer{}
保険計理人は、財産の状況に関し、以下を確認しなければならない。

① 将来にわたり、保険業の継続の観点から適正な水準(事業継続基準)を維持することができるかど
うか。

② 保険金等の支払能力の充実の状況が保険数理に基づき適当であるかどうか。
(ソルベンシー・マージン基準の確認)

<①の確認の概要>

・ 「将来の時点における資産の額として合理的な予測に基づき算定される額(イ)」が、「当該将来の
時点における負債の額として合理的な予測に基づき算定される額(ロ)」を上回ることを確認する
ことにより行う。

・ 上記(イ)とは、事業継続基準の確認に関する将来収支分析(3号収支分析)を行った場合の、資
産(時価評価)から

資産運用リスク相当額

(その他有価証券の評価差額金がマイナスの場合)当該評価差額金に係る繰延税金資産

を控除した額をいう。

・ 上記(ロ)とは、以下の合計額をいう。

事業継続基準に係る額(それぞれの保険契約もしくは保険契約群団について、全期チルメル式
責任準備金と解約返戻金相当額のいずれか大きい方の額を計算したものの合計額)

負債の部の合計額から、責任準備金、価格変動準備金、配当準備金未割当額、評価差額金に係
る繰延税金負債、劣後特約付債務(資産運用リスク相当額を限度とする) を控除した額

・ 3号収支分析は会社全体について毎年行うものとし、分析期間は少なくとも将来10年間とする。

・ 分析期間中の最初の5年間の事業年度末において、上記(イ)の額が(ロ)の額に不足する場合は、
その旨を意見書に記載しなければならない。ただし、


満期保有目的債券および責任準備金対応債券の含み損を算入しない場合に不足が解消される
ときは、分析期間を通じた十分な流動性資産の確保を条件に事業継続困難とはならない旨を、
意見書に記載することができる。


ただちに行われる経営政策の変更により不足を解消できることを、意見書に示すことができる。

<②の確認の概要>

・ ソルベンシー・マージン総額およびリスク合計額が、法令の規定に照らして適正であることを踏ま
えた上で、ソルベンシー・マージン比率が200%以上であることを確認することにより行う。

・ とくに、ソルベンシー・マージン総額が法令の規定に照らして適正であることの確認には、保険料
積立金等余剰部分控除額がソルベンシー・マージン基準の確認に関する将来収支分析(3号の2収
支分析)により算出される保険料積立金等余剰部分控除額の下限以上となっていることを確認しな
ければならない。

・ 3号の2収支分析は、会社全体について毎年行うものとし、分析期間は将来5年間とする。

・ 保険料積立金等余剰部分控除額の下限は、分析期間中の事業年度末に生じた事業継続基準に係る額
の不足額の現価の最大値とする。

・ ソルベンシー・マージン比率が 200%未満である場合には、その旨を意見書に記載しなければ
ならない。

\problem{2019 生保2問題 1(1)【監督指針】}

「保険会社向けの総合的な監督指針」【Ⅱ-3-9 資産負債の総合的な管理】について、以下の
A~Eの空欄に当てはまる適切な語句を記入しなさい。

Ⅱ-3-9-1 意義

資産及び負債、資産の運用方針及び負債の管理方針が、AやBの状況に適
合していることを確保するためには、資産負債全体の状況を把握し管理するための効果的な態
勢を整備し、資産負債全体を適切に管理することが求められる。

Ⅱ-3-9-2 主な着眼点

\begin{itemize}
 \item[(1): ]  (省略)
 \item[(2): ]  取締役会は、資産負債全体の総合的な管理に関する戦略目標を設定し、戦略目標の中でCに関する方針を明確化しているか。
 \item[(3): ]  (省略)
 \item[(4): ]  (省略)
 \item[(5): ]  資産負債を統合的に管理する際に、少なくとも、Dに対する潜在的な影響に関して重要と考えられるリスクは資産負債管理の枠組みにおいて評価されているか。なお、そのようなリスクとしては以下のリスクが含まれる。
 \item[①: ] 市場リスク(省略)
 \item[②: ] 保険引受リスク
 \item[③: ] リスク
\end{itemize}

(以下、省略)
\answer{}
\begin{itemize}
\item[ A: ] リスクの特性
\item[ B: ] ソルベンシー
\item[ C: ] リスク許容度
\item[ D: ] 経済価値
\item[ E: ] 流動性
\end{itemize}
\problem{2019 生保2問題 1(6)【監督指針】}

「生命保険会社の保険計理人の実務基準」における 1 号収支分析の結果、責任準備金不足相当
額が発生した場合において、保険計理人が責任準備金不足相当額の一部または全部を積み立て
なくてもよいことを意見書に示すことができるための条件である経営政策の変更を5つ列挙
しなさい。

\answer{}
\begin{itemize}
\item[]  一部または全部の保険種類の配当率の引き下げ
\item[]  実現可能と判断できる事業費の抑制
\item[]  資産運用方針(ポートフォリオ)の見直し
\item[]  一部または全部の保険種類の新契約募集の抑制
\item[]  今後締結する保険契約の営業保険料の引き上げ
\end{itemize}

\problem{2018 生保2問題 1(3)【監督指針】}
保険会社向けの総合的な監督指針」【Ⅱ-2-4 生命保険会社の区分経理の明確化】につい
て、以下のA~Eの空欄に当てはまる適切な語句を記入しなさい。

Ⅱ-2-4-1 意義 (省略)

Ⅱ-2-4-2 主な着眼点

各生命保険会社においては、適切な区分経理を行うため、例えば、以下のような考えに基づ
く区分経理に関する管理方針を策定しているか。また、区分経理の状況が、取締役会その他
これに準ずる機関に対して報告されているか。

(1)~(4) (省略)

(5) 資産の配賦方法及び管理基準
\begin{itemize}
\item[①] 運用資産の配賦方法\\
運用資産は、原則として、資産の購入時に配賦する資産区分を決める。
\item[②] 運用資産の管理\\
運用資産は、資産区分ごとに、次に掲げる方式の中から適切な方式を選択し管理する。
\item[ア.: ] A・・・ 個々の資産を銘柄ごとに、資産区分に直接帰属させる方式
\item[イ.: ] B・・・ 取引単位(例えば、不動産では物件ごと)ごとに、資産区分の持分で管理する方式
\item[ウ.: ]  資産持分管理方式・・・ 投資対象資産ごとのマザーファンドを設定し、各資産のマザーファンドに対する持分を管理する方式
(注)資産持分管理方式を用いる場合は、一般勘定資産(C保険に対応する資産を除く。)全体を一個のマザーファンドとして扱わない。
\item[③] 運用資産以外の配賦方法\\
再保険貸等、各資産区分に直課できるものは直課し、直課できないものは、区分経理に関
する管理方針に基づいて配賦する。
\item[④] 全社区分の資産\\
D、子会社・関連会社株式、E(E等の管理機能を持つ場合)、その他全社区分に配賦することが相応しい資産の全部又は一部を配賦するものとする。
\end{itemize}
(6)、(7) (省略)

Ⅱ-2-4-3 監督手法・対応 (省略)

\answer{}
\begin{itemize}
\item[ A: ] 資産分別管理方式
\item[ B: ] 資産単位別持分管理方式
\item[ C: ] 無配当
\item[ D: ] 営業用不動産
\item[ E: ] 現預金
\end{itemize}

\problem{2018 生保2問題 2(1)【監督指針】}

生命保険会社の保険計理人の実務基準に規定されている公正・衡平な配当の要件および公正・
衡平な配当の確認の概要について、簡潔に説明しなさい。

\answer{}
○公正・衡平な配当の要件

\begin{itemize}
\item[・] 剰余金の分配または契約者配当(以下「配当」という。)が、公正・衡平であるとは、以下の要件を満たすことである(実務基準第17条第2項)
\begin{itemize}
\item[①: ] 責任準備金が適正に積み立てられ、かつ、会社の健全性維持のための必要額が準備されている状況において、配当所要額が決定されていること
\item[②: ] 配当の割当・分配が、個別契約の貢献に応じて行われていること
\item[③: ] 配当所要額の計算および配当の割当・分配が、適正な保険数理および一般に公正妥当と認められる企業会計の基準等に基づき、かつ、法令、通達の規定および保険約款の契約条項に則っていること
\item[④: ] 配当の割当・分配が、国民の死亡率の動向、市場金利の趨勢などから、保険契約者の期待するところを考慮したものであること
\end{itemize}
\end{itemize}

○公正・衡平な配当の確認
\begin{itemize}
\item[・]配当が公正・衡平であることの確認として、保険計理人は以下の確認を行わなければならない(実務基準第18条第2項)。
\begin{itemize}
\item[①]会社全体について、以下の要件が満たされていること
\begin{itemize}
\item[イ. ] 翌期配当所要額が、相互会社では配当準備金繰入額と配当準備金中の未割当額の合計額、株式会社では当期末の配当準備金(分配済未支払および積立配当金を除く)以下であること(簿価ベースの確認とも言われる)
\item[ロ. ] 翌期の全件消滅ベースの配当所要額が会社の配当可能財源の範囲内であること
\item[ハ. ] 翌期配当所要額が、会社の配当可能財源から会社の健全性の基準を維持するために必要な額を控除した額の範囲内であること
\end{itemize}
\item[②] 区分経理の商品区分毎の翌期の全件消滅ベースの配当所要額が、当該商品区分の配当可能財源の範囲内であること。ただし、保険計理人が特に必要と判断する場合は、さらに細分化した保険契約群団毎に財源が確保されていることを確認しなければならない。また、保険計理人が合理的であると判断する場合は、複数の商品区分をまとめて、財源が確保されていることを確認することができる。
\item[③] 契約消滅時に最終精算として消滅時配当を行う保険種類においては、以下の要件が満たされていること
\begin{itemize}
\item[イ.] 代表契約の翌期配当額が、原則として当年度末のネット・アセット・シェアを超えていないこと(ヒストリカルな視点)
\item[ロ.] 代表契約の将来のネット・アセット・シェアが健全性の基準維持のための金額を下回っていないこと(プロジェクションの視点)
\end{itemize}
\end{itemize}
\end{itemize}

\problem{H29 生保2問題 1(1)【大蔵省告示】}

平成10年・大蔵省告示第231号に規定されている、危険準備金の取崩基準について、以
下の①~⑤の空欄に当てはまる適切な語句を記入しなさい。

(危険準備金の取崩基準)

第六条 危険準備金Ⅰ及び危険準備金Ⅳは、それぞれ①がある場合において、当該
①のてん補に充てるときを除くほか、取り崩してはならない。

2 危険準備金Ⅱは、②がある場合において、当該②のてん補に充てるときを除くほか、取り崩してはならない。

3 危険準備金Ⅲは、最低保証に係る③が負の場合において、当該③のてん補に充てるときを除くほか、取り崩してはならない。

4 その他前三項それぞれに共通する取崩基準として、前事業年度末の④の額が当該事業年度末の⑤を超える場合は、当該超える額を取り崩さなければならない。

\answer{}
\begin{itemize}
\item[ ①: ] 死差損
\item[ ②: ] 利差損
\item[ ③: ] 収支残
\item[ ④: ] 積立残高
\item[ ⑤: ] 積立限度額
\end{itemize}

\problem{H29 生保2問題 1(3)【実務基準】}

「生命保険会社の保険計理人の実務基準」に定めるソルベンシー・マージン基準の確認に関
する将来収支分析(3号の2収支分析)について、以下の①~⑤の空欄に当てはまる適切な
語句または数値を記入しなさい。

・3号の2収支分析は毎年行うものとし、3号の2収支分析を行う期間(以下「分析期間」とい
う。)は、将来①年間とする。

・3号の2収支分析のシナリオの各要素は、以下に定める通りとする(このシナリオを「3号の
2基本シナリオ」という。)。

 金利は、直近の長期国債応募者利回りが横ばいで推移するものとする。

 株式・不動産の価格や為替レートについては、変動しないものとする。

②の取崩しおよび含み益の実現による積立財源への充当は行わない。

 価格変動準備金・危険準備金等への繰入れは行わない。

 劣後性債務・社債・③
については、その約定に従って、利息を支払うこととする。

保険料積立金等余剰部分控除額の下限は、分析期間中の事業年度末に生じた事業継続基準に係
る額の不足額の④とする。なお、ソルベンシー・マージン比率の算出を行う日におい
て、保険業法施行規則第69条第5項の規定に基づき積み立てた⑤の額を積み立てて
いないものとして計算を行う。

\answer{}
\begin{itemize}
 \item[①: ]  5 
 \item[②: ]  評価差額金 
 \item[③: ]  基金 
 \item[④: ]  現価の最大値 
 \item[⑤: ]  保険料積立金 
\end{itemize}

\problem{H29 生保2問題 1(6)【監督指針】}
「保険会社向けの総合的な監督指針」【Ⅱ-3-5 リスクとソルベンシーの自己評価】につ
いて、以下の①~⑥の空欄に当てはまる適切な語句を記入しなさい。

Ⅱ-3-5-1 意義

保険会社は、経営戦略及びリスク特性等に応じ、自らのリスク管理の適切性と現在及び将来
にわたるソルベンシーの十分性を評価するために、①の責任の下、定期的にリスクとソルベンシーの自己評価を実施することが求められる。自己評価においては、将来の経済状況や他の外部要因の変化も考慮し、合理的に予見可能で関連性のある重大なリスクを含んでいる必要がある。

Ⅱ-3-5-2 リスクとソルベンシーの自己評価

(1) 保険会社は、将来の経済状況やその他の外部要因の変化を含めた合理的に予見可能で関
連性のある全ての重大なリスクを考慮し、資本の②と十分性の評価を実施している
か。

また、リスクの要因やリスクの重要性の程度を定期的に評価しているか。さらに、③に大きな変化があった場合には、速やかにリスクとソルベンシーの再評価を行っているか。

保険会社は、リスクとソルベンシーの自己評価に当たっては、中長期事業戦略(例えば3年から5年間)、特に新規事業計画に十分留意しているか。

(2) 保険会社は、必要な④及びソルベンシー・マージン規制に基づく資本の要件を満たしているかをモニタリングするために、リスクとソルベンシーの自己評価を定期的に行い、リスクと資本の管理プロセスを整備しているか。また、必要な④とソルベンシー・マージン規制に基づく資本の要件の違いについて、経営陣は適切に認識しているか。

(3) 保険会社は、リスクとソルベンシーの自己評価の結果を、例えば、リスクの特定及び③、リスク測定、リスク管理方針及びリスクとソルベンシーの自己評価の結果を踏まえた行動計画等とともに、適切に⑤しているか。

(4) 保険会社は、リスクとソルベンシーの自己評価の有効性について、内部(例えばリスク管理担当役員など)又は外部による全般的な評価を行っているか。

(5) ⑥部門は、統合的リスク管理及びリスクとソルベンシーの自己評価の有効性を独立した立場から検証し、必要に応じ経営陣に提言を行っているか。

Ⅱ-3-5-3 経営計画とソルベンシー評価

(1)、(2) (省略)

\answer{}
\begin{itemize}
\item[ ①: ]  取締役会
\item[ ②: ]  質
\item[ ③: ]  リスク・プロファイル
\item[ ④: ]  経済資本
\item[ ⑤: ]  文書化
\item[ ⑥: ]  内部監査
\end{itemize}

\problem{H28 生保2問題 1(2)【施行規則】}

生命保険会社の保険計理人の関与事項について、以下の①~⑤の空欄に当てはまる適切な語句
を記入しなさい。

生命保険会社の保険計理人の関与事項は、次に掲げるものに係る①に関する事項である。

\begin{itemize}
\item[ 1]  保険料の算出方法
\item[ 2]  ②の算出方法
\item[ 3]  契約者配当又は社員に対する剰余金の分配に係る算出方法
\item[ 4]  ③の算出方法
\item[ 5]  未収保険料の算出
\item[ 6]  ④の算出
\item[ 7]  ⑤に関する計画
\item[ 8]  生命保険募集人の給与等に関する規程の作成
\item[ 9]  その他保険計理人がその職務を行うに際し必要な事項
\end{itemize}
\answer{}
\begin{itemize}
\item[ ① ] 保険数理
\item[ ② ] 責任準備金
\item[ ③ ] 契約者価額
\item[ ④ ] 支払備金
\item[ ⑤ ] 保険募集
\end{itemize}
\problem{H28 生保2問題 1(3)【監督指針】}
ストレステストについて、以下の①〜⑤の空欄に当てはまる適切な語句を記入しなさい。

「保険会社向けの総合的な監督指針」の II -3-3-3 ストレステスト

保険会社は、将来の不利益が①に与える影響をチェックし、必要に応じて、追加的に
経営上又は財務上の対応をとって行く必要がある。そのためのツールとして、
②等を含むストレステスト(想定される将来の不利益が生じた場合の影響に関する分析)及び
③(経営危機に至る可能性が高いシナリオを特定し、そのようなリスクをコントロール
すべく必要な方策を準備するためのストレステスト)が重要である。特に、市場が大きく変動し
ているような状況下では、④によるリスク管理には限界があることから、ストレステストの活用は極めて重要である。保険会社においては、⑤等も勘案しつつ、財務内容及び
保有するリスクの状況に応じたストレステストを自主的に実施することが求められる。

\problem{H28 生保2問題 1(4)【業法, 監督指針】}
金融庁による事業費モニタリングについて、以下の①〜⑤の空欄に当てはまる適切な語句を記入しなさい。

○生命保険会社は、平成18年4月以降、次の5つの資料を金融庁宛定期報告することとなっている。
\begin{itemize}
\item[ ・]  5−5「予定事業費等の設定状況」
\item[ ・]  5−6「総合的な充足状況」
\item[ ・]  5−7「①の充足状況」
\item[ ・]  5−8「①の回収状況」
\item[ ・]  5−9「②の充足状況」
\end{itemize}

○5-7「①の充足状況」について
保険種類および③の区分ごとの新契約に係る事業費の効率等を見る資料で、定期的に金融庁宛報告を要する。報告対象は、原則として、当該期における新契約の全て。

①に関して、保険種類および③の区分ごとに、「④」、「事業費」、「⑤」を算出し、
「効率(事業費÷④)」および「回収予定平均年数(事業費÷⑤)」を報告する。

\begin{itemize}
\item[① : ] イニシャルコスト
\item[② : ] ランニングコスト
\item[③ : ] 販売経路
\item[④ : ] 予定事業費現価
\item[⑤ : ] 年換算予定事業費
\end{itemize}

\problem{H27 生保2問題 1(3)【実務基準】}
「生命保険会社の保険計理人の実務基準」に基づき保険計理人が行う責任準備金積立ての確認に
おける、1号収支分析を行わなくともよい保険契約について説明しなさい。
\answer{}
\begin{itemize}
\item[ ・] 責任準備金が特別勘定に属する財産の価額により変動する保険契約であって、保険金等の額を最低保証していない保険契約
\item[ ・] 保険料積立金を積み立てない保険契約
\item[ ・] 保険約款において、保険会社が責任準備金および保険料の計算の基礎となる係数(平成 13年 7 月 1 日または平成 13 年 4 月 1 日以降締結する保険契約については、責任準備金および保険料の計算の基礎となる予定利率)を変更できる旨を約してある保険契約
\item[ ・] その他標準責任準備金の計算の基礎となるべき係数の水準について、必要な定めをすることが適当でない保険契約
\end{itemize}

\problem{H27 生保2問題 1(5)【大蔵省告示】}
平成 8 年・大蔵省告示第 50 号別表第 6 の 2 に規定されている、変額年金保険等の最低保証リス
ク相当額の算出について、次のA~Eに適切な語句を記入しなさい。

II. 最低保証リスク相当額の算出
\begin{itemize}
\item[1.] 標準的方式
\begin{itemize}
\item[(1)]最低保証リスク相当額は、次のイに掲げる額からロに掲げる額を控除した額とする。
\begin{itemize}
\item[イ]Aの責任準備金の額(原則として法第 4 条第 2 項第 4 号に掲げる書類に記載された商品区分ごとに、次の①から④までに定める手順に基づき算出した額をいう。)
\begin{itemize}
\item[①] 次に掲げる区分に応じたリスク対象資産の額から、別表第 7 の 2 の区分によるそれぞれの対象取引残高の欄に掲げる額(別表第 7 の 2 によりリスクヘッジの有効性が確認できたものに限る。)を控除した残高に、次の表に掲げる区分に応じた下落率をそれぞれ乗じた額の合計額を算出する。(省略)
\item[②] 上記①に掲げる額から、その額に次に掲げる算式により計算したB係数を乗じた額を控除する。(省略)
\item[③] 上記②により算出した額を特別勘定資産の額の合計額で除した率を算出する。
\item[④] 上記③により算出した率に基づき資産下落が生じたとした場合の、一般勘定におけるCの額を算出する。
\end{itemize}
\item[ロ] 法第 4 条第 2 項第 4 号に掲げる書類に記載された方法に基づき算出された一般勘定におけるCの額
\item[(2)] (省略)
\item[(3)] (省略)
\item[2.] 代替的方式
次の①から⑬に定める基準を満たす保険会社、外国保険会社等又は免許特定法人(以下「保険会社等」という。)は代替的方式を用いることができる。ただし、代替的方式を用いた場合は、Dの結果、代替的方式の使用を継続することが不適当と認められ、代替的方式の使用を中断する旨又はEに重大な変更を加える旨をあらかじめ金融庁長官に届け出た場合を除き、これを継続して使用しなければならない。(以下、省略)
\end{itemize}
\end{itemize}
\end{itemize}
\answer{}
\begin{itemize}
\item[ A: ] 資産価格下落後
\item[ B: ] 分散投資効果
\item[ C: ] 最低保証に係る責任準備金
\item[ D: ] バック・テスティング
\item[ E: ] リスク計測モデル
\end{itemize}
\end{document}
