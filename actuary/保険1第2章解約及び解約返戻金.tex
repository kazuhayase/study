\documentclass[report,gutter=10mm,fore-edge=10mm,uplatex,dvipdfmx]{jlreq}

\usepackage{lmodern}
\usepackage{amssymb,amsmath}
\usepackage{ifxetex,ifluatex}
\usepackage{actuarialsymbol}
\usepackage[]{natbib}
\RequirePackage{plautopatch}

% maru suji ① etc.
\usepackage{tikz}
\newcommand{\cir}[1]{\tikz[baseline]{%
\node[anchor=base, draw, circle, inner sep=0, minimum width=1.2em]{#1};}}

\usepackage{comment}

\begin{comment}

\ifnum0\ifxetex1\fi\ifluatex1\fi=0 % if pdftex
  \usepackage[T1]{fontenc}
  \usepackage[utf8]{inputenc}
  \usepackage{textcomp} % provide euro and other symbols
\else % if luatex or xetex
  \usepackage{unicode-math}
  \defaultfontfeatures{Scale=MatchLowercase}
  \defaultfontfeatures[\rmfamily]{Ligatures=TeX,Scale=1}
\fi
% Use upquote if available, for straight quotes in verbatim environments
\IfFileExists{upquote.sty}{\usepackage{upquote}}{}
\IfFileExists{microtype.sty}{% use microtype if available
  \usepackage[]{microtype}
  \UseMicrotypeSet[protrusion]{basicmath} % disable protrusion for tt fonts
}{}
\makeatletter
\@ifundefined{KOMAClassName}{% if non-KOMA class
  \IfFileExists{parskip.sty}{%
    \usepackage{parskip}
  }{% else
    \setlength{\parindent}{0pt}
    \setlength{\parskip}{6pt plus 2pt minus 1pt}}
}{% if KOMA class
  \KOMAoptions{parskip=half}}
\makeatother
\usepackage{xcolor}
\IfFileExists{xurl.sty}{\usepackage{xurl}}{} % add URL line breaks if available
\IfFileExists{bookmark.sty}{\usepackage{bookmark}}{\usepackage{hyperref}}
\hypersetup{
  hidelinks,
  pdfcreator={LaTeX via pandoc}}
\urlstyle{same} % disable monospaced font for URLs
\usepackage{longtable,booktabs}
% Correct order of tables after \paragraph or \subparagraph
\usepackage{etoolbox}
\makeatletter
\patchcmd\longtable{\par}{\if@noskipsec\mbox{}\fi\par}{}{}
\makeatother
% Allow footnotes in longtable head/foot
\IfFileExists{footnotehyper.sty}{\usepackage{footnotehyper}}{\usepackage{footnote}}

\end{comment}
%\makesavenoteenv{longtable}
\setlength{\emergencystretch}{3em} % prevent overfull lines
\providecommand{\tightlist}{%
  \setlength{\itemsep}{0pt}\setlength{\parskip}{0pt}}
\setcounter{secnumdepth}{-\maxdimen} % remove section numbering

\author{kazuyoshi}
\date{}

\newcommand{\problem}[1]{\subsubsection{#1}\setcounter{equation}{0}}
%\newcommand{\answer}[1]{\subsubsection{#1}}
\newcommand{\answer}[1]{\subsubsection{解答}}

%Pdf%\newcommand{\wakumaru}[1]{\framebox[3zw]{#1}}
\newcommand{\wakumaru}[1]{#1}






%\newcommand*{problem}[3]{\subsubsection{#1 生保#2 #3}}
%\newcommand*{answer}{\subsubsection{解答}}

\begin{document}
\section{保険1第2章 解約および解約返戻金}
\section{2.2 解約および解約返戻金の意義と法規整}
\subsubsection{H20 生保1問題 1(3)}
「保険料計算基礎率」と「責任準備金計算基礎率」が異なっている場合において、「解約返戻金計
算基礎率」の設定方法について説明しなさい。
\subsubsection{解答}
解約返戻金計算基礎率は、以下の観点から、保険料計算基礎率と同じに設定することが考えられる。
\begin{itemize}
  \item 解約返戻金は、個々の契約者が会社財産の形成に貢献した金額を基準とするものであ  り、契約者が拠出した金額は保険料であることから、解約返戻金の算定は保険料に基
  づくことが整合的である。
  \item 一方、責任準備金は、ソルベンシーを考慮して会社が評価し積み立てるものであり、契約者価額ではない。
  \item 補完的な観点として、解約返戻金水準は契約時に約定が必要であるが、責任準備金は経済状況等により保険期間途中でも積み増しまたは削減があり得る。それをこうした
  約定に反映することは困難である。
\end{itemize}

\section{2.3 解約控除の理由}

\subsubsection{H4 生保1問題 1(2)}
  解約控除について簡潔に説明せよ。
\subsubsection{解答}
  責任準備金に控除率を適用して解約返戻金を算出する場合に、その控除を解約控除という。
  解約控除の理由としては、
  \begin{itemize}
    \item 新契約費の回収、
    \item 解約による逆選択の防止と被保険群団の維持、
    \item 解約に手間がかかる、
    \item 投資上の不利益、
    \item 数学的危険の不安定さの増加等が挙げられる。
  \end{itemize}

\subsubsection{H14 生保1問題 1(5)、H11 生保1問題 2(2)①}
  個人保険・個人年金保険において解約控除を行う理由を列挙し簡潔に説明せよ。
\subsubsection{解答}
解約控除の理由としては一般的に次の4つの理由が挙げられる。
\begin{enumerate}
  \item 新契約費の回収:新契約時にかかる生命保険の募集・締結のための経費は、営業保険料中に予定事業費として組み込んでいる。保険契約が解約された場合には以後の保険料が回収
  されず新契約費はすでに支出されている一方で、その財源である予定新契約費(保険料)の収入が完結していないことになる。このため、未回収部分(の一部)を解約返戻金の算式に反映するものである。
  \item 逆選択防止:一般に保険契約を解約する者は平均的に健康体であることが想定され、残された保険群団の死亡率が高まることが予想される。このため、残された保険群団の収悪化を補うものである。
  \item 投資上の不利益:解約を見込んで資産の流動性を図ることになるが、このことが資産運用利回りを低下させるため、これを補うものである。
  \item ペナルティー:解約に伴う上記の様々な不利益へのペナルティーという意味合いである。
\end{enumerate}

\section{2.4 解約返戻金に関する視点}

\subsubsection{H21 生保1問題 2(2)改}
  個人保険の解約返戻金を決めるにあたって考慮すべき視点を5つ\footnote{元の問題は4つ. H21以降で、教科書の改定で「収益性」の追加があったのかも? }挙げ、それぞれ簡潔に説明しなさい。
\subsubsection{解答}
  \begin{enumerate}
    \item 健全性
    \begin{itemize}
      \item 解約者に対して解約返戻金を支払うことは、保険群団に留保される金額、すなわち将来の保険債務に備えるための責任準備金積立財源となるものが少なくなるということである。
      \item このため、保険群団としての健全性の観点から、解約返戻金は、必要な責任準備金、及びソルベンシーマージンの積立財源を脅かさない金額以下で設定する必要があり、
      \item 具体的には少なくとも、解約返戻金は責任準備金額以下とすべきである。
      \item いわゆる「解約控除」は、健全性の視点で設定していると見ることもでき、
      \item 残存契約からの収益に期待せずに、新契約費の未回収部分を賄うためには、これが必要となる。
    \end{itemize}
    \item 公平性
    \begin{itemize}
      \item 例えば「保険契約を解約した者と継続する者」との間の公平性を考えてみると、
      \item 解約者の新契約費の未回収部分の負担を保険を継続した者に負わせることは公平性に欠くと思われ、その意味でもいわゆる「解約控除」は必要と考える。
      \item また、保険種類間の公平性を考えてみると、「解約控除」が利くものと、
      \item 元々責任準備金が少ないため、「解約控除」しきれない等により、何らかの形で残存者に負担を負わせている保険種類もあることに留意する必要がある。
    \end{itemize}
    \item 効率性
    \begin{itemize}
      \item 効率性は広い意味での契約者価額(保険料、配当、解約返戻金)を向上させ、保険会社の収益も向上させるものというものである。
      \item ここで、いわゆる「解約控除」の水準を低くしていくことが、効率性につながる。保険料計算上の予定新契約費を低く設定したり、解約控除の水準を低くしたりすることで、
      \item 保険会社が新契約費支出削減の経営努力を行い、効率性を高めるインセンティブになる。
      \item また、効率性は他社との競争力と見ることもでき、企業努力によりソルベンシーを損なわずに事業費支出を削減し、解約返戻金等の契約者価額を高く設定することは、他社との競争力を高めることにもなる。
    \end{itemize}
    \item 契約者の (合理的な) 期待
    \begin{itemize}
      \item 契約者の期待に明確な定義があるわけではなく、社会通念、言いかえれば「常識」に基づくことになり、
      \item 例えば、契約者の直接的な期待としては、「解約控除」は小さく、解約返戻金は高いほうが良いことは明確である。
      \item 前述の健全性、公平性および効率性は、究極的には契約者の利益のためであることは明確であるが、この理屈のみでは契約者の期待に応えることは十全ではない。
      \item したがって、契約者の理解を得難いことも事実であり、「解約控除」について、健全性・公平性等に基づく保険計理での妥当性を主張したとしても、消費者契約法の立場からも許容される保証はない。
      \item 以上はH21の公式解答。教科書では、他に以下3点ほどあったように思う。
      \item 社会通念は、その時点の社会環境、経済状態によって変化するので、「契約者の合理的な期待」は常に一定のものではない。
      \item また、法令等によって規定できる性格のものではない。
      \item 解約返戻金自体が、様々な外的要因と絡まって何らかの契約者行動を誘引し、収益の感応度に影響するという関係にある、という視点を理解することが重要である。
    \end{itemize}
   \item 収益性
    \begin{itemize}
      \item 生命保険商品の設計においてその収益性を検証するとき、大きく分けて二通りの視点がある。
      \item 実際の経験率が、契約者価格を決定するときの前提条件である予定死亡率・罹患率、予定利率、予定事業費率、予定解約率等などの基礎率からどの位乖離しているか、その乖離がどのような収益または損失をもたらしたか、または、もたらすか、という利源分析の視点と、
      \item これらの基礎率の変動が収益性にどのような変動を与えるか、という感応度の視点である。
      \item 解約返戻金の設定に限定すれば、予定解約率の設定と、解約控除に反映される予定新契約費用の設定が利源分析の視点、解約返戻金の規模と予定解約率の設定が収益性をどのように変動させるか、というのが感応度の視点となる。
      \item 解約返戻金の財源として保険年度末の保険料積立金を充てる伝統的な設計の場合、従来の利源分析上の解約益は新契約費用が償却できているか、という視点であり、解約率動向の分析ではない。
      \item また、このような商品の場合、解約返戻金の規模と予定解約率の設定による感応度は、予定解約率を設定していないことから検証の対象とはならない。
      \item 解約返戻金を保険年度末保険料積立金より低く抑える低解約返戻金もしくは無解約返戻金型商品における収益性の感応度は、解約返戻金の規模と予定解約率の設定が影響する。
      \item 解約返戻金が保険料払込満了時等の特定時に上昇する設計の場合、上昇する直前では解約が減少すると仮定することが合理的である。また、上昇直後には解約が集中すると仮定される。
      \item また、低・無解約返戻金型商品に限らず、一般に解約返戻金の規模自体が、もしくはその算出の根拠となった予定利率などの基礎率が解約率を変動させるという関係にある。市場の金利が解約返戻金の計算に使用した予定利率を大幅に超過した時に予想される資金の流出、一時払変額年金に想定される特別勘定残高と最低保証水準との比較から惹起される解約行動がその例である。このように金融市場などの外部環境の変動に敏感に反応して解約行動が変動する状況を「動的な解約」という。
    \end{itemize}
  \end{enumerate}
\subsubsection{H30 生保1問題 3(1)①}
  解約返戻金に関連する以下の2点について簡潔に説明しなさい。
  \begin{itemize}
    \item 解約返戻金額と死亡保険金額に起因する「契約者間の公平性」について簡潔に説明せよ。
    \item 低・無解約返戻金型商品の開発における一般的な留意点のうち「保険契約者の理解」
  \end{itemize}
  
\subsubsection{解答}
  \begin{itemize}
    \item 契約者間の公平性(死亡保険金額との関係)
    \begin{itemize}
      \item 解約返戻金額が死亡保険金額を超過している場合、死亡保険金を請求するより解約したほうが契約者および保険金受取人の受け取る金額の総額が大きくなる。
      \item この場合、解約のほうが有利であることを知っている契約者とそうでない契約者の間の公平性に問題が生じる。
      \item 保険会社には、「お客様を公平に扱う」との考えのもと、解約返戻金額が死亡保険金額を超過しないような商品設計が求められる。
    \end{itemize}
    \item 保険契約者の理解 (低・無解約返戻金型商品の開発において)
    \begin{itemize}
      \item 低・無解約返戻金型商品を開発する際の留意点で最も大事なのは保険契約者の理解である。解約返戻金の水準を小さくすればするほど保険料は低廉になる。加入時にきちんと説明し理解してもらって加入いただくので、そこに問題があるわけではない。
      \item しかし、長期間経過したのちに解約する場合というのは、やむを得ないケースが多くなる。契約時に低解約返戻金のことを説明し水準を示すとしても、10 年、20 年と経った場合も想定しなければならない。
      \item これに対して、商品の設計上の工夫として、例えば次のような方法が取り入れられている。
      \begin{itemize}
        \item 解約返戻金をあまり極端に小さくは設定しない。
        \item 契約の乗り換えが頻繁には起こりにくい比較的高齢者向け、あるいは持病のある被保険者向けに設計された契約に導入する。
        \item 保険期間を短く設定する。
        \item 解約返戻金を低く抑える期間を短く設定する。
        \item 元々が十分な解約返戻金があることを期待されていない掛け捨て型商品に導入する。
      \end{itemize}
    \end{itemize}
  \end{itemize}

\end{document}

