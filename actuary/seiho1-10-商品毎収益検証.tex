\documentclass[report,gutter=10mm,fore-edge=10mm,uplatex,dvipdfmx]{jlreq}

\usepackage{lmodern}
\usepackage{amssymb,amsmath}
\usepackage{ifxetex,ifluatex}
\usepackage{actuarialsymbol}
\usepackage[]{natbib}
\RequirePackage{plautopatch}

% maru suji ① etc.
\usepackage{tikz}
\newcommand{\cir}[1]{\tikz[baseline]{%
\node[anchor=base, draw, circle, inner sep=0, minimum width=1.2em]{#1};}}

\usepackage{comment}

\begin{comment}

\ifnum0\ifxetex1\fi\ifluatex1\fi=0 % if pdftex
  \usepackage[T1]{fontenc}
  \usepackage[utf8]{inputenc}
  \usepackage{textcomp} % provide euro and other symbols
\else % if luatex or xetex
  \usepackage{unicode-math}
  \defaultfontfeatures{Scale=MatchLowercase}
  \defaultfontfeatures[\rmfamily]{Ligatures=TeX,Scale=1}
\fi
% Use upquote if available, for straight quotes in verbatim environments
\IfFileExists{upquote.sty}{\usepackage{upquote}}{}
\IfFileExists{microtype.sty}{% use microtype if available
  \usepackage[]{microtype}
  \UseMicrotypeSet[protrusion]{basicmath} % disable protrusion for tt fonts
}{}
\makeatletter
\@ifundefined{KOMAClassName}{% if non-KOMA class
  \IfFileExists{parskip.sty}{%
    \usepackage{parskip}
  }{% else
    \setlength{\parindent}{0pt}
    \setlength{\parskip}{6pt plus 2pt minus 1pt}}
}{% if KOMA class
  \KOMAoptions{parskip=half}}
\makeatother
\usepackage{xcolor}
\IfFileExists{xurl.sty}{\usepackage{xurl}}{} % add URL line breaks if available
\IfFileExists{bookmark.sty}{\usepackage{bookmark}}{\usepackage{hyperref}}
\hypersetup{
  hidelinks,
  pdfcreator={LaTeX via pandoc}}
\urlstyle{same} % disable monospaced font for URLs
\usepackage{longtable,booktabs}
% Correct order of tables after \paragraph or \subparagraph
\usepackage{etoolbox}
\makeatletter
\patchcmd\longtable{\par}{\if@noskipsec\mbox{}\fi\par}{}{}
\makeatother
% Allow footnotes in longtable head/foot
\IfFileExists{footnotehyper.sty}{\usepackage{footnotehyper}}{\usepackage{footnote}}

\end{comment}
%\makesavenoteenv{longtable}
\setlength{\emergencystretch}{3em} % prevent overfull lines
\providecommand{\tightlist}{%
  \setlength{\itemsep}{0pt}\setlength{\parskip}{0pt}}
\setcounter{secnumdepth}{-\maxdimen} % remove section numbering

\author{kazuyoshi}
\date{}

\newcommand{\problem}[1]{\subsubsection{#1}\setcounter{equation}{0}}
%\newcommand{\answer}[1]{\subsubsection{#1}}
\newcommand{\answer}[1]{\subsubsection{解答}}

%Pdf%\newcommand{\wakumaru}[1]{\framebox[3zw]{#1}}
\newcommand{\wakumaru}[1]{#1}





\begin{document}
\chapter{保険1第10章 商品毎収益検証}
\section{10.1 商品毎収益検証の必要性}
\problem{H14 生保1問題 2(2)①改}
商品毎収益検証の必要性・目的について簡潔に説明せよ。

\answer{解答}
\begin{itemize}
 \item [ア)] 必要性
\begin{itemize}
 \item 
米国ではアセット・シェアの計算原理を応用した将来収支分析を行うことによって保
険料を求めるのが実務となっている。その際、現実的な死亡率・解約率・事業費率を使
い、経営者・株主から事前に提示された利益の目標およびソルベンシーの目標を満足す
るよう、キャッシュフローのアウトの現在価値とインの現在価値を一致させる保険料を
計算している。リスクの準備をも明示的に取り込んでいる。
 \item 
一方、日本では利益目標等を明示的に組み込まない代わり、保険料の計算基礎率に安
全割増を組み込んでおり、これが利益の源泉となるが、保険料を求める時点では利益が
いくら得られるかわからない。
 \item 
したがって、保険料の計算基礎に組み込まれている安全割増が将来どのように利益
(剰余)として発生するかの検証、保険料の算式に組み込めなかったリスク要素の検証、
責任準備金の健全性の検証が、商品単位で必要になってくる。
\end{itemize}

 \item [イ)] 目的

\begin{enumerate} [(i)]
 \item 生命保険商品および商品群のキャッシュフローの特性を知るとともに、
 \item 個々の生命保険商品の特性が、会社全体の収益性・健全性に与える影響を検証することである
\end{enumerate}

商品単位で保険料の計算基礎率に組み込まれている安全割増が将来どのように利益
(剰余)として発生してくるかの検証・保険料の算式に組み込めなかったリスクの要素
の検証・リスク準備の検証・責任準備金の健全性の検証・商品間の配当公平性検証を行
い、生命保険商品の再設計・販売計画の策定・経営の方針の決定などに利用する。
最終的には会社モデルを構築し、各種のシナリオが会社全体の収益性・健全性に与え
る影響を直接検証することになる。

\end{itemize}
\section{10.2 商品毎収益検証の目的}
\problem{2020 生保1問題 3(2)①, H19 生保1問題 3(2)①、H14 生保1問題 2(2)①、H11 生保1問題 2(3)①}
商品毎収益検証の目的及びそれを実施するための 3 つの手順について簡潔に説明せよ。

\answer{}

【商品毎収益検証の目的】

商品毎収益検証の目的は、
 生命保険商品および商品群のキャッシュフローの特性を知るとともに、
 個々の生命保険商品の特性が、会社全体の収益性・健全性に与える影響を検証することである。
そのためには、
 生命保険商品の収益性・健全性に影響を与えると考えられる金利のシナ
リオ、解約率のシナリオおよび死亡率のシナリオなどの各種シナリオを設定し、
生命保険商品の特性に応じた将来のキャッシュフローを算出するモデルを構築し、
 モデル・ポイントを選定する、
という3つの手順を経る必要がある。

これらを通じて、
 生命保険商品の収益性・健全性の感応度の分析、
 特異なシナリオを使ったストレステスト、
 商品間の収益性・健全性の相互比較を行うことが必要になる。
 最終的には会社モデルを構築し、各種のシナリオが会社全体の収益性・健全性に与える
影響を直接検証することになる。
これらの検証結果は、今後の商品設計、販売計画の策定、
経営方針の決定に利用されることになる。

【3つの手順】

 シナリオの設定

 商品毎収益検証を実施するにあたり、生命保険商品の収益に影響を与え得る金利・死亡率・解約率等の要素について、
シナリオの中心となる「最も確からしいシナリオ」を第一に設定し、
次に「その周辺のシナリオ」や、
保険価格の計算基礎率に組み込まれていない
「起こりそうにないシナリオ」を設定する。
シナリオの設定は、その変動が収益性・健全
性にどのように影響するかという商品ごとの特性を考慮しつつ、将来予測の理論の信頼度
を勘案して、シナリオ間の相関関係や商品設計・準備金などによるリスクヘの対応状況を
踏まえて、アクチェアリーとしての判断をもとに行うことが求められる。

 モデルの構築

商品毎収益検証においてモデルの役割は、現実に発生し得るであろうキャッシュフロー
の表現にあり、検証の目的と重要度に配慮しながらその選定・構築を行うこととなる。
モデル選定にあたっては、キャッシュフローのタイミングをどうとらえているか、検証項目
をどう選定するか、検証目的と見合っているか、実務的であるか、という点が論点となる。
このうちキャッシュフローのタイミングの精度は、保険年度単位のモデル、事業年度単位
のモデル、月単位のモデルと進むにつれて向上するが、検証目的や実務を踏まえてモデル
を選定することが必要となる。

 モデル・ポイントの選定

商品毎収益検証のモデルを利用して会社全体の収益検証等を行う場合、計算効率の上昇
を目的として各契約を一定の要件のもと群団化し、群団を代表する契約をモデル・
ポイントとして選定する方法が取られることがある。
その際、会社モデルの計算に要
する時間、コンピューターの効率、シナリオの数や検証の手法(確率論的手法、決定
論的手法)、分析に求められる精度、モデル・ポイントの選定に要する作業コスト・時
間が選定の要点となる。
また選定は、得られたモデル・ポイントがどれほどよく会社
全体の保有契約を代表しているかを見るために、ヴァリデーション(各種統計数値と
モデル・ポイントを利用して算出した数値の比較評価)をしながら、トライ・アンド・
エラーで行われる。

\section{10.3 商品毎収益検証に用いるシナリオの設定}
\problem{2021 生保1問題 1(6), H17 生保1問題 3(1)①}
商品毎収益検証の手法である「確率論的手法」および「決定論的手法」について、メリット・デメリットを含め簡潔に説明せよ。

\answer{}

【確率論的手法】

非常に多くのシナリオや確率分布を仮定し、その仮定に基づき将来のキャッシュフロー
を推計する手法。

シナリオ毎のキャッシュフロー(シミュレーション)から現在価値の期待値や分布など
を求める方法、確率分布をもとに解析的に求める方法がある。

メリット

複雑な商品であっても、現在価値の分布・分散を把握する事が容易である。

変額年金など、オプション性のある商品の収益検証に有用である。

分布の裾野が広い商品の収益検証に有用である。

確率による重み付けが可能である。


デメリット

計算負荷が大きい。

単純化されたシナリオでないと解析的に求めることが出来ない。

結果の再現が出来ない、または再現が容易ではない場合がある。

計算する度に結果が異なる場合には、客観性の担保が困難となる。

異常値として考慮しなかった極端なシナリオがある場合には、その影響を把握できない。

【決定論的手法】

たかだか数本のシナリオを用いて、将来のキャッシュフローを推計する手法。

シナリオの設定には、通常、最も確からしいシナリオとその周辺のシナリオおよび起こ
りそうにないシナリオが使用される。

メリット

計算負荷が少なく、簡便である。

再現が容易であり、客観性を担保しやすい。

極端なシナリオの与える影響を考慮しやすい。

デメリット

シナリオの違いにより計算結果が大きく異なる可能性があるため、シナリオが結論に大
きな影響を及ぼす。

現在価値の分布・分散が不明確である。確率による重み付けが出来ない。


\problem{H14 生保1問題 2(2)②}
感応度分析とストレス・テストについて簡潔に説明せよ。

\answer{}

感応度分析

「感応度分析」とは、シナリオによりどれほど生命保険商品の収益性・健全性が変動する
かを分析すること、である。シナリオのパラメーターを動かすことによって、最終的に判定
したい指標例えばIRRであるとか損益分岐年などがどのように変動するのかを見るも
のである。これは動かすパラメーターによって指標の動きに差異があることをとらえて行
われる。

ストレス・テスト

「ストレス・テスト」とは、収支が相当悪化するような状況が発生したとき、生命保険商
品がどのような収益性・健全性を示すであろうかを知ることである。これは特異なシナリ
オを用い、パラメーターを想定しうる特異値に設定することにより、想定されるリスク限
度を観測するものである。


\section{10.4 死亡率のシナリオの設定}
\problem{H28 生保1問題 2(1)}
商品毎収益検証において、死亡率のシナリオを設定する際に定性的に配慮すべき事項を5項目列挙し、それぞれ簡潔に説明しなさい。
\answer{}

会社の過去の経験値

会社の過去の経験値は、他の要因をすべて網羅した結果を含んでおり、この分析が第一にな
されなければならない。分析結果が統計的に安定でないと判断されるならば、他の方法をとる
べきである。

生命保険業界の経験値

新設会社等で、充分な保有契約・経過年数がなく、自社の経験値を利用しても信頼できる統
計値が得られない場合、生命保険業界の経験値を利用することもできる。この場合、販売商品
の特性、販売制度・販売チャネルおよび自社の査定基準など、会社独自の事情を勘案する必要
がある。

国民生命表

生命保険の被保険者は選択された集団であり、一般に国民全体が示す死亡率より良好な死亡
率を示すといわれている。しかし、日本では、ほとんどすべての国民が何らかの形で生命保険
の被保険者であり、生命保険を契約してから一定期間が経過した被保険者の死亡率は国民生命
表の死亡率に接近していると考えることもできる。

選択効果

生命保険を契約する際、保険会社は自社の引き受け基準に従って被保険者の体況などに関す
る選択を行うことから、被保険者集団は経過年数の浅いうちは良好な死亡率を示す傾向にある。
これには、自社の許容できるリスクの範囲内でリスクを保有し、被保険者集団が相互に請け合
うリスクを公平に扱おうとする意味がある。自社の過去の経験値から、この選択効果を統計的
に定量化することになる。

再保険会社からの情報

再保険会社には、再保険契約を取り扱うことによって、多くの保険会社の死亡率に関するデ
ータが蓄積されており、これらを元受会社に対するサービスとして提供する場合がある。

自社の査定基準

会社の査定基準のあり方によっては、自社に死亡率の高い被保険者集団が偏る可能性があり、
生命保険業界の平均的な経験値より高い死亡率の被保険者が自社に集中している可能性もあ
る。競合他社の査定基準と自社の査定基準とを比較検討し、使用する死亡率シナリオを適正に
設定する必要がある。また、有診査契約と無診査契約の別などの査定内容による差異も配慮す
べきである。

販売制度・販売チャネル

一般に販売制度が異なれば、被保険者集団の属性も異なると考えられる。健康管理のゆきと
どいた職域の被保険者集団は比較的死亡率の低い集団であると推定できるが、ダイレクトによ
り募集された集団は、モラルリスクが多く混入しており比較的死亡率の高い集団が形成される
可能性もある。自社の販売チャネルごとに死亡率のシナリオを設定することも考えられる。

商品特性

一般に、定期性商品にはモラルリスクが混入する可能性が高いと考えられる。年金保険契約
の加入者には、自分の健康に不安のない者が多いと推定される。

オプション

定期性商品で将来の更新を約定する場合、更新する契約の被保険者の死亡率は、更新しない
契約の被保険者の死亡率よりも高いと考えられる。更新後まで含めた商品毎収益検証を行う場
合、自社の過去の経験値等に基づき、この差を配慮する必要がある。

社会全体の動向

一般的に、経済が不況のときは、モラルリスクが混入する可能性が高くなると予想される。
また、現在の日本ではほとんど配慮する必要はないと考えられるが、エイズ等の特別な疾病の
影響を調査する必要もある。

新商品

優良体保険または非喫煙者保険が開発・販売された場合、同様の給付を行っている商品の自
社の契約者が買い替えをすることが予想され、既存商品の被保険者の死亡率が相対的に上昇す
る可能性がある。また、自社の新商品だけでなく、競合他社の新商品であっても、これに買い
替えられることにより、自社の契約の死亡率が上昇する可能性もある。

\section{10.5 金利のシナリオの設定}
\problem{H14 生保1問題 2(2)③}
(2)商品毎収益検証について、以下の問に答えよ。
③ 金利シナリオを設定する際に留意すべき点について所見を述べよ。なお、金利シナリオの
設定手順の内容についても必ず触れること。

\answer{}
ア)金利シナリオの設定手順

金利シナリオの設定は金利の動向を予測するステップと、その金利予測に従って、どう資
産運用を行なうかという方針決定のステップに分解される。金利の将来予測のステップは、
金利の期間構造の変化を予測することであり、運用方針の決定のステップは再投資政策を
立案することである。

○金利の期間構造の将来予測

(i)最も確からしいシナリオの設定

現在将来の金利の動向を完全に予測する財務投資理論が確立しているとは言い難い。
過去の経験値を用いて将来の金利の動向をさぐる手法、つまり、現在に至るまでの金利
の動向が何らかの形で将来の動向を予測する手段となるという考え方が基本となろう。
金利以外の経済指標との相関から金利モデルを構築する場合でも、その経済指標の将来
の動向については、過去の経験値を分析することによって推定するのが一般的であろう。

(ii)収益性悪化または健全化する方向のシナリオの設定

正規分布等を仮定して、或いは確率論的シナリオを用いるなどにより感応度分析、
ストレス・テストを行う。

○再投資政策

再投資政策として立案しなければならないものは、将来投資する債券の種類、投資
する債券のクーポンレート、投資する期間などの項目及び資産売却計画である。また、
中途償還、債務不履行、評価損益などを考慮する必要がある場合がある。資金の借り入
れを行なう方針があれば借入方法、借入資金の種類、借入利率の設定方針を設定する必
要がある。

イ)留意すべき点と所見

答案作成に際し、実際の収支分析の手順に沿って留意点について考察すると答案を作
成しやすいであろう。金利シナリオの設定方法が必ずしも確立されたものではないため、
人により意見の異なる部分も多く、ここで示す答案は一例である。

○目的に応じた将来収支分析

将来の金利を予測するときは財務投資理論を応用することになるが、死亡率等は統計
的に安定している等の理由により将来の動向は比較的正確に予測できると考えられる
のに対して、前述のとおり現在将来の金利の動向を予測する財務投資理論が確立してい
るとは言い難い。したがって、目的に応じて、金利モデルの理論の信頼度にあわせ、ま
た、将来収支分析全体の結果における金利シナリオの重要性に応じて、シナリオを充実
させることが重要である。

また、シナリオの設定は商品ごとの特性を考慮して行なう必要があり、例えば短期の
定期保険の収支分析を行なう場合のように、将来収支分析全体の結果における金利シナ
リオが重要でない場合は、金利シナリオの数は自ずと数が少なくなるであろう。極端な
話として金利は一定で変動しないシナリオでも、金利の予測を無理なく含み、再投資政
策も含まれているとみなすことができる場合もあろうと考えられる。

○金利の期間構造の将来予測

各作業ステップにおいて、次のような点に留意する必要があろう。

(i)決定論的シナリオか確率論的シナリオかの決定

収益検証の目的、将来収支分析全体の結果における金利シナリオの重要性、計算負
荷などを考慮して決める。

(ii)財務投資理論の選択

確立したものがあるとは言い難い状況であるので、それぞれの理論の特徴を理解し、
慎重に選択する必要がある。
また、株価・債券価格などの相互に影響し合う項目に留意し、その影響を考慮した
モデル・シナリオとする必要がある場合がある。

(iii)将来推定用の過去のデータの選択

自社の運用実績、過去の市場動向を用いて将来の金利を予測することとなるが、そ
のサンプル期間の長さ・時期により将来結果が大きく異なることとなるので注意が必
要である。また、アウトライヤー(通常の分布から大きく外れた値)がある場合、経
済構造変化が生じた場合には、その修正またはその期間の除外が望ましい。
データの信頼性・連続性にも注意が必要である。

(iv)シナリオの検証

入手可能な経済動向の分析結果、金利動向の調査結果と比較すること等が重要であ
る。最終的には説明可能なシナリオであることを確認する必要がある。

(v)保守的な設定

収益検証の目的によっては、保守的な見積もりとなるよう設定する必要がある。

(vi)収益性悪化または健全化する方向のシナリオの設定

感応度分析により、全体の状況を把握することができ、どのような収益指標におい
て金利の貢献度が高いかを知ることができる。感応度分析のシナリオは将来金利の正
規分布を仮定したり、単に1%上下した場合などのように分かり易さを優先したシナ
リオとすることもある。
ストレス・テストを行なうことにより、リスク量を明確にすることができる。このと
きの信頼度は収益検証の目的、内部留保などによるリスク対応状況を勘案することが必
要である。

○再投資政策

会社の運用方針を反映させたものとする必要があるので、経営者、上級管理者、財
務運用部門などとの連携をとる必要がある。負債のデュレーションなどの商品特性・将
来のキャッシュフローを踏まえた運用が行なわれる場合には、再投資政策にもそれらを
反映させたものとする必要がある。
ただし、計算を簡便化するため、或いは分かりやすくするために保有資産の比率を今
後とも一定にする場合がある。また、簡便化を図ると同時に保守的な再投資とするため、
すべて国債に再投資するなどの仮定をおくことがある。

○商品間の収益性、健全性の相互比較

収益性、健全性の考察及び感応度分析、ストレス・テストの考察は相対的なものであ
ることが多いので、これを商品毎に比較することが必要である。

○他シナリオとの関連性

分析の目的により、次のようなシナリオ間の相互作用をモデルに反映させることが
必要である場合がある。

(i)金利と解約率

解約率は金利のシナリオの影響を受ける。また、逆に、解約のキャッシュフローは
規模が大きいので、投資運用収益により大きな影響を与える。

(ii)金利と事業費率

インフレ時には高金利になるとともに事業費が上昇する。

○まとめ

金利シナリオの設定方法については確立されたものがあるわけではないため、設定者
の判断に依存する面が大きく、主観的な恣意性が介入しやすい。このため必要に応じて
経営者、上級管理者、財務運用部門などとの連携をとりながら、説得力のあるシナリオ
を設定する必要がある。
また、いたずらに精綴な計算に走ることなく、適切に手法を使い分け、常にどの前提
の基での計算であるかを意識し、目的にあった分析とする必要がある。

\section{10.6 解約率のシナリオの設定}
\problem{H19 生保1問題 3(2)②}
解約率のシナリオを設定するに際し、解約率の特性について言及しつつ、留意すべき点を挙げよ。

\answer{}
【解約率の特性】

解約率の特性については、以下の点が挙げられる。

解約は、契約者からの一方的通知で足りるので、事後的経営管理が困難であること

解約は契約者側からの一方的な通知で足り、事後的なコントロール可能な部分(事後的
経営管理できる部分)が少ない。投資政策の変更等によりある程度の事後的コントロール
が可能な金利に関するシナリオ、もしくは契約者の意志でコントロールすることが一般に
不可能であり、その発生が比較的安定している死亡率との特性の相違が存在する。

解約は、他のパラメーターの影響、他のシナリオとの連動が考えられること

解約は、多くのパラメーターの影響を受け、かつ他のシナリオに連動している。例えば、
貯蓄性の高い商品の解約率(市場金利との連動)、経済状況の悪化を起因とする解約(経
済状態との連動)、健常者め解約による残存集団の死亡率の悪化(死亡率への影響)等が
挙げられる。

解約率の変動幅は、死亡率及び金利の変動幅より大きく、投資運用収益に大きく影響
することが予想されること

解約率の変動のオーダーは一般に死亡率、および金利と比較すると大きい。更にそれが
投資運用収益・効率に影響を与える。

商品の特性が解約を誘引すること

商品の特性により解約が誘引される例として、次のものが挙げられる。

解約返戻金と払込保険料総額

既払込保険料総額を解約返戻金が上回った場合に解約を誘引する可能性がある。

死亡保険金を上回る解約返戻金

ある時点で解約返戻金が死亡保険金を上回る場合、死亡事故が発生したときには死亡
保険金が請求されるのではなく解約返戻金が請求されることとなり、そのような時点で
の解約率は増大する(死亡率は減少する)。

変額保険と死亡保障

変額保険においてインデックスが大幅下落した場合に、最低死亡保障の存在により、
かえって解約率が減少する可能性が考えられる。

解約率の一定方向への変動が、収益性・健全性を一定方向へ変動させるとは限らないこと

商品の特性・設計により、解約率の一定方向への変動が、収益性・健全性を一定方向へ
変動させるとは限らない。その例としては、ラプス・サポーテッド商品あるいは標準責任
準備金積立における予定利率と保険料計算基礎における予定利率が異なる場合の解約率
の増加等が挙げられ、ラプス・サポーテッド商品の場合は、解約率の改善が収益性・健全
性を悪化させる方向へ変動する。

【解約率シナリオ設定の際の留意点】

解約率のシナリオの設定は非常に困難である。金利だけではなく、非常に多くのパラメ
ーターが解約に影響を与える。そもそもパラメーターが何であるかもはっきりしない。よ
って、解約率のシナリオの設定の際には、まず、パラメーターを発見するところから始め
なければならない。

解約率に影響を与える項目として、経過年数、経済動向と市場金利、販売チャネル・販
売方法、加入目的、保険料の規模・変動、保険金額、保険料の払込方法(回数・経路)、
保険料払込期間・保険期間、年齢・性別、商品特性、特別条件・優良体保険、新商品販売、
税制、報酬制度といった数々のものが考えられる。この他にも解約率に影響を与える項目
がある可能性を無視してはいけない。項目中の判断も、状況によってはまったく異なるも
のになる可能性がある。

これらの項目と解約率の関連を示す統計は十分ではない。過去の統計だけからではシナ
リオの設定は不可能であるといってもよい。解約率は死亡率および金利と違って、予測理
論が希薄である。解約率の特性に記したとおり、解約率の将来動向は死亡率・金利の将来
動向とは異なる特性を持っている。

従って、解約率のシナリオ設定では、死亡率や金利のシナリオの設定とは異なるアプロ
ーチが必要となる。過去の経験値をべースにするにしても、アクチュアリーは、生命保険
商品のあらゆる側面を検討し、商品毎収益検証の目的を勘案しながら、総合的な判断を加
えることによって経験値を修正し、解約率のシナリオを設定しなければならない。


\problem{H12 生保1問題 2(3)①}
商品毎収益検証における解約率のシナリオの設定について、解約率に対して影響を与える項目を 5つ挙げ、説明せよ。

\answer{}

経過年数

一般に解約率は契約の初期において大きく、期問が経過するにつれて減
少する傾向にある。あるいは経過年数の相違により、解約率の水準に明確
な相違が見られる場合が多い。

経済動向と市場金利

貯蓄性の高い商品は他の金融商品の金利水準と比較され、相対的に見て
魅力的でない場合は解約が増加する可能性が高い。また一般に、金利が上昇
すると固定金利で締結された生命保険契約は解約される率が高まる。昨
今の生保会社を取りまく状況のように、経済状況の慢性的悪化により金融
機関に対する不信不安が高まり解約率が上昇する場合がある。不況時には
家計における収入減少や失業等が発生する機会が多く、既加入の保険内容
の見直しにより解約率が高まることが考えられる。

販売チャネル・販売方法

販売チャネルにより解約率が異なるであろうことが想定される。また同
じ販売チャネルでも、コンサルティングが徹底されている場合とそうでな
い場合、代理店における乗り合いの程度、あるいは販売されている商品や
チャネルの主要な顧客層といった要素によって解約率が異なってくる。ブ
ローカー制度やダイレクト・メールを含む通信販売チャネル等の解約率の
動向は、経験が十分でない。


加入目的

生存保障目的、短期貯蓄目的あるいや義理人情による加入等の加入目的
の違いにより解約率の動向は異なってくる。法人契約の場合は法人単位で
解約の意志決定がなされる場合があることから、解約が集中したり、解約
率が安定しない傾向が発生しがちであることを考慮する必要がある。この
いわゆるショック・ラプス(ある事由を契機とした集中解約)のシナリオ
も商品毎収益検証の対象とする必要性があり得ると考えられる。

保険料の規模・変動

保険料が高額であるか低額であるかが解約率に影響を与える可能性があ
る(例えば家族構成の変化、教育費等の支出の増大等により高額死亡保障
契約ほど解約率が高くなる可能性があること等)。また保険料の高低は販
売チャネル、商品特性に依存する場合が多く、予期せぬ販売話法等に注意
する必要がある。ステップ払契約、更新型定期特約等、解約率が不連続に
上昇する可能性がある商品が存在することにも注意する必要がある。

その他にも、解約率に影響を与える要素として以下のような項目が考えられ、
このような項目についても正解とした。

保険金額


保険料の払込方法(回数・経路)

保険料払込期間、保険期間

年齢、性別

商品特性

特別条件、優良体保険

新商品販売

税制

報酬制度

\problem{H12 生保1問題 2(3)②}
解約率の特性について、死亡率および金利の特性と異なる点を説明せよ。
\answer{}
解約は、契約者からの一方的通知で足るので、事後的経営管理が困難で
あること。

解約は契約者側からの一方的な通知であり、事後的にコントロール可能
な部分が少ない。投資政策の変更等によりある程度の事後的コントロール
が可能な金利に関するシナリオ、もしくは契約者の意志でコントロールす
ることが一般に不可能であり、その発生が比較的安定している死亡率との
特性の相違が存在する。

解約は、他のシナリオに連動していると考えられること。

解約は他のシナリオに連動している。例として、貯蓄性の高い商品の解
約率(市場金利との連動)、経済状況の悪化を起因とする解約(経済状態
との連動)、健常者の解約による残存集団の死亡率の悪化(死亡率への影
響)等が挙げられる。

解約率の変動幅は、死亡率及び金利の変動幅より大きく、投資運用岐益
に大きく影響することが予想されること。

解約率の変動のオーダーは一般に死亡率、および金利と比較すると大き
い。さらにそれが投資運用収益に影響を与える。

商品の特性が解約を誘引すること。

商品の特性により解約が誘因される例として、以下のものが挙げられる。

解約返戻金と払込保険料総額

既払込保険料総額を解約返戻金が上回った場合に解約を誘因する可能性がある。

死亡保険金を上回る解約返戻金

ある時点で解約返戻金が死亡保険金を上回る場合、死亡事故が発生し
たときには死亡保険金が請求されるのではなく解約返戻金が請求される
こととなり、そのような時点での解約率は増大する(死亡率は減少する)。

変額保険と死亡最低保障

変額保険においてインデックスが大幅下落した場合に、最低死亡保障
の存在により、かえって解約率が減少する可能性が考えられる。


解約率の一定方向への変動が、収益性・健全性を一定方向へ変動させる
とは限らないこと。

解約率の一定方向への変動が、収益性・健全性を一定方向へ変動させる
とは限らない。その例としては、ラプス・サポーテッド商品あるいは標準
責任準備金積立における予定利率と保険料計算基礎における予定利率が異
なる場合の解約率の増加等が挙げられる。

\problem{H27 生保1問題 2(1)}
商品毎収益検証のシナリオの設定において考慮すべき解約率の特性に関して、死亡率および金利の特性と異なる点を簡潔に説明しなさい。
\answer{}
解約は、契約者からの一方的通知で足るので、事後的経営管理が困難である。

保険契約は保険契約者と保険者との双務契約でありながら、保険者の側に解約権はなく、契約
者側からの一方的通知によって契約の解約が行われるため、どの様な政策方針を事後的に打ち
立てようとも解約の防止には限度があり、事後的にコントロール可能な部分が少ない。

投資政策の変更等によりある程度の事後的コントロールが可能な金利に関するシナリオ、もし
くは契約者の意思でコントロールすることが一般に不可能であり、その発生が比較的安定して
いる死亡率との特性の相違が存在する。

解約は、他のシナリオに連動していると考えられる。

特に貯蓄性の高い商品の解約は、他の金融商品との魅力の差により誘引される。例えば、市場
金利が上昇し、生命保険商品の付与利率がこれに追随できなければ、解約の増加が予測される。

ただし、低金利下においても、経済状況の悪化、金融機関への不満・不安等による解約増加も
考えられることから、解約率は金利シナリオに連動していると考えられるものの、そのモデル
化およびシナリオの設定は非常に困難である。

また、健康状態に自信のある者は安易に解約する傾向にあると考えられ、残存する被保険者集
団の平均的死亡率は悪化するため、解約は死亡率のシナリオに影響することもある。

解約率の変動幅は、死亡率および金利の変動幅より大きく、投資運用収益に大きく影響すると予
想される。

解約率は通常数パーセントのオーダーで変動するため、通常千分の1のオーダーである死亡率
および金利の変動幅よりも大きく、解約率の変動により生じるキャッシュフローが会社の投資
運用収益に与える影響は大きいと考えられる。

商品の特性が解約を誘引する。

商品の特性により解約が誘引される例として、以下の例が挙げられる。

単純に既払込保険料と解約返戻金を比較し、後者が前者を上回ったときに解約が誘引される
可能性がある。

ある時点で解約返戻金が死亡保険金を上回る場合、死亡事故が発生すると死亡保険金ではな
く解約返戻金が請求され、解約率が増加する可能性がある。

死亡給付や生存給付に最低保証がある変額商品において、特別勘定のインデックスが下落し
た場合、最低保証を期待して解約率が減少することが考えられる。

解約率の一定方向への変動が、収益性・健全性を一定方向へ変動させるとは限らない。

例えば、保険料の計算基礎率に予定解約率を入れた場合や、保険料計算基礎の予定利率を責任
準備金計算基礎の予定利率よりも高めに設定した場合などにおいて、解約率の増大は収益性・
健全性を改善する方向に働く可能性があるなど、生命保険商品の特性・設計によっては、解約
率の増大が必ずしも収益性・健全性の悪化につながらないこともある。

\problem{H21 生保1問題 2(3)}
商品毎収益検証を行なう際の解約率の設定にあたり、以下の商品において各ア~ウの要素が解約率に与える影響について、それぞれ簡潔に説明しなさい。

①無解約返戻金タイプの終身払終身入院保険について、ア.経過年数、イ.経済動向、ウ.加入年齢、が解約率に与える影響

②保険期間 10 年の全期払定期保険について、ア.加入目的、イ.保険金額、ウ.特別条件・優良体、が解約率に与える影響
\answer{}
①無解約返戻金タイプの終身払終身入院保険

経過年数

一般的な生保商品と同様、契約当初は解約率は高く、経過を経るにつれ減少傾向。

一定程度経過した場合、解約返戻金取得目的での解約がないことから、解約率は解
約返戻金のある商品に比べて同等以下となることが想定される。

経済動向

一般に、経済環境が悪化すれば解約率は上昇するものと想定される。

本商品は無解約返戻金タイプであることから経済動向の悪化により、解約返戻金取
得目的とした解約は生じないが、保険料の支払負担増から解約につながるケースが
想定される。ただし、保険料が低廉であるため、保険料負担面から経済動向に比較
的影響されにくいと想定される。

加入年齢

高齢者層は、老後の医療保障二一ズに対する意識が高いこと、生活が安定しており、
保険料支払能力が高いことから、継続率は若年層に比べ良好であると想定される。

加入年齢が高いほど、加入後に健康状態が悪化する可能性が高まり、新規加入が難
しくなることから、解約する動機が薄まり、継続率は良くなることが想定される。

若年層は、家族構成等により、将来のライフサイクルが変わる可能性があり、その
時々の保障内容の適合性の観点から、高齢層に比べ継続率が悪くなる傾向にある。

②保険期間 10 年の全期払定期保険

加入目的

個人契約の場合、一般に、定期保険においては、遺族に対する生活保障が加入目的
である。従って、この加入目的が明確であれば、継続率は良好と考えられる。

個人契約か法人契約によっても加入目的が異なる。後者では経営者保険が考えられ、
継続率も継続の必要性に応じてそれ相応に異なることが想定される。

法人契約や、特定の商品・チャネルに特化した販売チャネルにおいては、外的な要
因による、「ショック・ラプス」と呼ばれる集中的な解約が起こることがある。

保険金額

保険金額の高い契約は、契約者の独自の二一ズが反映されて高くなっている場合、
継続率は良好であると考えられる。一方、保障額が高額であるが故、料率の差異の影
響が大きいため、より低廉な料率を求めての解約も考えられる。また経過に伴い、経
済動向や必要保障額の変化等により、契約当初の二一ズが喪失しての解約も生じうる。

高額保障の場合、収入の変動により保険料負担が過大となり、解約することも想定
される。

特別条件・優良体

特別条件については、健康状態がより一層悪化していること等、他の保険契約には
解約後再加入が困難であれば、継続が見込まれる。一方、健康状態が改善し、より
有利な条件で再加入できれば、解約は高まる。

優良体では、各生命保険会社が競って優良体商品を提供する場合、より低廉な料率
を求めて既存商品からの乗り換えが生じ、解約率を悪化が想定される。

\section{10.7 事業費の仮定の設定}

\problem{H28 生保1問題 3(1)①}
商品毎収益検証の事業費の配賦における配賦単位について簡潔に説明しなさい。なお、保険料設定にお
ける予定事業費の設定では用いられていない配賦単位を取り扱う場合の留意点についても触れること。
(5点)
\answer{}
商品毎収益検証の事業費の配賦における配賦単位

 適切に商品毎収益検証を行うためには、会社の事業費を分析し、商品 1 件ごとが負担すべき
事業費を求める必要がある。ただし、事業費は保険料の収入や保険金の支払とは異なり、必
ずしも商品 1 件ごとに直課できる経費ばかりではない。商品 1 件ごとが負担すべき事業費を
求めるためには、商品毎の事業費特性を考慮し、実際の事業費を適切な配賦単位に分類し配
賦する必要がある。

 配賦単位の例としては、
「件数比例の費用」・「保険金額比例の費用」・「保険料比例の費用」・「責任準備金比例の費用」などがある。

 このうち「件数比例の費用」などは営業保険料に明示的に加味しにくいものであり、この場
合は保険料と事業費が適切に対応しない。例えば、性別・年齢ごとに詳細に収益検証を行う
場合や、保険料が相対的に小額な商品の収益検証を行う場合などにおいて、費差損益が過小
(過大)に算出され安定しないものとなることに留意が必要である。

\problem{H16 生保1問題 3(2)①}
事業費を新契約費と維持費に区分する必要性について説明し、新契約費として区分されるものにど
のようなものがあるかを列挙せよ。
(8 点)

\answer{}
【新契約費と維持費に区分する必要性】

生命保険商品の事業費にかかるキャッシュフローは、一般的に、契約当初の新契約獲得にかか
る費用負担が大きく、一方、収入たる付加保険料は保険期間を通じて平準的である。よって、商品
毎収益検証にあたっては、保険期間を通算し契約当初の費用負担を将来の付加保険料収入で賄
えるかどうか、維持にかかる収入・支出のバランスはどうかを確認する必要がある。このため、事業費は総額だけでなく、新契約にかかる経費と維持にかかる経費を区分する必要がある。

【新契約費として区分されるもの】

営業職員・代理店の報酬・手数料

新契約査定部門の人件費・物件費

医務部門の人件費・物件費

広告宣伝費、募集文書の作成、発行、商品パンフレットの作成、ダイレクト・メールの作成にか
わる費用およびこれに伴う人件費・物件費

商品開発にかかわる部門の人件費・物件費

支社・営業所、本社営業管理部門のうち、契約の保全業務以外のすべての経費

情報システム関連の経費のうち、新契約に関する業務と新商品開発に関する業務の経費

その他、支社・営業所・本社を含むすべての不動産の賃貸料、税金、修繕費、光熱費等の物
件費のうち、新契約にかかわる部分

\problem{H16 生保1問題 3(2)②}
事業費の配賦における配賦単位について説明せよ。
(7 点)

\answer{}

【事業費の配賦の必要性】

商品毎収益検証を行うためには、会社の事業費を分析して、商品1件ごとが負担すべき事業費
を求める必要がある。ただし、保険料の収入や保険金の支払とは異なり、事業費は必ずしも商品1
件ごとに直課できる経費ばかりではない。商品1件ごとが負担すべき事業費を求めるためには、実
際の事業費を適切な配賦単位に分類し配賦する必要がある。

【事業費の配賦単位】

配賦単位は、経費が何に比例または連動して支出されているかに基づき、経費ごとに決定される。
主な配賦単位としては、以下のようなものが挙げられる。(各項目の内容については教科書を確認のこと)

件数比例の費用

契約保全のための情報システム運用経費、契約保全

部門の人件費・物件費、新契約査定部門の人件費・物件費、および、申込書、

保険証券、募集資料、しおり付約款の作成費


保険金額比例の費用

新契約の査定基準に連動する経費

営業職員、営業担当職の報酬(保険金額を基準の場合)

保険金の支払業務も、保険金額の高低により経費が異なるような場合

支社・営業所経費、営業推進部門の人件費等、新契約の獲得に従事する部門の経費(営業予算が保険金額基準で策定されている場合)


保険料比例の費用

代理店の報酬と団体扱いの集金事務手数料

営業関連部門・部署の経費(営業予算が収入保険料または年換算保険料で策定されている場合)


保険料収納1件(1回)あたりの費用

収納コスト他、未納案内を作成・郵送する経費


責任準備金比例の費用

変額保険では特別勘定運営費,

一時払養老保険のように貯蓄性の高い商品にあつては、投資運用費用

責任準備金の充分性に関する業務・システム開発(標準責任準備金、責任準備
金の積み立て計画、保険計理人の実務基準、ソルベンシー・マージンの開示)


手数料・営業職員の報酬比例の費用

消費税 、本社営業管理部 門 の人件費等, コンピュー タ ー・ システム 開発 コス ト, 
営業所長等 の 管理者 の給与 (の一 部)


\problem{H25 生保1問題 2(3)}
商品毎収益検証を行う際の事業費のシナリオの設定について、以下の問に答えなさい。

① 事業費を新契約費と維持費に区分する必要性について説明しなさい。

② 配賦単位を用いる必要性について説明し、配賦単位の例を3つ挙げなさい。

③ 会社の経験値が使用できない場合および将来の「規模の経済」を考慮する場合の留意点を述べな
さい。なお、「規模の経済」を考慮する場合の留意点については、「生産性の向上」についても
言及すること

\answer{}
① 事業費を新契約費と維持費に区分する必要性について説明しなさい。

生命保険商品の事業費にかかるキャッシュフローは、一般的に、契約当初の新契約獲得にかかる
費用負担が大きく、一方、収入たる付加保険料は保険期間を通じて平準的である。
したがって、商品毎収益検証では、契約当初の費用負担を将来の付加保険料収入で賄えるかどう
か、維持にかかる支出・収入のバランスはどうかなどを検証する必要がる。このため、事業費は総額だけでなく、新契約にかかる経費(新契約費)と維持にかかる経費(維持費)に区分する必要がある。

② 配賦単位を用いる必要性について説明し、配賦単位の例を3つ挙げなさい。

適切に商品毎収益検証を行うためには、会社の事業費を分析し、商品 1 件ごとが負担すべき事業
費を求める必要がある。
しかしながら、事業費は保険料の収入や保険金の支払とは異なり、必ずしも商品 1 件ごとに直課
できる経費ばかりではない。商品 1 件ごとが負担すべき事業費を求めるためには、商品毎の事業費
特性を考慮し、実際の事業費を適切な配賦単位に分類し配賦する必要がある。
配賦単位の例としては、
「件数比例の費用」・「保険金額比例の費用」・「保険料比例の費
用」・「責任準備金比例の費用」などがある。

③ 会社の経験値が使用できない場合および将来の「規模の経済」を考慮する場合の留意点

○会社の経験値が使用できない場合の留意点

商品毎収益検証に用いる事業費は、会社の経験値を分析し、設定することが考えられるが、新設
間もない会社の販売する商品や新規の販売チャネルで販売される商品に対しては、経験値を使用することができない。
このような場合には、

生命保険業界の経験値または同規模の他社の経験値

会社全体の事業計画または新規チャネルの事業計画

各々の経費の積み上げ

などの情報が利用できる。

○将来の「規模の経済」を考慮する場合の留意点

主に商品 1 件あたりの維持費に対して、会社規模の拡大に伴い 1 件あたりの事業費は減少すると
いう「規模の経済」の原理が働く。新設間もない会社で、将来、会社規模の拡大が見込まれる場合
は、現時点の 1 件あたり事業費は相対的に大きくなっており、「規模の経済」を将来の事業費に考
慮することが考えられる。

ただし、規模の経済は将来の保有件数、新契約件数に依存するため、大きく予測を間違えることもあり得るので注意が必要である。

なお、
「規模の経済」に似た概念として、職員の処理能力の向上等によりもたらされる「生産性
の向上」があるが、これは、新設会社や新規チャネルの場合であっても、新たな技術革新でもない
限りは短期のうちに限界に達し、将来、大きく変動しないと思われる。「規模の経済」の考慮にあ
たっては、
「生産性の向上」と混同しない分析が必要である

\section{10.8 契約内容の検証}
(過去問での出題なし)
\section{10.9 収益性・健全性の指標の例}

\problem{H11 生保1問題 1(10)}
プロフィット・マージンの計算において現在価値を求める際、資産運用利回りを上回る率である危険
割引率で割り引く場合があるが、そのことの意味を 2 つ簡潔に説明せよ。
\answer{}
危険割引率で割り引く意味としては次の2つがあげられる。

①将来の予測値には不確実性が大きいので、その分を過小評価する意味。

②保険会社への投資家(株式会社の場合は株主、相互会社の場合は既存
の契約者)が期待する、一般の金融商品以上の収益率という意味。この
危険割引率で計算した税引後利益の現在価値がゼロ以上となれば、投資
に値する会社であると判断される。

\problem{H14 生保1問題 1(3)}
商品の収益検証の際の将来収支分析に関する次の①~⑤について、正しいものには○、誤りのあるものには×を付けよ。

① プロフィット・マージンでは、利益の発生の時期を判断できない。

② プロフィット・マージンがマイナスの場合でも、内部収益率がプラスの場合がある。

③ 割引率として将来収支分析に使用した資産運用利回りを用い、かつ利益の社外流出を考慮する場合プロフィット・マージンは責任準備金の積み立て水準に依存しない。

④ 割引率として将来収支分析に使用した資産運用利回りを上回る率を用いた場合、プロフィット・マージンは保守的に算出される。

⑤ 継続率が悪化すれば、収益の認識が後ろ倒しになるので投資回収年度は遅くなる。

\answer{}
①○②○③×④×⑤×

③ ...利益の社外流出を考慮\sout{する}しない場合

④分子・分母ともに小さくなる (過小評価) であるが、マージンとして保守的(小さく)になるとは限らない。

⑤継続率が悪いとき、投資回収年度が短くなる場合がある(シナリ
オ32参照)。これは、経過期間の浅いうちに発生した解約益で見
た目の利益が上がるからである。(10-104の5.)

\problem{H18 生保1問題 2(2)}
商品毎収益検証において用いられる指標である「プロフィット・マージン」「投資回収年度(Break
Even Year)」「内部収益率(Internal Rate of Return)」に関し、以下の設問に答えよ。

①プロフィット・マージンの計算において、「資産運用利回り」と「危険割引率」に同じ率を使用した
場合(すなわち、下記の記号において$i=j$の場合)、プロフィット・マージンは責任準備金の積立方法に関わらず一定となることを証明せよ。
ただし、プロフィット・マージンの計算の前提条件および解答に使用する記号は以下に示すものとする。(記号はすべて保険金額 1 あたりのものであり、解答も保険金額 1 あたりとして答えよ。)

〔前提条件〕

・保険年度単位のモデルを使用する。

・保障内容は死亡保険金額 1 の定期保険とする。

・保険料払込方法は年払とし、保険料払込期間は保険期間と同一とする。

・死亡は保険年度の中央で発生し、保険金は保険年度の中央で支払うものとする。

・解約は保険年度未で発生し、解約返戻金は保険年度未で支払うものとする。

・事業費は保険年度始に支出されるものとする。

〔記号〕

$\pi$:年払営業保険料

$E_t$:第$t$保険年度事業費率

$V_t$:第$t$保険年度末責任準備金率

$W_t$:第$t$保険年度末解約返戻金率

$q_t^d$:第$t$保険年度死亡率

$q_t^w$:第$t$保険年度解約失効率

$p_t = 1  - q_t^d - q_t^w$

$l_t$:第$t$保険年度末残存率($l_t=l_{t-1}\cdot p_t$, $l_0=1$)

$i$:資産運用利回り

$j$:危険割引率

$v$:$(1+j)^{-1}$

$n$:保険期間(年)

②投資回収年度(Break Even Year)および内部収益率(Internal Rate of Return)について、その定義ならびに活用方法について簡潔に説明せよ。
\answer{}

①
プロフィット・マージン($ProfitMargin$)は、

$ProfitMargin = \sum^n_{t=1} v^t l_{t-1}[(\pi-E_t)(1+i)-q_t^d(1+i)^{1/2}-q_t^wW_t+V_{t-1}(1+i)-p_tV_t)]\div \sum^n_{t=1}v^{t-1}l_{t-1}\pi$

ここで$i=j$ のとき、
$v^t l_{t-1}[(\pi-E_t)(1+i)-q_t^d(1+i)^{1/2}-q_t^wW_t+V_{t-1}(1+i)-p_tV_t)]
= v^{t-1}l_{t-1}(\pi-E_t) - v^{t-1/2}l_{t-1}q_t^d - v^tl_{t-1}q_t^w W_t + v^{t-1}l_{t-1}V_{t-1} - v^tl_tV_t$
より、

ProfitMarginの分子
$=v^0l_0(\pi-E_1)-v^{0.5}l_0q_1^d - v^1l_0q_1^wW_1 + v^0l_0V_0 - v^1l_1V_1$\\
$+v^1l_1(\pi-E_2)-v^{1.5}l_1q_2^d - v^2l_1q_2^wW_2 + v^1l_1V_1 - v^2l_2V_2$\\
$+v^2l_2(\pi-E_3)-v^{2.5}l_2q_3^d - v^3l_2q_3^wW_3 + v^2l_2V_2 - v^3l_3V_3$\\
$+\cdots$\\
$+v^{n-1}l_{n-1}(\pi-E_{n})-v^{n-0.5}l_{n-1}q_{n}^d - v^{n}l_{n-1}q_{n}^wW_{n} + v^{n-1}l_{n-1}V_{n-1} - v^{n}l_{n}V_{n}$
$-v^nl_nV_n$

$= \sum^n_{t=1}[v^{t-1}l_{t-1}(\pi-E_t)-v^{t-1/2}l_{t-1}q_t^d-v^tl_{t-1}q_t^wW_t] + v^0l_0V_0 - v^nl_nV_n$  = (※)

V積立方法によらず$V_0=V_n=0$ である(本問は定期保険)から、

(※) = $ \sum^n_{t=1}[v^{t-1}l_{t-1}(\pi-E_t)-v^{t-1/2}l_{t-1}q_t^d-v^tl_{t-1}q_t^wW_t]$

従って、ProfitMargin= $ \sum^n_{t=1}[v^{t-1}l_{t-1}(\pi-E_t)-v^{t-1/2}l_{t-1}q_t^d-v^tl_{t-1}q_t^wW_t] / \sum^n_{t=1}v^{t-1}l_{t-1}\pi$

となり、Profitmarginは、責任準備金の積立方法に関わらず一定となる。

②
○投資回収年度

・定義

利益の終価がプラスに転じ、その後もプラスであり続ける場合の、そのプラスに
転じる最初の保険(または事業)年度のこと(もしくは、ある期間までの利益の(契
約成立時点における)現在価値がプラスに転じ、その後もプラスであり続ける場合
の、最初の保険(または事業)年度のこと)

・活用方法

投資回収年度の指標を活用すると、プロフィット・マージンでは測定困難な、利
益の発生時期を判断することができる。

○内部収益率

・定義

初期投資がどのような利率で回収されるかを表す指標。将来の利益の現在価値(プ
ロフィット・マージン)をゼロとする割引率のことでもある。

・活用方法

株主的な観点からは投資判断の指標として、保険契約者の観点からは金融商品と
しての利回りのひとつの指標として活用される。

\problem{H22 生保1問題 2(3)}

商品毎収益検証およびアセット・シェアについて、次の①、②の各問に答えなさい。

① 商品毎収益検証において以下の前提条件により各パラメーターを定義するとき、次の(A)および(B)について各パラメーターを用いて表しなさい。

(A)第 $t$ 保険年度末の残存契約 1 件当たりの利益($profit_t$)

(B)第 $t$ 保険年度末の残存契約 1 件当たりのアセット・シェア($AS_t$ )

ここで、解約は保険年度末で発生するものとし、(B)アセット・シェア($AS_t$ )は漸化式で表しなさい。なお、解答にあたっては、保険金額 1 あたりとして示すこと。

〔前提条件〕

$\pi$:営業保険料、$E_t$ :第 $t$ 保険年度事業費率

$i_t$ :第 $t$ 保険年度資産運用利回り、$W_t$ :第 $t$ 保険年度解約返戻金率

$q^d_t$ :第 $t$ 保険年度死亡率、$q^w_t$ :第 $t$ 保険年度解約率

$p_t = 1 - q^d_t - q^w_t$

$V_t$ :第 $t$ 保険年度末の残存契約1件当たりの責任準備金

$AS_0=0$

② $NA_t$を$NA_t=AS_t-V_t$ と定義するとき、①の前提条件および結果を用いて次の等式を証明しなさい。

[等式]
$profit_t=NA_t-NA_{t-1}\frac{1+i_t}{p_t}$

\answer{}

①事業費は年始,死亡は年央とする

(A) $profit_t=\frac{(\pi-E_t)(1+i_t)-q^d_t(1+i_t)^{1/2}-W_tq_t^w+V_{t-1}i_t-(V_tp_t-V_{t-1})}{p_t}$

(B) $AS_t = \frac{(AS_{t-1}+\pi-E_t)(1+i_t)-q^d_t(1+i_t)^{1/2}-W_tq^w_t}{p_t}$

②
$profit_t=\frac{(\pi-E_t)(1+i_t)-q^d_t(1+i_t)^{1/2}-W_tq_t^w+V_{t-1}i_t-(V_tp_t-V_{t-1})}{p_t}$\\
$= \frac{(AS_{t-1}+\pi-E_t)(1+i_t)-q^d_t(1+i_t)^{1/2}-W_tq_t^w-V_tp_t-(AS_{t-1}-V_{t-1})(1+i_t)}{p_t}$\\
$= \frac{(AS_{t-1}+\pi-E_t)(1+i_t)-q^d_t(1+i_t)^{1/2}-W_tq_t^w-V_tp_t-NA_{t-1}(1+i_t)}{p_t}$\\
$= \frac{(AS_{t-1}+\pi-E_t)(1+i_t)-q^d_t(1+i_t)^{1/2}-W_tq_t^w}{p_t}-\frac{V_tp_t}{p_t}-\frac{NA_{t-1}(1+i_t)}{p_t}$\\
$= AS_t-V_t-NA_{t-1}\frac{(1+i_t)}{p_t} = NA_t - NA_{t-1}\frac{1+i_t}{p_t}$


\section{10.11 会社モデルへの応用}

\problem{H13 生保1問題 2(2)①}
①商品毎収益検証のモデルを選定する場合の論点について簡潔に説明せよ。

\answer{}
商品毎収益検証のモデルを選定する場合の論点は、以下の4点である。

1. キャッシュフローの発生のタイミングをどうとらえているか。

例えば、契約の発生時期の分布、死亡事故・解約等の発生時期の分布、
事業費の発生時期の分布、利息の付利の仕方などである。

2. 検証項目をどう選定するか。

例えば、消滅時配当を考慮するか否か、ソルベンシー・マージンを考慮
するか否か。また、事業年度単位で収益を検証する場合には、支払備金、
IBNR備金、危険準備金、追加責任準備金、およびその他諸積増を含ま
せることができる。

3. 検証目的と見合っているか。

商品ごとの収益性およびシナリオに対する収益性の感応度を、商品間で
比較・分析する場合、保険年度単位のモデルでも十分である。実際の決算
に与える影響まで検証する場合には、事業年度単位のモデルが必要になる
う。また、決算見込み、短期収支計画等の詳細な将来予想が必要とされる
場合は、月単位のモデルが不可欠となる場合もあろう。
理想的には商品毎収益検証のモデルをそのまま会社モデルとして使用で
きることが望ましい。

4. 実務的であるか。

例えば確率論的手法を用いて将来収支分析を行なう場合、数百本以上の
シナリオに対する利益の感応度を検証することになる。モデルの選定及び
モデル・ポイントの数によっては分析時間が実務的でなくなる場合がある。
必然的に簡便で効率的なモデルを用い、モデル・ポイントの数を減らして、
実務対応することとなる。
また、事業費と運用収益の仮定はよく変更が加えられるので、それらの
変更が容易なモデルとする必要がある場合がある。

\problem{2022 生保1問題 2(2), 29 生保1問題 2(1)}
商品毎収益検証のモデル・ポイントの設定において行うヴァリデーションについて、簡潔に説明しな
さい。なお、ヴァリデーションを行う主な項目を挙げるとともに、ヴァリデーションを行う際の留意点
についても触れること。
\problem{H24 生保1問題 2(3)}
商品毎収益検証に用いたモデルを利用して、会社全体の収益検証などを行う場合に、モデル・ポイン
トを使用することがある。モデル・ポイントについて簡潔に説明した上で、モデル・ポイントの設定上
の留意点を述べなさい。
\problem{H13 生保1問題 2(2)②}
商品毎収益検証について、モデル・ポイントの設定上の留意点をヴァリデーションも踏まえて述べ
よ。なお、モデル・ポイントおよびヴァリデーションについても簡潔に説明せよ。

\answer{}
商品毎収益検証に用いたモデルを利用して将来収支分析を行う場合、計算効率を上げることを目
的として、各契約を一定の要件のもと群団化し、群団を代表する契約を選定する方法が取られるこ
とがある。この代表契約のことをモデル・ポイントという。

モデル・ポイントの設定方法としては、モデル・ポイントを選定する単位(性別、年齢、保険期間、
保険料払込期間、基礎率等)で区分し、その「選定単位」の収支状況を代表していると考えられる
代表契約を選ぶ手法のほか、証券番号の下1桁が1のものだけを抽出するなどランダムかつ機械的
に設定する手法も考えられる。

モデル・ポイントは「生命保険会社の保険計理人の実務基準」におけるアセット・シェアの算定に
活用されているほか、MCEVなどにおける保証とオプションの時間価値算定の際には、確率論的
手法による計算を行うなど多数のシナリオでの計算を必要とすることから、その計算負荷を軽減す
るために活用される。

○ モデル・ポイントの設定上の留意すべき点

① 利用目的に応じた設定

モデル・ポイントの選定にあたっては、利用目的を考慮すべきである。求められる計算精
度は、モデル・ポイントの数に影響すると考えられ、利用目的は「選定単位」に影響する
可能性がある。例えば、金利上昇による動的解約の保証とオプションの時間価値算定にあ
たっては、解約率に影響のある単位で選定することが考えられる。一方で、莫大かつ多種
多様な契約を保有する会社全体の収支を試算する際には、効率的な計算を企図してランダ
ムに抽出することも考えられよう。

② 効率化を目指した選定

モデル・ポイントの選定は、ヴァリデーションを行いながらトライ・アンド・エラーで行
うこととなる。効率的な群団化とモデル・ポイントの選定を行なわないと、販売量の少な
かった商品の場合、選定されたモデル・ポイントの数が、該当する群団の保有契約の件数
より多くなってしまうことすらある。したがって販売実績やモデルを利用する目的に応じ
て、ケース・バイ・ケースで効率的に群団化することが重要である。

③ 恣意性の排除

モデル・ポイントの選定にあたっては、可能な限り恣意性を排除するために、客観的な選
定基準が明示的に示されていることが重要である。

④ ヴァリデーションの必要性

このモデル・ポイントが群団をいかに良く代表しているか評価することをヴァリデーションとい
う。具体的には、会社全体の各種統計数値とモデル・ポイントを利用して算出された数値の差また
は比を評価することをいう。

ヴァリデーションの結果、会社全体の各種統計数値とモデル・ポイントを利用して算出されたこれ
らの数値の差が小さければ良いモデル・ポイントといえるが、差を小さくするためにモデル・ポイ
ントの設定が過剰になって計算負荷を高めてしまうことがないよう、モデル・ポイントの設定を分
析の目的や影響度に応じて行い、ヴァリデーションにおいてもその状況を踏まえる必要がある。

モデル・ポイントで計算した結果が適切な水準となるようにヴァリデーションによる検証が必要不
可欠である。ヴァリデーションを行う項目は、保有件数、保険金額、事業年度末保険料積立金など
である。保険料収入、事業費、保険金等支払額、運用収益、年換算保険料および事業年度末保有契
約に対する解約返戻金などを追加すればより高い精度のヴァリデーションを行うことができる。
ヴァリデーションを行う項目は、損益に与える影響等、重要度に応じて選定することが必要であ
り、特に事業年度末保険料積立金の重要性は高い。

ある一時点における統計値に対してよい近似を与えるモデル・ポイントであっても、将来収支のよ
い近似を与えるとは限らないので、より精度の高いヴァリデーションを行うには、以下のような工夫をすることが重要である。

1.保険期間満了以外の死亡・解約等の脱退を見込まないで、保有契約全体の将来の数値およびモ
デル・ポイントによる将来の数値を計算し比較する方法

2.過去の統計数値について死亡の選択効果や解約率等の仮定からモデル・ポイントを利用して逆
算し、保有契約全体の過去の統計数値と比較する方法

確率論的手法による計算を行うなど、多数のシナリオでの計算を行う際は、ベストエスティメイト
のシナリオでのヴァリデーションのほか、ストレスシナリオでも統計数値が反映できるかも確かめ
ておく必要がある。

ヴァリデーションの結果、乖離があった場合には、「選定単位」の再検討などを行う必要がある。場
合によっては精度の低い契約群団のみをさらに細分化することも考えられる。
モデル・ポイントの選定は、ヴァリデーションを行いながらトライ・アンド・エラーで行うことと
なる。

将来収支分析は死亡の選択効果、解約、金利等の多くの前提に基づくものであるが、将来のこれら
の前提を確実に予想する理論は現在のところ存在しないため、シナリオを用いた将来収支分析を併
用することで最善を求めている。最善の分析を行うためには、将来収支分析の前提となるモデル・
ポイントの正確性をヴァリデーションにより確認することが必須である。

⑤ 実務スケジュールやコンピューターの処理速度など

実務的にどの程度に時間でアウトプットしなくてはいけないか、コンピューターの処理に
要する時間もモデル・ポイントの選定にあたり重要な視点である。モデル・ポイントの選
定にも時間を要することから、実務スケジュールや処理速度などを総合的に勘案する必要
がある。


⑥ 生命保険会社の保険計理人の実務基準の要請

商品毎収益検証を行なう中で、配当を検証する必要がある場合がある。
その際、保険業法に基づく、保険計理人による配当の公正・衡平の確認作業
との整合性を考慮する必要がある場合がある。
実務基準では、保険計理人は、消滅時配当を支払う保険契約に対して、
アセット・シェアに基づき、その公正・衡平を確認しなければならないと
されており、「選定単位は(a)区分経理の商品区分(b)保険事故の種類(c)契約
経過年度によって最低限区分しなければならない。さらに基礎書類上の保
険種類、販売経路、危険選択方法、性別、契約年齢、保険料払込方法、保
険金額、保険期間によって細かく区分できる」となっている。

また、実務基準の解説書によれば代表契約の選定基準の例として以下の
項目を挙げている。

(i)保険料および責任準備金の対保険金額比、費差損益および死差損益
発生状況が選定単位の平均から乖離しない契約

(ii)選定単位内で最も出率の高い契約

(iii)その他計理人が合理的かつ適正であると判断した契約

モデル・ポイントの選定にあたっては、これらと整合性を考慮する必要があ
る場合がある。


\section{10.12 保険計理人の実務基準への応用}
(過去問での出題なし)

\section{10.99 計算問題}
\problem{H12 生保1問題 1(1)}
次の表は、ある保険契約の新契約 1 件当たりの各年度の保険料収入、連用収益、保険金等支払額、事
業費、責任準備金繰入額および各年度の収益(profit)を表している。このとき、この契約の(A)プ
ロフィット・マージンおよび(B)内部収益率に最も近い値をそれぞれ次の①~⑧から選択せよ。ただ
し、保険期間は 3 年とし、保険料収入は年度始に、それ以外は年度末に発生するものとする。なお、現
在価値の算出のため定められた割引率を用いる必要がある場合には、年 2%を使用すること。

\begin{tabular}{|c|c|c|c|c|c|c|}
\hline
 保険年度&保険料収入& 運用収益&保険金等支払額&事業費&責任準備金繰入額&収益\\ \hline
 1年& 100&5 &35 &90 &45&▲65 \\ \hline
 2年& 90&5 &45 &25& 0&25 \\ \hline
 3年& 80&5 &55 &20&▲45&55 \\ \hline
\end{tabular}

①3% ②5% ③7% ④9% ⑤11% ⑥13% ⑦15% ⑧17%
\answer{}
A)Profit Margin

保険料の現価=100+90/1.02+80/1.022=265.12…

収益の現価=-65/1.02+25/1.022+55/1.023=12.13…

プロフィット・マージン=12.13/265.12=4.6%

(B)IRR
$-65\times(1+i)^2+25\times(1+i)+55=0$ を満たすiを求めると、
i=13.2%

\problem{H17 生保1問題 1(5)}
3 年満期の保険料年払の養老保険について、下記の前提の下でのプロフィット・マージンを計算する。
このとき、次の①~⑤の選択肢のうちプロフィット・マージンの値に最も近い値の記号を選択し、記号
で解答せよ。計算過程もあわせて記載せよ。

〔選択肢〕
①1.9% ②2.0% ③2.1%
④2.2% ⑤2.3%

【前提】

\begin{tabular}{|c|c|}
\hline 保険金額& 100万円\\ \hline
 年払営業保険料& 33万円\\ \hline
 死亡率&年0\%\\ \hline
 事業費支出(第1保険年度のみ)&保険金額1円に対して0.025円\\ \hline
 運用利回り&年3\%\\ \hline
 解約率&年0\%\\ \hline
 第t年度末責任準備金&t=1: 32.5万円, t=2: 65.6万円\\ \hline
キャッシュ・フロー発生のタイミング &年始:保険料収入,事業費支出, 年末:満期保険金支払 \\ \hline
\end{tabular}

計算にあたっては、税金、ソルベンシー・マージン、危険準備金積立、価格変動準備金積立等、表中に
記載のない項目は考慮せず、モデルは保険年度単位のモデルを使用することとする。なお、現価計算に
用いる割引率は年 5.5%とし、また、各年度の運用収益の計算は以下の計算式を使用するものとする。
運用収益=(前年度末責任準備金+営業保険料-事業費)×運用利回り

\answer{}
ず各保険年度の利益を求めると、

●第1保険年度
保険料:330,000円
事業費支出:1,000,000円×0.025=25,000円
V繰入:325,000円-0円=325,000円
運用収益:(0円+330,000円-25,000円)×0.03=9,150円
利益=330,000円-25,000円-325,000円+9,150円=-10,850円

○第2保険年度 
保険料:330,000円
事業費支出:0円V
繰入:656,000円-325,000円=331,000円
運用収益:(325,000円+330,000円-0円)×0.03=19,650円
利益=330,000円-0円-331,000円+19,650円=18,650円
●第3保険年度
保険料:330,000円
事業費支出:0円
V繰入:0円-656,000円=-656,000円
運用収益:(656,000円+330,000円-0円)×0.03=29,580円
(満期)保険金:1,000,000円
利益=330,000円-0円-(-656,000)+29,580円-1,000,000円=15,580円
次に、保険料および利益の新契約時における現価を求める。現価計算に用いる割引率
は、5.5%なので、
●保険料の現価
330,000円×〔1+1/1,055+1/$1.055^2$〕=939,285.5円
●利益の現価
-10,850/1,055+18,650/$1.055^2$+15,580/$1.055^3$=19,739.9円
よって、プロフィット・マージンは、
19,739.9円/939,285.5円=0.02102

\end{document} 