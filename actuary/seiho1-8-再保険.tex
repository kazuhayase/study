\documentclass[report,gutter=10mm,fore-edge=10mm,uplatex,dvipdfmx]{jlreq}

\usepackage{lmodern}
\usepackage{amssymb,amsmath}
\usepackage{mathtools}
\usepackage{ifxetex,ifluatex}
\usepackage{actuarialsymbol}
\usepackage[]{natbib}

%strike through 
%https://tex.stackexchange.com/questions/23711/strikethrough-text
%\usepackage[]{ulem}

\usepackage[normalem]{ulem}
\usepackage{enumerate}

% Tables
\usepackage{multirow}
\usepackage{tabularx}
%\usepackage{booktabs} % http://www.yamamo10.jp/yamamoto/comp/latex/make_doc/table/table.php

%Framedbox
%https://hakuoku.github.io/agakuTeX/tutorial/5_6framed/
\usepackage{framed}

%https://tgnx8810.wordpress.com/2014/11/29/latex%E3%81%A7%E8%A1%A8%E3%81%AE%E3%82%BB%E3%83%AB%E5%86%85%E6%94%B9%E8%A1%8C%E3%81%AFtabularx%E7%92%B0%E5%A2%83%E3%82%92%E4%BD%BF%E3%81%86%E3%81%A8%E6%A5%BD/
\usepackage{longtable}
\usepackage{booktabs}
\RequirePackage{plautopatch}

% maru suji ① etc.
\usepackage{tikz}
\newcommand{\cir}[1]{\tikz[baseline]{%
\node[anchor=base, draw, circle, inner sep=0, minimum width=1.2em]{#1};}}

%http://yamamo10.jp/yamamoto/comp/latex/make_doc/box/box.php
%枠付き文章
\usepackage{ascmac}
\usepackage{fancybox}

\usepackage{comment}

\begin{comment}

\ifnum0\ifxetex1\fi\ifluatex1\fi=0 % if pdftex
  \usepackage[T1]{fontenc}
  \usepackage[utf8]{inputenc}
  \usepackage{textcomp} % provide euro and other symbols
\else % if luatex or xetex
  \usepackage{unicode-math}
  \defaultfontfeatures{Scale=MatchLowercase}
  \defaultfontfeatures[\rmfamily]{Ligatures=TeX,Scale=1}
\fi
% Use upquote if available, for straight quotes in verbatim environments
\IfFileExists{upquote.sty}{\usepackage{upquote}}{}
\IfFileExists{microtype.sty}{% use microtype if available
  \usepackage[]{microtype}
  \UseMicrotypeSet[protrusion]{basicmath} % disable protrusion for tt fonts
}{}
\makeatletter
\@ifundefined{KOMAClassName}{% if non-KOMA class
  \IfFileExists{parskip.sty}{%
    \usepackage{parskip}
  }{% else
    \setlength{\parindent}{0pt}
    \setlength{\parskip}{6pt plus 2pt minus 1pt}}
}{% if KOMA class
  \KOMAoptions{parskip=half}}
\makeatother
\usepackage{xcolor}
\IfFileExists{xurl.sty}{\usepackage{xurl}}{} % add URL line breaks if available
\IfFileExists{bookmark.sty}{\usepackage{bookmark}}{\usepackage{hyperref}}
\hypersetup{
  hidelinks,
  pdfcreator={LaTeX via pandoc}}
\urlstyle{same} % disable monospaced font for URLs
\usepackage{longtable,booktabs}
% Correct order of tables after \paragraph or \subparagraph
\usepackage{etoolbox}
\makeatletter
\patchcmd\longtable{\par}{\if@noskipsec\mbox{}\fi\par}{}{}
\makeatother
% Allow footnotes in longtable head/foot
\IfFileExists{footnotehyper.sty}{\usepackage{footnotehyper}}{\usepackage{footnote}}

\end{comment}
%\makesavenoteenv{longtable}
\setlength{\emergencystretch}{3em} % prevent overfull lines
\providecommand{\tightlist}{%
  \setlength{\itemsep}{0pt}\setlength{\parskip}{0pt}}
\setcounter{secnumdepth}{-\maxdimen} % remove section numbering

\author{kazuyoshi}
\date{}





\begin{document}
\chapter{保険1第8章 再保険}
\section{8.1 はじめに}
\problem{2022 生保1問題 2(3)}
再保険の目的について、次の(ア)、(イ)の各問に答えなさい。(8点)

(ア)再保険会社が元受会社よりも低い死亡率・発生率を用いることで低廉な再保険料率を元受会
社に提供し、これによって元受会社は競争的な保険料率を顧客に提供できることがある。再保
険会社が低廉な再保険料率を提供できる理由・工夫を挙げて説明しなさい。(200字程度)(2点)

(イ)(ア)以外で、再保険を活用することで元受会社が競争的な保険料率を顧客に提供できる理
由・工夫を挙げて説明しなさい。また、(ア)を含めた、再保険を活用する際の留意点を挙げて
説明しなさい。(600字程度)(6点)

\answer{}
(ア)

再保険会社は、以下の理由・工夫により再保険料率を抑えることができる。
\begin{itemize}
\item[] 複数の元受会社から再保険契約を受再し、多くの契約を保有することでリスクを安定化させる。
\item[] 様々な再保険契約を受再し、リスク間の分散効果を働かせる。
\item[] リスク管理の側面から、再々保険を活用して、リスクを減らす。
\item[] リスクの種類によっては、元受会社に比べ、過去から多くの引受実績を有しており、引受査定力に差がある。
\item[] 条件体の評点について元受会社より競争力のある査定を行う。
\end{itemize}

(イ)

<元受会社が競争的な保険料率を顧客に提供できる理由>

\begin{itemize}
\item[] 再保険会社が保有する十分な実績データの活用により、予定発生率設定における安全割増を抑える
ことで、保険料率の競争力を上げられる。
\item[] 非伝統的な目的で再保険を活用し資本効率を向上させることによる資本コストの軽減を保険料水準
に反映させることで、保険料率の競争力を上げられる。例えば、いわゆるサープラスリリーフなど、
出再により契約初年度に収益を一時計上することで、一般に大きくなる契約初年度の事業費支出に
よる収益への影響を軽減させることができる。
\item[] 新設の元受会社など資本が十分でない場合、再保険会社の資本の活用により再保険料の水準を下げ
ることが考えられる。例えば、資本水準の違いから、よりリスクを取った資産運用を再保険会社が
行い、元受会社の資産運用利回りを上回ることが期待できる場合には、それをプライシングに反映
することで、保険料率の競争力を上げられる。
\item[] 総合収益の観点から保険料の水準を定める場合、保険引受リスク以外の様々なリスクシナリオ顕在
時の損失も出再によって抑えられるならば、その分、安全割増を小さく抑えることができ保険料の
競争力を上げられる。
\item[] 再保険の出再を条件として、競争的な査定基準を再保険会社から提供してもらうことで、販売量を
増やせると、固定費コストに対する1件当たり負担を減らせるので、保険料率の競争力を上げられ
る。
\end{itemize}

<再保険を活用する際の留意点>

\begin{itemize}
\item[] リスク管理の側面から、出再によって出再先の再保険会社の信用リスクを保有することに留意する。
そのため、出再先の財務状況や格付けの定期的なモニタリングの実施、担保設定や、出再先を複数
の再保険会社に分散することも考えられる。
\item[] 実質的にリスクを軽減していないケース(キャプティブなど)があることに留意する。
\item[] 再保険会社の引き受け制限に留意する。特に任意再保険ではそれまで出再していたリスクの受再を
断られる可能性がある。
\item[] 再保険契約の更新時などにおける再保険料率の値上げに留意する。
\item[] 再保険会社の破綻時や契約解除などの場合の代替案・計画(別の会社を探す等)を用意する必要性
に留意する。
\item[] 出再しない場合と比べて、収益性は再保険会社の収益等の分だけ減少することに留意する。そのた
め、再保険への依存度を将来にわたり下げていくことも考えられる。例えば、再保険への依存度を
下げた次期商品の開発や、非伝統的な目的の再保険活用を経営の安定化により解消していくなどが
考えられる。
\item[] 再保険に関する国内規制の変化にも留意する必要がある。規制変化によって出再の制限や、出再に
よる財務諸表の改善効果の低下を招く可能性がある。
\item[] 比例方式でない場合、出再割合が想定を超え、契約群団としての収益性が予想を下回る可能性に留
意する。逆に出再割合が想定を下回り、リスク移転が十分できない可能性にも留意する。
\item[] 再保険の方式によっては、事務が煩雑になり、事務体制の整備やそのためのコストに留意する。
\end{itemize}



\problem{H19 生保1問題 2(2)}
再保険の活用目的のうち、
「伝統的な目的」について簡潔に説明せよ。また、
「非伝統的な目的」を 2つ挙げ、それぞれについて項目を列挙せよ。
\answer{}

\noindent <伝統的な目的>\\
保険金支払の変動が収益および資本に与える影響を軽減すること、元受会社にとって
経験のないリスクが収益および資本に与える影響を軽減すること、および再保険料率
をもとに競争的な元受料率を提供することの3点が挙げられる。この目的のために移
軽されるリスクは死亡率・発生率などの保険引受リスクである。

\begin{itemize}
\item[]
 ①元受会社の保有限度額を超過する額を出再
\begin{itemize}
\item[]  元受会社は自己保有限度額を定めている。保険経営を安定させるために、自己保有
 限度額は、大数の法則が十分に機能し偶然の変動による収益および資本への影響を
 受容することができる水準に定められる。
\item[]
 一方、会社の最高引受保険金額は、競合
 上の観点から定められる。この差額を再保険に付すことによって保険金支払の変動
 が収益および資本に与える影響を軽減することが可能となる。
\item[]
 自己保有限度額は一律ではなく、標準体・条件体刑、さらには年齢群団別に定めら
 れることも多い。
\end{itemize}
\item[]
 ②巨大災害などに起因する保険支払の集積リスクを移転
\begin{itemize}
\item[]
 自己保有限度額の設定により均質でリスク発生が互いに独立した危険集団を形成し
 ていても、巨大災害が発生し広範囲に被害が生じた場合は、リスクの独立性は保た
 れず一時に多額の保険金支払が発生する可能性がある。
\item[]
 巨大災害が発生した場合の集積リスクを移転し、保険経営の安定を図るために、一
 定額以上の保険金支払が発生することを再保険事故と定義する再保険契約が活用さ
 れる。
\end{itemize}
\item[]
 ③経験のない保険引受リスクを移転
\begin{itemize}
\item[]
 元受会社は、市場の要請等により、保有したことのない保険引受リスクに晒される
 ことがある。新しい給付を提供する新商品を開発する場合やリスク細分化保険を発
 売する場合が典型的な事例である。このような場合、死亡率・発生率は、国民の統
 計または他の市場で活用されているものを必要に応じ修正して使用することが多い。
 このようにして作成された死亡率・発生率は、常にミスプライシングの可能性を包含
 している。このリスクの顕在化が収益および資本に与える影響を元受会社にとっ
 て受容できる範囲内に収まるようにリスクの一定割合を出再することが行われる。
\item[]
 また、再保険会社の情報をもとに新商品を開発した場合には、経験のない保険引受
 リスクを移転するという理由に加え、その再保険会社への報酬的な意味合いにより、
 出再することも多く見られる。
\end{itemize}
\item[]
 ④再保険料率をもとに競争的な元受料率を提供
\begin{itemize}
\item[]
一般的に、再保険会社の提供する再保険料率は、元受会社がプライシングで採用す
 る死亡率・発生率よりも低い。また、条件体の評点についても再保険会社の方が競
 争力のある査定を行うことが多い。元受会社では、保険引受リスクを保有する代わ
 りに低廉な再保険料を支払うことにより、顧客に競争的な保険料率を提供すること
 ができる。
\item[]
 我が国においても、任意再保険を活用することで、条件体契約で競争的な評点を提
 供することが広く行われている。
\end{itemize}
\end{itemize}

\noindent <非伝統的な目的>
\begin{itemize}
\item[]
①財務諸表の改善
\begin{itemize}
\item[]
新契約費の抑制、収益の安定、収益認識のタイミングの変更、
ソルベンシ]マージン比率の改善、ROE・IRR等の収益率の向上
\end{itemize}
\item[]
②特定のビジネスコールの達成
\begin{itemize}
\item[]
増資の抑制、課税所得の平準化、
格付けの引上げ・安定、円滑な買収・株式会社化
\end{itemize}
\end{itemize}

\problem{H11 生保1問題 1(4)【テキストに記載箇所なし】}
次の①~⑤を適当な語句で埋めよ。

再保険の有する機能の中で、最も重要なものは\wakumaru{①}
機能である。危険の種類を分類すると以下の
ようになる。

\begin{itemize}
\item[] \wakumaru{②}危険 …
\begin{itemize}
\item[ア.] 保険金額が高額による危険
\item[イ.]  被保険者の欠陥度合いが高度による危険
\end{itemize}

\item[] \wakumaru{③}危険 …
\begin{itemize}
\item[ア.]  地震、飛行機事故など、支払い保険金が\wakumaru{④}する危険
\item[イ.]  一定期間の\wakumaru{⑤}が変動する危険
\end{itemize}
\end{itemize}

\answer{}
\begin{itemize}
\item[ ①] …危険分散
\item[ ②] …個別
\item[ ③] …集団
\item[ ④] …集積
\item[ ⑤] …死亡率
\end{itemize}

\section{8.2 再保険の方式}
\subsection{8.2.1 比例式再保険:非比例式再保険}
\problem{H12 生保1問題 1(2)}
次の①~⑤を適当な語句または算式(文章中の記号を用いること)で埋めよ。

比例再保険方式で代表的なものとしては、サープラス方式と\wakumaru{①}
方式がある。通算保険金額を S、
保有額にかかる一定金額を R、一定割合をαとすれば、サープラス方式の出再額は
②と表されるのに対して、\wakumaru{①}
方式の出再額は\wakumaru{③}で表すことができる。一方、再保険金額が増減しても比例
的に再保険料が増減しない方式は
\wakumaru{④}
方式といい、その代表的なものの一つとして、「一事故」時、
ある契約集団の保険金支払総額が事前に定められている一定額を超えた場合に、再保険会社が元受会
社に対して超過額を支払う\wakumaru{⑤}
再保険が挙げられる。
\answer{}
\begin{itemize}
\item[①: ] クオーターシェア
\item[②: ] S−R
\item[③: ] (1一α)S
\item[④: ] 非比例再保険
\item[⑤: ] エキセス・オブ・ロス
\end{itemize}

\problem{}H27 生保1問題 1(6)
再保険の分類である「比例式再保険」、
「非比例式再保険」について、代表的な再保険種類を挙げなが
ら、簡潔に説明しなさい。
\answer{}
\begin{itemize}
\item[] 比例式・非比例式の分類は保険責任の分担方法から見た区分。
\item[] 再保険契約における保険金支払義務が元受契約の保険約款によって定義した保険金支払要件と
同一になっている再保険を比例式再保険という。
\item[] 元受契約と再保険契約の保険金支払要件が異なった形態で、元受契約群団の保険責任の一部を
移転する再保険を非比例式再保険という。
\item[] 比例式再保険の主なものは、危険保険料式再保険、共同保険式再保険、修正共同保険式再保険。
\item[] 非比例式再保険の主なものは、エクセスオブロス・カバー、ストップロス・カバー。
\end{itemize}

\subsection{8.2.2 自動(automatic)再保険:任意(facultative)再保険}
\problem{2021 生保1問題 1(5)}
自動再保険と任意再保険について、両者の違いが分かるように、簡潔に説明しなさい。

\answer{}
自動再保険と任意再保険は契約手続きから見た区分である。

自動再保険は、再保険協約で予め再保険の範囲・条件を定めておき、元受会社はそれに該当したリスクを義務的に出再し、再保険会社はそれを義務的に受再する再保険の形態をいう。

任意再保険では、1リスクごとに元受会社が出再するか判断し、再保険会社はそれを受再するか判断する権利を有している。

自動再保険は、出再時の事務が簡便なので大量のリスクを出再する場合に適している。任意再保険は、元受会社が再保険会社に条件体契約等の査定を依頼する場合に広く用いられている。



\subsection{8.2.3 超過額(surplus 又は excess)方式:比例(quata share)方式}
\problem{H3 生保1問題 1(5)}
再保険におけるサープラス方式について簡潔に説明せよ。
\answer{}
元受保険会社が一定金額Rを自社の請け負う危険金額として留保し、通
算保険金額SがRを超過する場合にその超過部分(S−R)を再保険会社に
出再する方法をサープラス方式という。この再保険により、元受け契約にお
ける少数の突出した高額保険金額の契約の保険金支払の危険を排除すること
ができ、被保険群団として保険金支払の安定性を高めることができる。

\problem{H8 生保1問題 1(3)}
比例式再保険の出再額を決定する代表的な 2 つの方式について簡潔に説明せよ。
\answer{}
比例式再保険の代表的な2つの方式として、サープラス方式とクォーターシ
ェア方式がある。両方式の相違点は元受保険金額を保有額と出再額に分ける方法の
違いである。

\begin{itemize}
\item[] サープラス方式では、一定金額Rを決めて、通算保険金額SがRを越えるまでは
 元受保険会社が保有し、Rを超過した部分(S−R)を再保険会社に出再する。
\item[] 一方、クォーターシェア方式では、通算保険金額Sの一定割合α・Sを元受保険会社
 が保有し、残りの(1一α)・Sを再保険会社に出再する。
\end{itemize}
\subsection{8.2.4 伝統的(traditional)再保険:非伝統的(nontraditional)再保険}
\problem{H26 生保1問題 1(3)}
再保険等について、次の①~⑤の空欄に当てはまる適切な語句を記入しなさい。

\begin{itemize}
\item[1.] 再保険はその目的から、伝統的再保険、非伝統的再保険に区分される。
 非伝統的再保険と同様の意味で
 \wakumaru{①}という用語が用いられることがある。また、新契約費の
 抑制を目的とする場合は
 \wakumaru{②}という用語が用いられることもある。

\item[2.] 
従来、保険によってリスク移転が行われていたものが、リスク・マネージメント技術の進
 歩により、保険以外の代替手段での対応も可能となってきている。このような保険の代替
 手段を総称して、 \wakumaru{③} という。代表的な \wakumaru{③} の例としては、
 \wakumaru{④}、 \wakumaru{⑤} などがある。
\end{itemize}
\answer{}
\begin{itemize}
\item[ ①: ] 財務再保険
\item[ ②: ] サープラス・リリーフ
\item[ ③: ] ART
\item[ ④⑤: ] 自家保険、キャプティブ、証券化、等
\end{itemize}

\problem{H28 生保1問題 1(2)}
大蔵省告示第 233 号(平成 10 年6月8日)について、次の①~⑤に適切な語句を記入しなさい。

\noindent \underline{第1条(財務再保険)}

保険業法施行規則(以下「規則」という。)第 71 条第2項に規定する金融庁長官が定める再保険は、
保険会社が保険契約を再保険に付した場合において、当該再保険に付した部分に係る
\wakumaru{①}を移転することを約し、
当該再保険に付した部分に係る保険契約から当該再保険に付した後に発生することが
見込まれる収益(以下…【中略】…という。)を
\wakumaru{②} (受再…【中略]…同じ。)としてあらかじめ収
受する再保険であって、次に掲げるすべての要件に該当するものをいう。

\begin{itemize}
\item[ 一 ]  【省略】
\item[ 二 ]  元受保険会社が受再保険会社から収受する\wakumaru{②}は\wakumaru{③}によるものであること。
\item[ 三 ]  【省略】
\item[ 四 ]  受再保険会社による\wakumaru{④}は、元受保険会社の再保険料の不払いによる場合を除き、
 できないものであること。
\item[]【省略】
\end{itemize}

\noindent \underline{第2条(種類)}

【第1項省略】

2.前項第1号に掲げる
\wakumaru{⑤}とは、受再保険会社が、元受保険契約(元受保険会社…【中略】…同じ。)
に係るリスクのうち、当該再保険に付された部分に係る\wakumaru{①}
を出再割合(受再保険会社…【中略】…同じ。)に応じて引き受け、
当該引き受けた部分に係る責任準備金(保険業法…【中略】…同じ。)
の積立て及び当該責任準備金に相当する額の資産の管理を行うものをいう。

【以下省略】

\answer{}
\begin{itemize}
\item[ ①: ] すべてのリスク
\item[ ②: ] 出再保険受入手数料
\item[ ③: ] 現金
\item[ ④: ] 一方的な解約
\item[ ⑤: ] 共同保険式再保険
\end{itemize}

\problem{H13 生保1問題 1(3)}
財務再保険の取り扱いは保険業法施行規則第 71 条第 2 項および平成 10 年大蔵省告示第 233 号に規
定されているが、これに関して次の①~⑤を適当な語句で埋めよ。

財務再保険とは、元受会社が保有する保険契約に係る\wakumaru{①}
のリスクを移転する再保険契約であって、
当該保険契約から将来にわたって発生することが見込まれる収益を
\wakumaru{②}としてあらかじめ一定額を収受するものをいい
(ただし、ある一定の条件を満たす必要がある)、元受会社は
\wakumaru{②}の金額を\wakumaru{③}として積み立てる。
財務再保険の種類は
\wakumaru{④}式再保険と
\wakumaru{⑤}式再保険の 2 種類がある。

\answer{}
\begin{itemize}
\item[ ①: ] すべて
\item[ ②: ] 出再保険受入手数料(初年度コミッション)
\item[ ③: ] 責任準備金
\item[ ④: ] 共同保険
\item[ ⑤: ] 修正共同保険
\end{itemize}

〔解説〕以下のような根拠規定によって問題文は組み立てられている。
\begin{itemize}
\item[] 保険業法71条第2項「…手数料を収受したときは、当該収受
 した金額を\underline{責任準備金}として積み立てなければならない。」
\item[] 平成10年大蔵省告示第233号第1条「…当該再保険に付した
 部分に係る\underline{すべて}のリスクを移転することを約し、当該再保険に付
 した部分に係る保険契約から当該再保険に付した後に発生するこ
 とが見込まれる収益を\underline{出再保険受入手数料}としてあらかじめ収受
 する再保険であって…」
\item[] 同第2条第1項「財務再保険の種類は、次に掲げる二種類とする。
 一 共同保険式再保険  二 修正共同保険式再保険」
\end{itemize}

\problem{H25 生保1問題 1(3)}
保険業法施行規則第71条および大蔵省告示第233号(平成10年6月8日)によって定義され
ている財務再保険について以下の①~③の空欄に当てはまる適切な語句を答え、下の表の各欄に
A~Cを埋めなさい。

\begin{tabular}{|l|c|c|c|}
 \hline
\multirow{2}{*}{}& \multicolumn{2}{c|}{将来収益に基づいた手数料を元受会社が受けるもの}
 &\multirow{2}{*}{その他}\\  \cline{2-3}
&告示第233号第1号の要件をすべて満たすもの&その他&\\ \hline
共同保険式  & & & \\ \hline
修正共同保険式 & & & \\ \hline
その他 & & & \\ \hline
\end{tabular}

\begin{itemize}
\item[ A:] 日本の法令上、「財務再保険」と定義されることにより、 \wakumaru{①}を\wakumaru{②}することができる。
\item[ B:] 日本の法令上、「財務再保険」と定義されないことにより、 \wakumaru{①}を\wakumaru{②}できず、 \wakumaru{③}として計上しなければならない。
\item[ C:] 日本の法令上、何の規制もない。
\end{itemize}
\answer{}

\begin{itemize}
\item[ ①: ] 手数料収入
\item[ ②: ] 収益認識
\item[ ③: ] 預かり金
\end{itemize}

\begin{tabular}{|l|c|c|c|}
 \hline
\multirow{2}{*}{}& \multicolumn{2}{c|}{将来収益に基づいた手数料を元受会社が受けるもの}
 &\multirow{2}{*}{その他}\\  \cline{2-3}
&告示第233号第1号の要件をすべて満たすもの&その他&\\ \hline
共同保険式  & A&B &C \\ \hline
修正共同保険式 &A &B & C \\ \hline
その他 & B & B & C\\ \hline
\end{tabular}



\problem{H21 生保1問題 1(1)}
次の再保険に関する説明文について、以下の①~⑦の空欄に適切な語句を解答用紙の所定の欄に記
入しなさい。

\begin{itemize}
\item[] 平成 10 年 大蔵省告示第 233 号においては、再保険契約のうち、元受会社が出再部分についてす
 べてのリスクを移転し、出再部分からの将来収益を出再保険受入手数料として収受するもので、か
 つ同告示第 1 条ならびに第 2 条の条件を同時に満たすものを\wakumaru{①}として定義している。
 この場合、手数料収入を\wakumaru{②}することができる。

\item[] 将来収益を出再保険受入手数料として収受するものであっても、同告示第 1 条ないし第 2 条の条
 件のうち少なくとも一つを満たさない再保険契約については、\wakumaru{①}とは認められない。
 この場合、手数料収入を\wakumaru{②}することができず、
 \wakumaru{③}として計上しなければならない。

\item[] この規制の趣旨は、元受会社が再保険を活用して収益の発生するタイミングを早期化することに
 伴う、将来の\wakumaru{④}にかかわるリスクをコントロールすることにある。

\item[] 集積リスクの規模や性質により、再保険会社だけでは引き受けられない場合、資本市場にリスクを
 移転するために、保険リスクを証券化したものを\wakumaru{⑤}と呼ぶ。

\item[] この証券化のために、\wakumaru{⑥}を設立し、この\wakumaru{⑥}が再保険会社となって、
 元受会社から保険リスク(Cat Cover)を引き受ける一方、\wakumaru{⑥}が資本市場において
 \wakumaru{⑤}を発行し、このCat Coverのリスクを\wakumaru{⑥}から投資家に移転する。
 \wakumaru{⑤}は、Cat Coverの保険事故が発生した場合、元受会社に再保険金を支払う原資として償還金
 や利息の一部または全部が支払われない一方で、そのリスクの対価として、\wakumaru{⑦}への上乗せが行
 われる。
\end{itemize}

\answer{}
\begin{itemize}
\item[ ①: ] 財務再保険
\item[ ②: ] 収益認識(「収益計上」も可)
\item[ ③: ] 預かり金
\item[ ④: ] 責任準備金(積み立て)不足
\item[ ⑤: ] カタストロフィーボンド(「Catボンド」も可)
\item[ ⑥: ] 特定目的会社(「SPC」も可)
\item[ ⑦: ] 利率
\end{itemize}
\subsection{8.2.5 経験割戻(experience refudまたは再保険配当)の有無}
(出題例なし?)

\section{8.3 再保険の種類}
\subsection{8.3.1 危険保険料式再保険(yearly renewable term, YRT)}
\problem{H1 生保1問題 1(2)}
危険保険料式再保険について簡潔に説明せよ。
\answer{}
元受保険金額S, 各t保険年度の責任準備金Vt、とするとき、各t保険
年度の再保険金額を危険保険金(S−Vt。)に比例して定める方法をいい、
出再額Aとするとき各t保険年度の再保険金額At、は
$$
A_t=A(S−V_t)/S
$$
で計算される。この方式の再保険は再保険金額が契約応当日ごとに更新される。

\problem{H29 生保1問題 1(5)}
危険保険料式再保険について、その活用目的も含めて簡潔に説明しなさい。
\answer{}
\begin{itemize}
\item[ ①: ] 元受契約の保険種類に関わらず再保険契約が自動更新一年定期保険となっている再保険である。つまり、再保険料率は一年定期保険の料率が用いられる。
\item[ ②: ] 再保険の付保の対象となるのは危険保険金額である。危険保険金額は、理論的には出再保険金額から消滅時の責任準備金を控除した金額であるが、単純化のために経過年数のみの関数による近似式を用いることが多い。
\item[ ③: ] 単純な仕組みであり、事務管理も容易である。
\item[ ④: ] 移転される保険責任は、死亡率等の発生率関係に限られる。
\item[ ⑤: ] 元受会社の保有限度額を超過する額を出再する場合や、条件体契約に対し再保険会社から競争的な評点の提供を受ける場合に活用される。
\item[ ⑥: ] 非伝統的な目的での活用方法としては、保険関係リスク相当額を減少させることによるソルベンシー・マージン比率の改善があげられる。
\end{itemize}

\subsection{8.3.2 共同保険式再保険(coinsurance)}
\problem{H4 生保1問題 2(1)}
共同保険式再保険について説明せよ。
\answer{}

共同保険式再保険は、危険保険料式などと同じ比例再保険の分野に属している。
危険保険料式が死亡保障のみを対象とするのに対し、共同保険式は死亡保障だけ
でなく満期保障もカバーする。再保険会社は解約返戻金の元受会社への支払など、
契約の一部を元受会社と同様に管理する。

再保険料は元受保険料から新契約事業費を控除した値であり、場合によっては
初年度再保険料がマイナスとなり、再保険会社が元受会社の事業費超過部分を
負担することもある。

新契約費を再保険会社が肩代わりしてくれるので設立間もない元受会社にはメ
リットがある。また、投資能力が低い元受会社は再保会社の運用力を利用できる。
一方、再保会社にとっては、投資的色彩が強いので元本の回収可能性を審査する
必要がある。
\problem{H7 生保1問題 1(2)}
危険保険料式再保険と共同保険式再保険の違いについて説明せよ。
\answer{}

危険保険料式再保険の対象となるのは死亡保障のみである。再保険金額は、元受
契約の危険保険金(保険金 - 責任準備金)に比例して定められ、契約応当日毎に更新
されるのが一般的である。

一方、共同保険式再保険は、元受会社が獲得した契約の一部を元受会社と全く同様に
維持管理する。したがって、満期保障や解約返戻金も支払対象となる。
再保険料は、元受保険料から新契約費を控除した値であり、設立間もな
い元受会社にとっては有効な再保険である。一方、再保険会社の保険収支は契約締結
後数年間は一般に赤字であるが、長期継続契約の割合が増大するにつれて黒字となる。
この意味で再保険会社にとっては、投資の一形態とも考えらるが、投資額を回収でき
ない危険もあるので、元受会社や契約条件の審査を危険保険料式再保険以上に慎重に
行う必要がある。

\problem{H23 生保1問題 2(1)}
A 社は設立後間もないが、設立当初から保障性商品の販売が好調であり、特にここ数年急速に保有
契約件数を伸ばしている。A 社は現在、保有契約について再保険割合(出再割合)80%の共同保険式再
保険で出再している。

①設立後間もない保険会社が共同保険式再保険を行う目的について、簡潔に説明しなさい。

②同社は今後の新契約について、再保険割合(出再割合)を 50%に下げることを検討している。
再保険割合(出再割合)を下げることのメリット・デメリットについて、簡潔に説明しなさい。
\answer{}
\noindent ①
設立後問もない保険会社については、新契約費支出が収支に大きな負担を与えるが、共同保険式
再保険を用いることにより、初年度に再保険料を超える出再保険受入手数料を受け取る形で新契約
費支出を緩和することが可能となる。この仕組みを通じて保険契約の収益認識のタイミングは一般
的に早まることとなり、早期に黒字転換を図ることができ、同時に資本の有効活用を通じてソルベ
ンシー・マージン比率をはじめ、ROE・IRRといった収益指標を改善することも期待できる。

資産運用リスクの面では、設立当初は運用する資産の規模が小さく、効率的な資産運用を行うこ
とが困難な場合が多いが、共同保険式再保険においては責任準備金の積立も出再先に移転すること
となり、再保険会社の資産運用力を活用することが可能となる。

\noindent ②
\noindent 【メリット】

契約件数の増加に伴い、ユニットコストの低下を通じて新契約費支出に伴う負担は軽減されてい
くため、収益認識のタイミングは自然に早期化していく。また、資産規模が一定大きくなれば、運
用効率も改善していく。このように共同保険式再保険の効用は徐々に薄れていくので、収支・財務
の状況をみながら再保険割合を引き下げることは望ましいといえる。

また、出再により保険引受に関する収益機会を出再割合分逸失することとなるが、定期保険の高
い死差益が期待できる場合、商品本来のもつ総合的な収益を享受するための再保険割合の引き下げ
は有効である。

さらに、再保険割合の引き下げにより再保険のカウンターパーティー・リスクも軽減される。

\noindent 【デメリット】

再保険割合の引き下げにより、その後の新契約における責任準備金の積立負担と対応する資産運
用リスクは増加する。全体として、ソルベンシー・マージン増加の効果は逓減する一方、保険引受
リスクを含めたリスクは増加するため、ソルベンシー・マージン比率は低下する。

再保険割合の見直しにあたっては出再保険受入手数料に係る初年度・次年度以降の事業費の設定
について効果的に再保険の目的が達成できるよう、再保険会社との交渉が必要となる。

また、共同保険式再保険については、保険金の支払いや解約時の返戻金の支払い、事業費支出に
関して出再割合に応じて元受会社と同じ管理が必要となるので事務管理は煩雑になる。再保険割合
を引き下げた場合においても本質的に事務面での煩雑さは残るうえ、契約時期ごとに出再割合を管
理する必要が生じるため、効率はさらに悪くなる。

\subsection{8.3.3 修正共同保険式再保険(Mod-Co, modified coinsurance)}
\problem{H22 生保1問題 1(1)}
次の再保険に関する説明文について、以下の①~⑤の空欄に当てはまる適切な語句を答えなさい。

\begin{itemize}
\item[] 修正共同保険式再保険の決済では、再保険会社と元受会社の間で四半期毎に現金を決済すること
 が一般的であるが、元受会社がソルベンシー・マージン比率の向上などの目的で再保険契約を行っ
 ており、かつ現金収受を必要としない場合、現金勘定の代わりに
 \wakumaru{①}勘定を建てることにより、 収益を計上する再保険形態がある。

\item[] この形態の再保険契約においては、その後
 \wakumaru{①}勘定が減少し、ゼロになった時点で再保険契約を解約する。この形態の再保険契約は
 \wakumaru{②}型修正共同保険式再保険と呼ばれる。
\item[] \wakumaru{②}型修正共同保険式再保険は、現金収受を伴う形態と比較すると、再保険コストは\wakumaru{③}。
\item[] また、この再保険の出再保険受入手数料に\wakumaru{④}が含まれない場合、元受会社はこの手数料を収
 益計上することができる。
\item[] 近年わが国で一般的に行われている非伝統的再保険は、上記の性質を全て満たし、一時払保険契約
 の\wakumaru{⑤}の補填のために活用されている。
\end{itemize}
\answer{}
\begin{itemize}
\item[ ①: ] 再保険貸
\item[ ②: ] 資産留保
\item[ ③: ] 低い
\item[ ④: ] 将来収益
\item[ ⑤: ] 新契約費支出
\end{itemize}

\problem{H20 生保1問題 1(1)}
次の文章は、修正共同保険式再保険について述べた文章である。以下の①~④について空欄に当ては
まる文言を解答用紙の所定の欄に記入しなさい。また、⑤と⑥について空欄に当てはまる文言を選択肢
から選択し、(a)または(b)の記号を解答用紙の所定の欄に記入しなさい。

修正共同保険式再保険において、元受会社と再保険会社の間で実際に四半期ごとに決済される金額
は次のとおりとなる。

再保険料 -( 保険金 + \wakumaru{①}  + 出再保険受入手数料 + \wakumaru{②} )

計算結果が正値であれば元受会社の支払、負値であれば再保険会社の支払となる。この結果、契約初年
度は、元受会社は再保険契約によって\wakumaru{③}を計上することができる。

通常、\wakumaru{②}は次の算式で求められる。

期末責任準備金 - 期始責任準備金 - 期始責任準備金 ×\wakumaru{④}

ここで、\wakumaru{④}は、投資リスクの移転の程度を定めるために重要な要素である。

最も基本的な\wakumaru{④}の設定は、
元受会社で保有された出再部分の責任準備金に対応する資産に対す
る当該期間の運用利回りを適用することである。
この場合、運用に関する収益・損失は、 \wakumaru{⑤}に帰属
する。これとは対照的に、責任準備金計算基礎に使用される予定利率を適用する場合もある。この場
合、運用に関する収益・損失は、\wakumaru{⑥}に帰属する。

[選択肢]⑤ (a)元受会社、(b)再保険会社    ⑥(a)元受会社、(b)再保険会社

\answer{}

\begin{itemize}
\item[ ①: ] 解約返戻金
\item[ ②: ] 修正共同保険準備金調整額
\item[ ③: ] 収益
\item[ ④: ] 運用利率(mod−co 利率)
\item[ ⑤: ] (b)再保険会社
\item[ ⑥: ] (a)元受会社
\end{itemize}

\problem{H9 生保1問題 2(4)}
再保険貸および再保険借を計上する意義について簡潔に説明せよ。

\answer{}
再保険契約において、事業年度末に再保険取引の債権債務が確定してい
て未決済の時には、債権を再保険貸、債務を再保険借の勘定科目で貸借対
照表上に計上する.これにより期間損益を把握することができる。
元受会社は未決済の再保険料は再保険借に計上し、払戻再保険料、再保険金、再
保険配当金の再保険収入の未決済額は再保険貸で計上する。再保険会社は
これと逆の計上処理を行う

\problem{2019 生保1問題 2(2)}
「共同保険式再保険」について、その仕組みの概要とこれが広く活用されない理由を簡潔に説明しな
さい。また、「修正共同保険式再保険」について、
「共同保険式再保険」からの修正点を簡潔に説明しなさい。さらに、
「資産留保型修正共同保険式再保険」について、資産留保型と呼ばれるその特徴を簡潔
に説明しなさい。なお、解答にあたって、これらの活用における代表的な目的・工夫・留意点について
は触れなくてよい。
\answer{}
共同保険式再保険とは、再保険会社が元受会社の収入した営業保険料のうち出再割合に応じた額
を再保険料として収受し、保険金や解約返戻金などの全ての支出に対して出再割合に応じた額を
元受会社に支払うという、再保険会社が個々の出再契約に関し元受契約の契約条件と同一の内容
で保険責任を引き受ける再保険の形態である。

従って、再保険会社には、死亡率等の保険リスクのみならず、投資リスク・解約失効リスクや事
業費支出にかかるリスク等、全ての保険責任が元受会社から移転されることになる。一方、例え
ば投資リスクについて、資産運用に係る技術や方針は保険会社によって異なるなど、元受会社と
全く同質の全てのリスクを引き受けることを好まない傾向にある場合が多いことや、責任準備金
の移転によって総資産の増加が抑制されることを元受会社が選択しない傾向、移転するリスク要
素の多さから事務管理が煩雑になることなどが、共同保険式再保険が広く活用されない理由であ
る。

この点を改良した修正共同保険式再保険は、元受会社と再保険会社との精算において「修正共同
保険準備金調整額」という調整科目を追加する点で共同保険式再保険と異なっている。「期末責
任準備金-期始責任準備金-期始責任準備金×mod-co 利率」で定義される修正共同保険準備金調
整額において、mod-co 利率として責任準備金計算基礎に使用される予定利率を適用することで再
保険会社は投資リスクを負わなくて済む。

さらに、両者との精算において契約初期に元受会社が計上する収益を再保険貸で建て、その後の
毎期の精算に応じて再保険貸を減少させ、ゼロになると解約になる契約形態を資産留保型修正共
同保険式再保険という。現金を計上せず再保険貸を建てることで実際の資産を移転させない特徴
が資産留保型と呼ばれる所以である。

\subsection{8.3.4 エクセスオブロス・カバー(ELC; excess of loss cover)}

\problem{2020 生保1問題 1(5)}
生命再保険におけるエクセスオブロス・カバーについて、次の①~⑤に適切な語句を記入しなさ
い。

エクセスオブロス・カバー(ELC)は\wakumaru{①}
式再保険の代表的な形態である。

ELCは、再保険事故発生の頻度と発生した場合の金額規模の観点から、カタストロフ・カバー(Cat
Cover)と\wakumaru{②}の2種類に分けられる。

このうち、生命再保険で\wakumaru{③}
リスクの移転のために活用されるのは Cat Cover である。

Cat Cover は災害を事由とした1事故の保険金支払総額のうち、元受会社の負担額とする一定金額
である\wakumaru{④}
額を超えた場合に、その超過額について再保険会社の支払いとなる。一方、再保険会社
の負担額は無限ではなく、この一定の上限額を\wakumaru{⑤}額という。

\answer{}
\begin{itemize}
\item[ ①: ]  非比例 
\item[ ②: ]  ワーキング・カバー 
\item[ ③: ]  集積 
\item[ ④: ]  自己保有 
\item[ ⑤: ]  填補限度
\end{itemize}

\problem{H24 生保1問題 1(2)}
集積リスクの移転を目的とする生命再保険について簡潔に説明しなさい。
\answer{}
\begin{itemize}
\item[] 集積リスクとは、巨大災害等により一時に多額の保険金支払が発生するリスクであり、リス
 ク分散等の通常の手法では管理が困難であることから、再保険によるリスク移転が用いられ
 ることがある。その際は、事務の簡便さやリスク移転効果の高さから、非比例式再保険の一
 種であるカタストロフ・エクセスオブロス・カバー(Cat Cover)が用いられることが多い。
\item[] Cat Cover は死亡率等の発生率のみを移転し、事由は災害に限定される。再保険料率は保有保
 険金額に応じて定められ、保険期間は1年である。引受可能な再保険者が限られていること
 から、再保険料率の高騰や引受再保険者の財務状況には留意が必要である。また、ソルベン
 シー・マージン基準のリスク計算には影響を与えない。
\end{itemize}

\subsection{8.3.5 ストップロス・カバー(stop loss cover)}

\problem{H15 生保1問題 2}
ストップ・ロス再保険について、次の設問に解答せよ。

以下の条件の下で、ある集団の契約をストップ・ロス再保険に出再したい。この再保険の年払純保険
料を計算し、下記の選択肢のなかから最も近い値の記号(①~⑤)を選んで解答せよ。また、あわせて
計算過程も明記せよ。

なお、この集団の各年の死亡者数の分布はポワソン分布で近似できるものとする。ポワソン分布は、
離散確率変数$X$について、$P\{X=k\}=e^{-\lambda}\lambda^k/k!$(ここでは $k$ は 0 以上の整数)と表される。ここで、$\lambda$は平均値を示し、$e=2.7183$ とする。

また、元受の保険料は付加保険料を考慮せず純保険料として計算せよ。
\begin{tabular}{ll}
被保険者数:&1,000 人\\
保険金額:&一律 100 万円\\
年払純保険料(元受):&一律 1,000 円\\
被保険者の死亡率:&0.00100\\
エクセスポイント(※1) :&元受総収入保険料の 1.0 倍\\
支払限度(※2):&元受総収入保険料の 2.0 倍\\
\end{tabular}

※1 ストップ・ロス再保険は、ある「一定額」を超過する場合にその超過額が再保険金の支払対象
となる再保険制度である。エクセスポイントとは、その「一定額」を示す。

※2 再保険会社の支払責任額の上限。

〔選択肢〕①22 万円 ②26 万円 ③30 万円 ④34 万円 ⑤38 万円

\answer{}

\noindent 〔計算過程〕
事故発生件数と再保険金の支払額を示すと次のようになる。(ΣP:元受総収入保険料、ΣS:元受の支払保険金額)

\begin{tabular}{cccc}
 件数& ΣP& ΣS&支払額\\
 0& 100万円& 0& 0\\
 1& 100万円& 100万円& 0\\
 2& 100万円& 200万円& 100万円\\
 3& 100万円& 300万円& 200万円\\
 4件以上& 100万円& --- & 200万円\\
\end{tabular}

再保険金の支払額に確率を乗じ和を取る(平均値を求める)ことによって本再保険の保険
料が算出される。次のようにして確率を求める。

ポワソン分布であることから、その確率は
$P\{X=k\}=e^{-\lambda}\lambda^k/k!$ で与えられる。

発生件数が0件の場合と1件の場合では確率の計算をする必要はないので、2件以上の
場合について計算することにする。λ=1,000×1.0‰=1であるので、
2件発生する場合は、$e^{-1}\times 1^2/2!$である。
3件以上発生する場合は、0件〜2件まで発生する確率を全確率から控除すればよいので、
$1−e^{-1}\times(1+1+1^2/2!)$ となる。

従って、保険料は、

$1,000,000\times e^{-1} \times 1^2/2! + 2,000,000\times \{1-e^{-1}\times(1+1+1^2/2!)\} =344,544$

④が解答。

〔解説〕この問題はまず支払額を求める部分が要となる。その後、200万円以上の支払額を
持つ場合の発生確率を求めることが次のステップである。その部分が、
$1−e^{-1}\times(1+1+1^2/2!)$ 
となっている。

\subsection{ELC と stop loss cover}

\problem{H10 生保1問題 1(4)}
次の①~⑤を適当な語句で埋めよ。

非比例再保険方式の代表的なものとして、一定期間における保険金支払総額が\wakumaru{①}
総額の\wakumaru{②}を超過した場合に再保険会社がその超過分を支払う
\wakumaru{③}再保険と、
「\wakumaru{④}」時の保険金支払総額が事前に定められている金額を超過した場合に
再保険会社がその超過額を支払う
\wakumaru{⑤}再保険がある。
\answer{}
\begin{itemize}
\item[ ①… ] 保険料収入
\item[ ②… ] 一定割合
\item[ ③… ] ストップ・ロス
\item[ ④… ] 一事故
\item[ ⑤… ] エキセス・オブ・ロス
\end{itemize}

\problem{H2 生保1問題 2(1)}
非比例再保険方式の代表的なものに「ストップ・ロス(Stop Loss)再保険」と「エキセス・オブ・
ロス(Excess of Loss)再保険」があるが、その各々の特徴と違いを簡潔に説明せよ。
\answer{}
ストップ・ロス再保険とは、元受契約集団の一定期間の保険金支払総額ΣSが
保険料収入総額などにより決定される一定金額αを超過した場合、その超過分
(ΣS一α)を再保険会社が、再保険金として支払うものである。なお、再保険
会社の支払責任額に関し、支払限度が設定されているケースが多い。

当該再保険は、経験死亡率が不安定な設立間もない生保会社にとって効果的で
ある。大規模会社については、団体保険に関し、個々の団体ごとに再保険を付す
ることにより団体保険の業績安定を図ることができる。また、新種商品などで未
経験の危険を対象とする事業を開始する場合の事業安定化にも効果的である。

日本では当該再保険は殆ど利用されていないが、欧米では、団体保険、医療保
険の分野で大いに活用されている。

エキセス・オブ・ロス再保険とは、「一事故」による保険金支払総額ΣSが、
ある一定金額Dを超過した場合に、超過額(ΣS−D)を再保険会社が元受会社
に支払うものである。この場合も支払限度を設定するのが一般的である。

保険金支払総額ΣSについては、ある定められた期間内に発生し、かつ、その
事故に困果関係がある全ての支払保険金を集計して決定される。当該再保険は、
巨大リスクや海外旅行保険等集積リスクがある場合に有効である。

\problem{H14 生保1問題 1(6)}
ストップ・ロス再保険とエキセス・オブ・ロス再保険について簡潔に説明せよ。
\answer{}
非比例再保険方式の代表的なものとして、①ストップ・ロス再保険と②エキセス・オブ・
ロス再保険の二つの再保険方式がある。

\noindent ①ストップ・ロス再保険

ある元受契約集団の一定期間の保険金支払総額ΣSが、保険料収入総額ΣPの一定割合α
を超過した場合、再保険金杜がその超過分(ΣS一ΣP×α)を再保険として支払う。

ただし、再保険会社の支払責任は無限ではなく支払限度が設定されている。この再保険は
設立間もない生命保険会社で死亡率が不安定な場合や団体保険分野で個々の団体ごとに
再保険に付す場合に有効である。

\noindent ②エキセス・オブ・ロス再保険

「一事故」のとき、ある契約集団の保険金支払総額ΣSが、事前に定められている金額D
を超過したときに、その超過額(ΣS−D)を再保険会社が元受会社に支払う再保険であ
り、この場合も支払限度を設定するのが一般的である。また、「一事故」については、事
故の種類ごとに時間が定められており、その時間内に発生した支払保険金が再保険の対象
となる。この再保険は海外旅行保険などのように集積危険のある場合に有効である。

\section{8.4 再保険と類似の機能}
\problem{H26 生保1問題 1(3)}
(再掲)
\problem{H21 生保1問題 1(1)後半}
(再掲)

\section{その他}
\problem{H18 生保1問題 1(4)}
再保険における「サープラス方式」と「ストップ・ロス再保険」について説明し、両者の違いについ
て述べよ。
\answer{}
サープラス方式とは、元受会社が契約を出再する場合、一定金額Rを決め、通算保
険金額SがRを超えるまでは元受会社が全額保有し、超過した場合には(S−R)を再
保険会社に出再する方式である。

ストップ・ロス再保険は、ある元受集団の一定期間の保険金支払総額ΣSが、保険料
収入総額ΣPの一定割合αを超過した場合、再保険会社がその超過分(ΣS - α・ΣP)
を再保険として支払う。ただし、再保険会社の支払責任額は無限ではなく、支払限度
が設定されている。本再保険方式は、設立間もない生命会杜で契約件数が少なく、死
亡率が不安定である場合に効果的である。

サープラス方式は、再保険料が再保険金額に比例する比例再保険方式である一方、
ストップ・ロス再保険は非比例再保険方式をとっている。

また、サープラス方式は個別危険が対象で、個々の生命保険契約に対し保険金額が
高額による危険や欠陥度合が高度による危険が対象となる。一方、ストップ・ロス再
保険は集団危険が対象で、契約集団に対し、地震・飛行機事故など支払い保険金が集
積する危険や一定期間の死亡率が変動する危険が対象となる。

\problem{H17 生保1問題 2(1)}
設立後間もない生命保険会社において活用しうる再保険の方式を 2 つ挙げ、それぞれの仕組みと活
用目的について簡潔に説明せよ。

\answer{}
\begin{itemize}
\item[1)] 共同保険式再保険
\begin{itemize}
\item[①] 仕組み\\
再保険割合に応じて、死亡保障だけでなく、満期保障もカバーし、解約返戻金も支払う
等、元受会社が獲得した契約の一部を元受会社と全く同様に維持管理する再保険である。
再保険料は元受保険料から新契約費を控除した値となり、場合によっては、初年度再保険
料はマイナスとなり、再保険料は受け取れず、その上再保険会社が元受会社の事業費超過
分を負担するケースもある。
\item[②] 活用目的\\
設立間もない保険会社においては、新契約費の負担の収支への影響が大きいが、共同保
険式再保険では再保険会社が新契約費を肩代わりしてくれるので、このような財政的圧迫
を避けることができる。また、再保険料中には、危険保険料部分の他に貯蓄保険料部分も
含まれており、再保険会社独自の資産運用により責任準備金を積み立てることができる。
それゆえ、設立間もない会社で効率的な資産運用ができない場合に、再保険会社の資産運
用力を利用することができる。
\end{itemize}
\item[2)] ストップ・ロス再保険
\begin{itemize}
\item[①] 仕組み\\
ある元受契約集団の一定期間、例えば一年間の保険金支払総額ΣSが、保険料収入総額
ΣPの一定割合αを超過した場合、再保険会社がその超過分(ΣS一ΣP・α)を再保険と
して支払う。ただし、再保険会社の支払責任額には支払限度が設定されている。
\item[②] 活用目的\\
設立間もない保険会社においては、契約件数が少なく、死亡率が不安定な状態にある。
ストップ・ロス再保険の活用により、死亡率の変動による保険金の支払額を一定範囲内に
止め、保有契約全体の収支を安定させることができる。
\end{itemize}
\item[3)] 財務再保険
\begin{itemize}
\item[①] 仕組み\\
再保険に付した部分に係るすべてのリスクを移転するかわりに、再保険に付した部分に
係る将来利益を出再保険受入手数料としてあらかじめ収受する再保険である。財務再保険
の種類は、共同保険式再保険、修正共同保険式再保険が挙げられる。
\item[②] 活用目的\\
財務再保険は、再保険契約から生じる将来の利益を担保とした資金調達的な要素が強い。
設立間もない保険会社においては、資本基盤が脆弱な状態にあるため、資本力の強化を目
的として利用される。
\end{itemize}
\end{itemize}

\problem{H16 生保1問題 1(2)}
再保険に関する次の①~⑤について、正しいものには○、誤りのあるものには×を付けよ。

\begin{itemize}
\item[ ①: ] 個々の生命保険契約ごとに危険が発生する個別危険には非比例再保険が、契約集団ごとに危険が発生する集団危険には比例再保険が適用されている。
\item[ ②: ] 再保険契約の形態で、元受会社の出再する契約範囲には義務出再と任意出再とあるが、再保険会社の引受形態は再保険会社独自の査定基準による任意引受のみである。
\item[ ③: ] 元受会社は未決済の再保険料があれば再保険借を計上し、未決済の再保険金や払戻再保険料は再保険貸で計上する。そして、再保険会社は逆の再保険貸借の計上処理を行う。
\item[ ④: ] 危険保険料式再保険の再保険金額は、元受保険金額を$S$、$t$年度の責任準備金を$V_t$ 、出再額を$A$としたとき、$A(S-V_t)/S$で計算される。
\item[ ⑤: ] ストップ・ロス再保険は、海外旅行保険等の集積危険において元受会社の支払責任額を設定し、
 一事故による保険金支払総額が支払責任額を上回ったとき、その超過額を再保険会社が元受会社に
 支払う仕組みであるが、再保険会社も支払限度を設定するのが一般的である。
\end{itemize}
\answer{}
①×②×③◯④◯⑤×

①比例と非比例が逆\\
個々の生命保険契約ごとに危険が発生する個別危険には比例再保険が、契約集団ごとに危険が発生する集団危険には非比例再保険が適用されている。

②(義務出再ではなく) 自動再保険と任意再保険。比例式再保険のみにある概念。
また、再保険会社は、自動再保険の場合は義務的に受再しなければいけない。\\
なお、任意再保険の派生的な形態で、元受会社は出再するか判断する権利を留保し、再保険会社はすべてを受再するように定めたものもある。(facultative - obligatory reinsurance 又は単に open cover)

⑤ストップ・ロスではなく、ELC (excess of loss over)エクセスオブロス・カバーのうちのcat coverのこと。また、支払責任額は自己保有額と言う。再保険会社の支払限度は補填限度額という。\\
ストップ・ロスは一定期間(1年間等)の保険金支払率が一定αを超えた場合、超過分をカバーするもの。


\end{document}