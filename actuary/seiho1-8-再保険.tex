\documentclass[report,gutter=10mm,fore-edge=10mm,uplatex,dvipdfmx]{jlreq}

\usepackage{lmodern}
\usepackage{amssymb,amsmath}
\usepackage{mathtools}
\usepackage{ifxetex,ifluatex}
\usepackage{actuarialsymbol}
\usepackage[]{natbib}

%strike through 
%https://tex.stackexchange.com/questions/23711/strikethrough-text
%\usepackage[]{ulem}

\usepackage[normalem]{ulem}
\usepackage{enumerate}

% Tables
\usepackage{multirow}
\usepackage{tabularx}
%\usepackage{booktabs} % http://www.yamamo10.jp/yamamoto/comp/latex/make_doc/table/table.php

%Framedbox
%https://hakuoku.github.io/agakuTeX/tutorial/5_6framed/
\usepackage{framed}

%https://tgnx8810.wordpress.com/2014/11/29/latex%E3%81%A7%E8%A1%A8%E3%81%AE%E3%82%BB%E3%83%AB%E5%86%85%E6%94%B9%E8%A1%8C%E3%81%AFtabularx%E7%92%B0%E5%A2%83%E3%82%92%E4%BD%BF%E3%81%86%E3%81%A8%E6%A5%BD/
\usepackage{longtable}
\usepackage{booktabs}
\RequirePackage{plautopatch}

% maru suji ① etc.
\usepackage{tikz}
\newcommand{\cir}[1]{\tikz[baseline]{%
\node[anchor=base, draw, circle, inner sep=0, minimum width=1.2em]{#1};}}

%http://yamamo10.jp/yamamoto/comp/latex/make_doc/box/box.php
%枠付き文章
\usepackage{ascmac}
\usepackage{fancybox}

\usepackage{comment}

\begin{comment}

\ifnum0\ifxetex1\fi\ifluatex1\fi=0 % if pdftex
  \usepackage[T1]{fontenc}
  \usepackage[utf8]{inputenc}
  \usepackage{textcomp} % provide euro and other symbols
\else % if luatex or xetex
  \usepackage{unicode-math}
  \defaultfontfeatures{Scale=MatchLowercase}
  \defaultfontfeatures[\rmfamily]{Ligatures=TeX,Scale=1}
\fi
% Use upquote if available, for straight quotes in verbatim environments
\IfFileExists{upquote.sty}{\usepackage{upquote}}{}
\IfFileExists{microtype.sty}{% use microtype if available
  \usepackage[]{microtype}
  \UseMicrotypeSet[protrusion]{basicmath} % disable protrusion for tt fonts
}{}
\makeatletter
\@ifundefined{KOMAClassName}{% if non-KOMA class
  \IfFileExists{parskip.sty}{%
    \usepackage{parskip}
  }{% else
    \setlength{\parindent}{0pt}
    \setlength{\parskip}{6pt plus 2pt minus 1pt}}
}{% if KOMA class
  \KOMAoptions{parskip=half}}
\makeatother
\usepackage{xcolor}
\IfFileExists{xurl.sty}{\usepackage{xurl}}{} % add URL line breaks if available
\IfFileExists{bookmark.sty}{\usepackage{bookmark}}{\usepackage{hyperref}}
\hypersetup{
  hidelinks,
  pdfcreator={LaTeX via pandoc}}
\urlstyle{same} % disable monospaced font for URLs
\usepackage{longtable,booktabs}
% Correct order of tables after \paragraph or \subparagraph
\usepackage{etoolbox}
\makeatletter
\patchcmd\longtable{\par}{\if@noskipsec\mbox{}\fi\par}{}{}
\makeatother
% Allow footnotes in longtable head/foot
\IfFileExists{footnotehyper.sty}{\usepackage{footnotehyper}}{\usepackage{footnote}}

\end{comment}
%\makesavenoteenv{longtable}
\setlength{\emergencystretch}{3em} % prevent overfull lines
\providecommand{\tightlist}{%
  \setlength{\itemsep}{0pt}\setlength{\parskip}{0pt}}
\setcounter{secnumdepth}{-\maxdimen} % remove section numbering

\author{kazuyoshi}
\date{}





\begin{document}
\chapter{保険1第8章 再保険}
\section{8.1 はじめに}
\problem{H19 生保1問題 2(2)}
再保険の活用目的のうち、
「伝統的な目的」について簡潔に説明せよ。また、
「非伝統的な目的」を 2つ挙げ、それぞれについて項目を列挙せよ。
\answer{}

\noindent <伝統的な目的>\\
保険金支払の変動が収益および資本に与える影響を軽減すること、元受会社にとって
経験のないリスクが収益および資本に与える影響を軽減すること、および再保険料率
をもとに競争的な元受料率を提供することの3点が挙げられる。この目的のために移
軽されるリスクは死亡率・発生率などの保険引受リスクである。

\begin{itemize}
\item[]
 ①元受会社の保有限度額を超過する額を出再
\begin{itemize}
\item[]  元受会社は自己保有限度額を定めている。保険経営を安定させるために、自己保有
 限度額は、大数の法則が十分に機能し偶然の変動による収益および資本への影響を
 受容することができる水準に定められる。
\item[]
 一方、会社の最高引受保険金額は、競合
 上の観点から定められる。この差額を再保険に付すことによって保険金支払の変動
 が収益および資本に与える影響を軽減することが可能となる。
\item[]
 自己保有限度額は一律ではなく、標準体・条件体刑、さらには年齢群団別に定めら
 れることも多い。
\end{itemize}
\item[]
 ②巨大災害などに起因する保険支払の集積リスクを移転
\begin{itemize}
\item[]
 自己保有限度額の設定により均質でリスク発生が互いに独立した危険集団を形成し
 ていても、巨大災害が発生し広範囲に被害が生じた場合は、リスクの独立性は保た
 れず一時に多額の保険金支払が発生する可能性がある。
\item[]
 巨大災害が発生した場合の集積リスクを移転し、保険経営の安定を図るために、一
 定額以上の保険金支払が発生することを再保険事故と定義する再保険契約が活用さ
 れる。
\end{itemize}
\item[]
 ③経験のない保険引受リスクを移転
\begin{itemize}
\item[]
 元受会社は、市場の要請等により、保有したことのない保険引受リスクに晒される
 ことがある。新しい給付を提供する新商品を開発する場合やリスク細分化保険を発
 売する場合が典型的な事例である。このような場合、死亡率・発生率は、国民の統
 計または他の市場で活用されているものを必要に応じ修正して使用することが多い。
 このようにして作成された死亡率・発生率は、常にミスプライシングの可能性を包含
 している。このリスクの顕在化が収益および資本に与える影響を元受会社にとっ
 て受容できる範囲内に収まるようにリスクの一定割合を出再することが行われる。
\item[]
 また、再保険会社の情報をもとに新商品を開発した場合には、経験のない保険引受
 リスクを移転するという理由に加え、その再保険会社への報酬的な意味合いにより、
 出再することも多く見られる。
\end{itemize}
\item[]
 ④再保険料率をもとに競争的な元受料率を提供
\begin{itemize}
\item[]
一般的に、再保険会社の提供する再保険料率は、元受会社がプライシングで採用す
 る死亡率・発生率よりも低い。また、条件体の評点についても再保険会社の方が競
 争力のある査定を行うことが多い。元受会社では、保険引受リスクを保有する代わ
 りに低廉な再保険料を支払うことにより、顧客に競争的な保険料率を提供すること
 ができる。
\item[]
 我が国においても、任意再保険を活用することで、条件体契約で競争的な評点を提
 供することが広く行われている。
\end{itemize}
\end{itemize}

\noindent <非伝統的な目的>
\begin{itemize}
\item[]
①財務諸表の改善
\begin{itemize}
\item[]
新契約費の抑制、収益の安定、収益認識のタイミングの変更、
ソルベンシ]マージン比率の改善、ROE・IRR等の収益率の向上
\end{itemize}
\item[]
②特定のビジネスコールの達成
\begin{itemize}
\item[]
増資の抑制、課税所得の平準化、
格付けの引上げ・安定、円滑な買収・株式会社化
\end{itemize}
\end{itemize}

\problem{H11 生保1問題 1(4)【テキストに記載箇所なし】}
次の①~⑤を適当な語句で埋めよ。

再保険の有する機能の中で、最も重要なものは\wakumaru{①}
機能である。危険の種類を分類すると以下の
ようになる。

\begin{itemize}
\item[] \wakumaru{②}危険 …
\begin{itemize}
\item[ア.] 保険金額が高額による危険
\item[イ.]  被保険者の欠陥度合いが高度による危険
\end{itemize}

\item[] \wakumaru{③}危険 …
\begin{itemize}
\item[ア.]  地震、飛行機事故など、支払い保険金が\wakumaru{④}する危険
\item[イ.]  一定期間の\wakumaru{⑤}が変動する危険
\end{itemize}
\end{itemize}

\answer{}
\begin{itemize}
\item[ ①] …危険分散
\item[ ②] …個別
\item[ ③] …集団
\item[ ④] …集積
\item[ ⑤] …死亡率
\end{itemize}

\end{document}