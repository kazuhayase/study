% Options for packages loaded elsewhere
\PassOptionsToPackage{unicode}{hyperref}
\PassOptionsToPackage{hyphens}{url}
%
\documentclass[]{article}

\usepackage{lmodern}
\usepackage{amssymb,amsmath}
\usepackage{ifxetex,ifluatex}
\usepackage{actuarialsymbol}
\usepackage[]{natbib}
\RequirePackage{plautopatch}

% maru suji ① etc.
\usepackage{tikz}
\newcommand{\cir}[1]{\tikz[baseline]{%
\node[anchor=base, draw, circle, inner sep=0, minimum width=1.2em]{#1};}}

\usepackage{comment}

\begin{comment}

\ifnum0\ifxetex1\fi\ifluatex1\fi=0 % if pdftex
  \usepackage[T1]{fontenc}
  \usepackage[utf8]{inputenc}
  \usepackage{textcomp} % provide euro and other symbols
\else % if luatex or xetex
  \usepackage{unicode-math}
  \defaultfontfeatures{Scale=MatchLowercase}
  \defaultfontfeatures[\rmfamily]{Ligatures=TeX,Scale=1}
\fi
% Use upquote if available, for straight quotes in verbatim environments
\IfFileExists{upquote.sty}{\usepackage{upquote}}{}
\IfFileExists{microtype.sty}{% use microtype if available
  \usepackage[]{microtype}
  \UseMicrotypeSet[protrusion]{basicmath} % disable protrusion for tt fonts
}{}
\makeatletter
\@ifundefined{KOMAClassName}{% if non-KOMA class
  \IfFileExists{parskip.sty}{%
    \usepackage{parskip}
  }{% else
    \setlength{\parindent}{0pt}
    \setlength{\parskip}{6pt plus 2pt minus 1pt}}
}{% if KOMA class
  \KOMAoptions{parskip=half}}
\makeatother
\usepackage{xcolor}
\IfFileExists{xurl.sty}{\usepackage{xurl}}{} % add URL line breaks if available
\IfFileExists{bookmark.sty}{\usepackage{bookmark}}{\usepackage{hyperref}}
\hypersetup{
  hidelinks,
  pdfcreator={LaTeX via pandoc}}
\urlstyle{same} % disable monospaced font for URLs
\usepackage{longtable,booktabs}
% Correct order of tables after \paragraph or \subparagraph
\usepackage{etoolbox}
\makeatletter
\patchcmd\longtable{\par}{\if@noskipsec\mbox{}\fi\par}{}{}
\makeatother
% Allow footnotes in longtable head/foot
\IfFileExists{footnotehyper.sty}{\usepackage{footnotehyper}}{\usepackage{footnote}}

\end{comment}
%\makesavenoteenv{longtable}
\setlength{\emergencystretch}{3em} % prevent overfull lines
\providecommand{\tightlist}{%
  \setlength{\itemsep}{0pt}\setlength{\parskip}{0pt}}
\setcounter{secnumdepth}{-\maxdimen} % remove section numbering

\author{kazuyoshi}
\date{}

\newcommand{\problem}[1]{\subsubsection{#1}\setcounter{equation}{0}}
%\newcommand{\answer}[1]{\subsubsection{#1}}
\newcommand{\answer}[1]{\subsubsection{解答}}

%Pdf%\newcommand{\wakumaru}[1]{\framebox[3zw]{#1}}
\newcommand{\wakumaru}[1]{#1}






\begin{document}

\hypertarget{ux4fddux967a1ux7b2c1ux7ae0-ux55b6ux696dux4fddux967aux6599}{%
\subsubsection{20230504
保険1第1章 営業保険料}\label{ux4fddux967a1ux7b2c1ux7ae0-ux55b6ux696dux4fddux967aux6599}}

\hypertarget{ux55b6ux696dux4fddux967aux6599ux6c7aux5b9aux306eux969bux306bux8003ux616eux3059ux3079ux304dux70b9}{%
\section{1.2
営業保険料決定の際に考慮すべき点}\label{ux55b6ux696dux4fddux967aux6599ux6c7aux5b9aux306eux969bux306bux8003ux616eux3059ux3079ux304dux70b9}}

\begin{verbatim}
 H27 問3 (1) 1,2; H19 問3 (1)1; H17 問3 (2)1
\end{verbatim}

\hypertarget{ux55b6ux696dux4fddux967aux6599ux6c7aux5b9aux306eux969bux306bux8003ux616eux3059ux3079ux304dux4e8bux98054ux3064}{%
\subsection{営業保険料決定の際に考慮すべき事項4つ}\label{ux55b6ux696dux4fddux967aux6599ux6c7aux5b9aux306eux969bux306bux8003ux616eux3059ux3079ux304dux4e8bux98054ux3064}}

\begin{enumerate}
\def\labelenumi{\arabic{enumi}.}
\tightlist
\item
  十分性

  \begin{itemize}
  \tightlist
  \item
    最も重要。
  \item
    会社の最終的な支払い能力を決定する。
  \item
    契約者からの直接の収入
  \item
    十分な検証が必要
  \end{itemize}
\item
  公平性

  \begin{itemize}
  \tightlist
  \item
    契約者のために考慮すべき点
  \item
    理論的に完璧な公平性を実現する必要はなく、
  \item
    実務の簡素化を念頭におきつつ、
  \item
    保険料における公平性の問題を考えるべき

    \begin{itemize}
    \tightlist
    \item
      個人保険の保険料は、保険種類・年齢・性別によって異なるのが一般的

      \begin{itemize}
      \tightlist
      \item
        団体保険や各種特約も同様であるべきか、
      \item
        年齢は歳別か群団料率か、といった点もある
      \end{itemize}
    \end{itemize}
  \end{itemize}
\item
  収益性

  \begin{itemize}
  \tightlist
  \item
    有配当の相互会社であれば、十分性が満たされていれば、収益性はさほど重要ではないという考えもある
  \item
    無配当の株式会社であれば、収益性の検証の重要性はより高まる
  \end{itemize}
\item
  標準責任準備金制度との関係

  \begin{itemize}
  \tightlist
  \item
    営業保険料の計算基礎率は各社の判断により決定すべき
  \item
    十分性を慎重に検証した上で、より低廉な営業保険料を設定することも可能

    \begin{itemize}
    \tightlist
    \item
      責任準備金の積立水準は収益の認識時期に影響するのみ

      \begin{itemize}
      \tightlist
      \item
        十分性の指標として保険期間満了時までの収益の単純合計を見るとすれば、
      \end{itemize}
    \item
      保険期間途中では積立負担が大きくなる

      \begin{itemize}
      \tightlist
      \item
        予定利率が標準利率より高い場合、特に一時払契約においては、契約初期にかなり大きな積立負担が発生
      \item
        積立負担を当該保険群団で賄えない場合は、

        \begin{itemize}
        \tightlist
        \item
          他の保険群団の剰余
        \item
          会社勘定(内部留保)で立て替えることになる。
        \end{itemize}
      \item
        標準責任準備金の積立は一種の初期投資、内部留保の水準から容認できる範囲で行うという考え
      \item
        結果として保険料の不足を引き起こす恐れもある
      \end{itemize}
    \item
      十分性が保たれているかどうかの判断には困難がつきまとうため、慎重に検討する必要がある
    \end{itemize}
  \end{itemize}
\end{enumerate}

\hypertarget{h19-ux554f3-1ux2460}{%
\subsubsection{H19 問3 (1)①}\label{h19-ux554f3-1ux2460}}

営業保険料決定の際に考慮すべき点のうち、「十分性」ならびに「標準責任準備金制度との関係」について簡潔に説明せよ。

===

\hypertarget{ux89e3ux7b54h19-ux554f3-1-ux2460}{%
\subsubsection{解答(H19 問3 (1)
①)}\label{ux89e3ux7b54h19-ux554f3-1-ux2460}}

\begin{itemize}
\tightlist
\item
  十分性

  \begin{itemize}
  \tightlist
  \item
    最も重要な点

    \begin{itemize}
    \tightlist
    \item
      会社の最終的な支払能力が決定される
    \item
      契約者からの直接の収入
    \end{itemize}
  \item
    十分な検証が必要

    \begin{itemize}
    \tightlist
    \item
      保険期間が長期にわたること
    \item
      利源分析等を参考に各基礎率の十分性についても検証し、
    
      \item
      その他の営業保険料決定の際に考慮すべき点とのバランスにも留意する必要がある。
    \end{itemize}
  \end{itemize}
\item
  標準責任準備金制度との関係

  \begin{itemize}
  \tightlist
  \item
    営業保険料の計算基礎率は、各社が各社の判断により決定すべきものであり、

  \item
      必ずしも標準責任準備金の評価基礎率(以下標準基礎率)にあわせる必要はない。

      \begin{itemize}
      \tightlist
      \item
        十分性を慎重に検証
      \item
        低廉な営業保険料を設定

        \begin{itemize}
        \tightlist
        \item
          保険期間の途中では積立負担が発生する。

          \item
            営業保険料およびその内訳である純保険料と対応しない
          \item
            その保険群団でまかなえない場合は、立て替えることになる。

                \item
              他の保険群団の剰余
            \item
              会社勘定(内部留保)
              \item
            恒常的に立替えが必要な状態は好ましくないと言える。
          \item
            内部留保の水準から容認できる範囲の初期投資
          \item
            結果として保険料の不足を引き起こす恐れ
          
        \end{itemize}
      \end{itemize}
    \item
      アクチェアリーとしで慎重に検討する必要がある。
    
  \end{itemize}
\end{itemize}

\hypertarget{h11-ux554f1-5}{%
\subsubsection{H11 問1 (5)}\label{h11-ux554f1-5}}

次の①~⑤を適当な語句で埋めよ。 生保標準生命表
1996(死亡保険用)の作成概要は以下のとおりである。 基礎データの収集
→①の決定→②→ 標準生命表
①の決定においては、標準生命表に求められる、死亡率の安定性・安全性の確保および経験死亡率の③の実態を勘案し、④および⑤を決定した。

=== \#\#\# 解答 (H11 問1 (5)) ①粗死亡率 ②補整 ③選択効果
④⑤観察年度、截断年数

\hypertarget{h13-ux554f1-9}{%
\subsubsection{H13 問1 (9)}\label{h13-ux554f1-9}}

===

\hypertarget{ux89e3ux7b54-h13-ux554f1-9}{%
\subsubsection{解答 (H13 問1 (9))}\label{ux89e3ux7b54-h13-ux554f1-9}}

生保標準生命表
1996(死亡保険用)では、截断年数については以下のとおり設定した。なぜこのような截断年数を設けたのか、「截断年数」自体の説明も含め簡潔に説明せよ。

\begin{itemize}
\tightlist
\item
  截断年数

  \begin{itemize}
  \tightlist
  \item
    経験表作成の際
  \item
    選択効果を排除し
  \item
    死亡率の安全性を確保するため、
  \item
    契約当初数年のデータを除外して作成する
  \end{itemize}
\item
  生保標準生命表1996以前の第5回全会社生命表

  \begin{itemize}
  \tightlist
  \item
    年齢・性別に関わらず1年裁断であった
  \end{itemize}
\item
  生保標準生命表1996の作成では、

  \begin{itemize}
  \tightlist
  \item
    年齢群団間で選択効果に差が認められる点を考慮し

    \begin{itemize}
    \tightlist
    \item
      年齢別に1年裁断から5年裁断とし
    \end{itemize}
  \item
    男女間でも選択効果に差異が認められたことから、

    \begin{itemize}
    \tightlist
    \item
      同じ裁断年数でも男女間で適用年齢に差異を設けた。
    \end{itemize}
  \end{itemize}
\item
  ただし、裁断年数を長くすれば安全な死亡率が作成できるわけではなく、
\item
  裁断年数を長くした場合その部分のサンプルが少なくなることからかえって死亡率データとしての信頼性が減少する
\end{itemize}

\begin{center}\rule{0.5\linewidth}{0.5pt}\end{center}

\hypertarget{h29-ux554f1-4}{%
\subsubsection{H29 問1 (4)}\label{h29-ux554f1-4}}

生保標準生命表 2007(年金開始後用)は、第 19 回生命表(2000
年)を基礎表とした上で、主に次のような処理を行って死亡率の安定性・安全性を加味している。

\begin{itemize}
\tightlist
\item
  将来の死亡率改善
\item
  ①の除去
\item
  将来死亡率の推定

  \begin{itemize}
  \tightlist
  \item
    毎年死亡率が改善していくとして推定
  \item
    原則として 1960 年生まれの人が各年齢に達する年とし、
  \item
    第 19 回生命表の死亡率に、2000
    年からその「将来」までの年数だけの死亡率の改善を加えたものを、将来の死亡率」とする。
  \item
    ただし、最低でも②年分の死亡率の改善を見込むこととする。
  \end{itemize}
\item
  生存リスク方向への補整:以下5点の観点から、死亡率の安全性を目的として、改善率反映後の死亡率に④\%が乗じられている。

  \begin{itemize}
  \tightlist
  \item
    「単年度のブレへの対応」
  \item
    「改善率の見込み差異の吸収」
  \item
    「母数(会社規模)の差による違いの吸収」
  \item
    「③の違いの吸収」
  \item
    「元データを国民表とすることへの対応」
  \end{itemize}
\end{itemize}

⑤ 生保標準生命表 2007(年金開始後用)における男性 60
歳の死亡率はいずれか? (A) 0.00006 (B) 0.00064(C) 0.00642(D)
0.06472

===

\hypertarget{ux89e3ux7b54-29-ux554f1-4}{%
\paragraph{解答 (29 問1 (4))}\label{ux89e3ux7b54-29-ux554f1-4}}

① コーホート効果 ② 20 ③ 代表生年 ④ 85 ⑤ C およその水準 0.64\%

\begin{center}\rule{0.5\linewidth}{0.5pt}\end{center}

\hypertarget{h17-ux554f13}{%
\subsubsection{H17 問1(3)}\label{h17-ux554f13}}

生保標準生命表
1996(年金開始後用)死亡率の作成方法について、次の①~④の空欄を適当な数値または数式で埋めよ。
生保標準生命表 1996(年金開始後用)死亡率は、第 10 回生命表(1955
年)と第 15
回生命表(1980年)とから将来の死亡率の改善を見込んで作成されたものである。具体的には、第
10 回生命表の x 歳の死亡率を\(q_x^{(10)}\)、第 15
回生命表のそれを\(q_x^{(15)}\)とし改善率を求めると、1
年あたりの改善率\(r_x\)= ① となる。
今後もこの改善率で毎年死亡率が改善していくとして将来の死亡率を推定する。推定する「将来」としては、原則として
1945 年生まれの人が各年齢に達する年とし、第 15 回生命表の死亡率に、1980
年からその「将来」までの年数だけ改善を加えたものを「将来の死亡率」とする。(ただし、最低
20 年分 の改善を見込む)
各年齢における死亡率の改善を見込む年数は下表のとおりとなる。

\begin{longtable}[]{@{}lll@{}}
\toprule
年齢 & 推定する「将来」 & 死亡率の改善を見込む年数\tabularnewline
\midrule
\endhead
50 歳 & 2000 年 & 20 年\tabularnewline
60 歳 & ②年 & ③年\tabularnewline
\bottomrule
\end{longtable}

60 歳の死亡率の推定値は、\(q_x^{(15)}\)、\(r_{60}\)
を使用し④と計算される。
このようにして求めた将来の死亡率について、死亡率を滑らかにするための補整と高年齢での補外を行ったものを年金開始後死亡率としている。

===

\hypertarget{ux89e3ux7b54-h17-ux554f13}{%
\paragraph{解答 (H17 問1(3))}\label{ux89e3ux7b54-h17-ux554f13}}

① \(1-\{q^{(15)}/q^{(10)}\}^{1/25}\);
q比率を(1980-1955=)25年幾何平均。改善率とするため1から減算 ② 2005;
基準年1980に、25年足す ③ 25 ④ \(q^{(15)}\_{60}\cdot(1-r\_{60})^{25}\);
改善率を1から減算したものを25乗

\begin{center}\rule{0.5\linewidth}{0.5pt}\end{center}

\hypertarget{h12-ux554f19}{%
\subsubsection{H12 問1(9)}\label{h12-ux554f19}}

生保標準生命表 1996 の年金開始後用死亡率の作成過程を簡潔に説明せよ。

===

\hypertarget{ux89e3ux7b54h12-ux554f19}{%
\paragraph{解答(H12 問1(9))}\label{ux89e3ux7b54h12-ux554f19}}

\begin{enumerate}
\def\labelenumi{\arabic{enumi}.}
\tightlist
\item
  基となる死亡率として、第15回生命表(1980年)の死亡率を用いる。
\item
  第15回生命表を第10回生命表と比較し、男女別・各年齢別ごとに死亡率が1年当たりどれだけの割合で減少しているか(改善率)を求める。
\item
  今後も2.で求めた改善率で毎午死亡率が改善していくとして、将来の死亡率を推定する。(推定する「将来」としては、原則として1945年生まれの人が各年齢に達する年とする。ただし、55歳以下の年齢については、推定する「将来」を2000年とする。)
\item
  このようにして求めた将来の死亡率について、死亡率を滑らかにする・ための補整と、高年齢での補外を行ったものを年金開始後用死亡率としている。
\end{enumerate}

\begin{center}\rule{0.5\linewidth}{0.5pt}\end{center}

\hypertarget{ux554f1-1}{%
\subsubsection{2019 問1 (1)}\label{ux554f1-1}}

第三分野標準生命表 2018
の作成過程について、次の①~⑤に適切な語句を記入しなさい。

\begin{itemize}
\tightlist
\item
  基礎データの決定
  第三分野保険の契約形態の変化(主契約・単品化)、死亡保険との診査手法の相違、同じ生存リスクに対応する年金開始後用との整合性等を踏まえ、基礎データとして①の死亡率を用いることとした。なお、第三分野標準生命表2018
  は②を含まない死亡率であるが、第三分野標準生命表2007は②を含む死亡率である。
\item
  死亡率改善の反映
  死亡率の改善状況等を踏まえ、基礎データに標準生命表の適用年までの死亡率改善を反映したものを補整前死亡率とした。具体的な改善率は、国民死亡率の実績が判明している
  2015 年までは、男性が年 2.5\%、女性が年 2.0%であり、2015
  年から標準生命表適用年である 2018 年までは男女ともに年③\%である。
\item
  数学的危険論による補整
  「単年度のブレヘの対応」、「母数(会社規模)の差による違いの吸収」、「将来の死亡率変動への対応」等を勘案し、数学的危険論に基づき、補整を行った。将来経験する死亡率が変動予測を超える確率を約
  2.28%とするように、2σ水準を補整前死亡率から減じた。ここで変動予測に用いる想定件数は、標準的な会社を想定し男女各々④件に設定した。
  また、特に高齢部分の「将来の死亡率変動への対応」を図る観点から、補整後死亡率に上限(補整前死亡率の⑤\%)を設けることとした。
\end{itemize}

===

\hypertarget{ux89e3ux7b542019-ux554f1-1}{%
\subsubsection{解答(2019 問1 (1))}\label{ux89e3ux7b542019-ux554f1-1}}

① 第21回生命表(2010 年)② 高度障害 ③ 1.0 ④ 100 万 ⑤ 85
なお、文脈から判断して適切な用語を埋めた場合も正解とした。

\begin{center}\rule{0.5\linewidth}{0.5pt}\end{center}

\hypertarget{h25-ux554f1-1}{%
\subsubsection{H25 問1 (1)}\label{h25-ux554f1-1}}

第三分野標準生命表2007の作成方法について、次の①~⑤に適切な語句または数字を答えなさい。

\begin{enumerate}
\def\labelenumi{\arabic{enumi}.}
\tightlist
\item
  基礎データ
  第三分野の加入者のリスク特性としては、健康に不安のある者が相対的に多い集団と思われ、基礎データとしては①用死亡率に比較的近いと思われる。
  ①においては②の除去を目的に截断を行ったが、第三分野用に関しては截断を行った場合、責任準備金の健全性を損なうこととなるため、截断は行われていない。一方、データの信頼性の観点から、
  ③部分については、
  ①の場合と同様、対象データの拡大や国民表の採用が行われている。
\item
  数学的危険論による④
  死亡率の安全性の点から、数学的危険論による④が行われている。具体的には、男女ごとに総人口⑤万人の正規分布の年齢構成を前提とし、将来の死亡率が変動予測を超える確率を約2.28%(2σ水準)におさえるように④した。この手法自体は、①の場合と同様であるが、安全をみる方向は反対である。
\end{enumerate}

===

\hypertarget{ux89e3ux7b54h25-ux554f1-1}{%
\subsubsection{解答(H25 問1 (1))}\label{ux89e3ux7b54h25-ux554f1-1}}

① 死亡保険 ② 選択効果 ③ 若年齢 ④ 補整 ⑤ 400

\hypertarget{ux5229ux7387}{%
\subsection{利率}\label{ux5229ux7387}}

\hypertarget{h15-ux554f4-2ux2461}{%
\subsubsection{H15 問4 (2)②}\label{h15-ux554f4-2ux2461}}

保険料計算基礎に用いる予定利率の設定方法について簡潔に説明せよ

===

\hypertarget{ux89e3ux7b54h15-ux554f4-2ux2461}{%
\subsubsection{解答(H15 問4
(2)②)}\label{ux89e3ux7b54h15-ux554f4-2ux2461}}

\hypertarget{ux57faux672cux7684ux306aux8003ux3048ux65b9}{%
\paragraph{<基本的な考え方>}\label{ux57faux672cux7684ux306aux8003ux3048ux65b9}}

\begin{itemize}
\tightlist
\item
  実績を元に

  \begin{itemize}
  \tightlist
  \item
    現時点における自社の運用利回り
  \item
    過去の運用利回りの推移をもとに、
  \end{itemize}
\item
  今後の運用方針を加味

  \begin{itemize}
  \tightlist
  \item
    資金特性を考慮する

    \begin{itemize}
    \tightlist
    \item
      解約等によるキャッシュアウトなど
    \end{itemize}
  \end{itemize}
\item
  他の基礎率に比べて特段の配慮が必要

  \begin{itemize}
  \tightlist
  \item
    リスク分散やコントロールが難しく
  \item
    将来的な予測も決して容易ではないことから
  \end{itemize}
\end{itemize}

\hypertarget{ux4fddux967aux671fux9593ux4fddux967aux6599ux6255ux8fbcux65b9ux5f0fux3068ux4e88ux5b9aux5229ux7387ux3068ux306eux95a2ux4fc2}{%
\paragraph{<保険期間・保険料払込方式と予定利率との関係>}\label{ux4fddux967aux671fux9593ux4fddux967aux6599ux6255ux8fbcux65b9ux5f0fux3068ux4e88ux5b9aux5229ux7387ux3068ux306eux95a2ux4fc2}}

\begin{itemize}
\tightlist
\item
  予定利率は保証料率

  \begin{itemize}
  \tightlist
  \item
    営業保険料は、保険給付の対価として契約時約定価格
  \end{itemize}
\item
  将来の運用利回りの予測

  \begin{itemize}
  \tightlist
  \item
    保険期間が長期になるほど困難
  \end{itemize}
\item
  利率の変動に影響を受けやすい貯蓄性の高い商品、

  \begin{itemize}
  \tightlist
  \item
    分割払は保守的にする必要:
    長期にわたり保険料のキャッシュインフローが見込まれる
  \item
    一時払は新契約時点での投資資産の運用利回りを基準:
    新契約時にのみキャッシュインフローが生じる

    \begin{itemize}
    \tightlist
    \item
      解約等による資金流出や解約返戻金の水準などを考慮することも必要
    \end{itemize}
  \item
    新契約の保険料率を機動的に変更できる体制:
    市中金利等の変動にキャッチアップするため
  \end{itemize}
\end{itemize}

\hypertarget{ux6709ux914dux5f53ux4fddux967aux7121ux914dux5f53ux4fddux967aux3068ux4e88ux5b9aux5229ux7387ux3068ux306eux95a2ux4fc2}{%
\paragraph{<有配当保険・無配当保険と予定利率との関係>}\label{ux6709ux914dux5f53ux4fddux967aux7121ux914dux5f53ux4fddux967aux3068ux4e88ux5b9aux5229ux7387ux3068ux306eux95a2ux4fc2}}

\begin{itemize}
\tightlist
\item
  有配当保険の場合、利率を保守的に見込んだことによる調整を配当により実施することが出来る
\item
  無配当保険の場合、

  \begin{itemize}
  \tightlist
  \item
    合理的に考えれば競争上の理由から、より実勢に近い設定が必要である。
  \item
    利率が相対的に硬直的であるので、金利の変動下では(新規の契約に対し)機動的な予定利率の変更が出来るよう、体制を整えておくことが望ましい。
  \end{itemize}
\end{itemize}

(参考)
上記の内容を踏まえ、具体的な保険種類を例示し、その予定利率の設定方法について論じる解答案でもよい。また、その他の基礎率の設定と異なる点を詳述するなどもよい

\begin{center}\rule{0.5\linewidth}{0.5pt}\end{center}

\hypertarget{h11-ux554f2-1ux2460}{%
\subsubsection{H11 問2 (1)①}\label{h11-ux554f2-1ux2460}}

保険料計算基礎率としての予定利率の設定について、予定利率設定の際に留意すべき一般的な事項をあげ、簡潔に説明せよ。

===

\hypertarget{ux89e3ux7b54h11-ux554f2-1ux2460}{%
\subsubsection{解答(H11 問2
(1)①)}\label{ux89e3ux7b54h11-ux554f2-1ux2460}}

\begin{itemize}
\item ~
  \hypertarget{ux57faux672cux7684ux306aux8003ux3048ux65b9-1}{%
  \paragraph{基本的な考え方}\label{ux57faux672cux7684ux306aux8003ux3048ux65b9-1}}

  \begin{itemize}
  \tightlist
  \item
    自社の運用利回りとその直前の期間における短期的トレンド・新規投資の運用利回り
  \item
    自社の将来の運用方針変更の有無
  \item
    短期的な将来の利回り予測
  \item
    特に長期のものについては、保守的なものを採用するのが普通

    \begin{itemize}
    \tightlist
    \item
      「保証利率」としての性格
    \item
      将来予測が困難
    \end{itemize}
  \item
    近年における低金利の長期化とそれに伴ういわゆる「逆ざや」の現状などから、

    \begin{itemize}
    \tightlist
    \item
      上に述べた考え方を基本としつつも、
    \item
      従来以上に商品特性や運用方針、配当政策などと一体化した予定利率の設定を行なう必要
    \end{itemize}
  \item
    標準責任準備金の積立利率である標準利率との関係についても考慮する必要がある。
  \end{itemize}
\item ~
  \hypertarget{ux9069ux5207ux306aux5b89ux5168ux6027ux3092ux78baux4fddux3057ux3066ux4e88ux5b9aux5229ux7387ux3092ux8a2dux5b9aux3059ux308bux5177ux4f53ux7684ux306aux7559ux610fux70b9}{%
  \paragraph{適切な安全性を確保して予定利率を設定する、具体的な留意点}\label{ux9069ux5207ux306aux5b89ux5168ux6027ux3092ux78baux4fddux3057ux3066ux4e88ux5b9aux5229ux7387ux3092ux8a2dux5b9aux3059ux308bux5177ux4f53ux7684ux306aux7559ux610fux70b9}}

  \begin{itemize}
  \tightlist
  \item
    キャッシュフローの特性と運用方針

    \begin{itemize}
    \tightlist
    \item
      キャッシュ・フロー特性およびその金利感応度

      \begin{itemize}
      \tightlist
      \item
        保険料の払込方法、
      \item
        保険期間および給付等
      \end{itemize}
    \item
      当該資産の運用方針、
    \item
      必要内部留保の水準

      \begin{itemize}
      \tightlist
      \item
        運用方針に基づく運用収益率、偏差および価格変動準備金等
      \end{itemize}
    \item
      キャッシュフロー・マッチング等のALM手法により資産運用を行なう商品の場合

      \begin{itemize}
      \tightlist
      \item
        当該資産ポートフォリオの期待収益率、
      \item
        市中金利や株価などの変動に対するリスク許容度、
      \item
        MVA等の方式により解約時のキャッシュ化に伴う費用を担保できるか否か、等。
      \end{itemize}
    \item
      運用関係以外の損益(死差益、費差益等)による金利リスクカバー。

      \begin{itemize}
      \tightlist
      \item
        商品特性・発生のタイミング
      \end{itemize}
    \item
      契約者に与えられたオプション(払済保険への変更等)とその特性
    \end{itemize}
  \item
    予定利率の設定方式と配当政策

    \begin{itemize}
    \tightlist
    \item
      予定利率変動型、ビルトイン方式等の予定利率の設定方式の違い
    \item
      約款上の基礎率変更権の有無

      \begin{itemize}
      \tightlist
      \item
        実際に変更が行える条件
      \item
        変更が会社等に与える影響度合
      \end{itemize}
    \item
      高料高配商品か低料低配商品か
    \item
      配当方針の違い(有配当、準有配、無配当;安定配当か実績還元型か)
    \end{itemize}
  \item
    責任準備金の積立水準(詳細は②で述べる)

    \begin{itemize}
    \tightlist
    \item
      標準責任準備金の計算基礎率である標準利率との関係
    \item
      健全性確保のために充分な責任準備金の積立を行なうことができるか
    \end{itemize}
  \end{itemize}
\end{itemize}

\hypertarget{h19-ux554f3-1ux2461}{%
\subsubsection{H19 問3 (1)②}\label{h19-ux554f3-1ux2461}}

営業保険料を決定する要素のうち予定利率の設定について基本的な考え方を説明し、貯蓄性商品について一時払及び平準払の各払方における予定利率の設定にあたり、アクチュアリーとして留意すべき点を挙げよ。

===

\hypertarget{ux89e3ux7b54h19-ux554f3-1ux2461}{%
\subsubsection{解答(H19 問3
(1)②)}\label{ux89e3ux7b54h19-ux554f3-1ux2461}}

\begin{itemize}
\tightlist
\item
  予定利率の設定について基本的な考え方

  \begin{itemize}
  \tightlist
  \item
    自社の運用利回りや新規投資の運用利回りなどをもとに、
  \item
    自社の将来の運用方針の変更の有無と将来の利回り予想などに基づき決定する

    \begin{itemize}
    \tightlist
    \item
      今後の運用方針を考える上では、該当する保険契約の解約等によるキャッシュアウトなど、キャッシュフローの特性も考慮する必要がある。
    \end{itemize}
  \item
    死亡率や事業費支出などと異なり、

    \begin{itemize}
    \tightlist
    \item
      運用利回りはリスク分散やコントロールが難しく、
    \item
      将来的な予測も容易でないことから、
    \end{itemize}
  \item
    予定利率の設定は他の基礎率と比較して特段の配慮が必要であり、アクチェアリーとして長期の予定利率は保守的なものを採用するのが一般的である。
  \end{itemize}
\item
  一時払商品の留意点

  \begin{itemize}
  \tightlist
  \item
    運用商品としての色彩が濃く、死差益、費差益等といった運用関係以外の収益によるバッファーがほとんどない。
  \item
    解約等による資金流動性が高く、また、一般的に効果的な解約控除機能がない。
  \item
    金利感応度が高く、市場金利の動向によっては解約増を招きやすい。
  \item
    他社商品、隣接業界の運用商品との競合。
  \item
    運用方針および配当政策との関係。(総合的なバランス型運用の場合は保守的な予定利率とし、実績還元型の配当が考えられ、ALM型運用の場合は期待される運用利率に近い予定利率を設定できる。)
  \item
    標準利率との関係。(一時払商品は、標準利率よりも高い予定利率を設定した場合、当初の標準責任準備金積立負担が重いため、十分性・収益性に留意が必要。)
  \end{itemize}
\item
  平準払商品の留意点

  \begin{itemize}
  \tightlist
  \item
    平準払については、毎年ニューマネーが入ってくるという点で、一時払とは状況が異なる。過去に締結した契約の保険料が毎年新規に入ってくるわけであり、現在の金利との差が逆ざやの要因になり得る。平準払の場合は将来の金利低下リスクがあるため、長期にわたる予定利率の設定には慎重な配慮が必要である。他に、以下のような点に留意する必要があると考えられる。

    \begin{itemize}
    \tightlist
    \item
      平準払の場合は一時払よりも死差、費差等他の利源が厚いため、これらのバッファーによりある程度金利リス.クをカバーできる。
    \item
      標準利率との関係。(予定利率が標準利率を上回っている場合、保険期間が超長期の場合には積増負担が大きい。)
    \item
      払済保険への変更等、契約者に与えられたオプションとその特性など。
    \end{itemize}
  \end{itemize}
\end{itemize}

\begin{center}\rule{0.5\linewidth}{0.5pt}\end{center}

\hypertarget{h13-ux554f23ux2460}{%
\subsubsection{H13 問2(3)①}\label{h13-ux554f23ux2460}}

団体年金保険について、一般の個人保険・個人年金保険との相違点のうち、予定利率設定にあたって考慮すべき点について述べよ。

===

\hypertarget{ux89e3ux7b54h13-ux554f23ux2460}{%
\subsubsection{解答(H13 問2(3)①)}\label{ux89e3ux7b54h13-ux554f23ux2460}}

団体年金保険と一般の個人保険・個人年金・との相違点の中で、以下の点については、団体年金保険の予定利率ρ設定にあたって考慮すべきである。

\begin{itemize}
\tightlist
\item
  終期のない保険契約であり基本的には永続的に継続する商品

  \begin{itemize}
  \tightlist
  \item
    満了等のある個人保険とは異なる
  \end{itemize}
\item
  資金流動性が高く、キャッシュフローが不規則

  \begin{itemize}
  \tightlist
  \item
    解約やシェア変更など
  \item
    大型団体の契約ではそれが高額となる。
  \item
    効果的な解約控除機能を有していないこと等にも起因
  \end{itemize}
\item
  運用商品としての色彩が強く、費差益・死差益等といった運用関係以外の収益がほとんどない。
\item
  契約者が企業であり、金利感応度が高く市場金利の動向によって解約増加等を招きやすい。

  \begin{itemize}
  \tightlist
  \item
    「一般的に保険契約に関する知識が豊富で理解力が高い」
  \item
    「選別意識が高い」
  \item
    「実績還元二一ズが強い」
  \item
    「ディスクロージャーの要求が強い」などの傾向があり、
  \end{itemize}
\item
  隣接業界にもわたり競合している

  \begin{itemize}
  \tightlist
  \item
    信託銀行、投資顧間会社
  \end{itemize}
\end{itemize}

\begin{center}\rule{0.5\linewidth}{0.5pt}\end{center}

\hypertarget{h3-ux554f1-1}{%
\subsubsection{H3 問1 (1)}\label{h3-ux554f1-1}}

予定利率のビルトイン方式について簡潔に説明せよ。

=== \#\#\# 解答(H3 問1 (1)) *
ひとつの保険期間をいくつかの期間に分割し、分割したそれぞれの期間毎に予定利率を設定する方式をいう
* 通常はその保守性から、先の期間ほど利率を低くする *
ビルトイン方式を取る理由 *
保険の数理計算に用いる予定利率の設定においては、 *
保険の長期性から、運用利回りを正確に予測することは困難であり、保守的に設定する必要がある。
* また、長期的に一定以上の利回りを確保することも困難である。

\begin{center}\rule{0.5\linewidth}{0.5pt}\end{center}

\hypertarget{h20-ux554f33}{%
\subsubsection{H20 問3(3)}\label{h20-ux554f33}}

契約者貸付利率の水準について、市中金利、予定利率等との関係を含め、設定にあたって留意すべき事項について説明しなさい。

===

\hypertarget{ux89e3ux7b54h20-ux554f33}{%
\subsubsection{解答(H20 問3(3))}\label{ux89e3ux7b54h20-ux554f33}}

\begin{enumerate}
\def\labelenumi{\arabic{enumi}.}
\tightlist
\item
  市中金利との関係
\end{enumerate}

\begin{itemize}
\tightlist
\item
  市中金利が契約者貸付利率よりも高い場合、保険金杜から資金流出が起きる。

  \begin{itemize}
  \tightlist
  \item
    契約者貸付を利用して金利の高い金融商品に資金移転することが考えられる
  \end{itemize}
\item
  市中金利が低下すると逆の現象が発生し、キャッシュフローの大きな変動が発生
\item
  保険会社が当初想定した運用ができなくなり運用効率の低下をもたらすこととなる。
\item
  従って、貸付利率は市中金利から大きく乖離しないように考慮する。
\end{itemize}

\begin{enumerate}
\def\labelenumi{\arabic{enumi}.}
\setcounter{enumi}{1}
\tightlist
\item
  予定利率との関係
\end{enumerate}

\begin{itemize}
\tightlist
\item
  契約者貸付は会社の資産運用の一形態
\item
  設定の段階で保険料計算基礎の予定利率を上回っていることが必要となる。

  \begin{itemize}
  \tightlist
  \item
    利差損が生じないため
  \end{itemize}
\item
  予定利率が異なる契約者間で不公平が生じないよう設定することが望ましい

  \begin{itemize}
  \tightlist
  \item
    予定利率は契約の年度、保険種類等により異なるため、
  \end{itemize}
\end{itemize}

\hypertarget{ux4ee5ux4e0aux5951ux7d04ux8005ux8cb8ux4ed8ux5229ux7387ux3092ux8a2dux5b9aux3059ux308bux969bux306bux306f}{%
\paragraph{以上、契約者貸付利率を設定する際には、}\label{ux4ee5ux4e0aux5951ux7d04ux8005ux8cb8ux4ed8ux5229ux7387ux3092ux8a2dux5b9aux3059ux308bux969bux306bux306f}}

\begin{itemize}
\tightlist
\item
  契約者貸付利率を相対的に高く設定することが考えられるが、過度に設定すべきではない。

  \begin{itemize}
  \tightlist
  \item
    市中金利、予定利率との関係を踏まえるとともに、
  \item
    継続性の観点から
  \end{itemize}
\item
  貸付残高が大きくなり、その結果失効消滅する契約が増えるため、
\item
  契約者に対するサービスの一貫であることから

  \begin{itemize}
  \tightlist
  \item
    同業地杜と競合できうる水準であることも望ましい。
  \end{itemize}
\item
  一方で、契約者貸付を行った場合に追加で掛かる事業費コスト

  \begin{itemize}
  \tightlist
  \item
    付加保険料の範囲内で賄えるのかも検証する必要があり、
  \item
    賄えない場合には不足分を利鞘で負担できるか否かも留意する必要がある。
  \end{itemize}
\item
  利差配当率の設定に

  \begin{itemize}
  \tightlist
  \item
    配当基準利回りは、資産運用利回りの範囲内で定める必要がある。
  \item
    契約者貸付を受けた契約者は、契約者貸付利率の範囲内で配当基準利回りを設定することも考えられる

    \begin{itemize}
    \tightlist
    \item
      契約者貸付利率と資産運用利回りが乖離している場合
    \end{itemize}
  \end{itemize}
\end{itemize}

\hypertarget{ux4ed8ux52a0ux4fddux967aux6599ux306bux5bfeux3059ux308bux76e3ux7763}{%
\subsection{付加保険料に対する監督}\label{ux4ed8ux52a0ux4fddux967aux6599ux306bux5bfeux3059ux308bux76e3ux7763}}

\hypertarget{h29-ux554f3-1-ux2460}{%
\subsubsection{H29 問3 (1) ①}\label{h29-ux554f3-1-ux2460}}

国内における付加保険料に対する現在の監督体制について、簡潔に説明しなさい。

===

\hypertarget{ux89e3ux7b54-h29-ux554f3-1-ux2460}{%
\subsubsection{解答: H29 問3 (1)
①}\label{ux89e3ux7b54-h29-ux554f3-1-ux2460}}

\hypertarget{ux5e74ux306eux4fddux967aux696dux6cd5ux65bdux884cux898fux5247ux304aux3088ux3073ux4fddux967aux4f1aux793eux5411ux3051ux306eux7dcfux5408ux7684ux306aux76e3ux7763ux6307ux91ddux306eux6539ux6b63}{%
\paragraph{2006年の「保険業法施行規則」および「保険会社向けの総合的な監督指針」の改正}\label{ux5e74ux306eux4fddux967aux696dux6cd5ux65bdux884cux898fux5247ux304aux3088ux3073ux4fddux967aux4f1aux793eux5411ux3051ux306eux7dcfux5408ux7684ux306aux76e3ux7763ux6307ux91ddux306eux6539ux6b63}}

目的: 3点

\begin{itemize}
\tightlist
\item
  監督の実効性の向上を図り、
\item
  保険料の合理性・妥当性・公平性を確保した上で、
\item
  保険商品の価格
\end{itemize}

\begin{enumerate}
\def\labelenumi{\arabic{enumi}.}
\tightlist
\item
  保険料及び責任準備金の算出方法書

  \begin{itemize}
  \tightlist
  \item
    予定事業費に関しては定性記載に留め、水準等の具体的な記述は不要となった。=
    どのような算出方法を用いるかは各保険会社の判断

    \begin{itemize}
    \tightlist
    \item
      すなわち、予定事業費の設定は、保険業法に定める合理性、妥当性、公平性を満たしていればよい
    \end{itemize}
  \end{itemize}
\item
  予定事業費の算出方法は社内規程等に定めることとなった。以下2点、従来通り適用される。

  \begin{itemize}
  \tightlist
  \item
    保険業法第5条第1項第4号(保険料における不当な差別的取扱いの禁止)、
  \item
    同第 300 条第1項第5号(その他特別の利益の提供の禁止)
  \end{itemize}
\item
  金融庁が事後モニタリング(区分ごとの定期報告の提出)

  \begin{itemize}
  \tightlist
  \item
    事業費の実績と付加保険料の関係を把握するため

    \begin{itemize}
    \tightlist
    \item
      商品別 or 販売経路・保険種類ごとに区分して測定し、
    \item
      事業費のうち特に

      \begin{itemize}
      \tightlist
      \item
        新契約時にかかる費用(イニシャルコスト)の回収状況
      \item
        その他契約維持・管理のために支出する事業費(ランニングコスト)の充足状況
      \end{itemize}
    \end{itemize}
  \item
    これをもって付加保険料の合理性、妥当性、公平性が事後的に検証される。
  \end{itemize}
\end{enumerate}

\begin{center}\rule{0.5\linewidth}{0.5pt}\end{center}

\hypertarget{h19-ux554f2-1}{%
\subsubsection{H19 問2 (1)}\label{h19-ux554f2-1}}

平成 18 年 (2006年) 4 月 1
日より保険業法施行規則において、「保険料及び責任準備金の算出方法書」の記載事項を一部削除する改正が実施された。本改正の趣旨・内容について簡潔に説明せよ。

===

\hypertarget{ux89e3ux7b54-h19-ux554f2-1}{%
\subsubsection{解答: H19 問2 (1)}\label{ux89e3ux7b54-h19-ux554f2-1}}

\hypertarget{ux8da3ux65e8}{%
\paragraph{趣旨}\label{ux8da3ux65e8}}

\begin{itemize}
\tightlist
\item
  保険金杜の経営効率化への取組み等の経営努力を保険料に適時適切に反映させる観点

  \begin{itemize}
  \tightlist
  \item
    (保険料のうち保険数理に直接よらない部分)=(事業費)を中心に、

    \begin{itemize}
    \tightlist
    \item
      商品審査を簡素化する
    \item
      充実したモニタリングを行う
    \end{itemize}
  \end{itemize}
\item
  監督の実効性の向上を図り、
\item
  保険料の合理性・妥当性・公平性を確保
\item
  保険商品の価格の弾力化を促進する。
\item
  予定事業費は、事前認可型から事後モニタリング型の監督体制となる。
\end{itemize}

\hypertarget{ux5185ux5bb9}{%
\paragraph{内容}\label{ux5185ux5bb9}}

\begin{enumerate}
\def\labelenumi{\arabic{enumi}.}
\tightlist
\item
  「保険料及び責任準備金の算出方法書」の記載事項より、予定事業費率に関する事項を削除し、予定事業費に係る具体的詳細な記述を不要とする。
\item
  予定事業費の算出方法は社内規定等に定めることとする。
\item
  金融庁が事業費の実績と保険料の関係を把握するために、事後モニタリングとして、

  \begin{itemize}
  \tightlist
  \item
    商品別等に細分化した定期報告を金融庁に提出する。販売経路や保険種類ごとに区分して測定し、これをもって付加保険料の十分性・公平性が事後的に検証される。
  \item
    このモニタリングにおいては、

    \begin{itemize}
    \tightlist
    \item
      事業費のうち特に新契約時にかかる費用(イニシャルコスト)の回収状況、
    \item
      その他契約維持・管理のために支出する事業費(ランニングコスト)の充足状況について、
    \end{itemize}
  \end{itemize}
\end{enumerate}

\begin{itemize}
\tightlist
\item
  なお、本改正は条文上、対象は損害保険業であるが、
\item
  生命保険業では「保険金杜向けの総合的な監督指針」で改正され、それに沿い「保険料及び責任準備金の算出方法書」を改定することとなった
\end{itemize}

\hypertarget{ux4ed8ux52a0ux4fddux967aux6599ux65b9ux5f0fux3068ux305dux306eux8003ux3048ux65b9}{%
\section{1.5
付加保険料方式とその考え方}\label{ux4ed8ux52a0ux4fddux967aux6599ux65b9ux5f0fux3068ux305dux306eux8003ux3048ux65b9}}

\hypertarget{ux8003ux3048ux65b9}{%
\subsection{考え方}\label{ux8003ux3048ux65b9}}

\hypertarget{h29-ux554f31ux2461-h20-ux554f41ux2460-h4ux554f11-h1-ux554f21}{%
\subsubsection{H29 問3(1)②, H20 問4(1)①, H4問1(1), H1
問2(1)}\label{h29-ux554f31ux2461-h20-ux554f41ux2460-h4ux554f11-h1-ux554f21}}

付加保険料の設定に際して留意すべき点を 4
つ挙げ、それぞれ簡潔に説明しなさい。(3つ、として問われる場合は4点目は補足として追加)
\#\#\# 解答: H29 問3(1)②, H20 問4(1)①, H4問1(1), H1 問2(1)

\begin{enumerate}
\def\labelenumi{\arabic{enumi}.}
\tightlist
\item
  十分性
  一定の保険群団の中において、その群団から入る保険料中の付加保険料をもってその群団の運営に必要な経費の全てを賄う必要がある。この十分性が満たされないような保険種類が存在すれば、保険種類間の公平性の問題が生ずる。従って、十分性を考慮して付加保険料を決定する際、将来におけるインフレ懸念や顧客サービスの高度化のためのコストをどのように織り込んでいくかは重要な問題である。
\item
  普遍性・公平性

  \begin{itemize}
  \tightlist
  \item
    普遍性:一つの方式でできるだけ多くの保険種類の付加保険料を矛盾無く表現する普遍性
  \item
    公平性:その一つの方式の中での保険種類間の公平性を確保
  \item
    「普遍性」と「公平性」は相反する関係

    \begin{itemize}
    \tightlist
    \item
      どう調和させるのか、両者のバランスを考慮した設定が重要となる。
    \item
      「普遍性」を重視し、一つの方式(例えばα-β-γ方式)を選択した場合

      \begin{itemize}
      \tightlist
      \item
        保険金額別や払方別等による公平性についても別途検討する必要がある。
      \end{itemize}
    \end{itemize}
  \end{itemize}
\item
  費用主義と効用主義

  \begin{itemize}
  \tightlist
  \item
    費用主義: 付加保険料を実際にかかる経費の型と大きさで賦課しようとする

    \begin{itemize}
    \tightlist
    \item
      特に間接費用をどのように分担させるか(保険種類毎の実際経費の決定において)に関して困難が伴う
    \end{itemize}
  \item
    効用主義:
    保険商品の提供する保障効用や貯蓄効用に比例した付加保険料を課そうとする

    \begin{itemize}
    \tightlist
    \item
      「効用」とは何か、またその指標として何が適当かが問題となる。
    \end{itemize}
  \item
    これらを共に一定程度満足させることが求められる。
  \item
    これらの費用主義と効用主義の是非については、

    \begin{itemize}
    \tightlist
    \item
      そのいずれかに基づいた付加保険料方式を考えるというのではなく、
    \item
      保険金杜における実際支出を十分にコスト分析したうえで、
    \item
      両者をバランスよくミックスさせた付加保険料方式を採用するべきである。
    \end{itemize}
  \end{itemize}
\item
  簡明性・実行可能性

  \begin{itemize}
  \tightlist
  \item
    実務的には、簡明・簡易な方式なほうが望ましい。
  \end{itemize}
\end{enumerate}

\begin{center}\rule{0.5\linewidth}{0.5pt}\end{center}

\hypertarget{ux554f14-h20-ux554f41-ux2461-h10-ux554f22-ux2460-h1-ux554f2-2}{%
\subsubsection{2019 問1(4), H20 問4(1) ②, H10 問2(2) ①, H1 問2
(2)}\label{ux554f14-h20-ux554f41-ux2461-h10-ux554f22-ux2460-h1-ux554f2-2}}

付加保険料方式において、α-β-γ方式の概要、優れている点および問題点について簡潔に説明しなさい。

===

\hypertarget{ux89e3ux7b54-2019-ux554f14-h20-ux554f41-ux2461-h10-ux554f22-ux2460-h1-ux554f2-2}{%
\subsubsection{解答: 2019 問1(4), H20 問4(1) ②, H10 問2(2) ①, H1 問2
(2)}\label{ux89e3ux7b54-2019-ux554f14-h20-ux554f41-ux2461-h10-ux554f22-ux2460-h1-ux554f2-2}}

\hypertarget{ux512aux308cux3066ux3044ux308bux70b9}{%
\paragraph{優れている点}\label{ux512aux308cux3066ux3044ux308bux70b9}}

\begin{itemize}
\tightlist
\item
  普遍性
  同一保険種類の中では、保険期間、加入年齢に無関係な付加保険料の算式であり、その中では普遍性が保たれている。
\item
  簡明性
  数少ないパラメータで付加保険料水準を決定でき、簡明性を満たしている。
\item
  費用主義および効用主義

  \begin{itemize}
  \tightlist
  \item
    主張を共に一定程度満たしている。
  \end{itemize}
\item
  収益管理の容易さ
  新契約費、維持費、集金費の各支出実態と予定事業費の対応が取りやすく、収益管理が容易である。
\item
  配当率設定 利源別配当方式においては、配当率の設定が容易である。
\item
  汎用性 さまざまな保険種類に対し、ある程度汎用的に適用できる。
\end{itemize}

\hypertarget{ux554fux984cux70b9}{%
\paragraph{問題点}\label{ux554fux984cux70b9}}

\begin{itemize}
\tightlist
\item
  費用主義の観点 1
  件あたりの経費が反映されない等、支出実態と乖離する部分がある。
\item
  効用主義の観点 貯蓄効用に対応する付加保険料は
  V(責任準備金)比例が適しているが、反映されない。
\item
  貯蓄性商品
  契約当初の利回りが低下し、商品性の面で問題が生じる場合がある。
\item
  満期を設定しない保険料建の貯蓄性商品等
  S(保険金額)比例の新契約費は馴染まない。
\item
  定期性の強い商品においては
  経過の浅い時期での解約の場合、会社持ち出しが生じ収益の悪化を招きやすい。
\item
  ユニバーサル保険や変額保険 普遍的な適用に無理が生じる場合がある。
\item
  年齢、保険期間によっては保険料の大小比較において矛盾が生じる場合がある。
  例えば、定期性商品で同一年齢では保険期間が短くなるにつれ保険料が高くなるケースがある。
\end{itemize}

\hypertarget{h10-ux554f13}{%
\subsubsection{H10 問1(3)}\label{h10-ux554f13}}

次の①~⑤を適当な語句で埋めよ。
付加保険料を賦課する場合の考え方として、「①」または「実費主義」と「②」の二つがある。①とは、付加保険料を実際にかかる経費の型と大きさで賦課しようというものであるが、保険
種類毎の実際経費の決定においては、特に③をどのように分担させるかに関して困難が伴うところである。また②は、保険商品の提供する「④」および「⑤」に比例した付加保険料を課そうというものである。

===

\hypertarget{ux89e3ux7b54-h10-ux554f13}{%
\subsubsection{解答: H10 問1(3)}\label{ux89e3ux7b54-h10-ux554f13}}

①\ldots 費用主義②\ldots 効用主義③\ldots 間接費用
④\ldots 保障効用⑤\ldots 貯蓄効用(④と⑤は逆順可)

\begin{center}\rule{0.5\linewidth}{0.5pt}\end{center}

\hypertarget{h13-ux554f14}{%
\subsubsection{H13 問1(4)}\label{h13-ux554f14}}

次の①~⑤を適当な語句で埋めよ。
現在日本における付加保険料体系はα-β-γ方式が広く用いられているが、それは次の点が優れているからである。
(a)
同一保険種類の中では、保険期間、加入年齢に無関係な付加保険料の算式であり、その中では①が保たれている。
(b) 数少ないパラメータで付加保険料水準を決定するという②がある。 (c)
③および効用主義の主張を共に一定限度満足している。 (d)
新契約費、維持費、集金費のそれぞれの支出実態と予定事業費の対応がとりやすく、④が容易である。
(e) ⑤配当方式においては、配当率の設定が容易である。

===

\hypertarget{ux89e3ux7b54-h13-ux554f14}{%
\subsubsection{解答: H13 問1(4)}\label{ux89e3ux7b54-h13-ux554f14}}

①普遍性②簡明性③費用主義④収益管理⑤利源別

\begin{center}\rule{0.5\linewidth}{0.5pt}\end{center}

\hypertarget{h12-ux554f15}{%
\subsubsection{H12 問1(5)}\label{h12-ux554f15}}

次の①~⑤を適当な語句で埋めよ。
付加保険料を賦課する場合の考え方として、「①」と「効用主義」の二つがある。
①とは、付加保険料を実際にかかる経費の型と大きさで賦課しようというものであるが、特に
②を保険種類毎にどのように分担させるかに関して困難が伴うところである。また効用主義は、保険商品の提供する「③」ならびに「④」に比例した付加保険料を課そうというものである。α-β-γ方式において、S
比例ローディングは③に、 ⑤は④に対応していると考えられる。

===

\hypertarget{ux89e3ux7b54-h12-ux554f15}{%
\subsubsection{解答: H12 問1(5)}\label{ux89e3ux7b54-h12-ux554f15}}

①費用主義(実費主義)②間接費用③保障効用④貯蓄効用⑤P比例ローディング

\begin{center}\rule{0.5\linewidth}{0.5pt}\end{center}

\hypertarget{h16-ux554f11}{%
\subsubsection{H16 問1(1)}\label{h16-ux554f11}}

x
歳加入、年払込ステップ払(m\textgreater10)の終身保険において保険料上昇後の保険金額
1
に対する月払営業保険料率の算式を記せ。なお、ここでのステップ払の保険料率とは加入後
10 年間の保険料率を 1 と した場合、その後の保険料率を l.5
とするものである。
なお、二見隆著「生命保険数学(上下)」に使用されている記号については何ら断ることなく使用して構わないが、予定事業費率を表す記号としては次のものを使用すること。
α:予定新契約費率(契約初年度に発生、保険金額比例)
β:予定集金費率(保険料払込期間中払込毎に発生、保険料比例)
γ:保険料払込期間中における予定維持費率(毎年の費用として保険料の払込毎に発生、保険金額比例)
γ':保険料払込期問満了後における予定維持費率(毎年の費用として保険料払込期間満了後の各契約年度始に発生、保険金額比例)
なお、これらの記号により難いときは必要な注釈を記すこと。

===

\hypertarget{ux89e3ux7b54-h16-ux554f11}{%
\subsubsection{解答: H16 問1(1)}\label{ux89e3ux7b54-h16-ux554f11}}

\[
\frac{1.5}{12}
\frac{\Ax*{x}+\alpha+\gamma \ax**{\endow{x}{m}}[(12)]+\gamma' \ax**[m|]{x}}{(\ax**{\endow{x}{10}}[(12)]+1.5\cdot\ax*[10|]{\endow{x}{m-10}}[(12)])(1-\beta)}
\]

\begin{center}\rule{0.5\linewidth}{0.5pt}\end{center}

\hypertarget{h9-ux554f21-h2-ux554f1-1}{%
\subsubsection{H9 問2(1), H2 問1 (1)}\label{h9-ux554f21-h2-ux554f1-1}}

フロント・エンド・ロードとバック・エンド・ロードについて簡潔に説明せよ。

===

\hypertarget{ux89e3ux7b54-h9-ux554f21-h2-ux554f1-1}{%
\subsubsection{解答: H9 問2(1), H2 問1
(1)}\label{ux89e3ux7b54-h9-ux554f21-h2-ux554f1-1}}

\begin{itemize}
\tightlist
\item
  フロント・エンド・ロード
  契約初年度に多額の募集手数料を支払うために初年度付加保険料を大きくするものである。これは必然的に契約者に対する利回りが悪化する。\\
\item
  バック・エンド・ロード
  アメリカにおいて、ユニバーサル保険が増え、販売各社間での競争が激化するに伴い、出てきた
  契約当初はノー・ロードとしてキャッシュ・バリューを高め、解約ないし一部引出時に解約控除の形で経費を徴収する
\end{itemize}

\begin{center}\rule{0.5\linewidth}{0.5pt}\end{center}

\hypertarget{ux55b6ux696dux4fddux967aux6599ux306eux8a08ux7b97}{%
\section{1.6
営業保険料の計算}\label{ux55b6ux696dux4fddux967aux6599ux306eux8a08ux7b97}}

\hypertarget{ux9ad8ux984dux5272ux5f15}{%
\subsection{高額割引}\label{ux9ad8ux984dux5272ux5f15}}

\hypertarget{h9-ux554f3-2-ux2460}{%
\subsubsection{H9 問3 (2) ①}\label{h9-ux554f3-2-ux2460}}

米英で一般的に用いられている高額割引の2方式を説明せよ。

===

\hypertarget{ux89e3ux7b54-h9-ux554f3-2-ux2460}{%
\subsubsection{解答: H9 問3 (2)
①}\label{ux89e3ux7b54-h9-ux554f3-2-ux2460}}

\begin{enumerate}
\def\labelenumi{\arabic{enumi}.}
\tightlist
\item
  Band method:
  保険金額を幾つかの区分に分け、その区分ごとの平均保険金額に基づき単位保険金額当たりの保険料を設定する方式
\item
  Policy fee method:
  単位保険金額当たりの率から求める額に、契約1件当たりの経費(Policy
  fee)を加えて保険料を算出する方式
\end{enumerate}

\hypertarget{ux4fddux967aux6599ux3092ux5de1ux308bux8b70ux8ad6}{%
\section{1.8
保険料を巡る議論}\label{ux4fddux967aux6599ux3092ux5de1ux308bux8b70ux8ad6}}

\hypertarget{ux4fddux967aux6599ux7387ux306eux7d30ux5206ux5316}{%
\subsection{保険料率の細分化}\label{ux4fddux967aux6599ux7387ux306eux7d30ux5206ux5316}}

\hypertarget{h30-ux554f3-2-ux2460-h15-ux554f4-1-ux2460}{%
\subsubsection{H30 問3 (2) ①, H15 問4 (1)
①}\label{h30-ux554f3-2-ux2460-h15-ux554f4-1-ux2460}}

保険料率の細分化の際に考底すべき「公平性」について説明しなさい。

===

\hypertarget{ux89e3ux7b54-2018h30-ux554f3-2-ux2460-h15-ux554f4-1-ux2460}{%
\subsubsection{解答: 2018(H30) 問3 (2) ①, H15 問4 (1)
①}\label{ux89e3ux7b54-2018h30-ux554f3-2-ux2460-h15-ux554f4-1-ux2460}}

\begin{itemize}
\tightlist
\item
  保険制度維持には、保険料負担にあたって被保険者間公平な取り扱い、
\item
  『保険技術的公平性』

  \begin{itemize}
  \tightlist
  \item
    同一の保険料で保障される被保険者集団は同一の危険性を有するべき
  \item
    適切なリスク区分に応じて料率が区分されるべきである
  \item
    保険料率が不適切になる危険を軽減するためリスク管理上も大切なこと
  \item
    リスクの均質化が現実には完全には不可能

    \begin{itemize}
    \tightlist
    \item
      保険技術的な公平性は実務において完全には達成されない点に留意する必要がある。
    \end{itemize}
  \end{itemize}
\item
  『社会的公平性』

  \begin{itemize}
  \tightlist
  \item
    保険料負担能力の面からの社会的な容認
  \item
    社会的コンセンサスに合致した料率設定
  \item
    社会的公平性を著しく阻害するものであれば、必ずしも容認されない場合もある

    \begin{itemize}
    \tightlist
    \item
      保険技術的公平性の観点から細分化をすすめ、適切な料率設定を行ったとしても、
    \item
      一部の契約者に対して高すぎる保険料を課すなど、

      \begin{itemize}
      \tightlist
      \item
        低リスクグループと高リスクグループ間の保険料格差が拡大して、
      \item
        真に保障を必要としている高リスクグループの保険料率が高くなり過ぎ、保険に加入できなくなるような場合
      \end{itemize}
    \end{itemize}
  \item
    私保険が社会保険を補完する役割を担う公共性の高い事業
  \end{itemize}
\item
  保険事業が社会性・公共性に基づいて行われていることを踏まえ、料率区分を設定する必要がある。

  \begin{itemize}
  \tightlist
  \item
    社会的公平性を確保した上で、
  \item
    保険技術的公平性の観点から適切に細分化を行い、
  \end{itemize}
\item
  社会政策的な観点から保険料率の区分はなんらかの法令等で規制されている
\end{itemize}

\begin{center}\rule{0.5\linewidth}{0.5pt}\end{center}

\hypertarget{h28-ux554f3-2-ux2460-h15-ux554f4-1-ux2461}{%
\subsubsection{H28 問3 (2) ①, H15 問4 (1)
②}\label{h28-ux554f3-2-ux2460-h15-ux554f4-1-ux2461}}

保険料率を区分するにあたって、留意すべき事項を挙げ、それぞれ簡潔に説明しなさい。

===

\hypertarget{ux89e3ux7b54-h28-ux554f3-2-ux2460-h15-ux554f4-1-ux2461}{%
\subsubsection{解答: H28 問3 (2) ①, H15 問4 (1)
②}\label{ux89e3ux7b54-h28-ux554f3-2-ux2460-h15-ux554f4-1-ux2461}}

\begin{itemize}
\tightlist
\item
  同質性
  料率の区分に用いる要素は、結果的に被保険者集団に同質性をもたらすものであること。
\item
  分離の必然性
  その要素を料率区分に使用することによって、実質的にリスクのレベルに差異をもたらすものであること。

  \begin{itemize}
  \tightlist
  \item
    たとえば、喫煙者、非喫煙者の違いは一般的に差異があるとみなされることが多い。
  \end{itemize}
\item
  測定可能性

  \begin{itemize}
  \tightlist
  \item
    実務的に測定可能であり
  \item
    信頼できるものであること
  \item
    又そのための費用があまりかからないこと。
    あまりに費用のかかる医的診査は高額契約以外なじまない。
  \end{itemize}
\item
  定義が明確であること

  \begin{itemize}
  \tightlist
  \item
    そのクラスに属することが明確に定義されること。
  \item
    契約当事者双方で納得が得られるものであることが望ましい。
  \item
    反例

    \begin{itemize}
    \tightlist
    \item
      非喫煙者の定義はこの点で若干あいまいさが残る。
    \item
      また、「適度な運動を行っていること」などはこの要件に反していると考えられる。

      \begin{itemize}
      \tightlist
      \item
        ただし、最近普及し始めているウェアラブル端末の発達などにより、将来は要件を満たすようになるかもしれない。
      \end{itemize}
    \end{itemize}
  \end{itemize}
\item
  将来に向けて予測可能であること

  \begin{itemize}
  \tightlist
  \item
    生命保険契約は長期にわたる契約であるものの
  \item
    保険加入時点の情報に基づいて保険料率を決定していること
  \item
    「将来の予測可能性」が薄い(リスクの差異が推定されるものの)

    \begin{itemize}
    \tightlist
    \item
      居住地域など。料率区分要素としては一般には採用されていない。
    \end{itemize}
  \end{itemize}
\item
  危険を減少させるインセンティブとなること

  \begin{itemize}
  \tightlist
  \item
    その要素の使用が被保険者にとってリスクを減少させるようなインセンティブをもたらす
  \item
    モラルリスクを排除する趣旨からもこの要件は重要である。
  \item
    個人年金で喫煙者割引を導入することは、被保険者自らが健康を害することに保険会社がインセンティブを与えることになり、問題があろう。
  \end{itemize}
\item
  制御可能性

  \begin{itemize}
  \tightlist
  \item
    各被保険者が帰属するその要素は意図的にコントロールできること。

    \begin{itemize}
    \tightlist
    \item
      ガン遺伝などの要素は制御不能ともいえ、この要件を満たさないかもしれない。
    \end{itemize}
  \end{itemize}
\item
  社会的に容認されること

  \begin{itemize}
  \tightlist
  \item
    その区分が社会的に容認されるようなものであること。

    \begin{itemize}
    \tightlist
    \item
      身体障害を加入不可にするなどは社会的要請に照らして検討されるべき
    \item
      対応は各国の社会要請や時代によって異なると考えられる。
    \end{itemize}
  \end{itemize}
\end{itemize}

\begin{center}\rule{0.5\linewidth}{0.5pt}\end{center}

\hypertarget{h10-ux554f2-3-ux2460}{%
\subsubsection{H10 問2 (3) ①}\label{h10-ux554f2-3-ux2460}}

年齢、性別等の保険料率設定パラメータにおいて、具備しなければならない要件について簡潔に説明せよ。

===

\hypertarget{ux89e3ux7b54-h10-ux554f2-3-ux2460}{%
\subsubsection{解答: H10 問2 (3)
①}\label{ux89e3ux7b54-h10-ux554f2-3-ux2460}}

\begin{enumerate}
\def\labelenumi{\arabic{enumi}.}
\tightlist
\item
  危険の公平性の保持
  当該パラメータにより区分される群団間において、死亡率に充分な差異があること。
\item
  社会的な容認
  当該パラメータにより保険料を区分することが社会的に容認されること。
\item
  危険の均一性 区分した群団の危険度に大きなバラツキがないこと。
\item
  基準の客観性 パラメータが客観的な基準により測定できること。
\item
  危険選択の簡便性 パラメータの測定、確認が比較的容易に行えること。
\item
  被保険群団の大きさ
  区分された被保険群団がある程度の大きさを有していないと、費用対効果の観点から区分の効果が小さくなる。また、支払実績の把握等の観点からも大数の法則が早期に成り立つ規模の被保険群団が形成されることが望まれる。
\end{enumerate}

\begin{center}\rule{0.5\linewidth}{0.5pt}\end{center}

\hypertarget{h22-ux554f2-1}{%
\subsubsection{H22 問2 (1)}\label{h22-ux554f2-1}}

保険料の細分化の根拠となり得る「公平性」について説明するとともに、料率の区分にあたって留意すべき点について簡潔に説明しなさい。

===

\hypertarget{ux89e3ux7b54-h22-ux554f2-1}{%
\subsubsection{解答: H22 問2 (1)}\label{ux89e3ux7b54-h22-ux554f2-1}}

\begin{itemize}
\item
  \textbf{保険制度を維持}するためには、
\item
  保険料の負担にあたって\textbf{被保険者間において公平な取り扱い}が行われることが要請される。
\item
  \textbf{『保険技術的公平性』}を確保する観点から\textbf{保険料の細分化}は行われる。

  \begin{itemize}
  \tightlist
  \item
    理論的には、\textbf{同一の保険料}で保障される被保険者集団は\textbf{同一の危険性}を有するべきことから、

    \begin{itemize}
    \tightlist
    \item
      \textbf{適切なリスク区分に応じて料率が区分されるべき}であり、
    \end{itemize}
  \item
    ただし、細分化によっても\textbf{リスクの均質化が現実には完全には不可能}なことから、
  \item
    保険技術的な公平性は\textbf{実務において完全には達成されない}点に留意する必要がある。
  \end{itemize}
\item
  \textbf{『社会的公平性』}についても留意する必要がある。

  \begin{itemize}
  \tightlist
  \item
    保険技術的公平性の観点から細分化をすすめ、適切な料率設定を行ったとしても、
  \item
    一部の契約者に対して高すぎる保険料を課すなど、
  \item
    \textbf{社会的公平性を著しく阻害}するものであれば、必ずしも容認されない場合もあり、
  \item
    \textbf{社会的コンセンサス}に合致した料率設定であるかといった観点からの検討も不可欠である。
  \item
    これは\textbf{私保険が社会保険を補完}する役割を担う\textbf{公共性の高い事業}であることからの要請でもある。
  \end{itemize}
\end{itemize}

このように保険事業が\textbf{社会性・公共性}に基づいて行われていることを踏まえ、\textbf{社会的公平性を確保}した上で、\textbf{保険技術的公平性の観点から適切に細分化}を行い、料率区分を設定する必要がある。
料率の区分においては、\textbf{プライバシー保護の確保}に関して、\textbf{契約者・被保険者の信頼}を得ることを前提として、以下のような事項に留意すべきと考えられる。

\begin{enumerate}
\def\labelenumi{\arabic{enumi}.}
\tightlist
\item
  同質性
  料率を区分することで、結果として被保険者集団に同質性をもたらすものであること
\item
  分離の必然性
  その要素を使用することで、リスクのレベルに違いをもたらすような分離の必然性があること
\item
  測定可能性 実務的に測定可能であり、信頼できるものであること
\item
  定義の明確性 そのリスク区分に属することが明確に定義されること。
\item
  予測可能性
  保険加入時点の情報に基づいて保険料率を決定していることから、将来に向けて予測可能な料率区分であること
\item
  危険を減少させるインセンティブ
  その要素を使用することで、被保険者に、リスクを減少させるインセンティブをもたらすこと
\item
  制御可能性 被保険者にとって制御可能なリスク要素であること
\item
  社会的容認 料率区分が社会的に容認されるような区分であること
\end{enumerate}

\begin{center}\rule{0.5\linewidth}{0.5pt}\end{center}

\hypertarget{h25-ux554f2-2}{%
\subsubsection{H25 問2 (2)}\label{h25-ux554f2-2}}

保険料率を区分するにあたって、料率区分要素が満たすべき要件について簡潔に説明しなさい。また、個人保険にいて保険料率を男女別に設定することについて、それらを踏まえて説明しなさい。

=== \#\#\# 解答: H25 問2 (2)

\begin{itemize}
\tightlist
\item
  料率区分要素が満たすべき要件

  \begin{enumerate}
  \def\labelenumi{\arabic{enumi}.}
  \tightlist
  \item
    同質性
    料率を区分することで、結果的に被保険者集団に同質性をもたらすものであること。
  \item
    分離の必然性
    その要素を料率区分に使用することによって、実質的にリスクのレベルに差異をもたらすものであること。
  \item
    測定可能性
    実務的に測定可能であり、信頼できるものであること。また、そのための費用があまりかからないこと。
  \item
    定義の明確性 そのリスク区分に属することが明確に定義されること。
  \item
    予測可能性
    保険加入時点の情報に基づいて保険料率を決定していることから、将来に向けて予測可能な料率区分であること。
  \item
    危険を減少させるインセンティブ
    その要素の使用が、被保険者に、リスクを減少させるインセンティブをもたらすこと。
  \item
    制御可能性
    被保険者にとって意図的にコントロールできるリスク要素であること。制御できない区分は不公平と考えられる傾向がある。
  \item
    社会的容認 料率区分が社会的に容認されるようなものであること。
  \end{enumerate}
\item
  男女別の保険料率とすること

  \begin{itemize}
  \tightlist
  \item
    性別によって死亡率や平均寿命に有意な差異があることは統計データから明確であり、また契約後に性別を変更することは困難であることから、「1.
    同質性」、「2. 分離の必然性」、「5.
    予測可能性」の観点からは合理性がある。
  \item
    性別は定義が明確、かつその確認が容易であることから、「4.
    定義の明確性」、「3. 測定可能性」が認められる。
  \item
    一方、被保険者が意図的に性別を変更することは困難であることから、「6.
    危険を減少させるインセンティブ」、「7. 制御可能性」は認められない。
  \item
    「8.
    社会的容認」については、現在の日本においては、男女別に保険料率を区分することは社会的に容認されていると考えられる。しかし、海外においては、男女別の保険料率が禁止されている国や地域もあり、必ずしも普遍的に容認されているものではない。
  \end{itemize}
\end{itemize}

\begin{center}\rule{0.5\linewidth}{0.5pt}\end{center}

\hypertarget{h23-ux554f1-2}{%
\subsubsection{H23 問1 (2)}\label{h23-ux554f1-2}}

個人保険における保険料率の設定にあたり、年齢・性別によらずに危険発生率を一律とする場合がある。生命保険商品において年齢・性別によらずに危険発生率を一律に設定する場合に想定される問題点について述べた上で、どのような場合に容認されるかについて、簡潔に説明しなさい。

===

\hypertarget{ux89e3ux7b54-h23-ux554f1-2}{%
\subsubsection{解答: H23 問1 (2)}\label{ux89e3ux7b54-h23-ux554f1-2}}

\hypertarget{ux4e00ux5f8bux306eux767aux751fux7387ux306eux554fux984cux70b9}{%
\paragraph{一律の発生率の問題点}\label{ux4e00ux5f8bux306eux767aux751fux7387ux306eux554fux984cux70b9}}

\begin{itemize}
\tightlist
\item
  \textbf{全体の保険料収入に不足}をきたす恐れがある。

  \begin{itemize}
  \tightlist
  \item
    年齢・性差により危険発生率に相違があり、
  \item
    被保険者群団の割合が想定から大きく乖離した場合等、
  \end{itemize}
\item
  \textbf{相対的に危険度の高い年齢・性別に契約が集中する}恐れがある。

  \begin{itemize}
  \tightlist
  \item
    保険料に年齢・性差のある他社同様商品と競争した場合、
  \end{itemize}
\item
  実態として加齢により危険発生率の増加が生じている場合、\textbf{責任準備金の不足}が生ずる恐れがある。
\item
  年齢・性別による危険発生率の差異が想定よりも大きい場合、結果的に\textbf{契約者間の公平性を損なう}度合いが高まる。
\end{itemize}

\hypertarget{ux5bb9ux8a8dux3055ux308cux308bux7406ux7531}{%
\paragraph{容認される理由}\label{ux5bb9ux8a8dux3055ux308cux308bux7406ux7531}}

\begin{itemize}
\tightlist
\item
  担保する危険について、\textbf{年齢・性別による違いが認められない}場合
\item
  年齢・性別による違いがある、もしくは、違いがあるかどうかはわからないが、\textbf{当該危険部分が保険商品全体の収支に対する影響が小さい}
  場合
\item
  年齢・性別による違いがあるが、

  \begin{itemize}
  \tightlist
  \item
    \textbf{被保険者群団を事前にある程度予想ができる}、または、
  \item
    保険料を一定程度保守的に設定する等により、\textbf{担保する危険に対して十分な保険料収入が見込まれる}場合(この場合は、有配当契約で、\textbf{配当により年齢・性別による差異の事後調整が可能}であることが望ましい)
  \end{itemize}
\end{itemize}

\begin{center}\rule{0.5\linewidth}{0.5pt}\end{center}

\hypertarget{h18-ux554f2-3}{%
\subsubsection{H18 問2 (3)}\label{h18-ux554f2-3}}

個人保険の保険料率を居住地域別に細分化することについての問題点を、料率区分要素の満たすべき要件を踏まえて説明せよ。

===

\hypertarget{ux89e3ux7b54-h18-ux554f2-3}{%
\subsubsection{解答: H18 問2 (3)}\label{ux89e3ux7b54-h18-ux554f2-3}}

\hypertarget{ux6599ux7387ux533aux5206ux8981ux7d20ux306eux8981ux4ef6ux306bux3064ux3044ux3066}{%
\paragraph{料率区分要素の要件について}\label{ux6599ux7387ux533aux5206ux8981ux7d20ux306eux8981ux4ef6ux306bux3064ux3044ux3066}}

\begin{enumerate}
\def\labelenumi{\arabic{enumi}.}
\tightlist
\item
  同質性
\item
  分離の必然性
\item
  測定可能性
\item
  定義が明確であること
\item
  将来に向けての予測可能性
\item
  危険を減少させるインセンティブとなること
\item
  制御可能性
\item
  社会的に容認されること
\end{enumerate}

\hypertarget{ux5c45ux4f4fux5730ux57dfux5225ux306eux4fddux967aux4e21ux7acbux306bux95a2ux3059ux308bux8003ux5bdfux3068ux8a55ux4fa1}{%
\paragraph{居住地域別の保険両立に関する考察と評価}\label{ux5c45ux4f4fux5730ux57dfux5225ux306eux4fddux967aux4e21ux7acbux306bux95a2ux3059ux308bux8003ux5bdfux3068ux8a55ux4fa1}}

\hypertarget{ux7279ux6bb5ux306eux554fux984cux304cux306aux3044ux89b3ux70b9}{%
\subparagraph{特段の問題がない観点}\label{ux7279ux6bb5ux306eux554fux984cux304cux306aux3044ux89b3ux70b9}}

\begin{itemize}
\tightlist
\item
  \textbf{2. 分離の必然性}の観点からは一定の合理性がある。

  \begin{itemize}
  \tightlist
  \item
    居住する都道府県などにより平均寿命に差異があることは、公表な統計データなどで示されており
  \end{itemize}
\item
  \textbf{3. 測定可能性}および\textbf{4.
  定義が明確であること}が認められる。

  \begin{itemize}
  \tightlist
  \item
    住民登録制度等の利用により居住地の確認は容易に行なえることから、
  \end{itemize}
\item
  \textbf{7. 制御可能}である。

  \begin{itemize}
  \tightlist
  \item
    被保険者の意思により居住地域の変更は可能であることから、
  \end{itemize}
\end{itemize}

\hypertarget{ux6599ux7387ux533aux5206ux8981ux7d20ux3068ux3057ux3066ux306eux63a1ux7528ux304cux56f0ux96e3ux3068ux3055ux308cux308bux89b3ux70b9}{%
\subparagraph{料率区分要素としての採用が困難とされる観点}\label{ux6599ux7387ux533aux5206ux8981ux7d20ux3068ux3057ux3066ux306eux63a1ux7528ux304cux56f0ux96e3ux3068ux3055ux308cux308bux89b3ux70b9}}

\begin{itemize}
\tightlist
\item
  \textbf{1. 同質性}をもたらすとは考えにくい。

  \begin{itemize}
  \tightlist
  \item
    保険加入に際して、低い保険料率が適用される地域に居住地を変更する\textbf{逆選択}が可能
  \item
    居住地域による平均寿命の差異は、\textbf{伝統的な生活習慣の差異}など起因するものと推測される
  \item
    が、\textbf{ライフスタイルが多様化}している現代においては、\textbf{居住地域以上に危険の大小に影響すると思われる要素が多数}存在しており、
  \end{itemize}
\item
  \textbf{6. 危険を減少させるインセンティブ}とはなっていない。

  \begin{itemize}
  \tightlist
  \item
    高い保険料が適用される地域から低い保険料が適用される地域に居住地を変更することが、
  \end{itemize}
\item
  \textbf{5. 将来に向けての予測可能性}が低い。

  \begin{itemize}
  \tightlist
  \item
    将来の居住地を過去の統計データから予測することは困難であり、
  \end{itemize}
\item
  \textbf{3. 測定可能性}の要件になじまない。

  \begin{itemize}
  \tightlist
  \item
    たとえば保険料払い込みの際に居住地域の確認をする等、居住地変更の都度、その事実を保険料に反映することは、費用面や手続き面を考慮すると
  \end{itemize}
\item
  さらには、料率の細分化により、\textbf{大数の法則}が機能するための群団構築が危惧される。
\item
  現在の我が国の生命保険では、居住地域により保険料率を区分した商品は販売されていないため、\textbf{8.
  社会的に容認される}か否かについても疑問が残る。
\end{itemize}

\begin{center}\rule{0.5\linewidth}{0.5pt}\end{center}

\hypertarget{ux751fux4fddux5546ux54c1ux306eux4fa1ux683cux5f3eux529bux6027}{%
\subsection{生保商品の価格弾力性}\label{ux751fux4fddux5546ux54c1ux306eux4fa1ux683cux5f3eux529bux6027}}

\hypertarget{h11-ux554f1-6}{%
\subsubsection{H11 問1 (6)}\label{h11-ux554f1-6}}

次の①~⑤ or (1) -- (5) を適当な語句、数値または算式で埋めよ。
\(p\)をある商品の保険料率、\(f(p)\)を保険料率 \(p\)
に対する新契約高とし、新契約に係る収入保険料が \(p \times f(p)\)
により表されるものとする。このとき、価格弾力性\(E(p)\)は保険料率の変動率に対する新契約高
の変動率として、次のように表される。

\[E(p) =  \cir{1} \]

また、保険料率の変動による新契約に係る収入保険料の変動率は、価格弾力性
\(E(p)\) を用いて、次のように表される。

\[\frac{d\{p \times f(p)\} }{dp}= \cir{2}\]

この式から、 \(E(p) > \cir{3}\) となる場合には、(4)
の引き下げにより、新契約に係る収入保険料は (5) することが分かる。

===

\hypertarget{ux89e3ux7b54-h11-ux554f1-6}{%
\subsubsection{解答: H11 問1 (6)}\label{ux89e3ux7b54-h11-ux554f1-6}}

\begin{enumerate}
\def\labelenumi{(\arabic{enumi})}
\tightlist
\item
  \(-\frac{df(p)}{f(p)} / \frac{dp}{p}\)
\item
  \((1-E(p))\cdot f(p)\)
\item
  1
\item
  保険料率
\item
  増加
\end{enumerate}

\begin{center}\rule{0.5\linewidth}{0.5pt}\end{center}

\hypertarget{h27-ux554f1-4}{%
\subsubsection{H27 問1 (4)}\label{h27-ux554f1-4}}

価格𝑝(\textgreater{}
0)に対する需要が\(f(p)=\frac{100}{p^2}\)、価格𝑝に対する利益率が\(g(p)=\frac{0.2p-2}{p}\)である保険商品を考えたとき、次の①、②の各問に答えなさい。
① この商品の価格弾力性𝐸(𝑝)を算出しなさい。 ②
この保険商品を販売したときに得られる利益が最大となる価格𝑝を算出しなさい。

===

① 2
\[ E(p) = -(df(p)/f(p)) / (dp/p) = -(\frac{-200}{p^3}/\frac{100}{p^2})/(1/p) = 2\]
② 20 P×需要×利益率より、
\[profit(p)=p\cdot\frac{0.2p-2}{p}\cdot\frac{100}{p^2}\]

微分 \[\frac{d}{dp} profit(p) = -\frac{20}{p^2}+\frac{400}{p^3}=0\]
となるpを求めると、\(p=20\)

\hypertarget{ux89e3ux7b54-h27-ux554f1-4}{%
\subsubsection{解答: H27 問1 (4)}\label{ux89e3ux7b54-h27-ux554f1-4}}

\begin{center}\rule{0.5\linewidth}{0.5pt}\end{center}

\hypertarget{h16-ux554f1-6}{%
\subsubsection{H16 問1 (6)}\label{h16-ux554f1-6}}

5 年満期、30 歳加入の定期保険で、保険金額 100
に対する年払営業保険料を𝑝とする。𝑓(𝑝)を価格𝑝に対する新契約件数を表わす需要関数とする。
価格弾力性を求めよ。更に、これを用いて保険料引下げによる収入保険料の影響について説明せよ。
ただし、\(f(p)=\frac{1}{p+p^2}\times 10^4\ \ \  (0.4\leq p \leq 0.6)\)
とする。

===

\hypertarget{ux89e3ux7b54-h16-ux554f1-6}{%
\subsubsection{解答: H16 問1 (6)}\label{ux89e3ux7b54-h16-ux554f1-6}}

\[E(p)=-1\times\frac{df(p)}{f(p)}\times\frac{p}{f(p)}
=\frac{1+2p}{(p+p^2)^2}\times p(p+p^2)
= 1 + \frac{p}{1+p}\]

このとき \(E(p) > 1\)
より、保険料引下げによる収入減の効果より保険料引下げによる新契約増の効果がおおきくなり、保険料を引き下げた方が収入保険料は増加する。

\begin{center}\rule{0.5\linewidth}{0.5pt}\end{center}

\hypertarget{h22-ux554f1-4-wbux3067ux306f3ux3068ux306aux3063ux3066ux3044ux305f}{%
\subsubsection{H22 問1 (4) **
WBでは(3)となっていた}\label{h22-ux554f1-4-wbux3067ux306f3ux3068ux306aux3063ux3066ux3044ux305f}}

保険商品の価格弾力性に関して、次の1.,2.の各問に答えなさい。

\begin{enumerate}
\def\labelenumi{\arabic{enumi}.}
\item
  一般に保障性商品の場合、価格が大幅に安くなったとしても、必要保障額を超えた需要を喚起することは難しいと考えられる一方で、貯蓄性商品の場合、価格が安くなる(利回りが良くなる)と理論的には需要は限りなく拡大することが考えられる。いま、価格pに対する需要関数として𝑓\(f(x)=\frac{1}{p}\)と\(g(p)=\frac{1-p}{1+p}\)(いずれも
  0<p<1)の 2
  つを考えたとき、それぞれの関数は保障性商品と貯蓄性商品のいずれの需要関数を表していると考えられるか、理由を付して述べなさい。
\item ~
  \hypertarget{ux9700ux8981ux95a2ux6570ux304cgpfrac1-p1pux3067ux8868ux3055ux308cux308bux5546ux54c1ux306bux3064ux3044ux3066ux4fa1ux683cux5f3eux529bux6027ux304c-1-ux3088ux308aux5c0fux3055ux304fux306aux308b-p-ux306eux7bc4ux56f2ux3092ux6c42ux3081ux306aux3055ux3044ux89e3ux7b54ux306bux3042ux305fux3063ux3066ux306fux8a08ux7b97ux904eux7a0bux3082ux7c21ux6f54ux306bux8a18ux8f09ux3059ux308bux3053ux3068}{%
  \section{\texorpdfstring{需要関数が\(g(p)=\frac{1-p}{1+p}\)で表される商品について価格弾力性が
  1 より小さくなる p
  の範囲を求めなさい(解答にあたっては、計算過程も簡潔に記載すること)。}{需要関数がg(p)=\textbackslash frac\{1-p\}\{1+p\}で表される商品について価格弾力性が 1 より小さくなる p の範囲を求めなさい(解答にあたっては、計算過程も簡潔に記載すること)。}}\label{ux9700ux8981ux95a2ux6570ux304cgpfrac1-p1pux3067ux8868ux3055ux308cux308bux5546ux54c1ux306bux3064ux3044ux3066ux4fa1ux683cux5f3eux529bux6027ux304c-1-ux3088ux308aux5c0fux3055ux304fux306aux308b-p-ux306eux7bc4ux56f2ux3092ux6c42ux3081ux306aux3055ux3044ux89e3ux7b54ux306bux3042ux305fux3063ux3066ux306fux8a08ux7b97ux904eux7a0bux3082ux7c21ux6f54ux306bux8a18ux8f09ux3059ux308bux3053ux3068}}
\end{enumerate}

\hypertarget{ux89e3ux7b54-h22-ux554f1-4}{%
\subsubsection{解答: H22 問1 (4)}\label{ux89e3ux7b54-h22-ux554f1-4}}

\begin{enumerate}
\def\labelenumi{\arabic{enumi}.}
\tightlist
\item
  f(p) が貯蓄性商品、g(p)が保障性商品 理由
  価格が安くなる場合の振る舞いを見るため、p-\textgreater+0の極限を考えると、
  \[\lim_{p\to+0}f(p)=+\infty, \lim_{p\to+0}g(p)=1\] となるため
\item
  \[E(p)=-\frac{dg(p)}{dp}\cdot\frac{p}{g(p)}=\frac{2p}{1-p^2}\]なので、
  \[E(p) < 1 \Leftrightarrow\frac{2p}{1-p^2}\Leftrightarrow p < \sqrt{2}-1\]
  よって、求めるpの範囲は、\(0 < p < \sqrt{2}-1\)
\end{enumerate}

\begin{center}\rule{0.5\linewidth}{0.5pt}\end{center}

\end{document}
