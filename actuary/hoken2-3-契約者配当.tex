\documentclass[report,gutter=10mm,fore-edge=10mm,uplatex,dvipdfmx]{jlreq}

\usepackage{lmodern}
\usepackage{amssymb,amsmath}
\usepackage{ifxetex,ifluatex}
\usepackage{actuarialsymbol}
\usepackage[]{natbib}
\RequirePackage{plautopatch}

% maru suji ① etc.
\usepackage{tikz}
\newcommand{\cir}[1]{\tikz[baseline]{%
\node[anchor=base, draw, circle, inner sep=0, minimum width=1.2em]{#1};}}

\usepackage{comment}

\begin{comment}

\ifnum0\ifxetex1\fi\ifluatex1\fi=0 % if pdftex
  \usepackage[T1]{fontenc}
  \usepackage[utf8]{inputenc}
  \usepackage{textcomp} % provide euro and other symbols
\else % if luatex or xetex
  \usepackage{unicode-math}
  \defaultfontfeatures{Scale=MatchLowercase}
  \defaultfontfeatures[\rmfamily]{Ligatures=TeX,Scale=1}
\fi
% Use upquote if available, for straight quotes in verbatim environments
\IfFileExists{upquote.sty}{\usepackage{upquote}}{}
\IfFileExists{microtype.sty}{% use microtype if available
  \usepackage[]{microtype}
  \UseMicrotypeSet[protrusion]{basicmath} % disable protrusion for tt fonts
}{}
\makeatletter
\@ifundefined{KOMAClassName}{% if non-KOMA class
  \IfFileExists{parskip.sty}{%
    \usepackage{parskip}
  }{% else
    \setlength{\parindent}{0pt}
    \setlength{\parskip}{6pt plus 2pt minus 1pt}}
}{% if KOMA class
  \KOMAoptions{parskip=half}}
\makeatother
\usepackage{xcolor}
\IfFileExists{xurl.sty}{\usepackage{xurl}}{} % add URL line breaks if available
\IfFileExists{bookmark.sty}{\usepackage{bookmark}}{\usepackage{hyperref}}
\hypersetup{
  hidelinks,
  pdfcreator={LaTeX via pandoc}}
\urlstyle{same} % disable monospaced font for URLs
\usepackage{longtable,booktabs}
% Correct order of tables after \paragraph or \subparagraph
\usepackage{etoolbox}
\makeatletter
\patchcmd\longtable{\par}{\if@noskipsec\mbox{}\fi\par}{}{}
\makeatother
% Allow footnotes in longtable head/foot
\IfFileExists{footnotehyper.sty}{\usepackage{footnotehyper}}{\usepackage{footnote}}

\end{comment}
%\makesavenoteenv{longtable}
\setlength{\emergencystretch}{3em} % prevent overfull lines
\providecommand{\tightlist}{%
  \setlength{\itemsep}{0pt}\setlength{\parskip}{0pt}}
\setcounter{secnumdepth}{-\maxdimen} % remove section numbering

\author{kazuyoshi}
\date{}

\newcommand{\problem}[1]{\subsubsection{#1}\setcounter{equation}{0}}
%\newcommand{\answer}[1]{\subsubsection{#1}}
\newcommand{\answer}[1]{\subsubsection{解答}}

%Pdf%\newcommand{\wakumaru}[1]{\framebox[3zw]{#1}}
\newcommand{\wakumaru}[1]{#1}






\begin{document}
\chapter{保険2第3章 契約者配当}
\section{3.1 序文}
\section{3.2 生命保険会社の利益と契約者配当}
\subsubsection{H20 生保2問題 3(1)①イ}
契約者配当を行う理由について簡潔に説明せよ。
\subsubsection{解答}
\paragraph{◇安全性の原則(安全割増の還元)}
保険料計算に用いる基礎率を高めに設定して、保険債務履行のために保険料の十分性を確保して
いる。このため、事後的に契約者(社員)配当を支払うことで、実費主義の理念にもとづき可及的
に安い費用で保障を提供する。契約からの時間経過とともに会社の実際上の経営諸効率が判明する
につれて、これらの事前に設定した諸率との水準差を契約者配当として還元することで、契約者間
の公平性を保つ必要がある。
\paragraph{◇保険料率の調整}
契約年齢別に保険料を細分化することや、経過年別の費消実態を反映した付加保険料(予定事業
費)・選択表を保険料計算に用いることが事務的に負担となる場合、保険料率あるいは計算基礎は経
過年別要素を反映せず、社員(契約者)配当を行うことで、配当による保険料の事後精算を行う必
要性が生じる。
\paragraph{◇経験料率の採用}
信頼できる統計データが不足している場合や、適切な危険率の測定が困難な場合、真の危険率に
料率を補正するために契約者(社員)配当を行う。

\subsubsection{演習問題 3.2.1}
生命保険における契約者配当は、高度な経営技術課題の一つといわれる。この理由を説明せよ。
(P.3-4〜5)
\subsubsection{解答}
\begin{enumerate}
 \item 長期性に基づく収支構造の理解
\begin{itemize}
 \item 収入と支出にタイムラグが生じるという期間構造
\begin{itemize}
 \item 費差収支面では、収入は平準化している一方で、支出は契約初期
に多く生じるため、契約初期は費差収支がマイナスとなることがある
 \item 危険差収支面では、選択効果を提えると、この効果は経過に応じて薄れていくため、危
険差収支が経過につれて減少することがある
\end{itemize}
 \item 予定基礎率は確率論・統計手法などを用いた予測のもとで算定される
\begin{itemize}
 \item 消滅時に初めて、設定した予定基礎率に応じた収支
が確定する
\end{itemize}
\end{itemize}
 \item 剰余の適正配分\\
収支構造を踏まえた上で、決算の剰余を、①契約者還元(契約者配当、社員配当)
②株主還元(株主配当)③社内留保(ソルベンシー確保)④社会還元(税金等)
に適正に配分する必要があること。
 \item 公平性と実務負荷のパランス\\
具体的な配当率は、契約者間の公平性に重点を置き、また事務処理の簡便性も考
慮する必要があること。
 \item 商品・価格政策\\
契約者配当は募集文書パンフレットにおける表示のように、商品政策、価格政策
の要素となり得ること。
 \item 多様化する収支構造の把握\\
近年においては、収支構造が多様化(特に、運用手法の多様化、第3分野商品の
開発による保険引受リスクの多様化)することにより複雑になり、その把握負荷が
大きくなってきていること。
\end{enumerate}
\subsubsection{演習問題 3.2.2}
契約者に契約者配当を還元するにあたって、決算利益の特性、契約者配当の特性の視点から留意す
べき点をあげ、配当財源をどのように決定するか説明せよ。
(P.3-6~12)
\subsubsection{解答}
\begin{enumerate}
 \item 決算利益の特性からの留意点
\begin{enumerate}
 \item 契約者配当財源との相違\\
決算利益の二部分
\begin{itemize}
 \item 即座に契約者に還元すべき部分
 \item 会社に留保しておくべき部分: 将来のソルベンシー確保と契約者配当
の向上・安定性維持のため。
\end{itemize}
保険業法での相互会社の分配財源
\begin{enumerate}
 \item 分配剰余の上限としての配当制限(いわゆる配当可能利益)、
 \item 当年度発生の
剰余から社員配当準備金に繰入れるべき最低限の水準を定める規定(いわゆる配当繰入下限)
\end{enumerate}
 \item 契約群団利益との相違
\begin{itemize}
 \item 各契約群団にとつての実際の利益は、責任準備金の積立方式とは独立に定まる
 \item 決算時の利益は責任準備金の関数であり、
 \item 契約期間を通じて両者は一致する
 \item (ことを考えれば、) 契約途中での決算時の利益は、あくまでも一つの責任準備金の
積立方式による評価数値でしかない
\end{itemize}
 \item 生命保険契約の長期性
\begin{itemize}
 \item 決算時の利益は、単年度における利益であって、
 \item 各契約群団にとっての真の利益はその契約群団が完全消滅して初めて判明するものである。
 \item 生命保険契約は長期のものであるから、契約期間中の特定の年度だけを取り上げた単年度の利益だ
けを見ても、当該契約の真の会社への利益の貢献度は判明しない。
\end{itemize}
 \item 配当開始期
\begin{itemize}
 \item 契約者配当に対応する財源は、保険年度式の利益が基本となるが、
 \item 実際に決算において把握できるのは事業年度式の利益である。
 \item 契約者配当とその財源となる利益の期間対応に留意しておく必要がある。
 \item 3年目配当の場合、利益が生まれた事業年度は、その決算で割り当てる契約者
配当に対応する保険年度から平均して半年遅れている。
 \item また、満期等の消滅時には、2年分の財源を準備することが必要となる点も留意が必要である。
\end{itemize}
 \item 決算利益の不安定性\\
決算利益自体、契約者配当に直接影響させることが適当でない様々な要因によ
って変動する可能性があり、安定的とならない一面を有している。例えば、以下
が挙げられる。
\begin{enumerate}
 \item 新契約が高進展した年度における(現行の利益の評価方式の下での)利益の圧迫
 \item 地震、災害等の異常危険の発生による損失の発生
 \item 経済環境の急激な変化による財務収益の大きな変動
 \item 会計基準等を含めた法令等の変更による収支の変動
\end{enumerate}
\end{enumerate}
 \item 契約者配当の特性からの留意点
\begin{enumerate}
 \item 契約者配当の公平性\\
契約者配当の特性を考える上での最大のポイントは、契約者間の公平性という
ことになるが、公平性の概念は極めて幅広く、柔軟なものと考えられる。
\begin{enumerate}
 \item 契約者間の公平性といっても、会社経営の健全性の確保と契約者全体の利益確保がこれに優先される。
 \item 同一世代間の公平性と世代を超えた公平性が求められるが、両者は相反す
る一面を有している。(例えば、初年度の契約手数料支出と付加保険料収入
のタイムラグによる期間損益への影響の取り扱いなど)
 \item 契約者の納得性が得られる公平性でなければならないが、それは数理的な
公平性と必ずしも一致しない。また、その要求するものは時代とともに変
化する。
 \item 実務上は簡明性が求められるが、それを失ってまで厳密な公平性を追求す
る意義は乏しい。
\end{enumerate}
 \item 契約者配当の安定性維持
\begin{itemize}
 \item 生命保険会社の利益の特質から、毎年の剰余は各種要因で変動する要素を含ん
でいる。
 \item その一方、保険契約は長期にわたる契約であることから、毎年の契約者
配当は安定性が求められるため、その実現は、生命保険契約の長期性を踏まえる
と重要な要素である。
 \item この場合、単年度の利益を全額還元してしまうことは好ましくない。
 \item また、配当方式を安定的に適用することも重要である。
 \item しかし、その長期性ゆえに、環境
変化への対応、及び収支貢献度に応じた適切な還元を行う観点から、
 \item 
同じ方式を用い続けることが適切でなくなることもあり得るであろう。
 \item 変更することとなる
場合には、十分な説明が必要になることから、契約者からも理解されやすい方式
で安定維持を図っていくことも重要であろう。
\end{itemize}
 \item 保険料率の先行指標
\begin{itemize}
 \item 契約者配当は、概算収入された保険料の割戻しという意味合いに加えて、
 \item 将来の保険料率設定の先行指標としての意味合いもある。
 \item 安定的に実施されてきた契約者配当部分は、保険料率見直しの際には、保険料率の引き下げ部分として考慮
されることとなろう。
 \item 従って、保険料率の見直しの時期や水準との関係にも留意をしておく必要はある。
\end{itemize}
 \item 既契約者の保険料の調整機能
\begin{itemize}
 \item  予定基礎率を改定して低料を行い、既契約に保険料改定を遡及せずに、料率差
 を調整配当\footnote{(「3.6.3 調整配当」参照)}という形で事後的に精算する場合、
 \item 一旦、利益という過程を通った料率差に該当する財源から、通常の契約者配当と同
 様に配当割当が行われる。
 \item この調整配当の財源となる利益は、他の利益と区分し て考えることも必要であろう。
\end{itemize}
\end{enumerate}
 \item 契約者配当財源の決定要因
\begin{enumerate}
 \item 責任準備金の評価方法\\
負債の大宗を占める責任準備
金の評価方法によつて、決算時の利益が大きく変動する\\
計算基礎率と積立方式に大きく依存する。\\
決算時の利益を、どのように振り分けるかのバランス
\begin{enumerate}
 \item 計算基礎率
\begin{itemize}
 \item 保険料計算に用いた基礎率と同じ計算基礎率を用いるか、別途の基礎率を用い
るか
 \item 現行実務は、標準責任準備金を積立てている契約に対しては、保険料の計算基
礎と異なる基礎率を用いているケースが多いが、責任準備金の計算基礎を保険料
の計算基礎と一致させることも考えられる。
 \item この場合に留意しなければならない
のはそれが保守的に設定されている必要があるということである。
 \item 責任準備金を手厚く積み立てるべきか、責任準備金の外枠に諸準備金\footnote{(EU諸国のソルベンシ
ー・マージン、フィンランドのイコーリゼーションリザーブ)}
を持つことを義務付けるかは、その国の税制等を含めた諸規制との兼ね合いもあって一概には云えない
 \item どれだけ内部留保を持つべきか、というのは、
将来の見通しにかかわることであるから利益自体を単年度ベースで一意的に定め
ることが難しい。
 \item 経過年数の浅い時点では会社に留保すべき金額は, 会社の実際の効率を反映したアセット・シェア法を
用いて算出される積立金より大きくなり、契約者配当財源が
捻出出来ないという結果を一般的に招来することになる。
 \item また、他方で配当開始
期を極端に遅らせることは解約返戻金の水準とも関連するが、(アセット・シェア
法との違い等から、)不整合をもたらすこともあろう。
\end{itemize}
 \item 積立方式; 例えば、平準純保険料式 or チルメル式のいずれで評価するか?\\
保険業法において、責任準備金の積立ては、積立方式は平準純保式 (計算基礎率は標準利率、標準生命表)\\
実務上は標準責任準備金ベース。健全性の確保が公平性に優先、標準責任準備金制度の目的がソルベンシー確保。
平準純保険料式での評価については、
\begin{itemize}
 \item 新契約については、初年度に経費を必要とする構造となっているため、新
契約が進展すればするほど、利益が圧迫されることになる。このため、契
約の進展構造によっては、毎年の利益が変動する。
 \item 新契約については、自力では平準純保険料式の責任準備金を積み立てるこ
とができないため、過年度の契約群団から利益が流用されることになる。
\end{itemize}
必ずしも望ましい積立方式ではないとの考え方もあった。ただ、多くの会社が平準純保険料式を達成・創設費の繰延(業法第113条)が認められる →課題は大きくない。
\end{enumerate}
 \item ソルベンシー確保\\
危険準備金等の諸準備金(純資産の部の準備金等を含む)をどれだけ確保するか \\
= 標準責任準備金
制度を前提としているので、これらの諸準備金の水準をどのようにするかが大きな要因
 \item 契約者配当の安定性維持・向上
\begin{itemize}
 \item 生命保険契約の長期性・契約者利益の観点
 \item 契約者配当の安定性維持は重要
\begin{itemize}
 \item 会社利益が安定的でない要素を抱えていることを踏まえると、
 \item 単年度の利益を全額還元してしまうことは好ましくない
\end{itemize}
 \item 契約者配当により資産を流動化する
\begin{itemize}
 \item 運用利回りが低下するこ
ととなり、
 \item 将来の利益の増加が見込めなくなる可能性もある。
 \item 具体的には、利益準備金や基金などの水準が決定要因
\end{itemize}
\end{itemize}
 \item 通常配当と特別配当
\begin{itemize}
 \item 生命保険会社の利益の特質から、毎年の剰余は変動する要素を含んでいる一方、
 \item 保険契約は長期にわたる契約であることから、毎年の契約者配当は安定性が求めら
れる。
 \item 従って、毎年の利益の全額のうち契約者配当として還元する残りの部分につ
いては、将来の安定配当のための配当平衡財源として会社に留保しておくべきと考
えられる。
 \item 一方で、この結果として、毎年の利益のうちで毎年の契約者配当には反映されな
い部分が発生することになる。
 \item また、株式含み益のように、毎年の利益には反映さ
れないが、実質的な会社の資産価値増加分として、会社の内部留保的性格を持つも
のとして形成されてくる部分もある。
 \item こうした毎年の通常配当では還元できない利
益の未精算部分や、
 \item 
毎年の通常配当の対象とならなかった部分(キャビタルゲイン
等)を契約者還元する部分として、特別配当が設けられている。
 \item 
この部分について、
資産形成の貢献度と(自契約脱退後の残存契約群団を含めた)契約群団の健全性確
保のための内部留保とのバランスにも留意する必要がある。
\end{itemize}
\end{enumerate}
\end{enumerate}


\subsubsection{H28 生保2問題 2(2)}
決算時の利益を契約者配当として契約者に還元するにあたっての、\underline{決算利益の特性の視点}からの留
意点について、簡潔に説明しなさい。
\subsubsection{解答}
\begin{itemize}
 \item 契約者配当財源との相違
\begin{itemize}
 \item  決算時の利益と契約者配当財源としての分配可能な利益は異なっている。利益には、即座に契
 約者に還元すべき部分と、将来のソルベンシー確保と契約者配当の向上・安定性維持のために
 会社に留保しておくべき部分がある。
\item  生命保険会社において生じた利益は、この利益が概算で収入した保険料の割戻しであるとの考
 え方にたてば、基本的には契約者に帰属する性格のものであると考えることができる。しかし、
 決算時の利益を全てその年度の契約者に還元してしまうことは、その利益の特質から好ましい
 ことではない。すなわち、ソルベンシー確保のために、あるいは、契約者配当の向上・安定性
 維持のために、その一部を会社に留保しておかなければならない。
\end{itemize} 
\item 契約群団利益との相違
\begin{itemize}
 \item  各契約群団にとっての実際の利益は、責任準備金の積立方式とは独立に定まるが、決算時の利
 益は責任準備金の関数であり、契約期間を通じて両者は一致することを考えれば、契約途中で
 の決算時の利益は、あくまでも一つの責任準備金の積立方式による評価数値でしかないとの考
 え方もできる。
\end{itemize} 
\item 生命保険契約の長期性
\begin{itemize}
 \item  決算時の利益は、単年度における利益であって、各契約群団にとっての真の利益はその契約群
 団が完全に消滅して初めて判明するものである。
\item  生命保険契約は長期のものであるから、契約期間中の特定の年度だけを取り上げた単年度の利
 益だけを見ても、当該契約の真の会社への利益の貢献度は判明しない。
\end{itemize} 
\item 配当開始期
\begin{itemize}
 \item  契約者配当に対応する財源は、保険年度式の利益が基本となるが、実際に決算において把握で
 きるのは事業年度式の利益である。
\item  3年目配当の場合、利益が生まれた事業年度は、その決算で割り当てる契約者配当に対応する
 保険年度から平均して半年遅れている。契約者配当とその財源となる利益の期間対応に留意し
 ておく必要がある。
\item  また、満期等の消滅時には、2年分の財源を準備することが必要となる点も留意が必要である。
\end{itemize} 
\item 決算利益の不安定性
\begin{itemize}
 \item  決算利益自体、契約者配当に直接影響させることが適当でない様々な要因によって変動する可
 能性があり、安定的とならない一面を有している。たとえば、以下が挙げられる。
\item 新契約が高進展した年度における(現行の利益の評価方式の下での)利益の圧迫
\item 地震、災害等の異常危険の発生による損失の発生
\item 経済環境の急激な変化による財務収益の大きな変動
\item 会計基準等を含めた法令等の変更による収支の変動
\end{itemize}
\end{itemize}

\subsubsection{H7 生保2問題 2(1)}
生命保険会社が公平な契約者配当を実施するにあたり、剰余金の分配に関して留意すべき事項(原
則)について、重要と思われる順に説明せよ。

\subsubsection{解答}

\begin{enumerate}[原則1.\ \ ]
 \item 
 会社の健全性と契約者利益の確保が個々契約者問の厳密な公平性に優先する。\\
 \paragraph{「説明」}
 ある保険種類で赤字を出した場合は他の種類の剰余で補う必要がある。ま
 た、将来にわたる支払能力を確保するためには配当率を一律に削減すること
 も必要になる。
 \item 
 原則1を充足している限りにおいて言十算基礎率の異なっている契約群間で実
 質的な公平性が維持されねばならない。\\
 各契約群はその群団からの剰余で出来る限り常に自立する必要があり、約定
 による債務の履行のための準備金を持たねばならない。
 \paragraph{「説明」}
 例えば、災害給付を行う保険種類では将来の損失に備えて準備金を持つ必
 要がある。また、同一保険種類で計算基礎の異なったものにっいては、将来
 損失の生ずる確率は異なると考えられるので危険準備金の持ち方等について
 工夫が必要である。
 \item 
 各群団内の契約の中では種類、加入年齢、経過年数等を考慮して概略剰余へ
 の寄与に比例して分配されるべきである。\\
 \paragraph{「説明」}
 利源分析、アセットシェア計算等によって剰余への寄与度を把握し、契約
 間の公平性を図る必要がある。
 \item 
 配当に関する契約者の通常持っている期待は上述の原則と矛盾しない範囲内
 でこたえられるべきである。
 \paragraph{「説明」}
 配当金が経過年数毎に上昇するといったことがこれにあたる。
 \item 
 実務的に得られるのは大まかな公平性である。
 \paragraph{「説明」}
 経費の個々契約への配分、利配収入の配分等については、ある程度裁量の
 余地がある。
\end{enumerate}

\section{3.3 保険業法における契約者配当の位置付け}
\subsubsection{演習問題 3.3.1}
保険業法における契約者配当の位置付けについて、保険相互会社と保険株式会社の違いを簡潔に説
明せよ。
\subsubsection{解答}
\paragraph{保険相互会社}
\begin{enumerate}
 \item 保険業法第55条; 基金利息の支払等の制限
\begin{itemize}
 \item 社員配当制限(配当可能利益). 
 \item (対比として) 会社法第461条 債権者保護のための株主配当制限
\end{itemize}
 \item 保険業法第55条の2; 剰余金の分配
\begin{itemize}
 \item 公正かつ衡平; 内閣府令で定める基準. 
 \item 最高決定機関は社員総会(または総代会);「社員自治」が企業理念
 \item 配当繰入下限; 社員である契約者の「自益権」.   \\
新業法施行時 80\% →2002年3月に引き下げ。現在は20\%.(規則第30条の6)
\end{itemize}
 \item 保険業法施行規則第30条の2; 剰余金の分配の計算方法\\
       区分ごとに剰余金の分配の対象となる金額を計算.  
 \item 相互会社における契約者配当原理;  
\begin{itemize}
 \item 実費(at cost)保険. 相互協力. 収支相等の原則. 原則,自己完結型. 
 \item 契約群が全体として再保険的機能を果たす. 会社の内部留保への貢献. 
 \item 剰余への寄与度に応じた配当. コントリビューション原則. 
 \item ただし簡易性・契約者理解・インフレヘッジ性といった、国の経営環境(経済,法制)・民族性→独自の方式
\end{itemize}
\end{enumerate}

\paragraph{保険株式会社}
\begin{enumerate}
 \item 保険業法第114条;契約者配当(相互会社における業法第55条の2に相当)
\begin{itemize}
 \item 配当財源\footnote{ 相互会社では配当可能利益 (業法第55条),  配当繰入下限 (業法第55条の2)}に関する規定はない
 \item 株主配当に対する配当可能利益(会社法の規定)は準用
 \item 公正かつ衡平; 内閣府令で定める基準. 
\end{itemize}
 \item 保険業法施行規則第62条; 契約者配当の計算方法\\
       区分ごとに契約者配当の対象となる金額を計算.  
 \item 契約者配当と株主配当
\begin{itemize}
 \item 契約者配当は契約内容の一部
 \item 契約者配当と株主配当のバランス(内部留保とのバランスは相互会社にもある問題)
\end{itemize}
\end{enumerate}

\paragraph{相互会社と株式会社の契約者配当}
\begin{itemize}
 \item 株式会社では費用性の債務である. 相互会社では剰余金の処分である。
 \item 相互会社では配当可能利益と最低繰入基準が定められている
 \item 株式会社では契約者配当と株主配当とのバランスが一番の問題
\end{itemize}

\subsubsection{H14 生保2問題 2(3)①}
日本における生命保険会社の配当(社員配当、契約者配当)について、配当に係る法令の規制につ
いて簡潔に説明せよ。
\subsubsection{解答}

\paragraph{相互会社と株式会社の違い}
生命保険相互会社の場合は社員配当は剰余金の分配であり、生命保険株式会杜の場合には費用
として契約者配当の分配を行うという違いがあるため法令は別々に規定されているものが多い
が、一部を除いて、本質的には同様の規定となっている。

\paragraph{生命保険相互会社に対する剰余金の分配に関する主な規定}

\begin{enumerate} [*]
\item 保険業法第58条(剰余金の分配)
\begin{itemize}
 \item 相互会社は、施行規則に定める基準に従い、公正かっ衡平な剰余金の分配を行わなければな
 らない。
\item 内閣総理大臣の認可を受け、かつ、定款を変更することにより、社員配当準備金への繰入割
 合を施行規則第29条に定める割合より下回ることができる。
\end{itemize}
\item 施行規則第25条(剰余金の分配の計算方法)
\begin{itemize}
 \item 保険契約の特性に応じて設定した区分ごとに剰余金の分配の対象となる金額を計算し、次の
 いずれかの方法(併用も可)により剰余金の分配を行わなければならない。

\begin{enumerate} [ ]
 \item 
  ①保険料およびその運用収益から保険金・返戻金等の支払給付金、事業費、その他の費用
 等を控除した金額に応じて分配する方法〔アセット・シェア方式〕
 \item 
 ②剰余金の分配の対象となる金額をその発生原因ごとに把握し、それぞれ責任準備金、保
 険金等の基準となる金額に応じて計算し、その合計額を分配する方法[利源別方式]
 \item 
 ③剰余金の分配の対象となる金額を保険期間等により把握し、責任準備金等の基準となる
 金額に応じて計算した金額を分配する方法[損保の配当方式]
 \item 
 ④その他の①〜③に準ずる方法
\end{enumerate}
\end{itemize}
\item 同第26条(積立勘定の設置)
\begin{itemize}
 \item 相互会社は、公正かつ衡平な剰余金の分配をするために、「積立勘定」を設けることができ
 る。(いわゆる損害保険の積立保険に対する規定であるが、生命保険金杜が取扱うことが可
 能な商品もあり、積立勘定の設置が可能である。)
\end{itemize}
\item 同第27条(剰余金のうち一定の比率を乗じる対象となる金額)
\begin{itemize}
 \item 剰余金の処分の対象となる金額を、当期未処分剰余金の額から次の合計額を控除した金額
 (ただし、保険業法第55条第2項に限度額の規定あり。)とすることを規定している。
\begin{enumerate} [ ]
 \item  ①前期繰越剰余金の額
 \item ②任意積立金目的取崩額
\item ③基金利息の支払額
\item ④損失てん補準備金として積み立てる額
\item ⑤基金償却積立金として積み立てる額
\item ⑥基金の償却に充てることを目的として資本の部に積み立てる任意積立金の額
\item ⑦商法第286条の3の規定により貸借対照表の資産の部に計上した金額
\item ⑧資産に時価を付したことにより増加した当期未処分剰余金の額
\item ⑨当該決算の剰余金に含まれる社員配当準備金の取崩額
\end{enumerate}
 \item 同第28条(剰余金の分配をするための準備金)
\begin{itemize}
\item 社員に対する剰余金の分配をするために積み立てる準備金は、①社員配当準備金、および②
 社員配当平衡積立金とする。
\item 社員配当準備金は、社員に対する剰余金の分配をするための準備金として、貸借対照表上、
 負債の部に計上する。
\item 社員配当準備金の積立限度は次の合計額とする。
\begin{enumerate} [ ]
\item   ①積立配当の額
\item  ②未払配当の額(決算期においては翌朝配当所要額を含む。)
\item  ③全件消滅時配当の額
\item  ④その他①〜③に準ずるものとして保険料及び責任準備金の算出方法書に定める方法によ
  り計算した額
\end{enumerate}
\item 社員配当平衡積立金は、社員に対する剰余金の分配の額を安定させることを目的とする任意
 積立金として、貸借対照表上、資本の部に計上する。
\item 社員配当準備金または社員配当平衡積立金を取崩した場合は、取崩額の合計から社員に対す
 る剰余金の分配に充てた金額を控除した残額を社員配当準備金または社員配当平衡積立金
 に積み立てなければならない。ただし、損失のてん補、基金利息の支払い、損失てん補準備
 金の積立または基金償報積立金の積立に充てる場合を除く。
\end{itemize}
 \item 同第29条(積立割合)
\begin{itemize}
 \item 施行規則第27条に定める額の20%以上の額を社員配当準備金又は社員配当平衡積立金
 に積み立てる旨を定款に定めることと規定している。
\end{itemize}
\end{itemize}
\end{enumerate}

\subsubsection{H10 生保2問題 2(1)①}
保険業法第 55 条の2第1項において規定されている「公正かつ衡平な分配」について簡潔に説明せ
よ。
\subsubsection{解答}

保険計理人の配当確認は、保険業法第121条の中で契約者配当又は社員に
対する剰余金の分配が
公正かつ衡平
に行われているかどうかを確認し、
その結果を記載した意見書を取締役会に提出し、その後、その写しを金融再
生委員会に提出しなければならないことが規定されており、法令により義務
づけられた保険計理人の確認業務の一つである。

この \"公正かつ衡平\"
に関して、日本アクチェアリー会が作成した『生命
保険金杜の保険計理人の実務基準』(以下、実務基準と略す)では、以下の
要件を満たすことであると定めている。

\begin{enumerate} [a.]
 \item 責任準備金が適正に積立てられ、かつ、会社の健全性維持のための必
 要額が準備されている状況において、配当所要額が決定されていること
 \item 
 配当の割当・分配が、個別契約の貢献に応じて行われていること
 \item  配当所要額の計算および配当の割当・分配が、適正な保険数理および
 一般に公正妥当と認められる企業会計の基準等に基づき、かつ、法令、
 通達の規定および保険約款の契約条項に則っていること
 \item  配当の割当・分配が、国民の死亡率の動向、市場金利の趨勢などから、
 保険契約者が期待するところを考慮したものであること
\end{enumerate}

公正・衡平な配当を実現するためには、個々の契約の剰余への貢献度に応
じた、いわゆるコントリビューション原則に則した配当の割当・分配を行う
ことが基本となる(b.)。ただし、それ以前に、責任準備金が適正に積み立
てられ必要な内部留保が行われていることが必要不可欠であり、会社の健全
性確保が前提条件となることに注意しなければならない(a.)。

法令では、保険契約の特性に応じて設定した区分ごとに剰余金の分配また
は配当の対象金額を計算し、いわゆるアセット・シェア方式または利源別配
当方式等規定された計算方式によって行うことが定められており(保険業法
施行規則第25条または第62条)、公正・衡平な配当を実現するためには、
これらの基準及びその他金融監督庁長官及び大蔵大臣が定める基準に従って
適正な分配がなされていることが必要となる(保険業法58条、同施行規則
第80条または158条)。

\subsubsection{H29 生保2問題 2(2)、H12 生保2問題 2(2)}
無配当保険について、その意義および利益の取扱いを簡潔に説明しなさい。なお、相互会社に
おける非社員契約の取扱いについて言及する必要はない。
\subsubsection{解答}
(意義)
\begin{itemize}
\item 低廉な保険料の提供\\
 有配当保険では、保険料の計算基礎率に予め安全性を見込み、その結果生じる利益は契約者配
 当で清算するという考え方に立っているが、保険料の計算基礎率に起因する利益が極めて安定し
 ている場合には、計算基礎率をできる限り実勢に近づけ(保険料を低廉化し)、過度な利益が発
 生しないようにすることで、はじめから契約者負担を軽減することができる。また、配当の割当、
 分配等の事務負荷が発生しないため、事業費負担も軽減することができる。
\item 内部留保の充実(契約者配当との関係)\\
 利益(損失)が年によって著しく変動する場合には、各年度の利益を毎期分配するのではなく、
 将来の損失に向けて備えておくべきであると考えられる。すなわち、生命保険会社の利益をすべ
 て契約者に還元するべきではないが、無配当保険においては、契約者配当を支払わないため、利
 益を保険群団の健全性維持や投資効率向上等のために使用可能となる。
\item 株式会社における利益相反への対応\\
 保険株式会社においては、株主の存在も考慮に入れなければならない。株主は、保険会社へ投
 資し、保険契約の履行ができないような不測の事態が起きた場合には、自己の財産が毀損するこ
 とを覚悟する一方で、その投資効果を享受する権利を有していると考えられる。無配当保険によ
 り契約者への配当還元が不要であり、契約者と株主の利益相反を回避できる
\end{itemize}

(利益の取扱い)
\begin{itemize}
\item 基本的には、無配当保険で区分経理して管理し、利益は内部留保しておき、損失が出た際には、
 当該区分に起因する内部留保を取り崩してその補填に充当すべきである。
\item 当該区分に起因する内部留保で補填できないような損失が発生した場合には、全社区分からの補
 填、または、有配当保険の商品区分との間で、一時的な貸借関係を発生させ、以後、当該無配当
 保険区分からの剰余は優先的にこの返済に充当すべきと考えられる。
\item 一方で、無制限に内部留保することは望ましいとは言えない。ある程度の水準を超える場合は、
 保険料率を見直すか、あるいは、その利益を有配当保険の商品区分に属する保険契約の契約者配
 当に流用することも考えられるであろう。
\item その場合でも、無制限に流用されるのではなく、法令等の内容に照らして妥当かどうか、公正か
 つ衡平な配当であるかどうか、また、区分経理等の状況に照らして妥当か等、配慮すべき事項に
 は十分な留意が必要である。
\end{itemize}

\section{3.4 生命保険会社の保険計理人の実務基準}

\end{document} 