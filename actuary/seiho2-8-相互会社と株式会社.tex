\documentclass[report,gutter=10mm,fore-edge=10mm,uplatex,dvipdfmx]{jlreq}

\usepackage{lmodern}
\usepackage{amssymb,amsmath}
\usepackage{ifxetex,ifluatex}
\usepackage{actuarialsymbol}
\usepackage[]{natbib}
\RequirePackage{plautopatch}

% maru suji ① etc.
\usepackage{tikz}
\newcommand{\cir}[1]{\tikz[baseline]{%
\node[anchor=base, draw, circle, inner sep=0, minimum width=1.2em]{#1};}}

\usepackage{comment}

\begin{comment}

\ifnum0\ifxetex1\fi\ifluatex1\fi=0 % if pdftex
  \usepackage[T1]{fontenc}
  \usepackage[utf8]{inputenc}
  \usepackage{textcomp} % provide euro and other symbols
\else % if luatex or xetex
  \usepackage{unicode-math}
  \defaultfontfeatures{Scale=MatchLowercase}
  \defaultfontfeatures[\rmfamily]{Ligatures=TeX,Scale=1}
\fi
% Use upquote if available, for straight quotes in verbatim environments
\IfFileExists{upquote.sty}{\usepackage{upquote}}{}
\IfFileExists{microtype.sty}{% use microtype if available
  \usepackage[]{microtype}
  \UseMicrotypeSet[protrusion]{basicmath} % disable protrusion for tt fonts
}{}
\makeatletter
\@ifundefined{KOMAClassName}{% if non-KOMA class
  \IfFileExists{parskip.sty}{%
    \usepackage{parskip}
  }{% else
    \setlength{\parindent}{0pt}
    \setlength{\parskip}{6pt plus 2pt minus 1pt}}
}{% if KOMA class
  \KOMAoptions{parskip=half}}
\makeatother
\usepackage{xcolor}
\IfFileExists{xurl.sty}{\usepackage{xurl}}{} % add URL line breaks if available
\IfFileExists{bookmark.sty}{\usepackage{bookmark}}{\usepackage{hyperref}}
\hypersetup{
  hidelinks,
  pdfcreator={LaTeX via pandoc}}
\urlstyle{same} % disable monospaced font for URLs
\usepackage{longtable,booktabs}
% Correct order of tables after \paragraph or \subparagraph
\usepackage{etoolbox}
\makeatletter
\patchcmd\longtable{\par}{\if@noskipsec\mbox{}\fi\par}{}{}
\makeatother
% Allow footnotes in longtable head/foot
\IfFileExists{footnotehyper.sty}{\usepackage{footnotehyper}}{\usepackage{footnote}}

\end{comment}
%\makesavenoteenv{longtable}
\setlength{\emergencystretch}{3em} % prevent overfull lines
\providecommand{\tightlist}{%
  \setlength{\itemsep}{0pt}\setlength{\parskip}{0pt}}
\setcounter{secnumdepth}{-\maxdimen} % remove section numbering

\author{kazuyoshi}
\date{}

\newcommand{\problem}[1]{\subsubsection{#1}\setcounter{equation}{0}}
%\newcommand{\answer}[1]{\subsubsection{#1}}
\newcommand{\answer}[1]{\subsubsection{解答}}

%Pdf%\newcommand{\wakumaru}[1]{\framebox[3zw]{#1}}
\newcommand{\wakumaru}[1]{#1}






% \newcommand*{problem}[3]{\subsubsection{#1 生保#2 #3}}
% \newcommand*{answer}{\answer{解答}}

\begin{document}
\chapter{保険2第8章 相互会社と株式会社}
\section{8.1 組織論}
\problem{2020 生保2問題 1(2)}
保険株式会社と保険相互会社の会計について、以下の①~⑤の空欄に当てはまる適切な語句また
は数を記入しなさい。なお、⑤は数で解答しなさい。

保険株式会社においては、契約者配当準備金繰入額は①処理されるが、保険相互会社
の社員配当準備金繰入額は②として扱われる。

保険株式会社は剰余金の配当をする場合、③と④の額を合わせた額が資本金
の額に達するまでは、剰余金の配当により減少する剰余金の額に⑤を乗じて得た額を
③または④として計上しなければならない。
\answer{解答}
① 費用  ② 剰余金処分  ③ 資本準備金  ④ 利益準備金  ⑤ 1/5

※③と④は順不同。
\problem{H27 生保2問題 1(2)}
保険相互会社と保険株式会社の契約者配当について、次の①~⑤に適切な語句を記入しなさい。\\
保険相互会社における契約者配当(社員配当)は、その設立理念である「①」を実現するため
の保険料の割り戻しである。よって、保険業法上も「剰余金の分配」とされ、分配方法は、社員相互
の利益・損失を考慮した「②」な分配が求められている。
「社員自治」が企業理念であるから社員
配当分配方法の最高意思決定機関は③である。\\
保険株式会社が契約者配当を行うこと自体は、④の一部である。契約者配当の決定方法は株式
会社の設立理念から直接帰結するものではなく、保験契約者をどう取り扱うか、という保険会社とし
ての態度から帰結するものである。ここでも保険業法は、保険株式会社が契約者配当を行う場合には、
「②」な分配を求めている。契約者配当準備金繰入額は会計上費用処理され、実質的な契約者配
当分配方法の最高意思決定機関は⑤である。
\answer{解答}
① 実費主義 
② 公正かつ衡平 
③ 社員総会または社員総代会 
④ 契約内容 
⑤ 取締役会

\problem{H14 生保2問題 1(2)}
保険株式会社と保険相互会社の会計上の相違点について、次の①~⑤を適当な語句で埋めよ。\\
基本的には保険株式会社と保険相互会社の会計方式に違いがあるわけではないが、設立理念上、主
に、配当準備金勘定と資本勘定の取扱いに相違が見られる。すなわち、保険株式会社においては、契
約者配当準備金繰入額は①されるが、保険相互会社の社員配当準備金繰入額は②される。勘
定科目を形式的に比較した場合、保険株式会社の資本の部の資本金、資本準備金、利益準備金に対応
するものとして、それぞれ、保険相互会社の資本の部の③、④、⑤がある。
\answer{解答}
① 費用処理 
② 剰余金処分 
③ 基金 
④ 基金償却積立金 
⑤ 損失てん補準備金 
\section{8.2 相互会社組織の今日的課題}
\problem{H25 生保2問題 1(3)}
相互会社の非社員契約に関し、以下の空欄を埋めなさい。
\begin{itemize}
 \item 相互会社は ① のない保険契約に係る保険契約者を、相互会社の社員としない旨を
 ② で定 めることができる。
 \item  非社員契約の引受けは、内閣府令で定める限度を超えてはならない。この限度は、おおむね、
社 員契約と非社員契約からの ③
 の合計額に対する非社員契約からの ③ の割合が ④ 以下 とされている。
 (実際には受再保険契約と出再保険契約の調整を入れる。)
 \item  非社員契約と社員契約とは内閣府令で定める方法で経理を区分しなければならない。
 \item  非社員契約に係る経理については、事業年度における ⑤ の状況を記載した書類を作成し、
事 業年度終了後 ⑥ 以内に金融庁長官に提出しなければならない。
\end{itemize}
\answer{解答}
① 剰余金の分配 ② 定款 ③ 保険料収入 ④ 20/100 ⑤ 収支 ⑥ 4ヶ月
\problem{H10 生保2問題 2(2)①}
非社員契約に関する法令規制について簡潔に説明せよ。
\answer{解答}
\paragraph{1.出題の視点}
\begin{enumerate}
 \item [1)]本間題は、非社員契約から生じる損益の取扱いについて所見を問うもの
 である。
 \item [2)]出題の趣旨としては、
\begin{itemize}
 \item  非社員契約に関する法令規制と当該規制の趣旨
 \item 非社員契約者と社員との法的地位比較
 \item 上記2束を踏まえた上での非社員契約から生じる損益の取扱い
 これらについての理解度および所見を期待した。
\end{itemize} 
 \item [3)]他の問題にも共通するが、枝間として①番を出題しているのは、単に
 部分点を与えるためではなく、②番の出題趣旨をより明確にし、①番の
 内容を踏まえた解答をリードすることが狙いであるため、受験者におい
 ては、この点に十分留意した解答を行うことを今後とも期待する。
\end{enumerate}
\paragraph{2.解答のポイント}
①非社員契約に係わる法令規制について簡潔に説明せよ。\\
当該規制について簡潔にまとめると以下のとおりである。\\
(根拠条文については、後段を参照のこと)
\begin{itemize}
 \item 非社員契約の対象保険種類:剰余金の分配のない保険契約(無配当保
 険)
 \item 量的制限:収入保険料(再保険契約に係わる調整あり)の20%以内
 \item 非社員契約の引受:引受を行う場合は定款で上記2点を規定
 \item 経理の区分:非社員契約に係わる経理を社員契約に係わる経理と区分
 \item 非社員の告知:非社員契約者になる者に対し非社員であることを告げ
 る
 \item 収支の提出:事業年度末における収支の状況を事業年度終了後4月以
 内に金融監督庁長官に提出
\end{itemize}

(法第63条, 規則第33条, 規則第34条, 規則第35条)を参照

\problem{H18 生保2問題 2(2)、H12 生保2問題 2(2)①}
相互会社が有配当保険と無配当保険を併売する場合における留意点を挙げ、公正・衡平な取扱いお
よび収益性・健全性確保の観点を含めて簡潔に説明せよ。
\answer{解答}
\paragraph{相互会社の無配当保険契約の取扱い}

相互会社が無配当保険を販売する場合、その契約を社員権を有する「無配当保険社員」
契約とする考え方と社員権を有さない「非社員」契約とする考え方がある。

しかし、「無配当保険社員」契約とする考え方は、「剰余金分配権」を持たない無配当
保険の契約者に有配当保険の契約者と同様の社員権(共益権)認めることとなり、剰余
金の分配にかかる革決を歪めるという問題があるため、一般的に無配当保険契約は「非
社員契約」として認められてきた。

法令上でも、保険業法第63条第1項において「相互会社は、剰余金の分配のない保
険契約その他の内閣府令で定める種類の保険契約について、当該保険契約に係る保険契
約者を社員としない旨を定款で定めることができる。」とあり、相互会社は、無配当保
険契約を「非社員契約」として定款で定めることができる。

無配当保険契約を「非社員契約」として定める場合、法令上では次の制限がある。

保険業法第63条第3項において「相互会社が行う第1項の保険契約に係る保険の引
受けは、内閣府令で定める限度を越えてはならない。」とあるように、相互会社の無配
当保険の引受けには「限度」が設けられている。具体的には、(再保険契約に係る保険
料を調整した後の、)社員契約と非社員契約からの保険料収入の合計額に対する非社員
契約からの保険料収入の割合が「20%」を越えてはならないと定められている [保険業
法施行規則第33条]。なお、非社員契約に係る保険の引受けの限度についても、定款に
定めることが求められている。

また、保険業法第63条第4項において「相互会社は、第1項の保険契約に係る保険
の引受けをする場合には、内閣府令で定めるところにより、当該保険契約に係る経理を、
社員である保険契約者の保険契約に係る経理と区分してしなければならない」とあるよ
うに、相互会社にあっては、非社員契約の無配当保険には区分経理を行うことが求めら
れている。

\paragraph{公正・衡平な取扱い}

相互会社が無配当保険を販売する場合の公正・衡平な取扱いについて考える場合、「非
社員契約」と「社員契約」との取扱いと、非社員契約である無配当保険から得られた利
益や損失を「社員契約」間で分配する際の取扱いの、両方の観点から考える必要がある。

「前者」について、無配当保険は、事後精算をしないことを前提に、より現実の期待
値に近い率を用いることで安価な保険料を実現することができる。それに対して社員配
当で事後精算を行なう有配当保険の「実費負担」を、どのように公正・衡平な形で設定
するかという点について留意する必要がある。

「後者」については、非社員契約である無配当保険から得られる利益・損益は、最終
的に社員である有配当契約者に帰属するものであり、社員である有配当契約間でどのよ
うに公正・衡平に分配するかという問題もある。しかし、後述のとおり、非社員契約の
無配当保険勘定でのセルフサポートが前提となっている以上、非社員契約の剰余はむや
みに社員契約へ流用するのではなく、非社員契約の勘定内に充分な内部留保を確保する
ことがまず求められる。

\paragraph{収益性・健全性確保}

保険相互会社の無配当保険から得られる利益・損益は、最終的に区分勘定を通じて有
配当保険の契約者に帰属し、有配当保険契約の収益性・健全性にも影響を与える。その
ため、経理の区分は、あくまで剰余金の分配のある有配当契約と、剰余金の分配のない
無配当契約とを相互会社の管理会計上区分して損益の管理及び剰余金の源泉を明確に
するためのものに過ぎないとも考えられる。

一方で、無配当保険の収益性・健全性は、第一義的に無配当保険区分の中で確保され
ること(セルフサポート)が要求されるため、無配当保険から生じた利益をむやみに流
用し有配当保険契約に振替えるのではなく、無配当保険の勘定内で一定の内部留保を行
なうことがまず求められる。

ただし、無配当保険は安価な保険料を提供できる代わりに、保険料に「配当」という
バッファが無く、収益性が安定しないことから、利益留保を意図的に厚くするなど、内
部留保の水準には充分な検討が必要である。

また、無配当保険の販売当初は、無配当保険からの利益だけでは必要なソルベンシー
を確保することができないばかりか、新契約費負担や(基礎率の設定水準によっては)
標準責任準備金に対する積み増し負担により損失が発生すると見込まれるため、自己資
本の財政的圧迫について問題ないか、また自己資本のうち、どの程度無配当保険に割り
振ることができるか確認が必要となる。

そして、無配当保険契約に対する通常の危険を超える危険に対するバッファ財源は、
主に「基金のうち無配当保険契約に割り振られたと想定される部分」、「無配当保険区分
内の内部留保(危険準備金など)」、「併売する有配当保険の剰余(出資)」などから構成
される。

\section{8.3 有配当保険と無配当保険}
\section{8.4 株式会社化と相互会社化(試験範囲外)}

\end{document} 