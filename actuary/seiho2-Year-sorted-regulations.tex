\documentclass[report,gutter=10mm,fore-edge=10mm,uplatex,dvipdfmx]{jlreq}

\usepackage{lmodern}
\usepackage{amssymb,amsmath}
\usepackage{mathtools}
\usepackage{ifxetex,ifluatex}
\usepackage{actuarialsymbol}
\usepackage[]{natbib}

%strike through 
%https://tex.stackexchange.com/questions/23711/strikethrough-text
%\usepackage[]{ulem}

\usepackage[normalem]{ulem}
\usepackage{enumerate}

% Tables
\usepackage{multirow}
\usepackage{tabularx}
%\usepackage{booktabs} % http://www.yamamo10.jp/yamamoto/comp/latex/make_doc/table/table.php

%Framedbox
%https://hakuoku.github.io/agakuTeX/tutorial/5_6framed/
\usepackage{framed}

%https://tgnx8810.wordpress.com/2014/11/29/latex%E3%81%A7%E8%A1%A8%E3%81%AE%E3%82%BB%E3%83%AB%E5%86%85%E6%94%B9%E8%A1%8C%E3%81%AFtabularx%E7%92%B0%E5%A2%83%E3%82%92%E4%BD%BF%E3%81%86%E3%81%A8%E6%A5%BD/
\usepackage{longtable}
\usepackage{booktabs}
\RequirePackage{plautopatch}

% maru suji ① etc.
\usepackage{tikz}
\newcommand{\cir}[1]{\tikz[baseline]{%
\node[anchor=base, draw, circle, inner sep=0, minimum width=1.2em]{#1};}}

%http://yamamo10.jp/yamamoto/comp/latex/make_doc/box/box.php
%枠付き文章
\usepackage{ascmac}
\usepackage{fancybox}

\usepackage{comment}

\begin{comment}

\ifnum0\ifxetex1\fi\ifluatex1\fi=0 % if pdftex
  \usepackage[T1]{fontenc}
  \usepackage[utf8]{inputenc}
  \usepackage{textcomp} % provide euro and other symbols
\else % if luatex or xetex
  \usepackage{unicode-math}
  \defaultfontfeatures{Scale=MatchLowercase}
  \defaultfontfeatures[\rmfamily]{Ligatures=TeX,Scale=1}
\fi
% Use upquote if available, for straight quotes in verbatim environments
\IfFileExists{upquote.sty}{\usepackage{upquote}}{}
\IfFileExists{microtype.sty}{% use microtype if available
  \usepackage[]{microtype}
  \UseMicrotypeSet[protrusion]{basicmath} % disable protrusion for tt fonts
}{}
\makeatletter
\@ifundefined{KOMAClassName}{% if non-KOMA class
  \IfFileExists{parskip.sty}{%
    \usepackage{parskip}
  }{% else
    \setlength{\parindent}{0pt}
    \setlength{\parskip}{6pt plus 2pt minus 1pt}}
}{% if KOMA class
  \KOMAoptions{parskip=half}}
\makeatother
\usepackage{xcolor}
\IfFileExists{xurl.sty}{\usepackage{xurl}}{} % add URL line breaks if available
\IfFileExists{bookmark.sty}{\usepackage{bookmark}}{\usepackage{hyperref}}
\hypersetup{
  hidelinks,
  pdfcreator={LaTeX via pandoc}}
\urlstyle{same} % disable monospaced font for URLs
\usepackage{longtable,booktabs}
% Correct order of tables after \paragraph or \subparagraph
\usepackage{etoolbox}
\makeatletter
\patchcmd\longtable{\par}{\if@noskipsec\mbox{}\fi\par}{}{}
\makeatother
% Allow footnotes in longtable head/foot
\IfFileExists{footnotehyper.sty}{\usepackage{footnotehyper}}{\usepackage{footnote}}

\end{comment}
%\makesavenoteenv{longtable}
\setlength{\emergencystretch}{3em} % prevent overfull lines
\providecommand{\tightlist}{%
  \setlength{\itemsep}{0pt}\setlength{\parskip}{0pt}}
\setcounter{secnumdepth}{-\maxdimen} % remove section numbering

\author{kazuyoshi}
\date{}






\begin{document}
\chapter{保険2 その他}
\section{1. 監督指針, 施行規則, 業法}

\problem{2022 生保2問題 1(1)【監督指針】}

「保険会社向けの総合的な監督指針」
【Ⅱ-2-1 責任準備金等の積立の適切性】について、以下
の(a)~(g)の空欄に当てはまる適切な語句または数値を記入しなさい。(7点)

\begin{itemize}
 \item [(ア)]「Ⅱ-2-1-2 積立方式(2)」においては以下が規定されている。\\
「第一分野及び第三分野において、保険会社の業務又は財産の状況及び保険契約の特性等
に照らし特別な事情がある場合に、保険数理に基づき、合理的かつ妥当なものとして、いわゆるチルメル式責任準備金の積立てを行っている場合には、 (a) に照らしチルメル歩合が妥当なものとなっているか。」
 \item [(イ)]「Ⅱ-2-1-2 積立方式(4)」においては以下が規定されている。\\
「特定の疾病による所定の状態、所定の身体障害の状態、所定の要介護状態その他の保険料
払込の免除事由に該当し、以後の保険料払込が免除されることとなった保険契約のうち、(b)の
可能な保険契約に係る責任準備金については、最終の保険期間満了日まで全て(b)が行われるものとして計算した金額を積み立てることとなっているか。」
\item[(ウ)]「Ⅱ-2-1-2 積立方式(5)」においては以下が規定されている。\\
「 (c) Ⅰ及びⅣにおける「 (d) 」に係る積立基準並びに積立限度の設定につい
ては、手術給付、介護給付その他の保険給付のリスクに応じたものとなっているか。
」
\item[(エ)]「Ⅱ-2-1-2 積立方式(7)④」においては以下が規定されている。\\
「ストレステスト及び (e) の基礎率を同じくする契約区分は同一のものを使用するこ
ととする。」
\item[(オ)]「Ⅱ-2-1-3-1 保険料積立金の積立(1)標準的方式①」においては以下が規定さ
れている。\\
「通常予測されるリスクに対応するものとして、標準的な計算式(
「一般勘定における最低
保証に係る保険金等の支出現価」から「一般勘定における最低保証に係る純保険料の収入現
価」を控除する形式の計算式)によって、概ね (f) %の事象をカバーできる水準に対
応する額を算出するものとなっているか。」
\item[(カ)]「Ⅱ-2-1-3-1 保険料積立金の積立(2)代替的方式④」において、平成8年2月
29日大蔵省告示第48号に列記する国内株式等の期待収益率及び (g) について、当
該告示に定めるものを使用する場合を除き、過去の実績や将来の資産運用環境の見通し、リ
スク中立の観点等から、合理的かつ客観的根拠に基づき定められる必要があることが規定
されている。
\end{itemize}
\answer{}
\begin{itemize}
\item[ (a): ] 新契約費水準
\item[ (b): ] 自動更新
\item[ (c): ] 危険準備金
\item[ (d): ] その他のリスク
\item[ (e): ] 負債十分性テスト
\item[ (f): ] 50
\item[ (g): ] ボラティリティ
\end{itemize}

\problem{2022 生保2問題 1(2)【業法】}
生命保険会社の保険計理人の職務(保険業法第121条)について、以下の(a)~(e)の空
欄に当てはまる適切な語句を記入しなさい。
(5点)

\begin{itemize}
\item[(ア)] 保険計理人は、毎決算期において、次に掲げる事項について、内閣府令で定めるところにより確認し、その結果を記載した(a)を(b)に提出しなければならない。
\begin{itemize}
\item[・]  内閣府令で定める保険契約に係る責任準備金が (c) に基づいて積み立てられているかどうか。
\item[・]  契約者配当又は社員に対する剰余金の分配が(d)に行われているかどうか。
\item[・]  その他内閣府令で定める事項
\end{itemize}
\item[(イ)] 保険計理人は、(ア)の(a)を(b)に提出した後、遅滞なく、その写しを(e)に提出しなければならない。
\item[(ウ)] (e)は、保険計理人に対し、(イ)の(a)の写しについてその説明を求め、その他その職務に属する事項について意見を求めることができる。
\item[(エ)] 上記に定めるもののほか、(ア)の(a)に関し必要な事項は、内閣府令で定める。
\end{itemize}

\answer{}
\begin{itemize}
\item[ (a): ] 意見書
\item[ (b): ] 取締役会
\item[ (c): ] 健全な保険数理
\item[ (d): ] 公正かつ衡平
\item[ (e): ] 内閣総理大臣
\end{itemize}
※(e)は「金融庁(長官)」も正答とした。

\problem{2022 生保2問題 1(4)【監督指針】}
「保険会社向けの総合的な監督指針」
【Ⅱ-2-4 生命保険会社の区分経理の明確化】について、
以下の(a)~(e)の空欄に当てはまる適切な語句を記入しなさい。
(5点)

Ⅱ-2-4-2 主な着眼点

各生命保険会社においては、適切な区分経理を行うため、例えば、以下のような考えに基づく
区分経理に関する管理方針を策定しているか。また、区分経理の状況が、取締役会その他これ
に準ずる機関に対して報告されているか。

(1)〜(6) (省略)

(7)各区分間の取引等
\begin{itemize}
\item[①] 資産区分間の取引\\
資金移動(流入・流出)管理、 (a) 確保、ポートフォリオの改善等、必要な取引とし、市場価格等の適正な価格をもって適切に管理する。
\item[②]商品区分と全社区分との取引\\
\begin{itemize}
\item[ア.] 現預金等の貸借
\begin{itemize}
\item[(ア)] 商品区分又は全社区分毎に区別して管理する。
\item[(イ)] (b)が継続しないよう限度額等を設ける。
\end{itemize}
\item[イ.] 現預金等以外の貸借
\begin{itemize}
\item[(ア)]  (c) から (d) への貸付は、異常な保険金の支払い、新商品の販売に伴う事業運営資金、その他やむを得ない事情がある場合に限る。
\item[(イ)]  (d) から (c) への貸付は、 (c) の規模が小さいために、その機能を十分に果たすことができない場合に限る。
\item[(ウ)] 上記の貸借は、金額、利率(貸付期間に応じた市中金利等を基に設定すること)、期限その他の返済条件をあらかじめ定める。
\item[(エ)] 貸付条件の緩和や債務免除は、回収が不可能な損失が発生している場合等、やむを得ない事情がある場合を除き、 (e) 。なお、貸付条件の緩和等を行った後に利益が生じた場合は、当該利益を返済に充てるものとする。
\end{itemize}
\item[ウ.] 出資 (省略)
\item[エ.] その他の取引 (省略)
\end{itemize}
\end{itemize}

\answer{}
\begin{itemize}
\item[ (a): ]  流動性
\item[ (b): ]  借越し
\item[ (c): ]  全社区分
\item[ (d): ]  商品区分
\item[ (e): ]  行わない
\end{itemize}

\problem{2021 生保2問題 1(1)【実務基準】}

生命保険会社の保険計理人に関する以下の①~④の文章について、下線部分が正しい場合は
○を、誤っている場合は×を記入するとともに、下線部分を正しい内容に改めなさい。

\begin{itemize}
\item[①] 「生命保険会社の保険計理人の実務基準」
(以下、実務基準)第19条によれば、会社全体
の\underline{翌期配当所要額}が、相互会社においては社員配当準備金繰入額(当年度末の未割当額を含
む)以下であること、株式会社においては当年度末の契約者配当準備金(分配済未払、積立
配当金を除く)以下であることを確認しなければならない。
\item[②] 実務基準第20条および実務基準第22条によれば、翌期の全件消滅ベースの配当所要額
が、配当可能財源の範囲内であることを確認しなければならないのは\underline{会社全体のみ}である。
\item[③] 実務基準第21条によれば、会社全体の翌期配当所要額が、会社の配当可能財源から、
\underline{危険準備金積立限度額}を維持するために必要な額を控除した額の範囲内であることを確認しな
ければならない。
\item[④] 保険業法施行規則第77条に定める保険計理人の関与事項には、第7号として「\underline{将来収支}に
関する計画」(解答欄④-1)および第8号として「生命保険募集人の\underline{給与}等に関する規程の作成」
(解答欄④-2)が規定されている。
\end{itemize}
 
\answer{}
\begin{itemize}
\item[ ①: ] ○
\item[ ②: ] ×会社全体および商品区分毎
\item[ ③: ] ×会社の健全性の基準
\item[ ④―1: ] ×保険募集
\item[ ④―2: ] ○
\end{itemize}


\problem{2021 生保2問題 1(4)【税制】}
生命保険会社の税制について、以下の①~⑤の空欄に当てはまる適切な語句または数値を記入し
なさい。

\begin{itemize}
\item[・] 責任準備金繰入額については、保険料積立金及び未経過保険料の部分に限り、保険料及び責
任準備金の算出方法書に定められている\wakumaru{①}の計算基礎を基として計算した額を限度
として損金算入できる。ただし、
\wakumaru{②}の対象契約については、平成8年の大蔵省告示
第48号に定められた計算基礎を基として計算した額を損金算入限度額とすることができる。
\item[・] 法人事業税(地方税)の課税標準は、生命保険業にあっては各事業年度の収入金額とされて
おり、生命保険業の各事業年度の収入金額は、収入保険料中の\wakumaru{③}相当額とするとの
考え方から、収入保険料に一定割合を乗じた金額と定められている。なお、法人事業税の一
部を分離した地方法人特別税は令和元年9月30日までに開始する事業年度をもって廃止さ
れた一方で、令和元年10月1日以後に開始する事業年度から\wakumaru{④}(国税)が創設さ
れている。
\item[・] 課税所得が当該事業年度の剰余金の額の\wakumaru{⑤}相当額に満たない場合は、契約者(社員)
配当準備金繰入額の損金算入を制限し、この剰余金の\wakumaru{⑤}相当額を課税標準とする制
度が設けられている。
\end{itemize}
\answer{}
\begin{itemize}
\item[ ①: ] 保険料
\item[ ②: ] 標準責任準備金
\item[ ③: ] 付加保険料
\item[ ④: ] 特別法人事業税
\item[ ⑤: ] 7%
\end{itemize}

\problem{2020 生保2問題 1(4)【監督指針】]}
金融庁による事業費モニタリングについて、以下の①~⑤の空欄に当てはまる適切な語句を記入
しなさい。

\noindent ○「5-7\wakumaru{①}の充足状況」

保険種類・\wakumaru{②}
の区分ごとの新契約に係る事業費の効率等を見る資料で、定期的に
金融庁宛報告を要する。報告対象は、原則として、当該期における新契約の全て。

\wakumaru{①}に関して、保険種類および\wakumaru{②}の区分ごとに、
「\wakumaru{③}」、「事業費」、「\wakumaru{④}」を算出し、
「効率(事業費÷\wakumaru{③})」および「回収予定平均年数(事業
費÷\wakumaru{④})」を報告する。

\noindent ○「5-9 \wakumaru{⑤}の充足状況」

保険種類・\wakumaru{②}
の区分ごとの契約維持・管理のために支出する事業費の回収状況を
見る資料で、定期的に金融庁宛報告を要する。報告対象は、当該期における全保有契約。
\answer{}
\begin{itemize}
\item[ ①: ] イニシャルコスト
\item[ ②: ] 販売経路
\item[ ③: ] 予定事業費現価
\item[ ④: ] 年換算予定事業費
\item[ ⑤: ] ランニングコスト
\end{itemize}

\problem{2020 生保2問題 1(6)【監督指針】}
変額年金保険等の最低保証に係る保険料積立金の積立てに際して予定解約率を使用する場合の留
意点について、保険会社向けの総合的な監督指針の記載を踏まえ、簡潔に説明しなさい。

\answer{}
\begin{itemize}
\item[]  予定解約率が過去の実績や商品性等から、合理的に定められたものとなっているか。
\item[]  例えば、以下の事例等に留意しているか。
\begin{itemize}
\item[]  特別勘定の残高が最低保証額を下回る状態にあるときの解約率が、特別勘定の残高が最低保証額を超える状態にあるときの解約率より低い率となっているか。
\item[]  解約控除期間における解約率が、解約控除期間終了後の解約率と比べ、低い率となっているか。
\item[]  最低年金原資保証が付された保険契約で、年金開始前における特別勘定の残高が最低保証額を下回る状態にある場合において解約率を保守的に設定しているか。
\item[]  設定された予定解約率について、解約実績との比較などにより、検証を行うこととなっているか。
\end{itemize}
\end{itemize}

\problem{2020 生保2問題 2(1)【業法, 施行規則】}
保険計理人の確認事項のうち、保険業法第121条第1項第3号および保険業法施行規則第79
条の2第1号に規定されている財産の状況の確認について、
「生命保険会社の保険計理人の実務基
準」を踏まえて、簡潔に説明しなさい。

\answer{}
保険計理人は、財産の状況に関し、以下を確認しなければならない。

① 将来にわたり、保険業の継続の観点から適正な水準(事業継続基準)を維持することができるかど
うか。

② 保険金等の支払能力の充実の状況が保険数理に基づき適当であるかどうか。
(ソルベンシー・マージン基準の確認)

<①の確認の概要>

・ 「将来の時点における資産の額として合理的な予測に基づき算定される額(イ)」が、「当該将来の
時点における負債の額として合理的な予測に基づき算定される額(ロ)」を上回ることを確認する
ことにより行う。

・ 上記(イ)とは、事業継続基準の確認に関する将来収支分析(3号収支分析)を行った場合の、資
産(時価評価)から

資産運用リスク相当額

(その他有価証券の評価差額金がマイナスの場合)当該評価差額金に係る繰延税金資産

を控除した額をいう。

・ 上記(ロ)とは、以下の合計額をいう。

事業継続基準に係る額(それぞれの保険契約もしくは保険契約群団について、全期チルメル式
責任準備金と解約返戻金相当額のいずれか大きい方の額を計算したものの合計額)

負債の部の合計額から、責任準備金、価格変動準備金、配当準備金未割当額、評価差額金に係
る繰延税金負債、劣後特約付債務(資産運用リスク相当額を限度とする) を控除した額

・ 3号収支分析は会社全体について毎年行うものとし、分析期間は少なくとも将来10年間とする。

・ 分析期間中の最初の5年間の事業年度末において、上記(イ)の額が(ロ)の額に不足する場合は、
その旨を意見書に記載しなければならない。ただし、


満期保有目的債券および責任準備金対応債券の含み損を算入しない場合に不足が解消される
ときは、分析期間を通じた十分な流動性資産の確保を条件に事業継続困難とはならない旨を、
意見書に記載することができる。


ただちに行われる経営政策の変更により不足を解消できることを、意見書に示すことができる。

<②の確認の概要>

・ ソルベンシー・マージン総額およびリスク合計額が、法令の規定に照らして適正であることを踏ま
えた上で、ソルベンシー・マージン比率が200%以上であることを確認することにより行う。

・ とくに、ソルベンシー・マージン総額が法令の規定に照らして適正であることの確認には、保険料
積立金等余剰部分控除額がソルベンシー・マージン基準の確認に関する将来収支分析(3号の2収
支分析)により算出される保険料積立金等余剰部分控除額の下限以上となっていることを確認しな
ければならない。

・ 3号の2収支分析は、会社全体について毎年行うものとし、分析期間は将来5年間とする。

・ 保険料積立金等余剰部分控除額の下限は、分析期間中の事業年度末に生じた事業継続基準に係る額
の不足額の現価の最大値とする。

・ ソルベンシー・マージン比率が 200%未満である場合には、その旨を意見書に記載しなければ
ならない。

\problem{2019 生保2問題 1(1)【監督指針】}

「保険会社向けの総合的な監督指針」【Ⅱ-3-9 資産負債の総合的な管理】について、以下の
A~Eの空欄に当てはまる適切な語句を記入しなさい。

Ⅱ-3-9-1 意義

資産及び負債、資産の運用方針及び負債の管理方針が、AやBの状況に適
合していることを確保するためには、資産負債全体の状況を把握し管理するための効果的な態
勢を整備し、資産負債全体を適切に管理することが求められる。

Ⅱ-3-9-2 主な着眼点

\begin{itemize}
 \item[(1): ]  (省略)
 \item[(2): ]  取締役会は、資産負債全体の総合的な管理に関する戦略目標を設定し、戦略目標の中でCに関する方針を明確化しているか。
 \item[(3): ]  (省略)
 \item[(4): ]  (省略)
 \item[(5): ]  資産負債を統合的に管理する際に、少なくとも、Dに対する潜在的な影響に関して重要と考えられるリスクは資産負債管理の枠組みにおいて評価されているか。なお、そのようなリスクとしては以下のリスクが含まれる。
 \item[①: ] 市場リスク(省略)
 \item[②: ] 保険引受リスク
 \item[③: ] リスク
\end{itemize}

(以下、省略)
\answer{}
\begin{itemize}
\item[ A: ] リスクの特性
\item[ B: ] ソルベンシー
\item[ C: ] リスク許容度
\item[ D: ] 経済価値
\item[ E: ] 流動性
\end{itemize}
\problem{2019 生保2問題 1(6)【監督指針】}

「生命保険会社の保険計理人の実務基準」における 1 号収支分析の結果、責任準備金不足相当
額が発生した場合において、保険計理人が責任準備金不足相当額の一部または全部を積み立て
なくてもよいことを意見書に示すことができるための条件である経営政策の変更を5つ列挙
しなさい。

\answer{}
\begin{itemize}
\item[]  一部または全部の保険種類の配当率の引き下げ
\item[]  実現可能と判断できる事業費の抑制
\item[]  資産運用方針(ポートフォリオ)の見直し
\item[]  一部または全部の保険種類の新契約募集の抑制
\item[]  今後締結する保険契約の営業保険料の引き上げ
\end{itemize}

\problem{2018 生保2問題 1(3)【監督指針】}
保険会社向けの総合的な監督指針」【Ⅱ-2-4 生命保険会社の区分経理の明確化】につい
て、以下のA~Eの空欄に当てはまる適切な語句を記入しなさい。

Ⅱ-2-4-1 意義 (省略)

Ⅱ-2-4-2 主な着眼点

各生命保険会社においては、適切な区分経理を行うため、例えば、以下のような考えに基づ
く区分経理に関する管理方針を策定しているか。また、区分経理の状況が、取締役会その他
これに準ずる機関に対して報告されているか。

(1)~(4) (省略)

(5) 資産の配賦方法及び管理基準
\begin{itemize}
\item[①] 運用資産の配賦方法\\
運用資産は、原則として、資産の購入時に配賦する資産区分を決める。
\item[②] 運用資産の管理\\
運用資産は、資産区分ごとに、次に掲げる方式の中から適切な方式を選択し管理する。
\item[ア.: ] A・・・ 個々の資産を銘柄ごとに、資産区分に直接帰属させる方式
\item[イ.: ] B・・・ 取引単位(例えば、不動産では物件ごと)ごとに、資産区分の持分で管理する方式
\item[ウ.: ]  資産持分管理方式・・・ 投資対象資産ごとのマザーファンドを設定し、各資産のマザーファンドに対する持分を管理する方式
(注)資産持分管理方式を用いる場合は、一般勘定資産(C保険に対応する資産を除く。)全体を一個のマザーファンドとして扱わない。
\item[③] 運用資産以外の配賦方法\\
再保険貸等、各資産区分に直課できるものは直課し、直課できないものは、区分経理に関
する管理方針に基づいて配賦する。
\item[④] 全社区分の資産\\
D、子会社・関連会社株式、E(E等の管理機能を持つ場合)、その他全社区分に配賦することが相応しい資産の全部又は一部を配賦するものとする。
\end{itemize}
(6)、(7) (省略)

Ⅱ-2-4-3 監督手法・対応 (省略)

\answer{}
\begin{itemize}
\item[ A: ] 資産分別管理方式
\item[ B: ] 資産単位別持分管理方式
\item[ C: ] 無配当
\item[ D: ] 営業用不動産
\item[ E: ] 現預金
\end{itemize}

\problem{2018 生保2問題 2(1)【監督指針】}

生命保険会社の保険計理人の実務基準に規定されている公正・衡平な配当の要件および公正・
衡平な配当の確認の概要について、簡潔に説明しなさい。

\answer{}
○公正・衡平な配当の要件

\begin{itemize}
\item[・] 剰余金の分配または契約者配当(以下「配当」という。)が、公正・衡平であるとは、以下の要件を満たすことである(実務基準第17条第2項)
\begin{itemize}
\item[①: ] 責任準備金が適正に積み立てられ、かつ、会社の健全性維持のための必要額が準備されている状況において、配当所要額が決定されていること
\item[②: ] 配当の割当・分配が、個別契約の貢献に応じて行われていること
\item[③: ] 配当所要額の計算および配当の割当・分配が、適正な保険数理および一般に公正妥当と認められる企業会計の基準等に基づき、かつ、法令、通達の規定および保険約款の契約条項に則っていること
\item[④: ] 配当の割当・分配が、国民の死亡率の動向、市場金利の趨勢などから、保険契約者の期待するところを考慮したものであること
\end{itemize}
\end{itemize}

○公正・衡平な配当の確認
\begin{itemize}
\item[・]配当が公正・衡平であることの確認として、保険計理人は以下の確認を行わなければならない(実務基準第18条第2項)。
\begin{itemize}
\item[①]会社全体について、以下の要件が満たされていること
\begin{itemize}
\item[イ. ] 翌期配当所要額が、相互会社では配当準備金繰入額と配当準備金中の未割当額の合計額、株式会社では当期末の配当準備金(分配済未支払および積立配当金を除く)以下であること(簿価ベースの確認とも言われる)
\item[ロ. ] 翌期の全件消滅ベースの配当所要額が会社の配当可能財源の範囲内であること
\item[ハ. ] 翌期配当所要額が、会社の配当可能財源から会社の健全性の基準を維持するために必要な額を控除した額の範囲内であること
\end{itemize}
\item[②] 区分経理の商品区分毎の翌期の全件消滅ベースの配当所要額が、当該商品区分の配当可能財源の範囲内であること。ただし、保険計理人が特に必要と判断する場合は、さらに細分化した保険契約群団毎に財源が確保されていることを確認しなければならない。また、保険計理人が合理的であると判断する場合は、複数の商品区分をまとめて、財源が確保されていることを確認することができる。
\item[③] 契約消滅時に最終精算として消滅時配当を行う保険種類においては、以下の要件が満たされていること
\begin{itemize}
\item[イ.] 代表契約の翌期配当額が、原則として当年度末のネット・アセット・シェアを超えていないこと(ヒストリカルな視点)
\item[ロ.] 代表契約の将来のネット・アセット・シェアが健全性の基準維持のための金額を下回っていないこと(プロジェクションの視点)
\end{itemize}
\end{itemize}
\end{itemize}

\problem{H29 生保2問題 1(1)【大蔵省告示】}

平成10年・大蔵省告示第231号に規定されている、危険準備金の取崩基準について、以
下の①~⑤の空欄に当てはまる適切な語句を記入しなさい。

(危険準備金の取崩基準)

第六条 危険準備金Ⅰ及び危険準備金Ⅳは、それぞれ①がある場合において、当該
①のてん補に充てるときを除くほか、取り崩してはならない。

2 危険準備金Ⅱは、②がある場合において、当該②のてん補に充てるときを除くほか、取り崩してはならない。

3 危険準備金Ⅲは、最低保証に係る③が負の場合において、当該③のてん補に充てるときを除くほか、取り崩してはならない。

4 その他前三項それぞれに共通する取崩基準として、前事業年度末の④の額が当該事業年度末の⑤を超える場合は、当該超える額を取り崩さなければならない。

\answer{}
\begin{itemize}
\item[ ①: ] 死差損
\item[ ②: ] 利差損
\item[ ③: ] 収支残
\item[ ④: ] 積立残高
\item[ ⑤: ] 積立限度額
\end{itemize}

\problem{H29 生保2問題 1(3)【実務基準】}

「生命保険会社の保険計理人の実務基準」に定めるソルベンシー・マージン基準の確認に関
する将来収支分析(3号の2収支分析)について、以下の①~⑤の空欄に当てはまる適切な
語句または数値を記入しなさい。

・3号の2収支分析は毎年行うものとし、3号の2収支分析を行う期間(以下「分析期間」とい
う。)は、将来①年間とする。

・3号の2収支分析のシナリオの各要素は、以下に定める通りとする(このシナリオを「3号の
2基本シナリオ」という。)。

 金利は、直近の長期国債応募者利回りが横ばいで推移するものとする。

 株式・不動産の価格や為替レートについては、変動しないものとする。

②の取崩しおよび含み益の実現による積立財源への充当は行わない。

 価格変動準備金・危険準備金等への繰入れは行わない。

 劣後性債務・社債・③
については、その約定に従って、利息を支払うこととする。

保険料積立金等余剰部分控除額の下限は、分析期間中の事業年度末に生じた事業継続基準に係
る額の不足額の④とする。なお、ソルベンシー・マージン比率の算出を行う日におい
て、保険業法施行規則第69条第5項の規定に基づき積み立てた⑤の額を積み立てて
いないものとして計算を行う。

\answer{}
\begin{itemize}
 \item[①: ]  5 
 \item[②: ]  評価差額金 
 \item[③: ]  基金 
 \item[④: ]  現価の最大値 
 \item[⑤: ]  保険料積立金 
\end{itemize}

\problem{H29 生保2問題 1(6)【監督指針】}
「保険会社向けの総合的な監督指針」【Ⅱ-3-5 リスクとソルベンシーの自己評価】につ
いて、以下の①~⑥の空欄に当てはまる適切な語句を記入しなさい。

Ⅱ-3-5-1 意義

保険会社は、経営戦略及びリスク特性等に応じ、自らのリスク管理の適切性と現在及び将来
にわたるソルベンシーの十分性を評価するために、①の責任の下、定期的にリスクとソルベンシーの自己評価を実施することが求められる。自己評価においては、将来の経済状況や他の外部要因の変化も考慮し、合理的に予見可能で関連性のある重大なリスクを含んでいる必要がある。

Ⅱ-3-5-2 リスクとソルベンシーの自己評価

(1) 保険会社は、将来の経済状況やその他の外部要因の変化を含めた合理的に予見可能で関
連性のある全ての重大なリスクを考慮し、資本の②と十分性の評価を実施している
か。

また、リスクの要因やリスクの重要性の程度を定期的に評価しているか。さらに、③に大きな変化があった場合には、速やかにリスクとソルベンシーの再評価を行っているか。

保険会社は、リスクとソルベンシーの自己評価に当たっては、中長期事業戦略(例えば3年から5年間)、特に新規事業計画に十分留意しているか。

(2) 保険会社は、必要な④及びソルベンシー・マージン規制に基づく資本の要件を満たしているかをモニタリングするために、リスクとソルベンシーの自己評価を定期的に行い、リスクと資本の管理プロセスを整備しているか。また、必要な④とソルベンシー・マージン規制に基づく資本の要件の違いについて、経営陣は適切に認識しているか。

(3) 保険会社は、リスクとソルベンシーの自己評価の結果を、例えば、リスクの特定及び③、リスク測定、リスク管理方針及びリスクとソルベンシーの自己評価の結果を踏まえた行動計画等とともに、適切に⑤しているか。

(4) 保険会社は、リスクとソルベンシーの自己評価の有効性について、内部(例えばリスク管理担当役員など)又は外部による全般的な評価を行っているか。

(5) ⑥部門は、統合的リスク管理及びリスクとソルベンシーの自己評価の有効性を独立した立場から検証し、必要に応じ経営陣に提言を行っているか。

Ⅱ-3-5-3 経営計画とソルベンシー評価

(1)、(2) (省略)

\answer{}
\begin{itemize}
\item[ ①: ]  取締役会
\item[ ②: ]  質
\item[ ③: ]  リスク・プロファイル
\item[ ④: ]  経済資本
\item[ ⑤: ]  文書化
\item[ ⑥: ]  内部監査
\end{itemize}

\problem{H28 生保2問題 1(2)【施行規則】}

生命保険会社の保険計理人の関与事項について、以下の①~⑤の空欄に当てはまる適切な語句
を記入しなさい。

生命保険会社の保険計理人の関与事項は、次に掲げるものに係る①に関する事項である。

\begin{itemize}
\item[ 1]  保険料の算出方法
\item[ 2]  ②の算出方法
\item[ 3]  契約者配当又は社員に対する剰余金の分配に係る算出方法
\item[ 4]  ③の算出方法
\item[ 5]  未収保険料の算出
\item[ 6]  ④の算出
\item[ 7]  ⑤に関する計画
\item[ 8]  生命保険募集人の給与等に関する規程の作成
\item[ 9]  その他保険計理人がその職務を行うに際し必要な事項
\end{itemize}
\answer{}
\begin{itemize}
\item[ ① ] 保険数理
\item[ ② ] 責任準備金
\item[ ③ ] 契約者価額
\item[ ④ ] 支払備金
\item[ ⑤ ] 保険募集
\end{itemize}
\problem{H28 生保2問題 1(3)【監督指針】}
ストレステストについて、以下の①〜⑤の空欄に当てはまる適切な語句を記入しなさい。

「保険会社向けの総合的な監督指針」の II -3-3-3 ストレステスト

保険会社は、将来の不利益が①に与える影響をチェックし、必要に応じて、追加的に
経営上又は財務上の対応をとって行く必要がある。そのためのツールとして、
②等を含むストレステスト(想定される将来の不利益が生じた場合の影響に関する分析)及び
③(経営危機に至る可能性が高いシナリオを特定し、そのようなリスクをコントロール
すべく必要な方策を準備するためのストレステスト)が重要である。特に、市場が大きく変動し
ているような状況下では、④によるリスク管理には限界があることから、ストレステストの活用は極めて重要である。保険会社においては、⑤等も勘案しつつ、財務内容及び
保有するリスクの状況に応じたストレステストを自主的に実施することが求められる。

\problem{H28 生保2問題 1(4)【業法, 監督指針】}
金融庁による事業費モニタリングについて、以下の①〜⑤の空欄に当てはまる適切な語句を記入しなさい。

○生命保険会社は、平成18年4月以降、次の5つの資料を金融庁宛定期報告することとなっている。
\begin{itemize}
\item[ ・]  5−5「予定事業費等の設定状況」
\item[ ・]  5−6「総合的な充足状況」
\item[ ・]  5−7「①の充足状況」
\item[ ・]  5−8「①の回収状況」
\item[ ・]  5−9「②の充足状況」
\end{itemize}

○5-7「①の充足状況」について
保険種類および③の区分ごとの新契約に係る事業費の効率等を見る資料で、定期的に金融庁宛報告を要する。報告対象は、原則として、当該期における新契約の全て。

①に関して、保険種類および③の区分ごとに、「④」、「事業費」、「⑤」を算出し、
「効率(事業費÷④)」および「回収予定平均年数(事業費÷⑤)」を報告する。

\begin{itemize}
\item[① : ] イニシャルコスト
\item[② : ] ランニングコスト
\item[③ : ] 販売経路
\item[④ : ] 予定事業費現価
\item[⑤ : ] 年換算予定事業費
\end{itemize}

\problem{H27 生保2問題 1(3)【実務基準】}
「生命保険会社の保険計理人の実務基準」に基づき保険計理人が行う責任準備金積立ての確認に
おける、1号収支分析を行わなくともよい保険契約について説明しなさい。
\answer{}
\begin{itemize}
\item[ ・] 責任準備金が特別勘定に属する財産の価額により変動する保険契約であって、保険金等の額を最低保証していない保険契約
\item[ ・] 保険料積立金を積み立てない保険契約
\item[ ・] 保険約款において、保険会社が責任準備金および保険料の計算の基礎となる係数(平成 13年 7 月 1 日または平成 13 年 4 月 1 日以降締結する保険契約については、責任準備金および保険料の計算の基礎となる予定利率)を変更できる旨を約してある保険契約
\item[ ・] その他標準責任準備金の計算の基礎となるべき係数の水準について、必要な定めをすることが適当でない保険契約
\end{itemize}

\problem{H27 生保2問題 1(5)【大蔵省告示】}
平成 8 年・大蔵省告示第 50 号別表第 6 の 2 に規定されている、変額年金保険等の最低保証リス
ク相当額の算出について、次のA~Eに適切な語句を記入しなさい。

II. 最低保証リスク相当額の算出
\begin{itemize}
\item[1.] 標準的方式
\begin{itemize}
\item[(1)]最低保証リスク相当額は、次のイに掲げる額からロに掲げる額を控除した額とする。
\begin{itemize}
\item[イ]Aの責任準備金の額(原則として法第 4 条第 2 項第 4 号に掲げる書類に記載された商品区分ごとに、次の①から④までに定める手順に基づき算出した額をいう。)
\begin{itemize}
\item[①] 次に掲げる区分に応じたリスク対象資産の額から、別表第 7 の 2 の区分によるそれぞれの対象取引残高の欄に掲げる額(別表第 7 の 2 によりリスクヘッジの有効性が確認できたものに限る。)を控除した残高に、次の表に掲げる区分に応じた下落率をそれぞれ乗じた額の合計額を算出する。(省略)
\item[②] 上記①に掲げる額から、その額に次に掲げる算式により計算したB係数を乗じた額を控除する。(省略)
\item[③] 上記②により算出した額を特別勘定資産の額の合計額で除した率を算出する。
\item[④] 上記③により算出した率に基づき資産下落が生じたとした場合の、一般勘定におけるCの額を算出する。
\end{itemize}
\item[ロ] 法第 4 条第 2 項第 4 号に掲げる書類に記載された方法に基づき算出された一般勘定におけるCの額
\item[(2)] (省略)
\item[(3)] (省略)
\item[2.] 代替的方式
次の①から⑬に定める基準を満たす保険会社、外国保険会社等又は免許特定法人(以下「保険会社等」という。)は代替的方式を用いることができる。ただし、代替的方式を用いた場合は、Dの結果、代替的方式の使用を継続することが不適当と認められ、代替的方式の使用を中断する旨又はEに重大な変更を加える旨をあらかじめ金融庁長官に届け出た場合を除き、これを継続して使用しなければならない。(以下、省略)
\end{itemize}
\end{itemize}
\end{itemize}
\answer{}
\begin{itemize}
\item[ A: ] 資産価格下落後
\item[ B: ] 分散投資効果
\item[ C: ] 最低保証に係る責任準備金
\item[ D: ] バック・テスティング
\item[ E: ] リスク計測モデル
\end{itemize}

\problem{H26 生保2問題 1(1)【業法, 施行規則】}

各種準備金に関する次の①~⑤の文章について、下線部分が正しい場合は○、誤っている場合
は×を記入するとともに、誤っている場合には下線部分を正しい表現に改めなさい。

① 保険業法第56条では、基金を償却(基金拠出者への返済)するときは、その償却する金額に相当する金額を、\underline{基金償却準備金}として積み立てなければならないと規定されている。
② 価格変動準備金の積立限度に関して、対象資産のうち国内法人発行の株式については、\underline{期末市場価額}に 100/1000 を乗じて計算する。
③ 危険準備金Ⅱの取崩しにあたっては、\underline{逆ざや}がある場合において、当該\underline{逆ざや}のてん補に充てるときを除くほか、取り崩してはならない。
④ 保険業法施行規則第30条の5で規定される\underline{社員配当準備金}とは、社員への剰余金分配の額を安定させるために積み立てる任意積立金である。
⑤ 生命保険株式会社の経理処理において、契約者配当準備金繰入額は\underline{費用処理}される。

\answer{}
\begin{itemize}
\item[ ①: ] ×基金償却積立金
\item[ ②: ] ×期末帳簿価額
\item[ ③: ] ×利差損
\item[ ④: ] ×社員配当平衡積立金
\item[ ⑤: ] ○
\end{itemize}

\problem{H25 生保2問題 1(1)【実務基準】}
「生命保険会社の保険計理人の実務基準」の規定に関する以下の①~⑥の文章について、下線
部分が正しい場合は○、誤っている場合は×を記入するとともに、誤っている場合には下線部
分を正しい表現に改めなさい。

① 保険計理人は、保険業法施行規則第82条第1項の定めるところにより、\underline{金融庁長官}に意見書を提出しなければならない。
② 1号収支分析(1)においては、各シナリオについて、分析期間中の最初の5年間の事業年度末に生じた責任準備金の\underline{不足額の最大値}を計算し、その値の上位10%を除いたもののうち最大値を責任準備金不足相当額とする。
③ 保険計理人は、\underline{配当を支払う全ての契約}について、代表契約を選定し、アセット・シェアに基づき配当を確認しなければならない。
④ 保険計理人は、3号収支分析の結果が、過去の分析の結果と著しく相違する場合は、その原因を\underline{意見書}に記載しなければならない。
⑤ 3号の2収支分析を行う期間は、将来\underline{5}年間である。
⑥ 保険計理人は、3号の2収支分析の結果、分析期間中の事業年度末において、事業継続基準に係る額の積立てが可能である場合には\underline{事業継続基準不足相当額}はゼロであると判断することができる。
\answer{}
\begin{itemize}
\item[ ①: ] ×: 取締役会
\item[ ②: ] ×: 不足額の現価の最大値
\item[ ③: ] ×: 最終精算として消滅時配当を支払う契約
\item[ ④: ] ×: 附属報告書
\item[ ⑤: ] ○ 
\item[ ⑥: ] ×: 保険料積立金等余剰部分控除額の下限
\end{itemize}

\problem{H24 生保2問題 1(2)【業法,施行規則】}

生命保険相互会社における剰余金の分配に関し、以下の①~⑤の空欄に当てはまる適切な語句
を記入しなさい。

・保険業法第55条の2では「剰余金の分配は、①な分配をするための基準として内閣府令で定める基準に従い、行わなければならない。」と定めている。「内閣府令で定める基準」として、保険業法施行規則第30条の2で以下のとおり規定している。

・相互会社が社員に対する剰余金の分配をする場合には、②に応じて設定した区分ごとに、剰余金の分配の対象となる金額を計算し、次の各号に掲げるいずれかの方法により、又はそれらの方法の併用により行わなければならない。

1.社員が支払った保険料及び保険料として収受した金銭を運用することによって得られる収益から、保険金、返戻金その他の給付金の支払、事業費の支出その他の費用等を控除した金額に応じて分配する方法(③方式)

2.剰余金の分配の対象となる金額をその発生の原因ごとに把握し、それぞれ各保険契約の責任準備金、保険金その他の基準となる金額に応じて計算し、その合計額を分配する方法(④方式)

3.剰余金の分配の対象となる金額を⑤等により把握し、各保険契約の責任準備金、保険料その他の基準となる金額に応じて計算した金額を分配する方法

4.その他前三号に掲げる方法に準ずる方法

\answer{}
\begin{itemize}
\item[ ①: ] 公正かつ衡平
\item[ ②: ] 保険契約の特性
\item[ ③: ] アセット・シェア
\item[ ④: ] 利源別配当
\item[ ⑤: ] 保険期間
\end{itemize}

\problem{H23 生保2問題 2(1)【監督指針】}
区分経理の意義および商品区分の設定について、「保険会社向けの総合的な監督指針」および「保険検査マニュアル」の内容を踏まえ、簡潔に説明しなさい。

\answer{}

\noindent ○区分経理の意義.
\begin{itemize}
\item[] 生命保険会社においては、利益還元の公平性・透明性の確保、保険種類相互間の内部補助の遮
 断、事業運営の効率化、商品設計や価格設定面での創意工夫などを図る観点から、一般勘定に
 ついて保険商品の特性に応じた区分経理を行うことが重要である。
\item[] 各生命保険会社において自己責任原則のもと、保険経理の透明性、保険契約者間の公平性確保
 等の観点から、適切な区分経理が行われる必要がある。
\item[] また、区分経理を導入するにあたっては、資産の配分方法、含み損益の配賦方法等について、
 アセットシェア等に基づき適切に配分方法が定められていることが重要である。
\item[] なお、区分経理は保険計理人の確認業務(責任準備金に関する事項、剰余金の分配または契約
 者配当に関する事項)にも関連している。
\end{itemize}
\noindent ○商品区分の設定
\begin{itemize}
\item[] 区分経理を活用する上では、その目的に応じた適切かつ有効な区分を設定することが重要であ
 り、保険の性質の相違等により理論的かつ合理的な区分とする必要がある。したがって、会社
 収支に重大な影響を与える場合等は、商品区分の新設や細分化をして管理することが望ましい。
 ただし、期間損益の安定性や事務負荷等を踏まえて重要性を検討した上で導入することが重要
 である。また、設定した商品区分については、商品ポートフォリオが大きく変化する場合等に
 は設定を見直す必要もある。
\item[] 保険検査マニュアルには、次のような留意点が記載されている。
\begin{itemize}
\item[①] 商品区分は、損益及び負債の管理を行うためのものであるが、商品の特性や契約の保有状況
に照らして、損益を把握する単位として適切なものとなっているか。\\
 例えば、「掛捨型の短期保険と貯蓄型の長期保険」、「無配当保険と有配当保険」、「予
 定利率固定型保険と予定利率変動型保険」、「個人保険と企業保険」などが、原則として、
 別区分で管理されているか。\\
 なお、主契約に付加された特約等は、原則として、主契約と同じ商品区分に帰属させてい
 るか。
\item[②] 新規商品の発売による当該保有契約の増大や、ある商品区分の中の一部の保険種類の契約
 の増大などにより、保険会社全体や商品区分の収支に重大な影響を与えるような場合に、
 新たな商品区分又は同種の細分化した商品区分を設定する際に、契約者間の公平性等に留意し、
合理的な方法で行っているか。
\item[③] 設定した商品区分について、合理的な理由(保有契約が減少し、商品区分の存在意義がな
 くなった場合等)がないにもかかわらず、その変更(他の商品区分に統合することを含
 む。)を行っていないか。
\end{itemize}
\end{itemize}
\noindent ○活用方法および留意点

\begin{itemize}
 \item[] 利益還元の公平性・透明性の確保
\begin{itemize}
\item[] 区分経理を行うことで、区分毎の損益の状況を明確にすることが可能となるため、利益還元の
 公平性・透明性を確保することができる。特に、有配当区分における契約者への配当の公平性・
 透明性を確保するために、有配当契約・無配当契約を区分するといった適切な区分経理の実施
 は重要である。
\item[] 法令では、剰余金の分配または契約者配当の計算は、
 「保険契約の特性に応じて設定した区分ご
 とに」計算することが規定されており、また、生命保険会社の保険計理人の実務基準では、公
 正・衡平な配当の確認における商品区分単位の配当可能財源の確認は「区分経理の商品区分毎
 に」行うことと規定されている。
\end{itemize}
 \item[] 保険種類相互間の内部補助の遮断
\begin{itemize}
\item[] 商品区分はセルフサポートが基本であり、その中で保険料および責任準備金の十分性を満たす
 必要がある。言い換えれば、区分毎の十分性の確保が、契約者間の公平性の確保および会社の
 健全性の確保につながる。特に、生命保険会社の保険計理人の実務基準では、責任準備金の十
 分性確認において区分経理の商品区分毎に将来収支分析を行うことと規定されている。
\item[] ただし、区分経理は現状ではあくまで内部管理会計であることもあり、最終的には全体で支払
 能力を裏付けていることにも留意する。
\item[] 商品区分の規模が小さくなると、分散効果の低下により毎年の保険収支が不安定化したり資産
 運用効率が低下したりすることから、安定的・効率的な保険制度の運営が難しくなる。このこ
 とから、無闇・頻繁な区分の変更は当然避けられるべきではあるが、一方で、発売間もない新
 商品の区分や販売停止して長期間経つなどして保有が減少した商品の区分など、小さすぎる区
 分は、他の商品区分への統合も検討するなど、区分の更新を検討する必要がある。また、この
 ような目的のために厳格な貸借・出資を前提として全社区分を活用することも考えられる。
\end{itemize}
 \item[] 事業運営の効率化
\begin{itemize}
\item[] 区分経理を行うことで、区分毎の効率性を把握することが可能となり、不採算区分の事業規模
 縮小・撤退などを検討する上での有効な判断材料となる。さらに、その区分の特性を把握でき、
 手数料などの販売政策、経営資源投入などの経営戦略の策定が可能となる。
\item[] 商品に対応する資産の運用特性に沿った区分とすることで、資産運用の効率性・資産負債マッ
 チングの向上や責任準備金対応債券の効果的な運用など、ALM を効果的に行うことができる。
\item[] 区分経理を行う上で算定・利用される保険関係収支などの各種の情報やインフラはリスク管理
 にも利用することができる。
\item[] 分析を行う上でも経営上の諸作を行う上でも区分経理が有効に働くように、商品特性・資産運
 用特性などに沿った商品区分とすることが必要である。例えば、配当の有無・保証性または貯
 蓄性・外貨建かどうかなどによって区分することは必要であろう。
\item[] ただし、区分経理の商品区分に基づく分析だけでなく、保険種類毎や販売チャネル毎など、更
 に細分化した分析や、単年度損益に加え、エンベディッドバリューや新契約価値などの評価手
 法を併用するなど、多面的な分析を行った上で経営判断に役立てていくことが重要である。
\item[] 区分毎の効率性を把握するためには事業費の配賦が不可欠ではあるが、間接経費の詳細な配賦
 は一般的に困難である。これらはあくまでも配賦によって得られた数字であり、常に精度改善
 の余地を持つことに留意が必要である。
\item[] また、一般に、細分化には情報収集コスト・インフラ整備が必要であり、費用対効果に留意が
 必要である。
\item[] 区分経理を効果的に経営に反映させるためにも、経営陣の区分経理に対する理解促進を図るこ
 と・アクチュアリー自身の説明能力の向上を図ることが必要である。
\end{itemize}
 \item[] 商品設計や価格設定面での創意工夫などを図る
\begin{itemize}
\item[] 例えば、独立した商品区分及び資産区分を設定することにより、利率変動型商品や外貨建商品
 などのような資産運用結果を契約者価額に反映させた商品の開発が可能となる。
\item[] また、他の金融商品に競合する商品を開発する場合には、資産区分の資産運用方針に基づき予
 定利率を定め、必要ならば解約返戻金を市場価格調整型とすることで、リスクコントロールを
 しつつ、魅力ある商品設計が可能になる。
\item[] 利源分析を区分毎に行うことで、計算基礎率の妥当性のチェックなどのより詳細な分析を行う
 こともでき、これを新商品開発時の計算基礎率に反映することができる。特に、商品区分別の
 事業費を把握し、それを保険種類別・販売チャネル別に按分することでそれぞれの保険種類・
 販売チャネルに必要な予定事業費率を把握することができるようになる。
\end{itemize}
\end{itemize}

\problem{H23 生保2問題 2(3)【実務基準】}
生命保険会社の保険計理人の実務基準に規定されている公正・衡平な配当の要件および公正・衡平な配当の確認の概要について、簡潔に説明しなさい。
\answer{}
生命保険会社の保険計理人は、保険業法第121条第1項において「契約者配当又は社員に対する
剰余金の分配が公正かつ衡平に行われているかどうか。」を確認し、その結果を記載した意見書
を取締役会に提出することが求められている。

配当が公正・衡平である要件および確認方法については、生命保険会社の保険計理人の実務基
準において以下の通り規定されている。

\begin{itemize}
\item[○] 公正・衡平な配当の要件\\
剰余金の分配または契約者配当(以下、配当という)が、公正・衡平であるとは、以下の要件を満たすことである(第17条第2項)。
\begin{itemize}
\item[①] 責任準備金が適正に積み立てられ、かつ、会社の健全性維持のための必要額が準備されている状況において、配当所要額が決定されていること
\item[②] 配当の割当・分配が、個別契約の貢献に応じて行われていること
\item[③] 配当所要額の計算および配当の割当・分配が、適正な保険数理および一般に公正妥当と認められる企業会計の基準等に基づき、かつ、法令、通達の規定および保険約款の契約条項に則っていること
\item[④] 配当の割当・分配が、国民の死亡率の動向、市場金利の趨勢などから、保険契約者が期待するところを考慮したものであること
\end{itemize}
\item[○] 公正・衡平な配当の確認\\
配当が公正・衡平であることの確認として保険計理人は以下の確認を行わなければならない(第18条第2項)。
\begin{itemize}
\item[①] 会社全体について、以下の要件が満たされていること
\begin{itemize}
\item[イ.] 翌朝配当所要額が、相互会社では配当準備金繰入額と配当準備金中の未割当額の合計、株式会社では当期末の配当準備金(割当済未支払および積立配当金を除く)以下であること
\item[口.] 翌朝の全件消滅べ一スの配当所要額が会社の配当可能財源の範囲内であること
\item[ハ .] 翌朝配当所要額が、会社の配当可能財源から会社の健全性の基準を維持するために必要な額を控除した額の範囲内であること
\end{itemize}
\item[②] 区分経理の商品区分毎の翌朝の全体消滅べ一スの配当所要額が、当該商品区分の配当可能財源の範囲内であること。ただし、保険計理人が特に必要と判断する場合は、さらに細分化した保険契約群団毎に財源が確保されていることを確認しなければならない。また、保険計理人が合理的であると判断する場合は、複数の商品区分をまとめて、財源が確保されていることを確認することができる。
\item[③] 契約消滅時に最終精算として消滅時配当を行う保険種類においては、以下の要件が満たされていること
\begin{itemize}
\item[イ .] 代表契約の翌朝配当額が、原則として当年度末のネット・アセット・シェアを超えていないこと
\item[口 .] 代表契約の将来のネット・アセット・シェアが健全性の基準維持のための金額を下回っていないこと
\end{itemize}
\end{itemize}
\end{itemize}

\problem{H22 生保2問題 2(1)【監督指針】}
保険契約を再保険に付した場合の責任準備金の不積立てについて、関連する法令および「保険会社向けの総合的な監督指針」における記載も踏まえ、簡潔に説明しなさい。

\answer{}
再保険に付した契約であっても、保険事故が発生した場合、保険金受取人からの請求に対して責
任を負うのは元受保険会社であり、再保険していることにより免責となるわけではない。受再会社
が破綻した場合のことを考慮すれば、再保険に付した部分も含めて元受保険会社が責任準備金を積
み立てることも考えられる。ただ、この場合、再保険方式によっては、元受保険金杜に過度な負担
を強いることになる。

このため、保険業法施行規則において、保険会社等に出再する場合は、健全な出再先への出再と
して、再保険を付した部分に相当する責任準備金を積み立てないことができるとされている。
また、「保険金杜向けの総合的な監督指針」においては、この取扱いの可否について、当該再保
険契約がリスクを将来にわたって確実に移転する性質のものであるかどうかや、当該再保険契約に
係る再保険金等の回収の蓋然性が高いかどうかに着目して判断すべきであるとされ、回収の蓋然性
の評価にあたっては、少なくとも再保険契約を引き受けた保険会社又は外国保険業者の財務状況に
ついて、できる限り詳細に把握する必要がある、とされている。

生命保険の再保険として代表的なものに、共同再保険方式や毎年の危険保険金部分を再保険する
方式があるが、これらの再保険に付した契約における責任準備金については、前者の場合は、保険
料、保険金等の全ての契約者との金銭を再保険割合に応じて分かち合うことになるため、責任準備
金も元受会社と再保険金杜で再保険割合に応じて積み立てなければ、収支に歪みが生ずることにな
る。一方、後者の場合は、元受保険金杜から再保険金杜に支払う再保険料も、貯蓄部分を含まない
その年の危険保険料に対応したものとなっており、この場合は、再保険金杜に必要な責任準備金は、
定期保険や一般の損害保険に近い方式のものであり、元受保険会社の保険料積立金から、出再部分
を控除することはできず、未経過保険料の出再分を控除することができる。

\problem{H22 生保2問題 2(4)【監督指針】}

保険会社向けの総合的な監督指針において、財務の健全性に関するリスク管理でのストレステス
トの実施を定めているが、生命保険会社がストレステストを行う意義およびストレスシナリオの設
定において留意すべき点について、監督指針に沿って簡潔に説明しなさい。
\answer{}
\begin{itemize}
\item[①] 生命保険会社がストレステストを行う意義
\begin{itemize}
\item[・] 保険会社は、将来の不利益が財務の健全性に与える影響をチェックし、必要に応じて、追加的に経営上又は財務上の対応をとって行く必要がある。そのためのツールとして、感応度テスト等を含むストレステスト(想定される将来の不利益が生じた場合の影響に関する分析)は重要である。
\item[・] 特に、市場が大きく変動しているような状況下では、VaRによるリスク管理には限界があることから、ストレステストの活用は極めて重要である。
\item[・] 保険会社においては、市場の動向等も勘案しつつ、財務内容及び保有するリスクの状況に応じたストレステストを自主的に実施することが求められる。
\end{itemize}
\item[②] ストレスシナリオの設定において留意すべき点
\begin{itemize}
\item[・] ヒストリカルシナリオ(過去の主な危機のケースや最大損失事例の当てはめ)のみならず、仮想のストレスシナリオによる分析も行っているか。
\item[・] 仮想のストレスシナリすについては、内外の経済動向に関し、株式の価格、金利、為替、信用スプレッドなど、保険会社の保有するリスクに応じて、複数の要素についてストレスシナリオを作成しているか。
\item[・] 複数の要素が同時に変動するシナリオについて、前提となっている保有資産間の価格の相関関係が崩れるような事態も含めて検討を行っているか。
\item[・] 保有する資産の市場流動性が低下する状況を勘案しているか。
\item[・] 変額年金保険の様なオプション・保証性の高い要素については、その特性を考慮した上で、適切なストレスシナリオを設定しているか。
\item[・] 再保険やデリバティブ等によるリスクのヘッジを行っている場合には、カウンターパーティリスクを考慮してストレスシナリオを設定しているか。
\item[・] ストレステストの設定に際しては、取締役会において、保険会社におけるリスク管理の方針として、基本的な考え方を明確に定めているか。
\item[・] その際、基本的な考え方は、統合リスク管理との間に矛盾がなく、かつ、統合リスク管理の計量化手法で把握できないリスクを捉えるとの観点からの配慮がなされているか
\item[・] また、取締役会等において、定期的に、かつ必要に応じ随時、保険会社の業務の内容等を踏まえ、設定内容を見直しているか。等
\end{itemize}
\end{itemize}

\problem{H21 生保2問題 2(3)【実務基準】}
生命保険会社の保険計理人の実務基準における事業継続基準の確認(3号収支分析)について概
要を簡潔に説明し、また、1号収支分析との主な相違点についても簡単に触れなさい。
\answer{}
事業継続基準の確認(3号収支分析)

事業継続基準の確認(3号収支分析)については、保険業法第121条第1項第3号および保険業法施行規則第80条第3号によって確認が求められており、会社全体として、将来の時点における資産の額として合理的な予測に基づき算定される額が、当該将来の時点における負債の額として合理的な予測に基づき算定される額を上回ることを確認することで、保険業の継続の観点から適正な水準を維持することを確認することとなっている。

分析期間中の最初の5年間の事業年度末において、事業継続基準不足相当額が発生する場合は、その旨を意見書に記載することとなっている。なお、経営政策の変更により不足額を解消できる場合には、意見書に示すことができる。事業継続基準不足相当額が発生し、かつこれを解消することのできる経営政策の変更をただちに実施できない場合で、資本調達等の経営政策が実施できなければ、事業継続困難の申出の基準に該当することとなる。

また、3号収支分析は、ソルベンシー・マージン比率における静的検証に加えて、動的検証の位置づけとして実施されていることとなる。

1号収支分析との主な相違点

3号収支分析は事業継続基準を維持できるかどうかの判定であり、責任準備金の適正性の確認である1号収支分析とは主に以下の点で異なる。

\begin{itemize}
\item[ ・] すでに締結されている保険契約だけでなく、将来締結される保険契約も含めて実行する方式(オープン型の将来収支分析)を用い、クローズド型は原則認められていない。
\item[ ・] 区分経理の商品区分ごとではなく会社全体の資産、負債、純資産について行う。
\item[ ・] 1号収支分析では対象外としてもよいとされている契約(変額保険や団体年金保険など)も収支分析に含める。
\item[ ・] 1号収支分析(2)(決定論的シナリオ)は、資産の評価について原価法を適用するが、3号収支分析は、資産の評価は時価で行う。
\end{itemize}


\problem{H20 生保2問題 1(1)【実務基準】}
生命保険会社の保険計理人の実務基準における1号収支分析に関し、次の①〜⑤の空欄にあては
まる最も適切な語句を記入しなさい。

1号収支分析の結果、責任準備金不足相当額が発生した場合において、保険計理人は、以下の
経営政策の変更により、責任準備金不足相当額の一部または全部を積み立てなくてもよいことを、
意見書に示すことができる。ただし、これらの経営政策の変更は、ただちに行われるものでなく
てはならない。

\begin{itemize}
\item[ イ] 一部または全部の保険種類の①の引き下げ
\item[ 口] 実現可能と判断できる②の抑制
\item[ ハ] ③の見直し
\item[ 二] 一部または全部の保険種類の④の抑制
\item[ ホ] 今後締結する保険契約の⑤の引き上げ
\end{itemize}

\answer{}
\begin{itemize}
\item[①:] 配当率
\item[②:] 事業費
\item[③:] 資産運用方針(またはポートフォリオ)
\item[④:] 新契約募集
\item[⑤:] 営業保険料
\end{itemize}

\problem{H20 生保2問題 1(2)【監督指針】}
\begin{itemize}
\item[・] 区分経理は、内部管理会計として行っている状況であるが、「保険会社向けの総合的な監督指針」(金融庁)には、一区分経理の明確化として内容が規定されている。
\item[・] 会社の損益等を区分する単位として、①及び②を設定する。①については、損益を把握する単位として適切なものとなっている必要があり、保険の性質の相違等により理論的・合理的な区分とする必要がある。②には例えば次のイから二の機能がある。
\begin{itemize}
\item[イ.] 死亡保障リスク等の③機能
\item[口.] 新商品開発に係る事業運営資金提供機能
\item[ハ.] 会社全体で共有する資産・共通する経費等の管理機能
\item[二.] 現預金等の管理機能
\end{itemize}
\item[・]運用資産は、資産区分ごとに、資産分別管理方式・④方式・⑤方式の中から適切な管理方式を選択し管理する。
\end{itemize}

\answer{}
\begin{itemize}
\item[①:] 商品区分
\item[②:] 全社区分
\item[③:] リスクバッファー
\item[④:] 資産単位別持分管理
\item[⑤:] 資産持分管理(またはマザーファンド)
\end{itemize}
※④と⑤は1頃不同

\problem{H19 生保2問題 1(2)【実務基準,業法,大蔵省告示】}

実質資産負債差額算出及び 3 号収支分析の「債務超過判定」に関し、以下の空欄を埋めよ。
3 号収支分析の「債務超過判定(事業継続基準の確認)」に係る「資産額」や「負債額」の定義は、
「保険業法第 132 条第 2 項に規定する区分等を定める命令」第 3 条や「平成11年金融監督庁・大蔵
省告示第 2 号」に規定される実質資産負債差額算出における「資産額」や「負債額」の定義とは若干
の相違がある。これを併せて表にまとめると以下のようになる。

\begin{tabularx}{\textwidth}{|X|X|X|}
\hline
 &資産額&負債額\\ \hline
 第132条第2項関係& 「時価評価資産額(満期保有目的債券等も時価評価)」\par 
- 「繰延税金資産(その他有価証券の評価差額が\framebox[3zw]{①}のときの計上金額)」
& 「負債の部計上額」\par - [「\framebox[3zw]{②}」+「危険準備金」\par 
+ 「責任準備金(危険準備金を除く)の解約返戻金相当額[注]超過額」\par 
+ 「配当準備金未割当額」\par  +「繰延税金負債 (その他有価証券の評価差額が\framebox[3zw]{③}のときの計上金額)」]
\\ \hline
3号収支分析 & 「時価評価資産額(満期保有目的債券等も時価評価)」\par 
- 「繰延税金資産(その他有価証券の評価差額が\framebox[3zw]{①}のときの計上金額)」\par 
- 「\framebox[3zw]{④}相当額」
& 「負債の部計上額」\par 
+「解約返戻金相当額[注]」\par 
- [「危険準備金を含んだ責任準備金」\par 
+  「\framebox[3zw]{②}」\par 
+ 「配当準備金未割当額」\par  
+「繰延税金負債 (その他有価証券の評価差額が\framebox[3zw]{③}のときの計上金額)」]
+「\framebox[3zw]{⑤}」
\\ \hline
\end{tabularx}
[注] 全期チルメル式責任準備金と比較しいずれか大きい方の額を計算したもの

\answer{}
① マイナス ② 価格変動準備金 ③ プラス ④ 資産運用リスク ⑤ 劣後特約付債務

\problem{H18 生保2問題 1(2)【施行規則】}
保険業法施行規則第六十九条(生命保険会社の責任準備金)について、以下の空欄を埋めよ。

第六十九条(生命保険会社の責任準備金)

生命保険会社は、毎決算期において、次の各号に掲げる区分に応じ、当該決算期以前に収入した①を基礎として、当該各号に掲げる金額を法第四条第二項第四号に掲げる書類に記載された方法に従って計算し、責任準備金として積み立てなければならない。

一 ②

保険契約に基づく将来の債務の履行に備えるため、③に基づき計算した金額(第二号の二の払戻積立金として積み立てる金額を除く。)

二 ④

未経過期間(保険契約に定めた保険期間のうち、決算期において、まだ経過していない期間をいう。次条及び第二百十一条の四十六において同じ。)に対応する責任に相当する額として計算した金額(次号の払戻積立金として積み立てる金額を除く。)

二の二 払戻積立金 (省略)\\
三 危険準備金 (省略)\\
2〜5(省略)

6 第一項第三号の危険準備金は、次に掲げるものに区分して積み立てなければならない。

一 第八十七条第一号に掲げる保険リスクに備える危険準備金\\
二 同条第二号に掲げる予定利率リスクに備える危険準備金\\
三 同条第二号の二に掲げる⑤に備える危険準備金
(省略)

7 (省略)
\answer{}

\begin{itemize}
\item[ ①:] 保険料
\item[ ②:] 保険料積立金
\item[ ③:] 保険数理
\item[ ④:] 未経過保険料
\item[ ⑤:] 最低保証リスク
\end{itemize}

\problem{H18 生保2問題 2(1)【大蔵省告示】}
平成8年・大蔵省告示第48号第5項に規定されている、変額年金保険等の最低保証に係る保険
料積立金の積立方式である「標準的方式」を簡潔に説明せよ。また、それを使用する場合における留
意点を挙げ、金融庁の「保険会社向けの総合的な監督指針」の記載内容を踏まえ簡潔に説明せよ。(10点)

\answer{}
<「標準的方式」の内容について>

「標準的方式」により計算される保険料積立金は次の①から②を控除した金額である。\\
①一般勘定における最低保証に係る保険金等の支出現価\\
②一般勘定における最低保証に係る純保険料の収入現価

<「標準的方式」を使用する場合における留意点>

[全般]: 通常予測されるリスクに対応するものとして、標準的な計算式によって、概ね
50%の事象をカバーできる水準に対応する額を算出するものとする。

[予定死亡率]: 最低死亡保険金保証が付された保険契約においては死亡保険用の標準死
亡率を、最低年金原資保証が付された保険契約については年金開始後用の標準死亡率を、
両方の保証が付された保険契約においては、保険料積立金の積立が保守的となる方の標
準死亡率を使用する。

[割引率]: 割引率として標準利率を使用する。

[期待収益率]: 期待収益率として標準利率を使用する。

[ボラティリティ]: 資産種類に応じて以下のボラティリティを使用する。なお、下記以
外の資産種類のボラティリティに関しては過去の実績等から合理的に定めたものを使
用する。\\
国内株式:18.4%、邦貨建債券:3.5%、\\
外国株式:18.1%、外貨建債券:12.1%

[予定解約率]: 以下の点に留意したうえで、過去の実績や商品性等から合理的に定める。\\
特別勘定の残高が最低保証額を下回る状態にあるときの解約率が、特別勘定の残高
が最低保証額を超える状態にあるときの解約率より低い率とする。\\
解約控除期間における解約率が、解約控除期間終了後の解約率と比べ、低い率とする。\\
最低年金原資保証が付された保険契約で、年金開始前における特別勘定の残高が最
低保証額を下回る状態にある場合において解約率を保守的に設定する。\\
設定した予定解約率について、解約実績との比較などにより、検証を行う。

[その他の基礎率]: 過去の実績や商品性等から合理的に設定する。

\problem{H17 生保2問題1(2)改題【大蔵省告示】}

平成 8 年・大蔵省告示第 48
号に規定されている予定利率(標準利率)の水準の設定について、以下の空欄を適切な語句または数値で埋めよ。④については計算過程も記載せよ。

<平成 11 年 4 月 1
日以降の標準利率の水準の設定の概要\textbf{(平準払保険の場合)}> 毎年
10 月 1 日を基準日として、基準日の属する月の前月から過去①年間または過去
10 年間に発行された利付国庫債券(10
年)の応募者利回りのそれぞれの平均値のいずれか②い方の値を下表に掲げる対象利率に区分して、それぞれの数値に安全率係数を乗じて得られた数値の合計値(基準利率)が、基準日時点で適用されている予定利率と比較して③\%以上乖離している場合には、基準利率に最も近い
0.25%の整数倍の利率(基準利率が 0.25%の整数倍の利率と
0.125%乖離している場合は、基準利率を超えず、かつ、基準利率に最も近い
0.25%の整数倍の利率とする。)を予定利率とし、基準日の翌年の 4 月 1
日以降締結する保険契約に適用する。

\begin{longtable}[]{@{}ll@{}}
\toprule
対象利率 & 安全率係数\tabularnewline
\midrule
\endhead
0%を超え、1.0%以下の部分 & 0.9\tabularnewline
1.0%を超え、2.0%以下の部分 & 0.75\tabularnewline
2.0%を超え、\textbf{4.0%}以下の部分 & 0.5\tabularnewline
\textbf{4.0%}を超える部分 & 0.25\tabularnewline
\bottomrule
\end{longtable}

<標準利率の水準の計算> 平成 11 年以降のある年の 4 月 1
日時点で適用されている標準利率を
1.50%とする。同年の基準日の属する月の前月から過去に
①年間の利付国庫債券(10 年)の応募者利回りの平均値が 2.90\%、過去 10
年間の応募者利回りの平均値が 4.25%とすると、その翌年の 4 月 1
日以降締結する保険契約の標準利率の水準は④\%となる。
※上記\textbf{強調部}が現行規制との整合のため修正した箇所

\answer{解答}

①3 ②低 ③0.5 ④1×0.9+1×O.75+0.9×0.5=2.10(1.50と0.5以上乖離)→2.0%

\problem{H16 生保2問題 1(4)【実務基準】}
事業継続基準の確認に関する保険計理人の実務基準について、次の①〜⑤に入る適切な語句を選
択肢から選び、ア〜タから該当する記号を解答せよ。

事業継続基準の確認の際に「将来の時点における負債の額として合理的な予測に基づき算定され
る額」とは、「第28条に定める事業継続基準に係る額」と「負債の部の合計額から①・価格
変動準備金・配当準備金未割当額・評価差額金に係る繰延税金負債・劣後特約付債務の合計額を
控除した額」の合計額である。

「事業継続基準に係る額」とは、それぞれの保険契約について、②と解約返戻金相当額のい
ずれか大きい方の額を計算したものの合計額である。

第31条第1項の事業継続基準不足相当額は、3号収支分析における分析期間中の最初の③年
間の事業年度末に生じた事業継続基準不足相当額の現価の④とする。

事業継続基準不足相当額が発生した場合に、経営政策の変更によりこの不足相当額が解消される
と保険計理人が判断した場合には、その旨を意見書に示すことができる。この場合の経営政策の
変更とは、資産運用方針の見直し、一部または全部の保険種類の新契約募集の⑤、実現可能
と判断できる事業費の抑制等がある。

〔選択肢〕
ア.3
イ.5
ウ.10
工.20
オ.推進
カ.抑制
キ.最大値
ク.最小値
ケ.合計値
コ.平均値
サ.平準純保険料式責任準備金
シ.5年チルメル式責任準備金
ス、全期チルメル式責任準備金
セ.危険準備金
ソ.貸倒引当金
タ.責任準備金

\answer{}
(4)①タ②ス③イ④キ⑤カ

\problem{H15 生保2問題 2(1)【業法,実務基準】}
○確認事項

保険業法第121条に基づき、毎決算期において以下の確認を行う。

・責任準備金が健全な保険数理に基づき積み立てられているかどうか

当年度末の責任準備金が法令(保険業法施行規則第69条)に従い適正に積み立てられており、将来収支分析により将来の責任準備金の積立水準が十分であることを確認している。

・契約者配当または社員に対する剰余金の分配が公正かつ衡平に行われているかどうか

会社ならびに商品区分単位で、翌朝配当所要額・全件消滅べ一スの配当所要額でそれぞれ財源確保されているおり、当年度末ならびに将来のネット・アセット・シエアが一定の金額を確保していること

・将来の時点における資産の額として合理的な予測に基づき算定される額が、当該将来の時点における負債の額として合理的な予測に基づき算定される額に照らして、保険業の継続の観点から適正な水準を満たさないと見込まれるかどうか

将来にわたり、資産(時価評価)から資産運用リスク相当額を控除した額が、全期チルメル式責任準備金と解約返戻金相当額のいずれか大きい方の類および負債(責任準備金・価格変動準備金・配当準備金未割当額などを除く)を上回っている。

以上「生命保険会社の保険計理人の実務基準」に従い、計算を行う。

保険計理人は、上記確認事項について意見書を取締役会に提出した後、遅滞なく内閣総理大臣(実際には金融庁長官)にその写しを提出しなければならない。

○関与事項

保険業法第120条に基づき、保険計理人は以下の事項について関与を行う。

\begin{itemize}
 \item[1: ] 保険料の算出方法
 \item[2: ] 責任準備金の算出方法
 \item[3: ] 契約者配当または社員に対する剰余金の分配に係る算出方法
 \item[4: ] 契約者価額の算出方法
 \item[5: ] 未収保険料の算出
 \item[6: ] 支払備金の算出
 \item[7: ] 保険募集に関する計画
 \item[8: ] 生命保険募集人の給与等に関する規程の作成、
 \item[9: ] その他保険計理人がその職務を行うに際し必要な事項
\end{itemize}

\problem{H14 生保2問題1(3)【実務基準】 }
「生命保険会社の保険計理人の実務基準」の規定に関する次の①〜⑤について、正しい
ものには○、誤りのあるものには×を付けよ。

\begin{itemize}
\item[ ①] 1号収支分析は標準責任準備金対象外契約の責任準備金についても確認しなくてはならない。
\item[ ②] 1号収支分析を行う期間(分析期間)は少なくとも将来5年間である。
\item[ ③] 1号収支分析は区分経理の商品区分ごとに行う必要があるが、保険計理人が合理的であると判断する場合には、複数の商品区分をまとめて行うことも可能である。
\item[ ④] 1号収支分析は新契約の募集を行う前提(オープン型)でなければならない。
\item[ ⑤] 1号収支分析の結果が過去の分析の結果と著しく相連する場合は、保険計理人はその原因を附属報告書に記載しなければならない。
\end{itemize}

\answer{}
①×[×:についても確認しなくてはならない。
→O:の一部は確認しなくてもよい。]

②×[×:将来5年間…
→O:将来10年間…]

③○

④×〔×…前提(オープン型)てなければならない。
→○…前提(オープン型)に限らず、すでに締結している保険契約のみで実行する方式(クローズド型)で行ってもよい。]

⑤○

\problem{H13 生保2問題1(2)【実務基準】 }
「生命保険会社の保険計理人の実務基準」の規定に関する以下の①〜⑤について、正
しいものには○、誤りのあるものについては×を付けよ。

①監督当局の認可を得て標準責任準備金(又は平準純保険料式責任準備金)以外の責任準備金を積み立てている場合、1号収支分析では責任準備金積立計画を考慮して責任準備金の確認を行わなければならない。
②1号収支分析(1号基本シナリオ)において、価格変動準備金、危険準備金の繰入については、原則として、それぞれのリスク量に応じて、法定最低繰入基準を下回らない範囲で、計画的に繰り入れる。
③1号収支分析において責任準備金不足額が発生しなかった場合、3号収支分析を行う必要は必ずしもない。
④全件消滅べ一スの配当所要額の配当可能財源の確認において、「その他有価証券」については含み損益を配当可能財源に算入するが、「満期保有債券」および「責任準備金対応債券」の含み損益は算入しない。
⑤配当可能財源の確認に使用する全件消滅べ一スの配当所要額は、次のとおりである。

全件消滅べ一スの配当所要額\\
=(2年目配当契約)翌年度に支払う通常配当(およびこれに準じる配当)\\
十(3年目配当契約)翌年度に支払う通常配当(およびこれに準じる配当)\\
十(3年目配当契約)翌々年度に支払う通常配当(およびこれに準じる配当)の1/2\\
+翌年度に全件消滅したと仮定した場合の消滅時配当

\answer{}
①×[×…場合、1号収支分析では責任準備金積立計画を考慮して責任準備金の確認を行わなければならない。
→○…場合でも、1号収支分析では責任準備金積立計画を考慮して責任準備金の確認を行う必要は必ずしもない。]

②○

③×[×…場合、3号収支分析を行う必要は必ずしもない。
→○…場合でも、3号収支分析を行わなければならない。]

④×[×…「満期保有債券」および「責任準備金対応債券」の含み損益は算入しない。
→○…「満期保有債券」および「責任準備金対応債券」の含み益は算入しないが、含み損は算入する。]

⑤×[×(2年目配当契約)翌年度に支払う通常配当(…)
→○(2年目配当契約)翌年度に支払う通常配当(…)の1/2]

\problem{H13 生保2問題2(2)【実務基準】 }
「生命保険金杜の保険計理人の実務基準」について、以下の問に答えよ。

①事業継続基準について、その確認方法を簡潔に説明せよ。

②事業継続基準未達となった場合、事業継続基準不足相当額を解消するために保険計理人が意見書に示すことができる経営政策の変更を5つ挙げよ。

(③「事業継続基準」が創設された趣旨を踏まえ、生命保険金杜経営におけるアクチュァリーの果たす役割について所見を述べよ。)

\answer{}
①事業継続基準について、その確認方法を簡潔に説明せよ。

保険計理人は、法第121条第1項第3号(および施行規則第80条第3号)の規定に基づき、将来にわたり、保険業の継続の観点から適正な水準(事業継続基準)を維持することができるかどうかを確認しなければならない。

<確認方法>

「将来の時点における資産の額として合理的な予測に基づき算定される額」が
「将来の時点における負債の額として合理的な予測に基づき算定される額」を上回
ることを確認する。

ここで、「将来の時点における資産の額として合理的な予測に基づき算定される
額」とは、事業継続基準の確認に関する将来収支分析を行った場合の時価評価した
資産から施行規則第87条第3号に定める額(資産運用リスク相当額)を控除した
額をいう。

また、「将来の時点における負債の額として合理的な予測に基づき算定される額」
とは、次のイと口の合計額をいう。

イ.事業継続基準に係る額、すなわちそれぞれの保険契約について、全期チルメル
式責任準備金と解約返戻金相当額のいずれか大きい方の額を計算したもの
の合計額。ただし、影響額が軽微であると判断される場合には、それぞれの
保険契約ごとに、全期チルメル式責任準備金と解約返戻金相当額のいずれか
大きい方を計算するのではなく、保険数理上妥当な範囲でまとめられた保険
契約群団ごとに計算することができる。

口.負債の部の合計額から、次に掲げる額の合計額を控除した額

(1)責任準備金\\
(2)価格変動準備金\\
(3)配当準備金未割当額\\
(4)評価差額金に係る繰延税金負債\\
(5)劣後特約付債務(ソルベンシー・マージン総額として計算される額に限る。)
ただし、資産運用リスク相当額を限度とする。

収支分析はオープン型とし、分析期間は少なくとも将来10年間とし、分析期間中の最初の5年間の事業年度末において上記の確認を行う。

<シナリオの設定の概要>

\begin{itemize}
\item[○] 金利は直近の長期国債応募者利回りが横ばいで推移する。
\item[○] 株式・不動産の価格や為替レートについては変動しない。
\item[○] 外貨建資産運用収益については直近の為替レートを使用し、資産運用収益は以下のとおり。
\begin{itemize}
\item[・] ニューマネーは全て国内長期国債に投資したものとし、オールドマネーについては直近の長期国債応募者利回りで運用収益が得られるものとする方法
\item[・] その他合理的な方法
\end{itemize}
\item[○] 新契約高・新契約の商品構成比、保険契約継続率、死亡率など保険事故発生率、事業費については、
\begin{itemize}
\item[・] 直近年度
\item[・] 直近年度を含む過去3年間の平均値
\end{itemize}
\item[○] 資産配分・資産構成比は直近年度における資産配分・資産構成比をもとに合理的なシナリオを設定する。
\item[○] 配当金は、原則として直近年度の配当率が据え置かれたものとする。
\item[○] 価格変動準備金、危険準備金の繰入れについては、原則としてそれぞれのリスク量に応じて、法定最低繰入基準を下回らない範囲で、計画的に繰り入れることとする。
\item[○] 配当準備金繰入額のうち積立配当金として留保されるもの等以外は、原則として、契約者に支払われることとし、その額を資産から減少させることとする。
\item[○] 配当準備金の残高は、原則として、前年度決算の配当準備金繰入額のうち積立配当金として留保されるもの、積立配当金の利息、および、積立配当金の引き出し分等を考慮して、計算することとする。なお、積立配当金の引き出し分は、その額を資産から減少させることとする。
\item[○] 劣後性債務・社債・基金については、その約定に従って、利息を支払うこととする。また、期限のあるものについては、期限到来時に約定に従って返済・償還または償却を行い、期限到来後は再調達しないものとする。
\item[○] その他の負債については、著しい変動の予想されるものを除き、原則として、直近の残高がそのまま推移するものとする。
\end{itemize}

②事業継続基準未達となった場合、事業継続基準不足相当額を解消するために
保険計理人が意見書を示すことができる経営政策の変更を5つ挙げよ。

「生命保険会社の保険計理人の実務基準」第31条(事業継続基準に関する意見書記載事項)に次の5つが規定されている。

\begin{itemize}
\item[ イ: ] 一部または全部の保険種類の配当率の引き下げ
\item[ 口: ] 実現可能と判断できる事業費の抑制
\item[ ハ: ] 資産運用方針(ポートフォリオ)の見直し
\item[ 二: ] 一部または全部の保険種類の新契約募集の抑制
\item[ ホ: ] 今後締結する保険契約の営業保険料の引き上げ
\end{itemize}

なお、これらの経営政策の変更は、直ちに実行き。れるものでなければならない。

% \problem{H17 生保2問題1(3)【業法】<8.4 試験範囲外> }
% 保険相互会社の株式会社化について、以下の空欄を埋めよ。

% ア)保険業法第92条では、組織変更を行う相互会社は定款において、①を定めなければならない、とされている。

% ①は、退社員の②総額とされ、保険業法施行規則第45条により以下の算式で定められる。

% 組織変更を行う相互会社の組織変更時における[蔓]額×A/(A+B)
% ここで、Aは退社員の[重コ類、Bは現社員の[夏コ類を表す。
% また、A+Bは[重コ計算と同様の方法により計算された[重コ類で、具体的には以下の1)か
% らi)、逝)および柵)を差し引くことにより計算される。
% 1)[重コ計算と同様の方法で評価した組織変更時の総資産の額
% i)社員に係る保険契約について保険契約上の債務を履行するために確保すべき資産の額
% 皿)[重コについて保険契約上の債務を履行するために確保すべき資産の額
% 柵)その他組織変更時における債務を履行するために確保すべき資産の類
% イ)組織変更後の株式会社は、貸借対照表上の[亘コ類から[亘コを控除した残額を超えて、
% [重コを行うことができない。
%\answer{}

%\problem{H21 生保2問題 3(2)【大蔵省告示】}
% 生命保険会社の利源分析については、金融庁により全社に一律の様式・基準が定められているが、個杜の状況や分析の局面・日的等によって、求められる利源分析はこれとは異なる場合があると考えられる。

% 今、下記ア〜ウのような状況にある会社があるとしたとき、必ずしも金融庁の基準にとらわれることなく、実際に積み立てている責任準備金の水準を考慮した上で各利源の損益の実態を把握するという視点から適切な情報を経営層に提供したい。

% \begin{itemize}
% \item[ ア.] 標準責任準備金の対象契約については標準責任準備金を積み立てているが、標準責任準備金の計算基礎として使用される予定利率(標準利率)を上回る予定利率を保険料計算基礎に用いている契約が相当程度存在する。
% \item[ イ.] 平成12年金融監督庁・大蔵省告示第22号に定める第三分野保険の負債十分性テストを実施した結果、不足額が生じており、不足額の現在価値に相当する金額を保険料積立金に積み増している。
% \item[ ウ.] 変額年金保険等の最低保証に係る保険料積立金の変動の影響が大きい。
% \end{itemize}

% この場合、アクチェアリーとして、どのような利源分析を実施することが考えられるか、所見を述べなさい。なお、下記A〜Cに沿って、ア〜ウの各々の状況について解答しなさい。

% \begin{itemize}
%  \item[A.] 金融庁への提出用として求められている利源分析上の取り扱いおよび各責任準備金の増減要因とそれに伴う損益の変動要因
%  \item[B.] A.以外に考えられる利源分析上の取り扱い、およびその取り扱い方法をA.の金融庁提出用と比較した時のメリット・デメリット
%  \item[C.] A、およびB.を踏まえた上で、貴君が考える利源分析上の取り扱い(選択した理由も記蔵すること)
% \end{itemize}

\end{document}
