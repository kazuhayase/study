\documentclass[report,gutter=10mm,fore-edge=10mm,uplatex,dvipdfmx]{jlreq}

\usepackage{lmodern}
\usepackage{amssymb,amsmath}
\usepackage{mathtools}
\usepackage{ifxetex,ifluatex}
\usepackage{actuarialsymbol}
\usepackage[]{natbib}

%strike through 
%https://tex.stackexchange.com/questions/23711/strikethrough-text
%\usepackage[]{ulem}

\usepackage[normalem]{ulem}
\usepackage{enumerate}

% Tables
\usepackage{multirow}
\usepackage{tabularx}
%\usepackage{booktabs} % http://www.yamamo10.jp/yamamoto/comp/latex/make_doc/table/table.php

%Framedbox
%https://hakuoku.github.io/agakuTeX/tutorial/5_6framed/
\usepackage{framed}

%https://tgnx8810.wordpress.com/2014/11/29/latex%E3%81%A7%E8%A1%A8%E3%81%AE%E3%82%BB%E3%83%AB%E5%86%85%E6%94%B9%E8%A1%8C%E3%81%AFtabularx%E7%92%B0%E5%A2%83%E3%82%92%E4%BD%BF%E3%81%86%E3%81%A8%E6%A5%BD/
\usepackage{longtable}
\usepackage{booktabs}
\RequirePackage{plautopatch}

% maru suji ① etc.
\usepackage{tikz}
\newcommand{\cir}[1]{\tikz[baseline]{%
\node[anchor=base, draw, circle, inner sep=0, minimum width=1.2em]{#1};}}

%http://yamamo10.jp/yamamoto/comp/latex/make_doc/box/box.php
%枠付き文章
\usepackage{ascmac}
\usepackage{fancybox}

\usepackage{comment}

\begin{comment}

\ifnum0\ifxetex1\fi\ifluatex1\fi=0 % if pdftex
  \usepackage[T1]{fontenc}
  \usepackage[utf8]{inputenc}
  \usepackage{textcomp} % provide euro and other symbols
\else % if luatex or xetex
  \usepackage{unicode-math}
  \defaultfontfeatures{Scale=MatchLowercase}
  \defaultfontfeatures[\rmfamily]{Ligatures=TeX,Scale=1}
\fi
% Use upquote if available, for straight quotes in verbatim environments
\IfFileExists{upquote.sty}{\usepackage{upquote}}{}
\IfFileExists{microtype.sty}{% use microtype if available
  \usepackage[]{microtype}
  \UseMicrotypeSet[protrusion]{basicmath} % disable protrusion for tt fonts
}{}
\makeatletter
\@ifundefined{KOMAClassName}{% if non-KOMA class
  \IfFileExists{parskip.sty}{%
    \usepackage{parskip}
  }{% else
    \setlength{\parindent}{0pt}
    \setlength{\parskip}{6pt plus 2pt minus 1pt}}
}{% if KOMA class
  \KOMAoptions{parskip=half}}
\makeatother
\usepackage{xcolor}
\IfFileExists{xurl.sty}{\usepackage{xurl}}{} % add URL line breaks if available
\IfFileExists{bookmark.sty}{\usepackage{bookmark}}{\usepackage{hyperref}}
\hypersetup{
  hidelinks,
  pdfcreator={LaTeX via pandoc}}
\urlstyle{same} % disable monospaced font for URLs
\usepackage{longtable,booktabs}
% Correct order of tables after \paragraph or \subparagraph
\usepackage{etoolbox}
\makeatletter
\patchcmd\longtable{\par}{\if@noskipsec\mbox{}\fi\par}{}{}
\makeatother
% Allow footnotes in longtable head/foot
\IfFileExists{footnotehyper.sty}{\usepackage{footnotehyper}}{\usepackage{footnote}}

\end{comment}
%\makesavenoteenv{longtable}
\setlength{\emergencystretch}{3em} % prevent overfull lines
\providecommand{\tightlist}{%
  \setlength{\itemsep}{0pt}\setlength{\parskip}{0pt}}
\setcounter{secnumdepth}{-\maxdimen} % remove section numbering

\author{kazuyoshi}
\date{}






% \newcommand*{problem}[3]{\subsubsection{#1 生保#2 #3}}
% \newcommand*{answer}{\subsubsection{解答}}

\begin{document}
\section{保険1第2章 解約および解約返戻金}
\section{2.2 解約および解約返戻金の意義と法規整}
\subsubsection{H20 生保1問題 1(3)}
「保険料計算基礎率」と「責任準備金計算基礎率」が異なっている場合において、「解約返戻金計
算基礎率」の設定方法について説明しなさい。
\subsubsection{解答}
解約返戻金計算基礎率は、以下の観点から、保険料計算基礎率と同じに設定することが考えられる。  
\begin{itemize}
\item 解約返戻金は、個々の契約者が会社財産の形成に貢献した金額を基準とするものであ  り、契約者が拠出した金額は保険料であることから、解約返戻金の算定は保険料に基
  づくことが整合的である。
\item 一方、責任準備金は、ソルベンシーを考慮して会社が評価し積み立てるものであり、契約者価額ではない。
\item 補完的な観点として、解約返戻金水準は契約時に約定が必要であるが、責任準備金は経済状況等により保険期間途中でも積み増しまたは削減があり得る。それをこうした
  約定に反映することは困難である。
\end{itemize}

\section{2.3 解約控除の理由}

\subsubsection{H4 生保1問題 1(2)}
解約控除について簡潔に説明せよ。
\subsubsection{解答}
責任準備金に控除率を適用して解約返戻金を算出する場合に、その控除を解約控除という。
解約控除の理由としては、
\begin{itemize}
\item 新契約費の回収、
\item 解約による逆選択の防止と被保険群団の維持、
\item 解約に手間がかかる、
\item 投資上の不利益、
\item 数学的危険の不安定さの増加等が挙げられる。
\end{itemize}

\subsubsection{H14 生保1問題 1(5)、H11 生保1問題 2(2)①}
個人保険・個人年金保険において解約控除を行う理由を列挙し簡潔に説明せよ。
\subsubsection{解答}
解約控除の理由としては一般的に次の4つの理由が挙げられる。
\begin{enumerate}
\item 新契約費の回収:新契約時にかかる生命保険の募集・締結のための経費は、営業保険料中に予定事業費として組み込んでいる。保険契約が解約された場合には以後の保険料が回収
  されず新契約費はすでに支出されている一方で、その財源である予定新契約費(保険料)の収入が完結していないことになる。このため、未回収部分(の一部)を解約返戻金の算式に反映するものである。
\item 逆選択防止:一般に保険契約を解約する者は平均的に健康体であることが想定され、残された保険群団の死亡率が高まることが予想される。このため、残された保険群団の収悪化を補うものである。
\item 投資上の不利益:解約を見込んで資産の流動性を図ることになるが、このことが資産運用利回りを低下させるため、これを補うものである。
\item ペナルティー:解約に伴う上記の様々な不利益へのペナルティーという意味合いである。
\end{enumerate}

\section{2.4 解約返戻金に関する視点}

\subsubsection{H21 生保1問題 2(2)改}
個人保険の解約返戻金を決めるにあたって考慮すべき視点を5つ\footnote{元の問題は4つ. H21以降で、教科書の改定で「収益性」の追加があったのかも? }挙げ、それぞれ簡潔に説明しなさい。
\subsubsection{解答}
\begin{enumerate}
\item 健全性
  \begin{itemize}
  \item 解約者に対して解約返戻金を支払うことは、保険群団に留保される金額、すなわち将来の保険債務に備えるための責任準備金積立財源となるものが少なくなるということである。
  \item このため、保険群団としての健全性の観点から、解約返戻金は、必要な責任準備金、及びソルベンシーマージンの積立財源を脅かさない金額以下で設定する必要があり、
  \item 具体的には少なくとも、解約返戻金は責任準備金額以下とすべきである。
  \item いわゆる「解約控除」は、健全性の視点で設定していると見ることもでき、
  \item 残存契約からの収益に期待せずに、新契約費の未回収部分を賄うためには、これが必要となる。
  \end{itemize}
\item 公平性
  \begin{itemize}
  \item 例えば「保険契約を解約した者と継続する者」との間の公平性を考えてみると、
  \item 解約者の新契約費の未回収部分の負担を保険を継続した者に負わせることは公平性に欠くと思われ、その意味でもいわゆる「解約控除」は必要と考える。
  \item また、保険種類間の公平性を考えてみると、「解約控除」が利くものと、
  \item 元々責任準備金が少ないため、「解約控除」しきれない等により、何らかの形で残存者に負担を負わせている保険種類もあることに留意する必要がある。
  \end{itemize}
\item 効率性
  \begin{itemize}
  \item 効率性は広い意味での契約者価額(保険料、配当、解約返戻金)を向上させ、保険会社の収益も向上させるものというものである。
  \item ここで、いわゆる「解約控除」の水準を低くしていくことが、効率性につながる。保険料計算上の予定新契約費を低く設定したり、解約控除の水準を低くしたりすることで、
  \item 保険会社が新契約費支出削減の経営努力を行い、効率性を高めるインセンティブになる。
  \item また、効率性は他社との競争力と見ることもでき、企業努力によりソルベンシーを損なわずに事業費支出を削減し、解約返戻金等の契約者価額を高く設定することは、他社との競争力を高めることにもなる。
  \end{itemize}
\item 契約者の (合理的な) 期待
  \begin{itemize}
  \item 契約者の期待に明確な定義があるわけではなく、社会通念、言いかえれば「常識」に基づくことになり、
  \item 例えば、契約者の直接的な期待としては、「解約控除」は小さく、解約返戻金は高いほうが良いことは明確である。
  \item 前述の健全性、公平性および効率性は、究極的には契約者の利益のためであることは明確であるが、この理屈のみでは契約者の期待に応えることは十全ではない。
  \item したがって、契約者の理解を得難いことも事実であり、「解約控除」について、健全性・公平性等に基づく保険計理での妥当性を主張したとしても、消費者契約法の立場からも許容される保証はない。
  \item 以上はH21の公式解答。教科書では、他に以下3点ほどあったように思う。
  \item 社会通念は、その時点の社会環境、経済状態によって変化するので、「契約者の合理的な期待」は常に一定のものではない。
  \item また、法令等によって規定できる性格のものではない。
  \item 解約返戻金自体が、様々な外的要因と絡まって何らかの契約者行動を誘引し、収益の感応度に影響するという関係にある、という視点を理解することが重要である。
  \end{itemize}
\item 収益性
  \begin{itemize}
  \item 生命保険商品の設計においてその収益性を検証するとき、大きく分けて二通りの視点がある。
  \item 実際の経験率が、契約者価格を決定するときの前提条件である予定死亡率・罹患率、予定利率、予定事業費率、予定解約率等などの基礎率からどの位乖離しているか、その乖離がどのような収益または損失をもたらしたか、または、もたらすか、という利源分析の視点と、
  \item これらの基礎率の変動が収益性にどのような変動を与えるか、という感応度の視点である。
  \item 解約返戻金の設定に限定すれば、予定解約率の設定と、解約控除に反映される予定新契約費用の設定が利源分析の視点、解約返戻金の規模と予定解約率の設定が収益性をどのように変動させるか、というのが感応度の視点となる。
  \item 解約返戻金の財源として保険年度末の保険料積立金を充てる伝統的な設計の場合、従来の利源分析上の解約益は新契約費用が償却できているか、という視点であり、解約率動向の分析ではない。
  \item また、このような商品の場合、解約返戻金の規模と予定解約率の設定による感応度は、予定解約率を設定していないことから検証の対象とはならない。
  \item 解約返戻金を保険年度末保険料積立金より低く抑える低解約返戻金もしくは無解約返戻金型商品における収益性の感応度は、解約返戻金の規模と予定解約率の設定が影響する。
  \item 解約返戻金が保険料払込満了時等の特定時に上昇する設計の場合、上昇する直前では解約が減少すると仮定することが合理的である。また、上昇直後には解約が集中すると仮定される。
  \item また、低・無解約返戻金型商品に限らず、一般に解約返戻金の規模自体が、もしくはその算出の根拠となった予定利率などの基礎率が解約率を変動させるという関係にある。市場の金利が解約返戻金の計算に使用した予定利率を大幅に超過した時に予想される資金の流出、一時払変額年金に想定される特別勘定残高と最低保証水準との比較から惹起される解約行動がその例である。このように金融市場などの外部環境の変動に敏感に反応して解約行動が変動する状況を「動的な解約」という。
  \end{itemize}
\end{enumerate} 
\subsubsection{H30 生保1問題 3(1)①-1}
解約返戻金に関連する以下の点について簡潔に説明しなさい。
\begin{itemize}
\item 解約返戻金額と死亡保険金額に起因する「契約者間の公平性」について簡潔に説明せよ。
\end{itemize}

\subsubsection{解答}
\begin{itemize}
\item 契約者間の公平性(死亡保険金額との関係)
  \begin{itemize}
  \item 解約返戻金額が死亡保険金額を超過している場合、死亡保険金を請求するより解約したほうが契約者および保険金受取人の受け取る金額の総額が大きくなる。
  \item この場合、解約のほうが有利であることを知っている契約者とそうでない契約者の間の公平性に問題が生じる。
  \item 保険会社には、「お客様を公平に扱う」との考えのもと、解約返戻金額が死亡保険金額を超過しないような商品設計が求められる。  
  \end{itemize}
\end{itemize}

\section{2.5商品開発における留意事項}
% \label{sec:2.5}

\subsection{低・無解約返戻金型商品}

\subsubsection{H30 生保1問題 3(1)①-2}
解約返戻金に関連する以下の点について簡潔に説明しなさい。
\begin{itemize}
\item 低・無解約返戻金型商品の開発における一般的な留意点のうち「保険契約者の理解」
\end{itemize}

\subsubsection{解答}
\begin{itemize}
\item 保険契約者の理解 (低・無解約返戻金型商品の開発において)
  \begin{itemize}
  \item 低・無解約返戻金型商品を開発する際の留意点で最も大事なのは保険契約者の理解である。解約返戻金の水準を小さくすればするほど保険料は低廉になる。加入時にきちんと説明し理解してもらって加入いただくので、そこに問題があるわけではない。
  \item しかし、長期間経過したのちに解約する場合というのは、やむを得ないケースが多くなる。契約時に低解約返戻金のことを説明し水準を示すとしても、10 年、20 年と経った場合も想定しなければならない。
  \item これに対して、商品の設計上の工夫として、例えば次のような方法が取り入れられている。
    \begin{itemize}
    \item 解約返戻金をあまり極端に小さくは設定しない。
    \item 契約の乗り換えが頻繁には起こりにくい比較的高齢者向け、あるいは持病のある被保険者向けに設計された契約に導入する。
    \item 保険期間を短く設定する。
    \item 解約返戻金を低く抑える期間を短く設定する。
    \item 元々が十分な解約返戻金があることを期待されていない掛け捨て型商品に導入する。
    \end{itemize}
  \end{itemize}
\end{itemize}

\subsubsection{H22 生保1問題 2(4)}

解約返戻金を責任準備金の一定割合に抑えた低解約返戻金タイプ(および無解約返戻金タイプ)の終身保険について、次の①、②の各問に答えなさい。

① 以下の空欄(ア)~(オ)に当てはまる適切な数式を解答用紙の所定の欄に記入しなさい。

\begin{itemize}
\item 死亡時には保険金1を、解約時にはそのときの責任準備金のα倍(0≦α≦1)を、それぞれ即時に支払う保険料連続払の終身保険を考える。このとき、年あたりの純保険料 $\bar{\Px{x}[\alpha]}$ について、次の(A)式が成立する。

$$
 \bar{\Px{x}[\alpha]}=\frac{\int_{0}^{\omega-x}\mu_{x+t}\cdot\px[t]{x}\cdot e^{-(\delta+(1-\alpha)\lambda) t}dt}{\int_{0}^{\omega-x}\px[t]{x}\cdot e^{-(\delta+(1-\alpha)\lambda)}dt} \eqno(A)
$$

ここで$\mu_{x+t}=-\frac{d}{dt}\ln(\px[t]{x}), \delta, \lambda$ はそれぞれ保険料計算基礎として予定される死力、利力、解約力(利力、解約力は定数)とする。

\item (A)式において、$\lambda=0$の場合(解約を見込まない)を考えると、

$$
 \bar{\Px{x}[\alpha]}=\frac{\int_{0}^{\omega-x}\mu_{x+t}\cdot\px[t]{x}\cdot e^{-\delta t}dt}{\int_{0}^{\omega-x}\px[t]{x}\cdot e^{-\delta t}} \eqno(B)
$$

\end{itemize}

(A)式と(B)式を比較すると、(A)式は(B)式において予定利力δを(ア)とした場合で
ある。
すなわち、低解約返戻金タイプにおいて予定解約力をλと設定することと、解約を見込まない場合の予定利力δを(イ)だけ高く設定することは、純保険料に与える効果が同等であることを意味する。

・(A)式の証明をおこなう。
経過tにおける責任準備金を、$\bar{\Vx[t]{x}[\alpha]}$ とするとき、t からt + dt までの間の収支を考えることにより、

$$
(\bar{\Vx[t]{x}[\alpha]}+\bar{\Px{x}[\alpha]}\cdot dt)(1+\delta\cdot dt)
=\mu_{x+t}\cdot dt+\lambda\cdot dt\cdot\alpha\cdot\bar{\Vx[t+dt]{x}[\alpha]}+(\text{ウ})\cdot\bar{\Vx[t+dt]{x}[\alpha]}
$$

2次の項を無視して整理すると、

$$
\frac{d}{dt}(\bar{\Vx[t]{x}[\alpha]})=\bar{\Px{x}[\alpha]}-\mu_{x+t}+(\text{ェ})\cdot\bar{\Vx[t]{x}[\alpha]} \eqno{(C)}
$$

を $\bar{\Px{x}[\alpha]}$ は満たす。このとき、

$$
f(t)\equiv \px[t]{x}\cdot e^{-(\delta+(1-\alpha)\lambda)t} 
= \exp\left\{ -\int_{0}^{t}(\mu_{x+s}+\delta+(1-\alpha)\lambda)ds\right\}
$$

とおくと、$\frac{d}{dt}f(t)=-(\mu_{x+t}+\delta+(1-\alpha)\lambda)f(t)$であることから、(C)式の両辺に$f(t)$ を乗じ、部分積分を用いて整理することにより、

$$
\frac{t}{dt}(f(t)\cdot\bar{\Vx[t]{x}[\alpha]})=f(t)\cdot(\text{オ})
$$

$f(0)=1$, $f(\omega-x)=0$, $\bar{\Vx[0]{x}[\alpha]}=0$, $\bar{\Vx[\omega-x-0]{x}[\alpha]}=1$を用いると、$\int_{0}^{\omega-x} f(t)\cdot(\text{オ})dt=0$となることから、(A)式が導かれる。

② 無解約返戻金タイプの平準払終身保険において、払済保険への変更を取り扱うことが可能かどうか、理由を付したうえで、簡潔に説明しなさい。

\subsubsection{解答}

①
(ア) $\delta+(1-\alpha)\lambda$ (イ) $(1-\alpha)\lambda$ (ウ) $1-\mu_{x+t}\cdot dt - \lambda\cdot dt$
(ェ) $\mu_{x+t}+\delta+(1-\alpha)\lambda$ (オ) $\bar{\Px{x}[\alpha]}-\mu_{x+t}$

②

無解約返戻金タイプに払済保険への変更を可能とした場合、解約と払済保険変更を比較すると、

\begin{itemize}
 \item 解約:その後の「保険料払込負担なし」+「返戻金も保障も一切なし」
 \item 払済変更:その後の「保険料払込負担なし」+「小なりとはいえ保障が残る」
\end{itemize}

となるため、解約と払済変更とで公平性の観点から問題が生じる。この場合、解約する契約者はいなくなり、残存契約群団の維持に支障が生じる可能性があることから、払済保険への変更を取り扱うことはできないと考えられる。
なお、変更後の解約返戻金も「0」とすることで、保険数理上、変更は不可能ではないと考えられるが、この場合、契約者に対しては無解約返戻金であるが、実態としては、払済時に解約返戻金を保障するという商品となり、契約者は解約せずに払済を選択するであろう点を踏まえれば、保険料の割引に用いる予定解約率を織り込む妥当性に関して検討が必要となる。

(私的考察)
①で無解約返戻金とする(α=0)と、低解約返戻金タイプでの予定解約力分δだけ予定利力を高くすることが同等であることが示されている。従って、予定解約力を0としない限りは、予定利力を高くすることと同等となり、「予定解約率を織り込む妥当性」を慎重に検討することが必
要となる。


\subsection{変額年金の解約控除}
\subsubsection{H28 生保1問題 1(6)}
金融商品取引法の行為規制の一部が準用される保険契約として、保険業法第 300 条の2に規定されている「特定保険契約」について、具体的に含まれる保険種類を挙げて、簡潔に説明しなさい。

\subsubsection{解答}

金融商品取引法の行為規制の一部が準用される、市場リスクを有する生命保険(金利、通貨の
価格、金融市場における相場その他の指標に係る変動により損失が生ずるおそれがある保険契約)のことであり、保険業法第300条の2において、「特定保険契約」と規定されている。
具体的には、変額保険、変額年金保険、外貨建て保険、MVAの機能を有する保険などがこれ
に含まれる。
特定保険契約の販売・勧誘に当たっては、顧客の知識、経験、財産の状況および特定保険契約
を締結する目的を的確に把握の上、顧客属性などに則した適正な販売・勧誘の履行を確保することが必要とされている。

\subsection{市場価格調整型(Market Value Adjustment)}
\subsubsection{H16 生保1問題 2(1)}

個人保険についての解約返戻金における市場価格調整型(Market Value Adjusted)について簡潔に説明せよ。
\subsubsection{解答}


解約返戻金における市場価格調整型とは、解約時における保険契約の簿価価格と投資対象資産の市場価格との調整を行うもので、1988年にニューヨーク不没収価格法に追加され、ユニバー
サル保険やSPDA(一時払据置年金)等のいわゆる金利感応型商品に適用される。その基本的な
考え方は、解約に伴うキャッシュ・アウトの際に顕在化する金利リスクを解約に伴うコストとみなすということである。具体的には、
①他業態の金融商品も意識した解約価格設定を行う。
②解約時の金利水準により、"解約控除"が変動する。
という仕組みになっている。
米国では、金融革命といわれた高金利・金利変動時代に、保険の保障機能と貯蓄機能を分け
て考えるいわゆるアンバンドリングが進展した結果、貯蓄性商品を他の金融商品と同列に扱うべき社会的要請(ディスクロージャー等)があって、この手法を導入する環境が整備されていた。日本で導入する際には、このような環境の相違や社会的要請を勘案して議論をすることが必要である。

\subsubsection{H25 生保1問題 2(1)}
MVA(Market Value Adjustment)は一般的には次の計算式を契約者価額に乗じたものを契約者価額より控除する形で適用する。

$$
MVA_{t+\frac{\theta}{12}}=1-(\frac{1+i_1}{1+i_2+\alpha})^{n-\frac{12t+\theta}{12}}
$$

$i_1,i_2,\alpha$についてそれぞれ簡潔に説明した上で、$\alpha, i_2$の設定にあたり留意すべき点を述べなさい。ただし、

\[
 \begin{array}[t]{ccl}
  MVA_{t+\frac{\theta}{12}}&:&t+\frac{\theta}{12}\text{時点のMVA} \\
  n&: &\text{保険期間(年)} \\
  t&: &\text{経過年数} \\
  \theta&: &\text{経過月数} \\
 \end{array}
\]

とする。

\subsubsection{解答}

〇 $i_1$ 、 $i_2$ 、$\alpha$の説明
\begin{itemize}
 \item  $i_1$ : 契約時の利率
 \item  $i_2$ : 解約時の利率
 \item  $\alpha$ : 保険会社が利率を設定する時期と保険契約者が解約する時期のタイムラグに対応するマージン
\end{itemize}

◯ $\alpha$設定の留意点

解約時に適用する利率( $i_2$ )は、通常、当日の市中金利ではなく、例えば前月の市中金利等を基に設定し一定期間適用される。このため、保険会社が利率を設定する時期と保険契約者が解約を決断する時期にタイムラグが生じる。このタイムラグに対するマージンとして$\alpha$が設定される。

保険会社にとっては、$\alpha$を大きく設定する方が保守的であるが、一方で解約する者に対して高い負担を求めることになる。また、$\alpha$は金利上昇時も低下時も一律に適用される。

金利上昇時における解約の発生に対応するという目的を踏まえつつ、過度に保守的な設定とならないように留意することが望ましい。

◯ $i_2$ 設定の留意点
契約時の利率 $i_1$ は期間n年の金利であるのに対し、解約時の利率 $i_2$ はどの期間の金利を用いるべきかという問題がある。

裏付けとなる資産価格の変動を反映するという観点からは、$i_2$ は残存期間に対応する利率を用いることになる。この場合には、金利変動だけでなく、金利の期間構造の影響も解約返戻金に反映させることになり、契約者からは理解されにくくなる。

一方、 $i_2$ を保険期間に対応する利率とした場合、その時点で加入した契約に適用される利率と同じとなるため、契約者からは理解されやすくなる。しかし、一般に金利の期間構造は順イールドであるので、(より高い利率を参照することにより)契約者に不利な取扱いとなる。
これらのメリット・デメリットを踏まえ、また、想定する契約者の理解度(個人か法人かなど)等も考慮し、対応を検討する必要がある。


\subsubsection{2019 生保1問題 2(1)}
MVA(市場価格調整)について、次の①、②の各問に答えなさい。
①ある経過年月数における市場価格調整後の解約返戻金が
(解約時の契約者価額) × $(1 − MVA_{t+\frac{\theta}{12}})$で表されるとき、$MVA_{t+\frac{\theta}{12}}$を以下の記号を用いて一般的な算式で記述しなさい。

\[
 \begin{array}[t]{ccl}
  i_1&: & \text{契約時の利率}\\
  i_2&: & \text{解約時の利率} \\
  \alpha&: &\text{タイムラグに対応するマージン} \\
  n&: &\text{保険期間(MVA 適用期間)(年)} \\
  t&: &\text{経過年数} \\
  \theta&: &\text{経過月数} \\

 \end{array}
\]

②MVA の概要、意義および$i_2$ の設定における留意点について説明しなさい。

\subsubsection{解答}

1① 市場価格調整によって契約者価額から控除される部分を表すので、以下の算式となる。
$$
MVA_{t+\frac{t+\theta}{12}} = 1-\left(\frac{1+i_1}{1+i_2+\alpha}\right)^{n-\frac{12t+\theta}{12}}
$$

②[概要・意義]
\begin{itemize}
 \item MVAは、契約時の金利と解約時の金利の差を解約返戻金額の算出に反映させる機能である。
 \item MVAにより、金利上昇時は解約返戻金が下落し、逆に金利下落時は解約返戻金が上昇する。
 \item MVAの意義は、解約が起こった時点の金利に基づき解約返戻金額を調整することで資産(債券) の流動化に伴う売却損が発生するリスクを保険契約者に移転することである。
 \item MVAを解約控除の一種ととらえるならば、投資上の不利益を補うために用いられると解釈され る。
 \item 資産売却によるリスクを保険契約者に移転することに加え、急激な金利上昇時に多量の解約が発 生することを防ぐためにもMVAが活用されている。
 \item 会社にとっては金利上昇局面における資産売却損が発生するリスクや流動性リスクから解放さ れるというメリットがあるとともに、リスクが軽減されることで、保険契約者に対して予定利率 を高めに設定することができる。
 \item MVAを有する商品は保険業法(第300条の2)に規定する「特定保険契約」に該当する。こ のため、販売・勧誘に当たっては、顧客属性などに即した適正な販売・勧誘の履行を確保するこ とが必要とされている。
\end{itemize}

[解約時の利率の設定における留意点]
\begin{itemize}
 \item $i$ を当日の市中金利を反映して毎日更新することは、実務上大きな負荷がかかる。したがって、通常、前月の市中金利等を基に設定し、一定期間(例えば一カ月間)適用する。
 \item しかし、その場合、保険会社が𝑖 を設定する時期と保険契約者が解約を決断する時期のタイムラグが生じることに留意しなければならない。
 \item $i$ の設定をする際、裏付けとなる資産価格の変動を反映するという観点から、残存期間に対応する金利を用いることが考えられる。この場合、金利変動だけでなく、金利の期間構造の影響も解約返戻金に反映させることになり、契約者からは理解されにくくなる。
 \item 一方、$i$ を保険期間に対応する利率を用いることも考えられる。その時点で加入した契約に適用される利率と同じとなるため、契約者からは理解されやすくなるが、金利の期間構造が順イールドの場合は、(より高い利率を参照することにより)契約者に不利な取扱いとなる。
 \item 上記のメリット・デメリットを踏まえ、また、想定する契約者の理解度(個人か法人かなど)等も考慮し、対応を検討する必要がある。
 \item 不当に解約返戻金額をコントロールしないように、契約者の保護の観点等から、恣意性のない合理的なルールを定める必要がある。
\end{itemize}


\end{document} 