\documentclass[report,gutter=10mm,fore-edge=10mm,uplatex,dvipdfmx]{jlreq}

\usepackage{lmodern}
\usepackage{amssymb,amsmath}
\usepackage{mathtools}
\usepackage{ifxetex,ifluatex}
\usepackage{actuarialsymbol}
\usepackage[]{natbib}

%strike through 
%https://tex.stackexchange.com/questions/23711/strikethrough-text
%\usepackage[]{ulem}

\usepackage[normalem]{ulem}
\usepackage{enumerate}

% Tables
\usepackage{multirow}
\usepackage{tabularx}
%\usepackage{booktabs} % http://www.yamamo10.jp/yamamoto/comp/latex/make_doc/table/table.php

%Framedbox
%https://hakuoku.github.io/agakuTeX/tutorial/5_6framed/
\usepackage{framed}

%https://tgnx8810.wordpress.com/2014/11/29/latex%E3%81%A7%E8%A1%A8%E3%81%AE%E3%82%BB%E3%83%AB%E5%86%85%E6%94%B9%E8%A1%8C%E3%81%AFtabularx%E7%92%B0%E5%A2%83%E3%82%92%E4%BD%BF%E3%81%86%E3%81%A8%E6%A5%BD/
\usepackage{longtable}
\usepackage{booktabs}
\RequirePackage{plautopatch}

% maru suji ① etc.
\usepackage{tikz}
\newcommand{\cir}[1]{\tikz[baseline]{%
\node[anchor=base, draw, circle, inner sep=0, minimum width=1.2em]{#1};}}

%http://yamamo10.jp/yamamoto/comp/latex/make_doc/box/box.php
%枠付き文章
\usepackage{ascmac}
\usepackage{fancybox}

\usepackage{comment}

\begin{comment}

\ifnum0\ifxetex1\fi\ifluatex1\fi=0 % if pdftex
  \usepackage[T1]{fontenc}
  \usepackage[utf8]{inputenc}
  \usepackage{textcomp} % provide euro and other symbols
\else % if luatex or xetex
  \usepackage{unicode-math}
  \defaultfontfeatures{Scale=MatchLowercase}
  \defaultfontfeatures[\rmfamily]{Ligatures=TeX,Scale=1}
\fi
% Use upquote if available, for straight quotes in verbatim environments
\IfFileExists{upquote.sty}{\usepackage{upquote}}{}
\IfFileExists{microtype.sty}{% use microtype if available
  \usepackage[]{microtype}
  \UseMicrotypeSet[protrusion]{basicmath} % disable protrusion for tt fonts
}{}
\makeatletter
\@ifundefined{KOMAClassName}{% if non-KOMA class
  \IfFileExists{parskip.sty}{%
    \usepackage{parskip}
  }{% else
    \setlength{\parindent}{0pt}
    \setlength{\parskip}{6pt plus 2pt minus 1pt}}
}{% if KOMA class
  \KOMAoptions{parskip=half}}
\makeatother
\usepackage{xcolor}
\IfFileExists{xurl.sty}{\usepackage{xurl}}{} % add URL line breaks if available
\IfFileExists{bookmark.sty}{\usepackage{bookmark}}{\usepackage{hyperref}}
\hypersetup{
  hidelinks,
  pdfcreator={LaTeX via pandoc}}
\urlstyle{same} % disable monospaced font for URLs
\usepackage{longtable,booktabs}
% Correct order of tables after \paragraph or \subparagraph
\usepackage{etoolbox}
\makeatletter
\patchcmd\longtable{\par}{\if@noskipsec\mbox{}\fi\par}{}{}
\makeatother
% Allow footnotes in longtable head/foot
\IfFileExists{footnotehyper.sty}{\usepackage{footnotehyper}}{\usepackage{footnote}}

\end{comment}
%\makesavenoteenv{longtable}
\setlength{\emergencystretch}{3em} % prevent overfull lines
\providecommand{\tightlist}{%
  \setlength{\itemsep}{0pt}\setlength{\parskip}{0pt}}
\setcounter{secnumdepth}{-\maxdimen} % remove section numbering

\author{kazuyoshi}
\date{}





\begin{document}
\chapter{保険1第6章 団体生命保険}
\section{6.1 はじめに,  6.2 日本における団体生命保険の沿革}
(過去問での出題なし)
\section{6.3 団体生命保険の危険選択}
\subsection{団体保険の危険選択の目的と選択理論}
\problem{H13 生保1問題 1(5)}
団体保険に関して、選択を行う際のグレッグによる原則について、次の①~⑤を適当な語句で埋めよ。

\begin{itemize}
\item[ (a)]  社会的な利害関係によって結成された個人の集団ではなく、 \wakumaru{①}を目的として組織された場合は、逆選択の危険性がある。
\item[ (b)]  若く健康な新規加入者があり、老齢で健康を害したものが団体から脱退すれば団体の\wakumaru{②}は安定する。
\item[ (c)]  各個人の保険金額は、例えば、全加入者について同一金額または各人の給与・勤続年数・資格等により\wakumaru{③}な基準で決定されるべきである。
\item[ (d)]  団体の従業員の\wakumaru{④}または\wakumaru{④}に近い者が加入することにより逆選択を防止できる。
\item[ (e)]  \wakumaru{⑤}が簡単である。
\end{itemize}

\answer{}
\begin{itemize}
\item[ ①: ] 保険加入
\item[ ②: ] 死亡率
\item[ ③: ] 客観的
\item[ ④: ] 全員
\item[ ⑤: ] 管理
\end{itemize}

\problem{H27 生保1問題 1(3)、H18 生保1問題 1(2)、H10 生保1問題 1(10)}
団体生命保険の危機選択に関し、団体による選択では各団体のリスクの均質性が前提であるが、ほか
に考慮する必要のある点を 5 つ列挙しなさい。(グレッグの挙げた原則)
\answer{}
\begin{itemize}
\item[]  保険加入目的のための団体ではないこと
\item[]  団体に加入脱退があること
\item[]  保険金額が客観的に決まること
\item[]  団体の一定以上の割合が加入すること
\item[]  管理が簡単であること
\end{itemize}

\problem{H14 生保1問題 2(1)①}
団体による危険選択を行うことの趣旨および留意点を述べよ。
\answer{}

危険選択の目的は、損失の発生がある程度予想できる程度に同種類の危険体を数多く集め
ることである。しかし、選択の結果、死亡率は低く抑えられても保険加入者が少なく危険単
位の量が少なければ危険の予測が困難となる。したがって、選択基準の厳格性と多数の危険
単位の引受の必要性との調和が必要である。団体生命保険の選択基準も危険単位を1団体と
考え、この調和に留意している。

【趣旨】

\begin{itemize}
\item[ ア)] 将来の結果が予測できるよう契約の同質化を図るとともに契約の量との調和を図ることが必要である。
\item[ イ)] 大多数の団体が標準料率で契約できるような基準を設ける。
\item[ ウ)] 種々の組分けの中にできるだけ多く一定水準以上の団体を含ませる。
\end{itemize}

【留意点】

\begin{itemize}
\item[ ア)] 保険加入目的のための団体でないこと。
\item[ イ)] 団体に加入、脱退があること。
\item[ ウ)] 保険金額が客観的に決まること。
\item[ 工)] 団体の一定以上の割合が加入すること。
\item[ オ)] 管理が簡単であること。
\item[ カ)] 危険論の見地からの留意点
\begin{itemize}
 \item[ ] 加入者の最低数の制限
 \item[ ] 最高保険金額の制限
 \item[ ] 最低保険金額に対する最高保険金額の倍数の制限
\end{itemize}
\item[ キ)] 団体に職業病、業務上の事故等による特別な危険がある場合は標準より高い保険料率を課す。
\end{itemize}

\section{6.4 団体定期保険の税務}
(過去問での出題なし)

\section{6.5 団体保険の種類}
\problem{H26 生保1問題 1(2)}

団体保険に関する以下の説明に関し、次の①~⑤の空欄に当てはまる適切な語句を記入しなさい。

平成 8 年 11 月より、従来の全員加入型の\wakumaru{①}の見直しを行った\wakumaru{②}が発売された。本商品
は団体が定める弔慰金、死亡退職金の財源を目的としたものである。また、併せて従業員の死亡、
高度障害に伴い団体が負担すべき代替雇用者の採用、育成費用等に対する財源を目的とする
\wakumaru{③}が発売された。

団体信用生命保険は、信用供与機関または信用保証機関が契約者となり、ローン等の借手である
賦払債務者を被保険者として契約するもので、原則として\wakumaru{④}と同一の金額を保険金額として、
賦払債権保全を目的とする団体保険である。

昭和 59 年 10 月、健康保険法改正による受益者負担引上げと退職者医療導入という改革が行わ
れ、医療分野における自助努力の要請が強まる中、\wakumaru{⑤}は、公的医療保険制度の補完的な役割を
担う保険で昭和 61 年に創設された。
\answer{}
\begin{itemize}
\item[ ①: ]  団体定期保険
\item[ ②: ]  総合福祉団体定期保険
\item[ ③: ]  ヒューマン・ヴァリュー特約
\item[ ④: ]  未払債務残高
\item[ ⑤: ]  医療保障保険(団体型)
\end{itemize}
\section{6.6 団体生命保険の数理}
\subsection{保険料}
\problem{H25 生保1問題 1(4)}

団体生命保険に関し、以下の①~⑤の空欄に当てはまる適切な算式を答えなさい。
ただし\wakumaru{⑤}は和の記号Σを使わないで表現すること。

被保険者$n$人で構成され、保険金年末払、保険料年始1回払の団体定期保険を契約している団体とそ
の団体定期保険契約を引き受ける保険会社を考える。

この団体の死亡法則が年齢・性別によらず一律に死亡率$q$の二項分布に従い、各被保険者の保険金額
が全て1と仮定し、割引率を$v$とすると、

\begin{itemize}
\item[] 死亡がなければ支払保険金の現価は0であり、その確率は①と表せる。
\item[] 死亡が1人であれば支払保険金の現価は$v$であり、その確率は②と表せる。
\item[] 死亡が$k$人であれば支払保険金の現価は③であり、その確率は④と表せる。
\item[] すると、支払保険金の現価の平均値は、以下の算式で表せる。
\end{itemize}

$$
\sum\limits_{k=0}^{n}\left( \text{\wakumaru{③}} \times\text{\wakumaru{④}} \right) = \text{\wakumaru{⑤}}
$$

\answer{}
\begin{itemize}
 \item [①] $(1-q)^n$
 \item [②] $_nC_1 q (1-q)^{n-1}$
 \item [③] $kv$
 \item [④] $_nC_k q^{k} (1-q)^{n-k}$
 \item [⑤] $nvq$
\end{itemize}

\problem{H15 生保1問題 1(2)改}
次の空欄に当てはまるもっとも適当な語句、記号または算式を解答せよ。

\noindent{} 1)被保険者$n$人で構成され、保険料は年始に1回払の団体定期保険(各被保険者の保険金額は1、保険金年未払)を契約している団体とその団体定期保険契約を引き受ける保険会社を考える。

この団体において1年間に$k$人 ($k = 0,1,2, \cdots , n$)が死亡すると、一件あたりの保険金額は1なの
で年末に保険会社は$1 \times k$支払うことになり、これを現価で考えると$1 \times kv$を支払うこととなる
($v$は割引率)。そして、この団体の死亡法則が年齢・性別によらず一律に死亡率$q$の二項分布であると仮定するとき$n$人のうち$k$人が死亡する確率は
\wakumaru{①}と表せる。すると、支払保険金の現価が二項分布であるような危険の引受に対し保険会社はその平均値

$$
\sum\limits_{k=0}^{n}(1\times kv\times\text{\wakumaru{①}})=\text{\wakumaru{②}}
$$
\hspace{5zw}(ただし\wakumaru{②}は和の記号Σを使わないで表現すること)

\noindent{}を年払純保険料として収入すればよい。

\noindent{} 2)この団体の被保険者$n$人は十分大きいとすると、この団体に対する支払保険金の分布は平均値\wakumaru{③}、標準偏差\wakumaru{④}
の正規分布で近似できる。

\noindent{} 3)団体定期保険の死差配当金は次式により計算される。

配当金 $= (P-S) \times \gamma(n)$

(ただし、$(P-S)<0$ のときは $(P-S)=0$ とする。)

ここで$P$は純保険料、$S$は支払保険金、$\gamma(n)$は配当係数で経験による\wakumaru{⑤}、危険準備金を満たすように被保険者数の大きさにより定められている。
\answer{}

\begin{itemize}
 \item [①: ]  $_nC_kq^k(1-q)^{n-k}$
 \item [②: ]  $nvq$
 \item [③: ]  $nq$
 \item [④: ]  $\{nq(1-q)\}^{1/2}$
 \item [⑤: ]  保険金プール費用
\end{itemize}

\subsection{平均保険料率}
\problem{H28 生保1問題 1(5)②、H20 生保1問題 1(2)、H8 生保1問題 1(2)、H3 生保1問題 1(4)}
団体定期保険における平均保険料率について説明しなさい。\\
団体生命保険における平均保険料率について、適用する目的と算出方法を簡潔に説明しなさい。
\answer{}
\begin{itemize}
\item[ (目的)] 中途加入や脱退等の移動があるたびに再計算をすることによる事務の煩雑さを回避するため。
\item[ (計算方法)] 被保険者ごとに計算した保険料の合計額を総保険金額で除して算出する。
\end{itemize}

\subsection{経験料率}
\problem{H16 生保1問題 1(4)}
団体生命保険に関する次の文章の空欄を埋めなさい。

団体生命保険における経験料率とは、初年度は被保険団体等の区分による[ ① ]別、
[ ② ]別に定められた料率を基準にして団体の保険料を定めるが、次年度以降はその団体の
[ ③ ]に応じて保険料率を増減して調整する方式である。この経験料率方式には、過去の経験から将来を予想して[ ④ ]の保険料率を定める方式と、過去の経験により配当を支払いその年度の実質保険料を調整する[ ⑤ ]方式とがある。
\answer{}
\begin{itemize}
\item[ ①: ] 男女
\item[ ②: ] 年齢
\item[ ③: ] 死亡実績
\item[ ④: ] 次期以降
\item[ ⑤: ] 配当精算
\end{itemize}

\problem{H14 生保1問題 2(1)②、H1 生保1問題 1(1)}
団体定期保険の経験料率について簡潔に説明せよ。
\answer{}
団体保険は個人保険とは異なり、「個々の保険金の支払の発生によっても契約は消滅せず継続すること」、「団体単位に保険料が払い込まれ団体ごとに死亡発生状況が把握できること」といった特質を有する。

経験料率とは、これらの特質に基づき、団体ごとの死亡実績に応じて次年度以降のその団体の保険料を増減させる方式をいう。技術的には、過去の経験から将来を予測して次期以降の保険料率を定める方式と、過去の経験により配当を支払いその年度の実質保険料を調整する配当精算方式とがある。

わが国においては従来広く後者の方式が行なわれてきたが、死亡実績の優良な団体に対しては前者の方式にあたる優良団体割引制度が実施されてきている。

また、保険料における経験料率方式は1年もしくは数年の個々の団体における支払実績に基づいたものにプーリングの概念を導入して将来の支払額を予測するもので、クレディビリティーの公式によって算出される。

\problem{H12 生保1問題 1(8)}
次の①~⑤について、正しいものには○、誤りのあるものについては×をつけよ。

\begin{itemize}
\item[ ① : ] 経験料率方式には、技術的には過去の経験から将来を予想して次期以降の保険料率を定める方式と、過去の経験により配当を支払いその年度の実質保険料を調整する方式とがある。
\item[ ② : ] 配当清算法では、「保険料+発生利息-保険金費用-剰余金に対する費用」なる算式による剰余金が、経験配当金として契約者に還元される。
\item[ ③ : ] クレディビリティーの公式は、クレディビリティー・ファクターをZとして、
$$Z \times \text{(その団体の実際発生保険金)} +(1 - Z) \times \text{(保険会社の経験から割り出されたその団体の予想発生保険金)}
$$
で表される。
\item[ ④ : ] プール方式の損失限度方式において、団体の発生保険金が定めた一定額に達するまでは、保険金費用は「発生保険金 − プーリング保険料」とし、団体の発生保険金が定めた一定額を超えた場合の保険金費用は、「定めた一定額 − プーリング保険料」となる。
\item[ ⑤ : ] 経験料率を用いて配当清算を行う際に、剰余がマイナスの場合、その団体の損として翌年度以降に持ち越しを行うことをキャリー・フォワードという。
\end{itemize}

\answer{}
\begin{enumerate}
 \item ○
 \item ×  「保険料 + 発生利息 - 保険金費用 - 事業費 - 剰余金に対する費用」
 \item ○ 
 \item × 「-プーリング保険料」→ 「+ プーリング保険料」→
\item ○
\end{enumerate}
\problem{H7 生保1問題 1(4)}
団体定期保険における代表的な 2 つの経験料率方式について説明せよ。
\answer{}
経験料率方式には、①過去の経験から将来を予想して次期以降の保険料率を定め
る方式と、②過去の経験により配当を支払いその年度の実質保険料を調整する配当精
算方式がある。前者は、個々の団体の支払実績に基づき、プーリングの概念を導入し
て将来の支払額を予測するもので、クレディビリティーの公式によって算出される。

一方、後者では、

(保険料+発生利息-事業費-保険金費用-剰余金に対する費用)

なる算式による剰余金が、経験配当金として還元されるが、ここにおいても保険金費
用の決定にはプール方式が導入される。このプール方式には、基本的な2つの方法と
して、信頼度方式と損失限度方式がある。わが国では、当初より損失限度方式の考え
方による配当精算方式が行われてきたと言える。また、3千人以上の収支良好な団体
に対して純保険料の最大30%を割引く特優団体割引制度も行われている。
なお現在では、各社の基礎書類である「保
険料及び責任準備金の算出方法書」の規定に従い取扱われている。

\problem{H10 生保1問題 1(6)}
次の①~⑤を適当な語句で埋めよ。

団体生命保険における経験料率方式のうち、配当精算法では、

保険料 + 発生利息 - 事業費- \wakumaru{①}費用 - \wakumaru{②}に対する費用

から求められる剰余金が経験配当金として契約者に還元される。ここに、
\wakumaru{①}費用の決定には\wakumaru{③}方式が導入され、これには
\wakumaru{④}方式と\wakumaru{⑤}方式の2つの基本的な方式がある。
\answer{}
\begin{itemize}
\item[ ①: ] …保険金
\item[ ②: ] …剰余金
\item[ ③: ] …プール
\item[ ④: ] …信頼度
\item[ ⑤: ] …損失限度(④と⑤は逆順一可)
\end{itemize}

\problem{H9 生保1問題 2(3)}
団体保険における損失限度方式(ストップロス・プーリング)を簡潔に説明せよ。
\answer{}
団体保険の経験料率方式における配当精算法では、
「保険料 + 発生利息 - 事業費 - 保険金費用 - 剰余金に対する費用」による剰余金が、
配当として還元される。この際の保険金費用の決定にはプール方式が使わ
れ、損失限度方式はその一つである。

団体ごとにその規模、経験年数に応じて一定額を定め、団体の発生保険
金が定めた一定額に達するまでは、保険金費用として、発生保険金 + プーリング保険料とし、団体の発生保険金が定めた一定額を超えた場合の保険金費用は、定めた一定額 + プ一リング保険料となる。
この時、発生保険金 - 定めた一定額がプール保険金として取り出されている。
プーリング保険料は、剰余金(保険料)×一定率で決められる。
\problem{H23 生保1問題 2(4)}
わが国の団体生命保険は有配当方式が一般的であり、死差配当部分については、その団体固有の保
険金発生経験を反映させる経験料率の配当精算法を採用し、配当による保険料の遡及的な調整を行な
っている。この団体生命保険数理の特徴である経験料率について説明し、将来的な調整すなわち、経
験料率法による無配当団体生命保険の保険料を設定する場合の設定方法および留意すべき点について
述べなさい。
\answer{}

\begin{itemize}
\item[] 経験料率\\
 経験料率とは、初年度は被保険団体等の区分による男女別、年齢別に定められた料率を基準にし
 て団体の料率を定め、次年度以降はその団体の死亡実績を過去の経験値とし、それに応じて保険
 料率を増減して調整する方法である。この経験値については、以下の2つがべ一スとなる。
\begin{itemize}
\item[①] 保険会社の経験から割り出された支払経験率…団体定期保険の被保険団体を団体の被保険者の規
 模や団体の危険程度等によりクラス分けした各クラスの総合経験によるもの。(1)プーリング部
 分と称される。
\item[②] 団体固有の支払経験率…その団体の過去の経験によるもの
\end{itemize}
\item[] 保険料の設定方法
\begin{itemize}
\item[(A)] 通常の小規模単体、標準下体集団、危険職種集団に適用する料率\\
 「保険料を全額プーリングにより定める」
\item[(B)] 通常の大規模団体に適用する料率
\begin{itemize}
\item[ ① ] 「信頼性理論を利用し、保険料の一部をプーリング、残りをその団体自身の経験により定める」\\
 「信頼性理論の概要」\par
 保険料率=Z×T+(1-Z)×M\\
\begin{itemize}
\item[  T: ] 団体固有の支払経験率
\item[ M: ]   保険会社の経験から割り出された支払経験率
\item[ Z: ]   信頼度(Credibility factor)…団体の大きさ、経験年数等により保険会社ごとの独自の
 裁量で決められる。通常被保険者数が多いほど1に近い数値となる。信頼度の算出は、様々
 な方法が存在するが、統計の区間推定を使う「有限変動信頼性理論」やベイズ統計的な手
 法を応用した「ビュールマンモデル」等を使用する。
\end{itemize}
\item[ ② ]「優良体割引数理の考え方の応用」\\
 大規模団体については、有配当団体生命保険の優良体割引の考え方を応用し、死亡実績の優良な
 団体において安全率を極力小さくし、事後の配当清算を無くし保険料を低廉化した形のプライシ
 シグも考えられる。
\end{itemize}
\end{itemize}
 保険料設定上の留意点
\begin{itemize}
\item[]契約団体の構成の困難さ\par
 プーリング団体の被保険者数範囲に明確な基準がないため、被保険者数範囲の設定について一部
 優良団体で各社判断による扱いに差が生じ、結果的に契約団体の構成がひずみ易い。
\item[]逆選択防止\par
 配当による事後精算がないため、有配当保険より逆選択が働きやすいと考えられるため、全員加
 入団体のみへの販売にするもしくは、任意加入団体においては加入率要件、倍率制限をさらに厳
 しくする等の対応が必要となる。
\item[]保険料水準\par
 無配当保険といえども必ず安全割増を必要とし、かつ剰余を返還しないので、原理的に正味掛金
 は割高となる。一方、配当還元前の営業保険料水準としては当然に有配当保険よりも低廉でなく
 てはならない。
\item[]事業費の設定\par
 配当支払い事務コストを無配当保険により軽減し、その事業費コストを反映させる場合、団体ご
 との経験料率の適用にともなう追加コストが配当支払い事務コストを上回ることのないようにし
 なければならない。
\item[]死差損団体への対応\par
 理論的には死亡実績が悪い団体では保険料の見直しにおいて高料を求めることになるが、予め特
 別保険料率の適用基準等を作成し、保険料の見直しルールについて事前に団体と話し合い、よく
 納得された合意をとっておく必要がある。販売上等の問題から、保険料の高料が難しい場合には、
 ストップロス方式の再保険も有用と考えられる。
\item[]危険準備金積立てコストの考慮\par
 販売当初においては、信頼度Zの設定において危険準備金積み立てのコストを勘案し保守的に設
 足する必要がある。
\end{itemize}
\end{itemize}
%Anki kokomade
\subsection{優良体割引}

\problem{H28 生保1問題 1(5)①、H19 生保1問題 1(3)、H2 生保1問題 1(4)}
団体保険の優良団体割引制度について、以下の問いに答えよ。

\begin{itemize}
\item[ ①: ]  優良団体割引制度について簡潔に説明せよ。
\item[ ②: ]  団体保険の優良団体割引率を設定するにあたっての留意点を3つ挙げよ。
\item[ ③: ]  以下の記号および前提を用いて優良団体割引制度の適用の算式を示し、前提にもとづく割引率を計算せよ。解答にあたり、団体の死亡率は正規分布に従うものと仮定し、その信頼区間の片側20%点を割引率設定の基準とせよ。なお、優良団体割引の適用団体と非適用団体の間で支払率を同水準に保つための安全割増は考慮しないものとし、割引率は%単位で小数点以下第2位を四捨五入して小数点以下第1位とする。
\end{itemize}

【記号および前提】

\begin{itemize}
\item[] 団体の実績に基づく死亡率:γ=1.2‰
\item[] 団体の年齢構成等を基に保険金杜の経験から割り出された死亡率:γ’=2.0‰
\item[] 標準正規分布の片側20%の値:u(20)=0.84
\item[] 被保険者数:N=7,500人
\item[] 割引率:α
\end{itemize}
\answer{}

①

死亡実績の優良な大団体について保険料を引き下げる制度であり、団体保険の経験料率の適用例である。
昭和49年の団体定期保険の運営基準の改訂により、5,000人以上の団体について、
純保険料の20%を限度とする割引制度が導入された。その後、基準の弾力化が行われ、3,000人以上の団体について、
純保険料の30%限度まで適用範囲が拡大されている。また、昭和59年度からは、団体信用保険にも当制度が適用されている。

②
\begin{itemize}
\item[] 一旦割引を行ったら死差損にならない限りその割引率を継続すること、また死差損を翌保険年度に繰り越すことができないことなどを考慮して、できるだけ大きい人数規模とすること。
\item[] 割引を適用する団体と適用しない団体との間で費用負担の公平性が図られていること。すなわち、適用団体と非適用団体の間で支払率を同水準に保つこと。
\item[] 割引を行うことに実質的な意味があること。
\end{itemize}

③

算式

$$
\gamma+u(20)\sqrt{\frac{\gamma}{N}} = \bar{\gamma}(1-\alpha)
$$

計算結果

\begin{align}
\alpha & = 1-\left(\gamma+u(20)\sqrt{\frac{\gamma}{N}}\right)/\bar{\gamma}\\
&= 1-\left(0.0012 + 0.84\sqrt{\frac{0.0012}{7500}}\right)/0.002=23.2\%
\end{align}

\section{6.7 配当}
\problem{H4 生保1問題 2(2)}
団体定期保険の現在の配当方式について説明せよ。
\answer{}
団体定期保険の配当は,団体毎の保険金発生経験を反映させる経験料率による
配当精算方式を採用しているため、団体毎の保険年度毎の収支が重要となり、個
人保険の配当方式と異なり2年目配当を採用している。

配当金は(純保険料一発生保険金)X配当係数により計算される。配当係数は、
経験による保険金プール費用、危険準備金積立を満たすように,また,利息によ
る増加も考慮して被保険者数の大きさにより定められる。被保険者数が大きいほ
ど、収支が安定しプーリング部分の割合が少なくてすむため、配当係数は大きく
なる。また,団体毎の収支が負の場合はそれをゼロとして計算し、損を次年度に
繰り起さずに、できるだけ団体の大きさが同一のランク内でカバーする考えをと
っている。
\problem{H11 生保1問題 1(3)}
次の①~⑤を適当な語句で埋めよ。

一般に、我が国の団体生命保険の死差配当は、\wakumaru{①}方式の考え方による配当精算方式をとってい
るが、不足額の\wakumaru{②}は行われていない。具体的には、次の算式により計算される。

配当金 = (P - S) × r(n)

ここでPは\wakumaru{③}、Sは\wakumaru{④}、r(n)は配当係数で経験による\wakumaru{⑤}および
危険準備金積立を満たすように被保険者数の大きさにより定められる。

\answer{}

\begin{itemize}
\item[ ①: ] ストップロス・プーリング(損失限度)
\item[ ②: ] キャリー・フォワード
\item[ ③: ] 純保険料
\item[ ④: ] 発生(支払)保険金
\item[ ⑤: ] 保険金プール費用
\end{itemize}


\end{document}