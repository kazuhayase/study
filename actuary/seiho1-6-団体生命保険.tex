\documentclass[report,gutter=10mm,fore-edge=10mm,uplatex,dvipdfmx]{jlreq}

\usepackage{lmodern}
\usepackage{amssymb,amsmath}
\usepackage{ifxetex,ifluatex}
\usepackage{actuarialsymbol}
\usepackage[]{natbib}
\RequirePackage{plautopatch}

% maru suji ① etc.
\usepackage{tikz}
\newcommand{\cir}[1]{\tikz[baseline]{%
\node[anchor=base, draw, circle, inner sep=0, minimum width=1.2em]{#1};}}

\usepackage{comment}

\begin{comment}

\ifnum0\ifxetex1\fi\ifluatex1\fi=0 % if pdftex
  \usepackage[T1]{fontenc}
  \usepackage[utf8]{inputenc}
  \usepackage{textcomp} % provide euro and other symbols
\else % if luatex or xetex
  \usepackage{unicode-math}
  \defaultfontfeatures{Scale=MatchLowercase}
  \defaultfontfeatures[\rmfamily]{Ligatures=TeX,Scale=1}
\fi
% Use upquote if available, for straight quotes in verbatim environments
\IfFileExists{upquote.sty}{\usepackage{upquote}}{}
\IfFileExists{microtype.sty}{% use microtype if available
  \usepackage[]{microtype}
  \UseMicrotypeSet[protrusion]{basicmath} % disable protrusion for tt fonts
}{}
\makeatletter
\@ifundefined{KOMAClassName}{% if non-KOMA class
  \IfFileExists{parskip.sty}{%
    \usepackage{parskip}
  }{% else
    \setlength{\parindent}{0pt}
    \setlength{\parskip}{6pt plus 2pt minus 1pt}}
}{% if KOMA class
  \KOMAoptions{parskip=half}}
\makeatother
\usepackage{xcolor}
\IfFileExists{xurl.sty}{\usepackage{xurl}}{} % add URL line breaks if available
\IfFileExists{bookmark.sty}{\usepackage{bookmark}}{\usepackage{hyperref}}
\hypersetup{
  hidelinks,
  pdfcreator={LaTeX via pandoc}}
\urlstyle{same} % disable monospaced font for URLs
\usepackage{longtable,booktabs}
% Correct order of tables after \paragraph or \subparagraph
\usepackage{etoolbox}
\makeatletter
\patchcmd\longtable{\par}{\if@noskipsec\mbox{}\fi\par}{}{}
\makeatother
% Allow footnotes in longtable head/foot
\IfFileExists{footnotehyper.sty}{\usepackage{footnotehyper}}{\usepackage{footnote}}

\end{comment}
%\makesavenoteenv{longtable}
\setlength{\emergencystretch}{3em} % prevent overfull lines
\providecommand{\tightlist}{%
  \setlength{\itemsep}{0pt}\setlength{\parskip}{0pt}}
\setcounter{secnumdepth}{-\maxdimen} % remove section numbering

\author{kazuyoshi}
\date{}

\newcommand{\problem}[1]{\subsubsection{#1}\setcounter{equation}{0}}
%\newcommand{\answer}[1]{\subsubsection{#1}}
\newcommand{\answer}[1]{\subsubsection{解答}}

%Pdf%\newcommand{\wakumaru}[1]{\framebox[3zw]{#1}}
\newcommand{\wakumaru}[1]{#1}





\begin{document}
\chapter{保険1第6章 団体生命保険}
\section{6.1 はじめに,  6.2 日本における団体生命保険の沿革}
(過去問での出題なし)
\section{6.3 団体生命保険の危険選択}
\subsection{団体保険の危険選択の目的と選択理論}
\problem{H13 生保1問題 1(5)}
団体保険に関して、選択を行う際のグレッグによる原則について、次の①~⑤を適当な語句で埋めよ。

\begin{itemize}
\item[ (a)]  社会的な利害関係によって結成された個人の集団ではなく、 \wakumaru{①}を目的として組織された場合は、逆選択の危険性がある。
\item[ (b)]  若く健康な新規加入者があり、老齢で健康を害したものが団体から脱退すれば団体の\wakumaru{②}は安定する。
\item[ (c)]  各個人の保険金額は、例えば、全加入者について同一金額または各人の給与・勤続年数・資格等により\wakumaru{③}な基準で決定されるべきである。
\item[ (d)]  団体の従業員の\wakumaru{④}または\wakumaru{④}に近い者が加入することにより逆選択を防止できる。
\item[ (e)]  \wakumaru{⑤}が簡単である。
\end{itemize}

\answer{}
\begin{itemize}
\item[ ①: ] 保険加入
\item[ ②: ] 死亡率
\item[ ③: ] 客観的
\item[ ④: ] 全員
\item[ ⑤: ] 管理
\end{itemize}

\problem{H27 生保1問題 1(3)、H18 生保1問題 1(2)、H10 生保1問題 1(10)}
団体生命保険の危機選択に関し、団体による選択では各団体のリスクの均質性が前提であるが、ほか
に考慮する必要のある点を 5 つ列挙しなさい。(グレッグの挙げた原則)
\answer{}
\begin{itemize}
\item[]  保険加入目的のための団体ではないこと
\item[]  団体に加入脱退があること
\item[]  保険金額が客観的に決まること
\item[]  団体の一定以上の割合が加入すること
\item[]  管理が簡単であること
\end{itemize}

\problem{H14 生保1問題 2(1)①}
団体による危険選択を行うことの趣旨および留意点を述べよ。
\answer{}

危険選択の目的は、損失の発生がある程度予想できる程度に同種類の危険体を数多く集め
ることである。しかし、選択の結果、死亡率は低く抑えられても保険加入者が少なく危険単
位の量が少なければ危険の予測が困難となる。したがって、選択基準の厳格性と多数の危険
単位の引受の必要性との調和が必要である。団体生命保険の選択基準も危険単位を1団体と
考え、この調和に留意している。

【趣旨】

\begin{itemize}
\item[ ア)] 将来の結果が予測できるよう契約の同質化を図るとともに契約の量との調和を図ることが必要である。
\item[ イ)] 大多数の団体が標準料率で契約できるような基準を設ける。
\item[ ウ)] 種々の組分けの中にできるだけ多く一定水準以上の団体を含ませる。
\end{itemize}

【留意点】

\begin{itemize}
\item[ ア)] 保険加入目的のための団体でないこと。
\item[ イ)] 団体に加入、脱退があること。
\item[ ウ)] 保険金額が客観的に決まること。
\item[ 工)] 団体の一定以上の割合が加入すること。
\item[ オ)] 管理が簡単であること。
\item[ カ)] 危険論の見地からの留意点
\begin{itemize}
 \item[ ] 加入者の最低数の制限
 \item[ ] 最高保険金額の制限
 \item[ ] 最低保険金額に対する最高保険金額の倍数の制限
\end{itemize}
\item[ キ)] 団体に職業病、業務上の事故等による特別な危険がある場合は標準より高い保険料率を課す。
\end{itemize}
\section{6.4 団体定期保険の税務}
(過去問での出題なし)

\section{6.5 団体保険の種類}
\problem{H26 生保1問題 1(2)}

団体保険に関する以下の説明に関し、次の①~⑤の空欄に当てはまる適切な語句を記入しなさい。

平成 8 年 11 月より、従来の全員加入型の\wakumaru{①}の見直しを行った\wakumaru{②}が発売された。本商品
は団体が定める弔慰金、死亡退職金の財源を目的としたものである。また、併せて従業員の死亡、
高度障害に伴い団体が負担すべき代替雇用者の採用、育成費用等に対する財源を目的とする
\wakumaru{③}が発売された。

団体信用生命保険は、信用供与機関または信用保証機関が契約者となり、ローン等の借手である
賦払債務者を被保険者として契約するもので、原則として\wakumaru{④}と同一の金額を保険金額として、
賦払債権保全を目的とする団体保険である。

昭和 59 年 10 月、健康保険法改正による受益者負担引上げと退職者医療導入という改革が行わ
れ、医療分野における自助努力の要請が強まる中、\wakumaru{⑤}は、公的医療保険制度の補完的な役割を
担う保険で昭和 61 年に創設された。
\answer{}
\begin{itemize}
\item[ ①: ]  団体定期保険
\item[ ②: ]  総合福祉団体定期保険
\item[ ③: ]  ヒューマン・ヴァリュー特約
\item[ ④: ]  未払債務残高
\item[ ⑤: ]  医療保障保険(団体型)
\end{itemize}
\section{6.6 団生命保険の数理}
\subsection{保険料}
\problem{H25 生保1問題 1(4)}

団体生命保険に関し、以下の①~⑤の空欄に当てはまる適切な算式を答えなさい。
ただし\wakumaru{⑤}は和の記号Σを使わないで表現すること。

被保険者$n$人で構成され、保険金年末払、保険料年始1回払の団体定期保険を契約している団体とそ
の団体定期保険契約を引き受ける保険会社を考える。

この団体の死亡法則が年齢・性別によらず一律に死亡率$q$の二項分布に従い、各被保険者の保険金額
が全て1と仮定し、割引率を$v$とすると、

\begin{itemize}
\item[] 死亡がなければ支払保険金の現価は0であり、その確率は①と表せる。
\item[] 死亡が1人であれば支払保険金の現価は$v$であり、その確率は②と表せる。
\item[] 死亡が$k$人であれば支払保険金の現価は③であり、その確率は④と表せる。
\item[] すると、支払保険金の現価の平均値は、以下の算式で表せる。
\end{itemize}

$$
\sum\limits_{k=0}^{n}\left( \text{\wakumaru{③}} \times\text{\wakumaru{④}} \right) = \text{\wakumaru{⑤}}
$$

\answer{}
\begin{itemize}
 \item [①] $(1-q)^n)$
 \item [②] $_nC_1 q (1-q)^{n-1}$
 \item [③] $kv$
 \item [④] $_nC_k q^{k} (1-q)^{n-k}$
 \item [⑤] $nvq$
\end{itemize}

\end{document}