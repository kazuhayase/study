\documentclass[report,gutter=10mm,fore-edge=10mm,uplatex,dvipdfmx]{jlreq}

\usepackage{lmodern}
\usepackage{amssymb,amsmath}
\usepackage{ifxetex,ifluatex}
\usepackage{actuarialsymbol}
\usepackage[]{natbib}
\RequirePackage{plautopatch}

% maru suji ① etc.
\usepackage{tikz}
\newcommand{\cir}[1]{\tikz[baseline]{%
\node[anchor=base, draw, circle, inner sep=0, minimum width=1.2em]{#1};}}

\usepackage{comment}

\begin{comment}

\ifnum0\ifxetex1\fi\ifluatex1\fi=0 % if pdftex
  \usepackage[T1]{fontenc}
  \usepackage[utf8]{inputenc}
  \usepackage{textcomp} % provide euro and other symbols
\else % if luatex or xetex
  \usepackage{unicode-math}
  \defaultfontfeatures{Scale=MatchLowercase}
  \defaultfontfeatures[\rmfamily]{Ligatures=TeX,Scale=1}
\fi
% Use upquote if available, for straight quotes in verbatim environments
\IfFileExists{upquote.sty}{\usepackage{upquote}}{}
\IfFileExists{microtype.sty}{% use microtype if available
  \usepackage[]{microtype}
  \UseMicrotypeSet[protrusion]{basicmath} % disable protrusion for tt fonts
}{}
\makeatletter
\@ifundefined{KOMAClassName}{% if non-KOMA class
  \IfFileExists{parskip.sty}{%
    \usepackage{parskip}
  }{% else
    \setlength{\parindent}{0pt}
    \setlength{\parskip}{6pt plus 2pt minus 1pt}}
}{% if KOMA class
  \KOMAoptions{parskip=half}}
\makeatother
\usepackage{xcolor}
\IfFileExists{xurl.sty}{\usepackage{xurl}}{} % add URL line breaks if available
\IfFileExists{bookmark.sty}{\usepackage{bookmark}}{\usepackage{hyperref}}
\hypersetup{
  hidelinks,
  pdfcreator={LaTeX via pandoc}}
\urlstyle{same} % disable monospaced font for URLs
\usepackage{longtable,booktabs}
% Correct order of tables after \paragraph or \subparagraph
\usepackage{etoolbox}
\makeatletter
\patchcmd\longtable{\par}{\if@noskipsec\mbox{}\fi\par}{}{}
\makeatother
% Allow footnotes in longtable head/foot
\IfFileExists{footnotehyper.sty}{\usepackage{footnotehyper}}{\usepackage{footnote}}

\end{comment}
%\makesavenoteenv{longtable}
\setlength{\emergencystretch}{3em} % prevent overfull lines
\providecommand{\tightlist}{%
  \setlength{\itemsep}{0pt}\setlength{\parskip}{0pt}}
\setcounter{secnumdepth}{-\maxdimen} % remove section numbering

\author{kazuyoshi}
\date{}

\newcommand{\problem}[1]{\subsubsection{#1}\setcounter{equation}{0}}
%\newcommand{\answer}[1]{\subsubsection{#1}}
\newcommand{\answer}[1]{\subsubsection{解答}}

%Pdf%\newcommand{\wakumaru}[1]{\framebox[3zw]{#1}}
\newcommand{\wakumaru}[1]{#1}







\begin{document}
\chapter{保険1第4章 生命保険の商品開発}
\section{4.1 生命保険商品の開発・改定・運営にあたっての基本的な考え方}
\section{4.2 商品開発プロセス}
\problem{2018 生保1問題 1(2)}
生命保険商品の開発・改定における「事後モニタリングと改善アクション」の目的について、
簡潔に説明しなさい。また、事後モニタリングにおいて、ある商品の給付発生指数(予定発生率
に対する実績発生率の割合)を確認したところ、予定発生率設定時の見通しより高いことが判明
した。このとき、考えられる要因、保険料率変更以外の販売継続を目的とした改善アクション、
および副作用について、それぞれ1つずつ挙げなさい。

\answer{解答}
\paragraph{目的}
商品開発時だけでなく、販売後のモニタリング結果に応じて機動的に販売
施策や価格を調整していくPDCAサイクルによって、長期間にわたる商
品事業の健全性をより強固にする。\\

\paragraph{考えられる要因}
\begin{enumerate}
 \item 公式解答: 想定外の高リスク集団の混入 
 \item 教科書事例:価格設定時の保険給付見通しが、楽観的であったか
増加トレンドを十分に想定できていなかった。
\end{enumerate}

\paragraph{改善アクション}
\begin{enumerate}
 \item 公式解答: 引受査定基準の見直し
 \item 教科書事例:保険料率の見直し、または危険選択基準の見直し。
\end{enumerate}

\paragraph{副作用}
\begin{enumerate}
 \item 公式解答: 販売量の低下
 \item 教科書事例:競合環境における競争力の低下を通じた、販売量の減少。
\end{enumerate}

\problem{2019 生保1問題 1(2)}
\footnote{* WBでは「その他」 1. 監督指針にあり。後で移す}
保険会社向けの総合的な監督指針「Ⅱ-2-5商品開発に係る内部管理態勢」について、次
の①~⑤に適切な語句を記入しなさい。

Ⅱ-2-5 商品開発に係る内部管理態勢\\
Ⅱ-2-5-1 意義\\
保険商品の内容は「普通保険約款」及び「①」に、料率については「②」に
記載されており、新商品の開発、商品内容の変更は、これらの変更を通じて行われている。

保険会社より商品の③申請が行われた場合、監督当局としては、契約内容が保険契約
者等の保護に欠けるおそれがないか、不当な差別的取扱いをするものではないか、契約内容が公
序良俗を害するものではないか等の保険業法に定める基準に適合するものであるか審査を行い、
適当と認められたものについて、これを③することとしている。

近年、保険商品には、わが国における社会の構造的変化・経済活動の多様化等に伴い、国民の
生活保障ニーズの高まり、新たなリスクの発生など、保険契約者ニーズに対応すべく多様化が求
められている。

こうしたニーズに応え、保険会社が商品開発を行うにあたっては、保険業法等の法令等を踏ま
え、④に基づき、リスク面、財務面、⑤、法制面等あらゆる観点から検討する
内部管理態勢の整備が求められているところである。
\answer{解答}
① 事業方法書
② 保険料及び責任準備金の算出方法書
③ 認可
④ 自己責任原則
⑤ 募集面

\section{4.3 生命保険商品の開発・事業運営の構成要素(商品設計)}
\problem{2022 生保1問題2(1)(イ)}
商品設計や計算基礎率の設定によっては、保険料全期払の定期保険(第三分野を含む)にお
ける平準純保険料式責任準備金が期中でマイナスになることがある。このようないわゆる負値
責任準備金が発生する具体例を2つ挙げ、それが発生する理由をそれぞれ説明しなさい。また、
負値責任準備金が発生した場合の課題や留意すべき点(会計の観点は除く)について簡潔に説
明しなさい。(500字程度)(5点)

\answer{解答}
\paragraph{<具体例>} 
\begin{enumerate} [◯]
 
\item 保険期間の途中で、予定死亡率や予定発生率が年齢の上昇とともに逓減する場合

 \begin{itemize}
 \item 一般に、予定死亡率等が年齢とともに上昇すると、危険保険料も年齢とともに上昇する。将来の高くなる危険保険料に備えて、平準払純保険料の一部である貯蓄保険料を経過が浅い期間に保険料積立金として積み立てるので、責任準備金は正値となる。
 \item ところが予定発生率等が年齢の上昇につれて逓減する場合は、純保険料は平準化されて収入することから、契約初期の危険保険料が純保険料を超過した状態が発生しうる。つまり、将来の平準払純保険料から契約初期の危険保険料を賄っている状況(貯蓄保険料がマイナス)となり、責任準備金は負値となることがある。
 \end{itemize}

\item 保険金額逓減型や収入保障特約のような、保障額が一定ではなく時間の経過とともに減少するような商品設計の場合

 \begin{itemize}
 \item 一般に、保障額が一定ならば、危険保険料は年齢とともに上昇し、将来の高くなる危険保険料に備
 えて、平準払純保険料の一部である貯蓄保険料を経過が浅い期間に保険料積立金として積み立てる ので、責任準備金は正値となる。
 \item ところが保障額が逓減する場合は、予定発生率等が年齢とともに上昇しても、危険保険金が減少し
 ているために危険保険料が年齢とともに上昇しないことがあり、純保険料は平準化されて収入する ことから、契約初期の危険保険料が純保険料を超過した状態が発生しうる。つまり、将来の平準払
 純保険料から契約初期の危険保険料を賄っている状況(貯蓄保険料がマイナス)となり、責任準備 金は負値となることがある。
 \end{itemize}

 \item 一定時期以降の保険料が上昇するような商品設計の場合

 \begin{itemize}
 \item 一定時期以降の保険料が高い分、その時期を経過した後の方が、経過する前に比べて、純保険料の
 一部をより多く保険料積立金として積み立てられるので、保険期間の満期で責任準備金が 0 となる
 定期保険においては、満期の前の時点で責任準備金が負値となることがある。
 \end{itemize}
\end{enumerate}

\paragraph{<留意点>}

\begin{itemize}
 \item 責任準備金が負値の状態で解約された場合、解約返戻金は 0 で処理され、不足分を契約者から回収 することが出来ず、収入保険料が不足した状態で契約が消滅するため、保険契約全体での収益(収 支バランス)の悪化や、解約者と継続者での公平性の問題が生じる可能性がある。
 \item 負値の責任準備金が生じている状況は、契約時年齢が高まるにつれて平準払営業保険料が逓減する 可能性があり、責任準備金が負値となる間に解約、加入しなおした方が保険料を安くできるケース
 や必要以上に長い保障期間で契約しつつ必要な保障期間で解約することで解約返戻金を受け取れ るケースが想定される。このとき、解約の増加や、解約者と継続者での公平性の問題が生じる可能
 性がある。
\end{itemize}

\problem{2021 生保1問題 1(4)}
転換制度について、
「転換価格」の構成要素を3つ挙げなさい。また、
「転換価格」以外に転換前
契約から引き継がれるべき要素を3つ挙げなさい。

\answer{解答}
\paragraph{構成要素}
転換時点での契約者価額としての保険料積立金、支払を据え置いた給付金、配当の積立残高\\
※この他にも「契約者貸付の精算」など正しい記述に対して適宜加点した。
\paragraph{引き継がれるべき要素}
自殺免責ルールの起算点、危険選択における転換前の保障額、解約控除\\
※この他にも「契約者配当の権利」など正しい記述に対して適宜加点した。

\problem{2020 生保1問題 2(2)}

生命保険の商品における更新制度について、次の①~③の各問に答えなさい。
\begin{enumerate}
 \item [①] 更新制度について、簡潔に説明しなさい。また、終身保障の商品と比較した場合の更新型商品の
 保険会社および保険契約者におけるそれぞれのメリットについて、簡潔に説明しなさい。(4点)
 \item [②] 将来収支の健全性や制御可能性における論点のうち、価格設定つまり更新後の保険料水準の十分
 性について留意すべき事項を挙げ、簡潔に説明しなさい。(4点)
 \item [③] 更新型として適している商品を2つ挙げ、そのように考える理由を簡潔に説明しなさい。 (2点)
\end{enumerate}

\answer{解答}
①\\
生命保険の更新制度とは、保険期間が有期の保険契約が満期を迎える時、契約者の保障継続ニーズに
応える機能として、同じ保障内容で契約を継続できる権利を保険会社が契約者に提供するものである。
終身保障の商品と比較した場合の更新型商品の保険会社のメリットは、保険会社が更新時に保険料を
変更できることにより、不確実性の高い給付事由に対し、実態を反映することができるため、死差損
リスクを抑制しつつ、過度に保守的でない保険料で保障することが可能となることである。契約者の
メリットは、まとまった保障を確保しつつ当面の保険料負担を抑えることができることである。また、
更新前と同じ保障内容で更新できるほか、
「更新しない」「保険金額を減らして更新する」など保障を
効率的に見直すことができることである。

②\\
更新後の保険料水準の十分性は、以下の点に留意する必要がある。
\begin{itemize}
 \item 更新時の事業費\\
 全く新しい顧客から新契約を獲得する場合と比べ、更新手続にかかる事業費は低いことが想定され
 る。
 \item 危険選択の有無\\
 更新の際に改めて危険選択を行う場合は、更新後の契約群団の保険引受リスク量をコントロールで
 きることになるが、更新の際に被保険者の健康状態に応じて割増保険料の適用や謝絶することが許
 容されるのかに留意する必要がある。一方、更新の際に危険選択を行わない場合は、更新手続きに係
 る事業費がさらに低廉となることが期待されるが、相対的に保険給付リスクの高い契約がより更新
 される(リスク濃縮)傾向となる可能性・程度に留意する必要がある。
 \item 価格設定\\
 更新後の保険料水準の十分性は、上記の事業費効率や危険選択の有無による保険給付リスクの程度
 を測って検証されることとなる。さらに、リスク濃縮を保険料に反映することによる更なるリスク
 濃縮の懸念や、競合環境において他社の類似商品の保険料よりも高い場合、保険料水準によって引
 き起こされる契約者行動の有無・程度を留意する必要がある。
\end{itemize}
③
\begin{itemize}
 \item 先進医療特約\\
 給付対象となる先進医療技術は入れ替えがあり、保険会社は現時点の給付実績で将来の給付率を適
 切に予測することが難しいため。
 \item 定期保険(死亡保障)\\
 契約者は、契約当初に低廉な保険料で大きな保障を得ることができ、更新時にライフサイクルの変化
 に応じて、保障内容を見直すことができるため。
\end{itemize}
※③はあくまで解答例であり、この他にも正しい記述に対して適宜加点した。

\section{4.4 生命保険商品の開発・事業運営の構成要素(商品設計以外)}

\end{document} 