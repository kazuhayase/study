\documentclass[report,gutter=10mm,fore-edge=10mm,uplatex,dvipdfmx]{jlreq}

\usepackage{lmodern}
\usepackage{amssymb,amsmath}
\usepackage{ifxetex,ifluatex}
\usepackage{actuarialsymbol}
\usepackage[]{natbib}
\RequirePackage{plautopatch}

% maru suji ① etc.
\usepackage{tikz}
\newcommand{\cir}[1]{\tikz[baseline]{%
\node[anchor=base, draw, circle, inner sep=0, minimum width=1.2em]{#1};}}

\usepackage{comment}

\begin{comment}

\ifnum0\ifxetex1\fi\ifluatex1\fi=0 % if pdftex
  \usepackage[T1]{fontenc}
  \usepackage[utf8]{inputenc}
  \usepackage{textcomp} % provide euro and other symbols
\else % if luatex or xetex
  \usepackage{unicode-math}
  \defaultfontfeatures{Scale=MatchLowercase}
  \defaultfontfeatures[\rmfamily]{Ligatures=TeX,Scale=1}
\fi
% Use upquote if available, for straight quotes in verbatim environments
\IfFileExists{upquote.sty}{\usepackage{upquote}}{}
\IfFileExists{microtype.sty}{% use microtype if available
  \usepackage[]{microtype}
  \UseMicrotypeSet[protrusion]{basicmath} % disable protrusion for tt fonts
}{}
\makeatletter
\@ifundefined{KOMAClassName}{% if non-KOMA class
  \IfFileExists{parskip.sty}{%
    \usepackage{parskip}
  }{% else
    \setlength{\parindent}{0pt}
    \setlength{\parskip}{6pt plus 2pt minus 1pt}}
}{% if KOMA class
  \KOMAoptions{parskip=half}}
\makeatother
\usepackage{xcolor}
\IfFileExists{xurl.sty}{\usepackage{xurl}}{} % add URL line breaks if available
\IfFileExists{bookmark.sty}{\usepackage{bookmark}}{\usepackage{hyperref}}
\hypersetup{
  hidelinks,
  pdfcreator={LaTeX via pandoc}}
\urlstyle{same} % disable monospaced font for URLs
\usepackage{longtable,booktabs}
% Correct order of tables after \paragraph or \subparagraph
\usepackage{etoolbox}
\makeatletter
\patchcmd\longtable{\par}{\if@noskipsec\mbox{}\fi\par}{}{}
\makeatother
% Allow footnotes in longtable head/foot
\IfFileExists{footnotehyper.sty}{\usepackage{footnotehyper}}{\usepackage{footnote}}

\end{comment}
%\makesavenoteenv{longtable}
\setlength{\emergencystretch}{3em} % prevent overfull lines
\providecommand{\tightlist}{%
  \setlength{\itemsep}{0pt}\setlength{\parskip}{0pt}}
\setcounter{secnumdepth}{-\maxdimen} % remove section numbering

\author{kazuyoshi}
\date{}

\newcommand{\problem}[1]{\subsubsection{#1}\setcounter{equation}{0}}
%\newcommand{\answer}[1]{\subsubsection{#1}}
\newcommand{\answer}[1]{\subsubsection{解答}}

%Pdf%\newcommand{\wakumaru}[1]{\framebox[3zw]{#1}}
\newcommand{\wakumaru}[1]{#1}






\begin{document}
\chapter{保険1 その他}
\section{1. 監督指針}

\problem{2019 生保1問題 1(2)【監督指針】}
保険会社向けの総合的な監督指針「Ⅱ-2-5 商品開発に係る内部管理態勢」について、次の①~⑤に適切な語句を記入しなさい。

\noindent{}Ⅱ-2-5 商品開発に係る内部管理態勢\\
Ⅱ-2-5-1 意義

保険商品の内容は「普通保険約款」及び「\wakumaru{①}」に、料率については「\wakumaru{②}」に記載されてお
り、新商品の開発、商品内容の変更は、これらの変更を通じて行われている。

保険会社より商品の\wakumaru{③}申請が行われた場合、監督当局としては、契約内容が保険契約者等の
保護に欠けるおそれがないか、不当な差別的取扱いをするものではないか、契約内容が公序良俗を害
するものではないか等の保険業法に定める基準に適合するものであるか審査を行い、適当と認めら
れたものについて、これを\wakumaru{③}することとしている。

近年、保険商品には、わが国における社会の構造的変化・経済活動の多様化等に伴い、国民の生活
保障ニーズの高まり、新たなリスクの発生など、保険契約者ニーズに対応すべく多様化が求められて
いる。

こうしたニーズに応え、保険会社が商品開発を行うにあたっては、保険業法等の法令等を踏まえ、
\wakumaru{④}に基づき、リスク面、財務面、\wakumaru{⑤}、法制面等あらゆる観点から検討する内部管理態勢の整備が求められているところである。

\answer{}
\begin{itemize}
\item[ ① :] 事業方法書
\item[ ② :] 保険料及び責任準備金の算出方法書
\item[ ③ :] 認可
\item[ ④ :] 自己責任原則
\item[ ⑤ :] 募集面
\end{itemize}

\problem{H18 生保1問題 1(1)【監督指針】}

金融庁の『保険会社向けの総合的な監督指針』のうち「Ⅳ保険商品審査上の留意点等 Ⅳ-5保険数
理」に規定されている、保険料及び責任準備金の算出方法書の審査に当たっての留意点のうち、保険料
に係わる項目(抜粋)について下記の空欄を埋めよ。

(1)保険料の算出方法については、 \wakumaru{①}や公平性等を考慮して、合理的かつ妥当なものとなっているか。

(2)保険料については、被保険者群団間及び保険種類間等で、不当な差別的扱いをするものとなっていないか。

(3)予定発生率・損害額又は予定解約率等については、基礎データに基づいて合理的に算出が行われ、かつ、基礎データの\wakumaru{②}に応じた補整が行われているか。(以下略)

(4)予定利率については、保険種類、保険期間、保険料の払方、運用実績や将来の利回り予想等を基に、合理的かつ\wakumaru{③}な観点から適切な設定が行われているか。

(5)(略)

(6)付加保険料(事業費の割増引を含む。)の設定について、係数によらずに定性的な表現で記載するときは以下の条件を満たしているか。

① 保険種類間の公平性が損なわれておらず、事業費の支出見込額に対して妥当であるなど適切なレベルとすることを明確にしているか。

② [Ⅱ-2-7-2(5)④] の主旨に則り、明確に社内規定等で定めることとしているか。

③ (1)(2)の観点を踏まえ、付加保険料の設定に応じ、その重要度を勘案した上で分類した保険種類及び\wakumaru{④}などの別ごとのに\wakumaru{⑤}を添付しているか。また、 \wakumaru{⑤}の基礎となる資料を添付しているか。(以下略)

\answer{}
\begin{itemize}
\item[ ①:] 十分性
\item[ ②:] 信頼度
\item[ ③:] 長期的
\item[ ④:] 販売経路
\item[ ⑤:] モニタリング資料
\end{itemize}

\end{document}
