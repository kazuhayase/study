\documentclass[report,gutter=10mm,fore-edge=10mm,uplatex,dvipdfmx]{jlreq}

\usepackage{lmodern}
\usepackage{amssymb,amsmath}
\usepackage{mathtools}
\usepackage{ifxetex,ifluatex}
\usepackage{actuarialsymbol}
\usepackage[]{natbib}

%strike through 
%https://tex.stackexchange.com/questions/23711/strikethrough-text
%\usepackage[]{ulem}

\usepackage[normalem]{ulem}
\usepackage{enumerate}

% Tables
\usepackage{multirow}
\usepackage{tabularx}
%\usepackage{booktabs} % http://www.yamamo10.jp/yamamoto/comp/latex/make_doc/table/table.php

%Framedbox
%https://hakuoku.github.io/agakuTeX/tutorial/5_6framed/
\usepackage{framed}

%https://tgnx8810.wordpress.com/2014/11/29/latex%E3%81%A7%E8%A1%A8%E3%81%AE%E3%82%BB%E3%83%AB%E5%86%85%E6%94%B9%E8%A1%8C%E3%81%AFtabularx%E7%92%B0%E5%A2%83%E3%82%92%E4%BD%BF%E3%81%86%E3%81%A8%E6%A5%BD/
\usepackage{longtable}
\usepackage{booktabs}
\RequirePackage{plautopatch}

% maru suji ① etc.
\usepackage{tikz}
\newcommand{\cir}[1]{\tikz[baseline]{%
\node[anchor=base, draw, circle, inner sep=0, minimum width=1.2em]{#1};}}

%http://yamamo10.jp/yamamoto/comp/latex/make_doc/box/box.php
%枠付き文章
\usepackage{ascmac}
\usepackage{fancybox}

\usepackage{comment}

\begin{comment}

\ifnum0\ifxetex1\fi\ifluatex1\fi=0 % if pdftex
  \usepackage[T1]{fontenc}
  \usepackage[utf8]{inputenc}
  \usepackage{textcomp} % provide euro and other symbols
\else % if luatex or xetex
  \usepackage{unicode-math}
  \defaultfontfeatures{Scale=MatchLowercase}
  \defaultfontfeatures[\rmfamily]{Ligatures=TeX,Scale=1}
\fi
% Use upquote if available, for straight quotes in verbatim environments
\IfFileExists{upquote.sty}{\usepackage{upquote}}{}
\IfFileExists{microtype.sty}{% use microtype if available
  \usepackage[]{microtype}
  \UseMicrotypeSet[protrusion]{basicmath} % disable protrusion for tt fonts
}{}
\makeatletter
\@ifundefined{KOMAClassName}{% if non-KOMA class
  \IfFileExists{parskip.sty}{%
    \usepackage{parskip}
  }{% else
    \setlength{\parindent}{0pt}
    \setlength{\parskip}{6pt plus 2pt minus 1pt}}
}{% if KOMA class
  \KOMAoptions{parskip=half}}
\makeatother
\usepackage{xcolor}
\IfFileExists{xurl.sty}{\usepackage{xurl}}{} % add URL line breaks if available
\IfFileExists{bookmark.sty}{\usepackage{bookmark}}{\usepackage{hyperref}}
\hypersetup{
  hidelinks,
  pdfcreator={LaTeX via pandoc}}
\urlstyle{same} % disable monospaced font for URLs
\usepackage{longtable,booktabs}
% Correct order of tables after \paragraph or \subparagraph
\usepackage{etoolbox}
\makeatletter
\patchcmd\longtable{\par}{\if@noskipsec\mbox{}\fi\par}{}{}
\makeatother
% Allow footnotes in longtable head/foot
\IfFileExists{footnotehyper.sty}{\usepackage{footnotehyper}}{\usepackage{footnote}}

\end{comment}
%\makesavenoteenv{longtable}
\setlength{\emergencystretch}{3em} % prevent overfull lines
\providecommand{\tightlist}{%
  \setlength{\itemsep}{0pt}\setlength{\parskip}{0pt}}
\setcounter{secnumdepth}{-\maxdimen} % remove section numbering

\author{kazuyoshi}
\date{}






\begin{document}
\chapter{保険2第4章 リスク管理}
\problem{2023 生保2問題 3(2)}
生命保険会社のALM(資産負債管理)について、次の(ア)~(ウ)の各問に答えなさい。(計25点)

(ア)生命保険会社のALMにおいて対象となる市場リスクについて、IAISのALM論点書に準拠すると4つのリスクが含まれるが、このうち「金利リスク」以外の3つを列挙し、それぞれ簡潔に説明しなさい。なお、ALM論点書においては、市場リスクに「流動性リスク」は含めていない。(解答の制限字数は1,000字)(3点)

(イ)流動性リスクについて簡潔に説明し、生命保険会社にとって流動性の問題を引き起こす潜在的な要因について列挙しなさい。(解答の制限字数は1,000字)(5点)

(ウ)あなたは、個人向けの貯蓄性商品を主に販売する生命保険会社に所属するアクチュアリーである。長期金利は、これまで低下トレンドが 20 年程度継続したが、経済情勢の変化等により足元では上昇に転じており、今後もさらなる上昇の可能性がある。このとき、ALMを実施するうえで、アクチュアリーとして留意すべき点を挙げ、所見を述べなさい。なお、解答にあたっては、次の観点を含めること。(解答の制限字数は3,500字)(17点)

A.考慮すべき保険契約者のオプション

B.今後の資産ポートフォリオ・運用方針

C.流動性の管理および、継続的なリスク管理の上で留意すべき事項

\answer{}
(ア)「金利リスク」以外にALMの対象となる市場リスク3つ 

・ 「株式、不動産およびその他の資産のリスク」

株式およびその他の資産の市場価値の変動から生じる損失リスクのこと。保有する株式、不動産およびその他の資産が、負債の動きと連動して変動していない場合、経済的に不利益を被ることがある。

・ 「外国為替リスク」

為替レートの変動から生じる損失リスクのこと。キャッシュフロー、資産、および負債が異なる通貨建ての場合、通貨の変動により、不利益を被ることがある。

・ 「市場リスクに関連する信用リスク」

市場リスクのエクスポージャーを調整した結果、カウンターパーティーの信用リスクのエクスポージャーが高まる可能性がある。(例えば、市場リスクのエクスポージャー調整のために、長期のデリバティブ取引を行う場合が該当する。)

(イ)流動性リスクの説明、流動性の問題を引き起こす潜在的な要因 

<流動性リスクの説明>

流動性リスクとは、支払事由が発生した際、それに充当するためのキャッシュフローに相当する流動性資産を、その負債の対応資産から用意できない場合に生じる損失の大きさのことである。このとき、保険会社は不本意な価格で他の資産を売却せざるを得ない場合もある。保険会社の流動性特性は、資産と負債によって決定され、市場環境に応じて変化する。

<流動性の問題を引き起こす潜在的な要因>
\begin{itemize}
 \item [ ・]  意図的なミスマッチング戦略
 \item [ ・]  グループ会社への投資リスク(関連グループ会社に投資された資金は流動化が困難であったり、そのグループ会社が保険会社の金融資源や事業資源を費消してしまうようなリスク)
 \item [ ・]  資金調達リスク(保有資産に流動性がなく、外部資金が必要な時に、十分な外部資金を得られないリスク)
 \item [ ・]  清算価値リスク(その時点で現金化すれば損失が出てしまうにも関わらず、予期せぬ時点または金額により、資産の現金化が必要となるリスク)
 \item [ ・]  否定的な評判(解約契約が増加することがある)
 \item [ ・]  予想外の巨額の損失(費用)となる即時の支払い
 \item [ ・]  再保険会社からの支払いの遅れ
 \item [ ・]  保険契約者の行動
 \item [ ・]  市場の変動が異常に高いか、市場の緊迫による経済の悪化
 \item [ ・]  予測不能な、規制や裁判所の裁定の変化から生じる政治や法規上のリスク
 \item [ ・]  他社との協力(取引)関係により売却できない投資
 \item [ ・]  複数の保険会社が同時に、予測不能な巨額の流動化が必要になり、会社の資産ポートフォリオの一部を清算する必要性が生じる場合(この場合、これらの資産を売却したくても市場は、保険会社にとって不本意な価格でしか受け入れてくれない。)
\end{itemize}


(ウ)金利上昇トレンドを踏まえたALMの留意点・所見 

<前提>

・ ALMとは、資産と負債に関する企業の判断や行動が調和のとれたものとなるよう事業を管理する手法である。

保険会社は、自社の事業特性に応じて、最適なALM戦略と技法を選択することとなるが、一般にリスクとリターンの間には様々なトレードオフが生じるため、金利リスクを抑える戦略は、経済的な純資産の増加(リターン)の可能性の減少を導く

特に、所属する会社(以下、当社)のように貯蓄性商品を主に販売する場合は、

\begin{itemize}
 \item [○]  保障責任の全うという意味で顧客に約束した予定利率の確保が重要になること
 \item [○]  予定利率を超えて利ざやを得ることが主な利益の源泉となりうること
 \item [○]  利ざやを利益の源泉としていることから、金利低下時の逆ざやリスクや金利上昇時の解約リスク等が、深刻な結果となりうることを踏まえると、ALM戦略は経営管理上、極めて重要である。
\end{itemize}

・ 貯蓄性商品の種類によっても、ALM上の性質は異なる。特に契約者行動の変化はALMに大きな影響を与えることから、これらの差異を意識した分析・モデル化が有用である。

\begin{itemize}
 \item [○]  有配当/無配当
 \item [○]  定額商品/変額商品
 \item [○]  円貨建て/外貨建て
 \item [○]  一時払/平準払
 \item [○]  利率保証期間(終身/有期、予定利率変動型/全期間一律)
 \item [○]  (保証利率における)信用スプレッドの有無
 \item [○]  MVA(市場価格調整)の有無
 \item [○]  解約控除の有無や元本回復(払込保険料を解約返戻金が上回ること)までの期間
 \item [○]  契約者の特性(金利感応度など)および満期や年金受取時等の契約者オプション
\end{itemize}

・ 20年程度長期金利が低下してきたことを考えると、負債側については、段階的に予定利率を引き下げてきたことにより、比較的高予定利率の古い保険契約と、低予定利率の新しい保険契約が混在している可能性がある。また、資産側も、十分なマッチング運用を前提とすれば、古い高利回りの債券と新しい低利回りの債券が混在するポートフォリオとなっていることが想定される。結果的に、この低金利下においては、経済価値ベースにおける負債の含み損に対して、同水準もしくはそれ以上の資産の含み益が計上されていることが想定される。

一方で、償還再投資等を前提に負債よりも短いデュレーションで運用している場合は、この限りではなく、負債の含み損が過大となっている可能性がある。特に、終身保険や終身年金等の超長期の商品を販売していた場合、対応する資産の不足から、この傾向は強くなる。

・ また、現行のロックインを前提とする法定会計についても、留意が必要である。責任準備金はロックイン方式で計算されるため、長期金利の変動によって評価は変わらない。ただし、長期にわたって長期金利が低下していることから、責任準備金の十分性を担保する観点から、過去の高予定利率契約に対して、保険計理人の1号収支分析等に基づき、追加責任準備金を積み立てている可能性がある。その財源として、例えば債券の売却益等が当てられている場合、資産サイドの含み益は減少している可能性があろう。

・ 今後、金利が上昇トレンドに転じ、継続する場合、将来的な運用可能性が高まり、契約の継続を前提とすれば、経済価値ベースの負債は現在よりも改善する。同時に資産側の含み益は減少するものの、仮に負債よりも短いデュレーションで運用している場合は、損失は相対的に小さくなるため、経済価値指標は一定改善する。ただし、特に資産側が大幅な含み損となる場合、以下で述べるような解約等の契約者のオプションの動向には、特段の留意が必要となる。また、減損などの発生等により、会計上の自己資本が棄損する点にも留意が必要となる。

A.考慮すべき保険契約者のオプション

\begin{itemize}
 \item [・]  満期時以前に保険契約を解約し、保険料の払い込みを停止して解約返戻金を受け取ることができる解約のオプションが考えられる。金利が上昇に転じた場合、競争上の観点から、自社商品を含め、新規加入時の予定利率が引き上げられる。これにより、予定利率が低い契約者を中心に乗り換えニーズが高まる。また、債券投資等、他の運用手段の魅力が相対的に高まるため、解約返戻金を振り向ける可能性もあろう。この契約者行動は、予定利率が固定の(変額商品でない)定額商品にて、顕著となる可能性が高く、チャネル特性や加入目的によっても影響は変動する。
 \item [・]  この契約者行動は、MVAがある商品では、金利上昇時に解約返戻金額が減少し、解約・再契約の経済合理性が低下するため一定弱まる可能性がある。また、仮に解約した場合においても、金利上昇による資産側の含み益の減少幅は解約返戻金の減少幅と同水準になるよう設計されている可能性が高く、(解約によって将来獲得するはずだった利益は減少するが)ALM上の問題とはならない可能性もある。ただし、MVAによる変動に一定の上限を設けている場合などは、運用資産の状況も含めて個別に検討する必要があろう。
 \item [・]  この契約者行動は、解約控除のある商品や、新契約費等により元本回復までの期間が長い商品では、一定弱まる可能性がある。特に、乗り換えの可能性が高い予定利率の低い契約は、直近の低金利下で契約しているため、解約控除の対象や元本が回復していない可能性が高い。ただし、元本回復のタイミングや解約控除がなくなるタイミングで、解約率が上昇する可能性があるため留意が必要である。
 \item [・]  保険金受取時や年金開始時に、支払方法(年金/一時金)を選択できるオプションにも影響が考えられる。年金選択時の予定利率が低く設定されている場合、想定よりも一時金での受け取りが多くなり、キャッシュアウトの増加やデュレーションのミスマッチが懸念される。ただし、年金額を支払方法選択時に再計算できる契約では、予定利率の設定次第で影響を抑えられる可能性がある。
 \item [・]  保険契約者が、いつでも定められた条件で保険契約に対する解約返戻金を担保に借入れができる、契約者貸付のオプションにも影響が考えられる。例えば、契約者貸付利率よりも相対的に有利な市中での運用手段がある場合、急激な資金流出の増加要因となりうる。現在では、契約者貸付利率を相対的に高い水準に設定することが一般的であり、大きな問題とはならない可能性もある。ただし、金利上昇のスピードや金利上昇の幅、その他の利率(新契約の予定利率や据置利率、配当積立利率等)との乖離、契約者の合理的期待等を踏まえ、契約者貸付利率の引き上げが困難なケースも考えられるため、留意が必要である。
 \item [・]  また、保険金据置制度や積立配当制度でも、同様にキャッシュアウトが増加する可能性がある。その他、既存の低予定利率契約への払い込みを止める払済の増加等が考えられる。
\end{itemize}

B.今後の資産ポートフォリオ・運用方針

\begin{itemize}
 \item [・]  資産運用における運用方針は、自社の健全性やリスク許容度等を踏まえ、負債特性や市場環境に応じて決定される。市場リスクを低減する観点からは、資産負債のデュレーションを合わせた同通貨建て債券での運用が基本となるが、健全性が維持できる範囲において、他の資産クラスの組み入れや意図的なミスマッチング戦略が検討できる。特に、金利上昇の蓋然性が高い局面では、資産の短期化入れ替えも選択肢となろう。ただし、金利がさらに低下した場合の再投資リスクは高まるため、リスクシナリオを設定し、リスク・リターンの効率や健全性への影響の検討が必要である。
 \item [・]  金利上昇に伴い想定以上に解約の増加する局面では、(MVA付商品等で一部例外はあるが)当該契約に対応する債券の時価が下落しており、売却しても解約返戻金額を賄えない事態が発生する。予定利率の低い契約を中心に、解約率上昇の想定を行いながら、年限の短い債券の一定確保や現金比率の増加などにより手元流動性を確保する必要があろう。外貨建商品の場合、将来の為替変動により会社の予定事業費収入が変動するなど、事業費支出を含めたALMを行っている場合には、ALM前提が変わることで、事業費支出を賄えない可能性も考えられる。
 \item [・]  予定利率の比較的高い契約については、解約・再契約のインセンティブが弱く、すぐには金利上昇の影響を受けないことが考えられる。その後の金利上昇の機会を捉え、償還再投資や、利回りの低い債券を損切りしながら高利回り債券へ入れ替え、債券の長期化など、健全な資産ポートフォリオの構築を検討・計画できる。場合によっては、株式等の資産クラスから投資魅力の増した債券へ移すことや、外国債券から投資魅力の増した国内債券へ移すことなども、リスク・リターンに応じて検討しうる。ただし、金利低下に転じた場合の逆ざやリスクも高いため、平均予定利率を踏まえながら、利回りの確保を優先的に図ることが考えられる。
 \item [・]  一方で、金利上昇時には、その他有価証券で保有している場合には純資産が減少し、特に急激な金利上昇の場合に、保有する債券が含み損に転じるなか、大量解約が発生し、売却を余儀なくされ、損失が拡大する可能性がある。特に金利上昇等の変化が急激に生じた場合、保有していたヘッジ手段が有効でない可能性や、かえってリスクやコストが増大する可能性も考えられる(例えば、カウンターパーティーの信用リスクやベーシスリスク等の増加などが挙げられる)。責任準備金対応債券での運用をしている場合などは、環境変化によってマッチング要件を満たさなくなる可能性もある。
 \item [・]  これらの問題は、特に資産の含み益や会計上の自己資本が低水準の場合、経営上の重大なリスクとなり得る。負債側について市中金利や新契約の予定利率に対する契約者行動をモデル化するとともに、資産側の時価変動やキャッシュフローについても適切にモデル化し、経済前提を複数設定したストレステストを実施する等によって、自社に発生しうるキャッシュフローや資産負債の状況、各種指標を把握することによって、金利の上昇に備えた運用ポートフォリオへの入れ替えを検討することが必要となる。
 \item [・]  市場リスクを低減する目的で、ポートフォリオのイミュナイゼーションを行った場合においても、イールドカーブのパラレルシフト以外の変動が起きた場合や保有契約に変化が起きた場合には金利変動時には資産と負債のデュレーションが徐々に離れていくことがあり得ることから、金利上昇に備えた運用ポートフォリオへ入れ替えを実施した後においても、デュレーションやコンベクシティ等は継続的にモニタリングし、常に資産の入れ替えなどの微調整をするべきであろう。なお、これらの入れ換えにおいては、取引コストや売却損益が発生することや、保有目的区分によっては売却に一定の制限(特に、満期保有目的で保有している債券を売却する場合には保有目的変更等が必要)があることに留意が必要である。
\end{itemize}

C.流動性の管理および、継続的なリスク管理の上で留意すべき事項

\begin{itemize}
 \item [・]  金利環境及び解約動向について、モニタリングの必要性が増す局面といえる。要因分析や当社経営へのインパクトを常に確認することや経営層にタイムリーに報告することが重要となる。
 \item [・]  想定外の解約が重大な影響を生じるため、解約のモデル化についても、精緻化が必要となる。20年間低金利が続いたので、動的解約率に適切なモデルが存在しない恐れがある。諸外国の過去事例や他業態での資金流出事例を参考に、当社のマーケット特性や投資信託等の他の運用手段の動向、金融リテラシーや税・規制等の社会環境の変化も踏まえた検討が考えられる。また、金利上昇と解約動向の実績を基に、逐次的な検証を行い、特に悪い方向に想定を外れる事象が発生した場合には素早い分析と適切な報告、モデル改正の検討が必要となろう。
 \item [・]  流動性の管理の観点からは、流動性比率やキャッシュフローモデリング等を通じて、全社ベースでの流動性資産の確保状況の確認が重要となろう。解約率のモデルを参照しながら、日々・月次・四半期・年間単位等でキャッシュフローを想定し、十分な流動性が確保できているか、検証することが必要となる。また、経済前提のストレスシナリオを用いた解約想定やリバースストレステストを用いた検証も、重要となろう。一方で、流動性資産は相対的に運用利回りが低いため、過大保有は投資機会を喪失する。顧客からの解約申出から実際に送金するまでのタイムラグや、1日に集中しうる契約数など短期的な解約行動・キャッシュフローのシナリオを立てながら、一定のマージンを確保しつつ、適切な水準の流動性資産の確保の検討を実施する。
 \item [・]  また、急激な金利上昇時には意思決定のスピードが追い付かない可能性があることから、金利上昇局面のシナリオごとに、とるべき対応策を予め合意しておく必要がある。シナリオ決定においては、イールドカーブのシフト、ツイスト、およびベンドのシナリオや、為替や株式などの相関性を踏まえた様々な経済前提のシナリオについて、それぞれ、かつ、それらの現実的な組み合わせを複数含めるべきだろう。また、対応策については、特に、急激な金利上昇による資産価値の変動や大量の解約が発生したときに、どのような対応策を取るか、経営層で議論・決定しておくことが重要である。
 \item [・]  例えば、緊急時の流動性の確保策として、資本調達や借入、レポ取引、銀行とのコミットメントライン等を活用しながら、対応時間を確保し、非流動性資産についても優先度を決めて流動化を進めるなど、決定しておくことが考えられる。ただし、金利急上昇や大量解約時には、信用不安による短資市場の不安定化など、機能しない可能性もある。非流動性資産についても、大幅な価格下落に見舞われている可能性もあろう。大口のコミットメントライン等については、手数料やカウンターパーティーの集中リスクにも留意が必要である。計画した手段が市場の混乱により実効性を失う可能性も視野に入れ、予め複数の対応策を検討・確保しておく必要がある。
 \item [・]  資産のより良いマッチングの実現やリスク移転(会計上の諸リスク含む)等の目的で再保険を締結することも考えられる。ただし、再保険はリスクを移転する一方で、カウンターパーティリスク、集中リスクを派生させる。カウンターパーティリスクは、再保険会社が保険会社に対する義務を履行できない、もしくは再保険会社の信用力が悪化するという状況において生じる。特に、再保険会社のデフォルトをもたらす要因は、保険会社自身を財務上の困難に陥らせる可能性がある要因と高い相関がある可能性があり、カウンターパーティリスクの中でも相応の留意が必要である。再保険会社の信用力を継続的にモニタリングするとともに、再保険契約に担保要件や格付けトリガー条項を設定することによって減少させることも検討しうる。
 \item [・]  急激な金利上昇に伴う、解約集中や債券価格下落に対応するため、危機対応のための自己資本の強化も重要となる。特に、追加責任準備金の積み立て等の対応のため、会計上の自己資本積み立て財源が十分に確保できていなかった場合、今後は自己資本の充足に軸足を移すことも検討しうる。一方で、収支改善を目的として、新契約の予定利率の抑制や年金開始後の予定利率等の抑制、有配当契約への配当の抑制等を行った場合、新契約獲得の低迷や既契約の解約が進むことで、資金流出が更に促進されることとなる。魅力的な乗り換え商品の開発や利差配当の引き上げなど、足元の金利環境の改善について、契約者の期待も踏まえた、適切な顧客還元と自己資本充足のバランスをとる必要があろう。
\end{itemize}

\problem{2023 生保2問題 2(2)}
リスク管理における、リスクモデリングの限界とその対応方法について簡潔に説明しなさい。なお、リスクモデルの具体的な構築手法や構築過程について説明する必要はない。(解答の制限字数は1,000字)(10点)

\answer{}
<リスクモデリングの限界>

\begin{itemize}
 \item [・]  大部分のモデルは過去の経験に根ざしており、明示あるいは暗黙の前提を用いている。特にインプットを取り巻く不確実性が存在する場合にはモデルの頑健性の理解が必要不可欠であり、ストレステスト・シナリオテスト等を通じて主要パラメータの変化による影響度や多様な環境下での市場価格の再現性を把握の上、アウトプットに対する判断に生かす必要がある。
 \item [・]  モデルを過信する原因には次のものが考えられる。
\begin{itemize}
 \item [✓ ] 自信過剰:結果を予見する自分の能力や、リスクや不確実性に対する自分の統制力の過大評価。
 \item [✓ ] 後知恵のバイアス:技術・社会変化・環境変化による状況変化や、過去の経験が必ずしも起こり得た事象の平均値ではない可能性の考慮不足。
 \item [✓ ] 生き残りバイアス:リスク評価のためのパラメータ決定において、生存者のみの経験値の使用や、破綻者のデータの除外による、リスクの過小評価。
 \item [✓ ] 判断の放棄:実際の状況に適合しない技法の使用(分析者の理解不十分による、標準的な技法の使用等)や、技法使用の正当性を証明しようとしないこと。
 \item [✓ ] テールへの外挿:パラメータ特定に使用した経験領域の範囲外にある事象に対してモデル使用することにより、アウトプットの信頼水準が著しく低下すること。
\end{itemize}
 \item [・]  モデルの結果が信頼できなくなる状況には次のものが考えられる。
\begin{itemize}
 \item [✓ ] データがもとの状況を十分に代表していない。
 \item [✓ ] リスクの市場価格とモデル価格が乖離している(利益機会・モデルの欠陥の両方がありうる)。
 \item [✓ ] モデル構築の基礎となった暗黙の前提が有効性を失っている。
 \item [✓ ] 環境変化等により、モデルの明示的な前提が有効性を失っている。
 \item [✓ ] 顧客の行動や構成が、以前の前提と大きく異なるようになった。
\end{itemize}
\end{itemize}

<限界への対応方法>

\begin{itemize}
 \item [・]  利用者はモデルが定量化している対象を明瞭に把握するだけでなく、自分の知識とモデルの限界を理解し、その前提が使用対象となる特定の状況にどの程度よく当てはまるかを考慮することが肝心である。
 \item [・]  モデル化の対象が未知事項であるにも関わらず既知と誤認した場合、モデルに対する誤った安心感と自信過剰が生じかねないため危険であり、過信が生じないよう十分な注意を払う必要がある。
 \item [・]  モデルの限界を踏まえない使い方をするとそれ自体がリスクの源泉となるため、モデルの具体的な開発や活用を検討する前にそのことを認識しておくことが重要である。
 \item [・]  モデルの結果が信頼できなくなる状況とそのシナリオを前もって特定し、実世界の変化がそうした状況に向かっている場合には、モデルに変更を加えて、少しでも信頼性を高めるようにすることが重要である。そうした手順をとらないと、変化の蓄積によってリスク評価の有効性が完全に損なわれてしまうことが認識されないままになる可能性がある。なお、変更の際には、実務負荷の観点も踏まえつつ、外部データを用いた出力結果の妥当性確認、内部・外部による第三者レビューを行うことが望ましい。
 \item [・]  モデルの限界の伝達にあたっては、その重大性、明瞭性、モデルとリスク評価が適切な状況と適切でない状況、考えられる影響等の論点を十分に検討し、対応する能力と権限のある関係者に適切に伝達することが重要である。
\end{itemize}





\problem{2022 生保2問題 2(2)}
リスク計測に使用する以下のモデルを簡潔に説明しなさい。(解答の制限字数は1,000字)(10点)
\begin{itemize}
\item[] 単純なファクターモデル
\item[] 標準ショック(ストレステスト)
\item[] 個別ショック(ストレステスト)
\item[] 部分モデル
\item[] 完全内部モデル
\end{itemize}
\answer{}
モデル化される潜在的なリスクの重要度や複雑性などにより、相応しいモデルの洗練度は異なる。

<単純なファクターモデル>

\begin{itemize}
\item[] リスク測定に用いられる最も単純な形態のモデル。
\item[] エクスポージャー等の基準額に所定のファクターを乗じて、リスク量を算出するモデル。
\item[] 米国の規制上の資本モデルや、EUソルベンシーⅡの標準方式の簡便計算法、日本のソルベンシー・マージン比率等で用いられている。
\item[] 例えば、保有資産の価格に格付け別クレジットデフォルトチャージを適用し、資産デフォルトリスクを算出する。
\end{itemize}

<標準ショック(ストレステスト)>

\begin{itemize}
\item[] 所定の単一あるいは複数のリスクファクターストレスの財務への影響を評価することでリスクを計測するモデル。
\item[] EUソルベンシーⅡの標準方式等で用いられている。
\item[] 例えば、死亡率の最良推定値が所定の割合で上昇した場合の財務への影響を算定することで死亡率リスクを評価する。
\end{itemize}

<個別ショック(ストレステスト)>

\begin{itemize}
\item[] 所定のストレステストの実施や、規制当局が定めた慎重な業界標準のストレステストを用いる代わりに、自社で設定したストレステストの財務への影響を評価することでリスクを計測するモデル。自社特有のリスク特性に合わせてキャリブレートしたストレステストを行うことが可能。
\item[] 例えば、死亡率リスクを評価する際、自社の実績や計測対象の商品ラインに特有の実績に基づき、要求される信頼水準を適切に反映していることを立証したうえで、所定の割合の代わりに自社のシナリオを用いて、ストレステストを行う。
\end{itemize}

<部分モデル>

\begin{itemize}
\item[] 単純なモデルでは正確な計測を行えないと判断した場合、それらの特定のリスクに対してより複雑なモデルを開発することがある。このモデルは、確率分布かシナリオの分布のいずれかに基づいて、確率論的あるいは決定論的に定めることができる。
\item[] 他のリスクに対するより簡便なモデルと組み合わせて部分モデルを使用することで、会社のリスクを総合的に計測することができる。
\end{itemize}

<完全内部モデル>

\begin{itemize}
\item[] 保険会社のリスクを計測する最も総合的(かつ最も複雑)な手法。
\item[] このモデルを開発する一つの方法は、全てのリスクを同時に計測する基盤として多変量確率分布関数を用いることである。他の方法は、各リスクを別々にモデル化し、統合の手法としてコピュラを用いて、それらの結果を統合することである。
\item[] テール部分の薄い引受リスクに完全内部モデルを開発する意義は低いかもしれないが、特にテール部分のリスク依存が大きいリスクについては、総合的なモデルがより適切である。
\item[] 一度モデルが開発されれば、基盤となる一連の確率論的な、あるいは決定論的なシナリオに基づいて、リスク評価ができる。
\end{itemize}
\problem{2021 生保2問題 3(2)}
生命保険会社のリスク管理について、次の①、②の各問に答えなさい。

① リスク管理プロセス(一般に6段階)について簡潔に説明しなさい。なお、解答にあたっては、リスク対応における4つのカテゴリーについて触れること。(8点)

② あなたの所属する生命保険会社では、これまで保険期間1年の無配当定期保険のみを販売しており、第三分野の保険を取り扱っていない。今般、新たに保険期間終身の無配当平準払医療保険の販売を開始することとなった。このような会社のリスク管理部門に所属するアクチュアリーとして、リスク管理を行うにあたっての留意点を挙げた上で、所見を述べなさい。なお、解答にあたっては、次の論点を含めること。(17点)

\begin{itemize}
\item[ A.] 医療保険の販売開始に伴う会社のリスク特性・リスクプロファイルの変化
\item[ B.] A.に対するリスク対応
\item[ C.] リスク管理の高度化に向けて検討すべき点
\end{itemize}
\answer{}
① リスク管理プロセス

以下の6つのステップに分けて説明する。

1.リスク特定
\begin{itemize}
\item[]  リスクを財務リスクや保険リスクに限らず、戦略リスク、風評リスクやその他のリスクも考慮した上で、潜在リスクを特定し、カテゴリー化し、追跡するプロセス。
\item[]  社内の多くの人員の関与によりリスクを特定するボトムアップ、主要リスクカテゴリーを上級経営陣が特定するトップダウン、あるいはその組み合わせがあり得る。
\item[]  洗い出されたリスクは、明確化されたリスク分類法によりカテゴリー分けされ、発生確率や影響といったリスク評価等の情報と共に、リスク記録簿に記録される。
\end{itemize}

2.リスクアセスメント
\begin{itemize}
\item[] リスクの蓋然性の評価やリスク発生時の影響の評価等を通じて、特定されたリスクの評価、プロファイリングを行うプロセス。
\item[] 固有リスクと残余リスクの両方を評価し、コントロールの有効性と信頼性の程度に対する知見を形成できるようにする。
\item[] リスクアセスメント、リスクプロファイルは、「詳細なリスクの説明」、「リスクによってもたらされる結果」、「適切な分類」、「蓋然性・影響度を考慮した固有リスクの評価」、「軽減戦略の有効性の評価」、「軽減後の残存リスクの評価」、「残存リスク抑制に必要なアクションの説明」等を含む。
\end{itemize}

3.リスク計測
\begin{itemize}
\item[] 企業の意思決定や、資本管理とパフォーマンス計測を含む諸プロセスの支援に活用するため、リスクの定量情報を提供するプロセス。
\item[] リスク計測に用いる技術は、対象となるリスクの性質、規模および複雑性によって決まることが多く、リスク計測に費やす努力の程度がリスクの大きさに見合うかについても考慮すべき。
\item[] 計測に使用するリスク尺度は、分析の目的、利害関係者、データとモデリングの制約等の基準を元に選択する。
\item[] リスクモデルでは、モデルの目的を踏まえて、リスクの規模、広がり、不確実性に応じてモデルを選定する。洗練度やリスク種類等によっても適切なモデルの種類は異なる。
\item[] データの収集とガバナンスの適切性や、モデルガバナンスについても、考慮が必要。
\end{itemize}

4.リスク対応
\begin{itemize}
\item[] 取締役会のリスクへの対応は、そのリスクアペタイト、リスク許容度、リスクリミットに反映される。また、軽減・移転(共有)を選択する場合、それ自体がモニタリングを必要とする新たなリスクを生むことが多い。
\item[] リスク対応の意思決定に際しては、当該リスクに係るリスク・リターンプロファイルや企業全体のリスク・リターンプロファイルへ与える影響を考慮する必要がある。
\item[] リスク対応は、以下の4つのカテゴリーに分類されることが多い。
\begin{itemize}
\item[i.] 回避
\begin{itemize}
\item[] リスクに対するエクスポージャーを減らす、新規開発は行わない等の行動。
\item[] ただし、リスクを完全に回避するのは難しい。
\end{itemize}
\item[ii.] 受容
\begin{itemize}
\item[] リスクを現在の形で受け入れるもの。
\item[] リスクをモニタリングし、適切な技術的準備金と資本が確保されていることを確認する。
\end{itemize}
\item[iii.] 軽減
\begin{itemize}
\item[] リスクの発生の可能性を減らすか、あるいは、リスクの影響を軽減させるか、がある。
\item[] ただし、この対応に伴うリスクの変化が異なるリスクを生み出すおそれがある点に注意が必要。
\end{itemize}
\item[iv.] 移転(共有)
\begin{itemize}
\item[] 保険や再保険等を利用してリスクを第三者に移転するもの。
\item[] カウンターパーティリスクの配慮が必要。
\end{itemize}
\end{itemize}
\end{itemize}

5.リスクモニタリング
\begin{itemize}
\item[] リスクモニタリングには、企業が用いる全てのリスク尺度のモニタリングが含まれる。
\item[] モニタリングは、十分な情報が与えられた上で意思決定が行われ、アクションが取れるような頻度で行われるべき。
\item[] モニタリングすべき事項には、「リスク評価のアウトプット(結果)」、「リスクコントロールの自己評価」、「リスクリミット、許容度、アペタイトの遵守」、「外部環境」、「主要リスク指標(KRI)」、「リスク管理のアクションプラン」等がある。
\end{itemize}

6.リスク報告
\begin{itemize}
\item[] リスク管理情報は、タイムリーに、包括的に、整合的に、正確に、監査可能な形で、フォワードルッキングに、報告される必要がある。
\item[] また、情報に関して保証すること、社内外の報告対象者のニーズに対し個別に配慮をすること、適切な情報開示への配慮をすること、にも留意が必要。
\end{itemize}

② リスク管理を行うにあたっての留意すべき事項、所見

\begin{itemize}
\item[] 保険会社のリスク管理では、各社の保有するリスク特性やリスク量、おかれた経営環境、保有する経営資源、経営層による判断およびこれに基づくリスクアペタイト等のリスク戦略等に応じて、必要となるリスク管理体制は様々である。
\item[] 所属する会社(以下、当社)は、現在、保険期間1年の定期保険のみを販売しており、貯蓄性商品を中心とした会社に比べ、運用資産は限定的と考えられる。また、普通死亡保障のみを扱っており、医療分野の引受リスクは限定的である。よって、以下のリスク管理体制が十分に整備されていない可能性がある。
\begin{itemize}
\item[] 資産運用に対するリスク管理体制(運用リスクモニタリング体制等)
\item[] ALM体制(デュレーション管理、解約集中時の流動性の確保等)
\item[] 医療給付発生に対するリスク管理体制(引受・支払査定管理、発生率動向モニタリング体制等)
\item[] 現行商品以外を販売した際のオペレーショナルリスクの管理体制
\end{itemize}
\item[]  よって、今回の新商品発売により、当社の現在のリスク管理体制ではカバーできないリスクが発生する。リスク管理プロセス(リスク特定、リスクアセスメント、リスク計測、リスク対応、リスクモニタリング、リスク報告)の中で、必要に応じて、リスクアペタイト、リスク許容度、リスクリミットの見直しを検討(所管部署への働きかけを含む)し、経営層や経営企画部門、販売部門、資産運用部門を含む、全社的な議論をするべきである。
\item[]  なお、リスク管理部門は、事業部門(第1ライン)から独立した第2ラインとして、ビジネスのサポートやモニタリングを行い、業務監督の責任を負う。議論の前提として、第2ラインは、第1ラインの活動に対して異議申し立てが出来ることが重要である。一方で、それを実効的なものにするためには、第2ラインが第1ラインと信頼で結ばれた関係を保つことも必要であり、リスク管理部門としては、時としてバランスの維持が難しいことに留意を要する。
\end{itemize}

A.リスク特性・リスクプロファイルの変化

\begin{itemize}
\item[] 当社において、新商品を発売した場合に、保有するリスク特性に、以下の変化が想定される。
\item[] 保険リスクについては、医療給付の変動・解約率の変動について、新たなリスクを保有。
\begin{itemize}
\item[] 普通死亡は比較的安定的に改善傾向を持つことが多い一方で、医療給付については、医療技術の発展による手術頻度の上昇や検出率の高い検査法の確立、社会環境の変化等で、給付が増加するリスクがある。特に新商品は終身にわたる保障であり、医療給付水準が高い高齢層を含めて中長期的に保障を継続する必要がある。保険会社側からは、基礎率変更権などのオプション設定は可能であるものの、実務上のハードルが高いため、発生率動向に関するリスク管理上の重要性は非常に高い。一方で、医療技術の進歩は入院日数の短期化を招き、入院日数×日額を保障する商品では給付が減少する場合がある。また、普通死亡率の減少は医療給付を増加させる傾向があるなど、影響が相殺しあう要素もあり、留意を要する。
\item[] 引受査定の手法については、販売チャネルの特性等を踏まえた配慮が必要である。また、他社との引受査定基準の差異による逆選択のリスクに留意が必要であり、査定コストも踏まえた検討が必要となる。医療保険の発売に際しては、引受査定について、新たなノウハウの蓄積とモニタリングが必要になる。支払査定についても、死亡保障ほど客観的な基準策定が難しい場合もあるため、ノウハウが必要となろう。
\item[] 解約率についても、保険期間終身の商品を発売する場合、より長期での解約の発生率動向に新しいリスクが発生する。これまで保険期間1年のみを販売していたため、長期契約の知見は薄いと思われる。新商品が低(無)解約返戻金型である場合、企図した解約率を下回る場合に解約損が発生する。
\item[] 一方で、企図した継続率を下回る場合には、契約初期のコスト(代理店手数料等)を回収できない可能性や目標利益水準を確保できない可能性が生じる。
\end{itemize}
\item[]  市場リスクについては、中長期的に運用資産が増加し、リスクが大幅に増加。
\begin{itemize}
\item[] 新商品は終身医療であり、既存商品と比較して、保険料積立金が増加する商品と考えられる。
\item[] 平準払いであるため、急激ではないものの、徐々に運用リスクが増大する。
\item[] 運用先に応じて、金利リスクや株式リスク、為替リスク等を受ける。特に、運用期間の長期化によって、金利低下局面における逆ざやリスクを抱える。また、過去のリーマンショックのように世界的に市場が連動して悪化する場合、分散効果が効きにくくなり、大きな損失を計上する可能性がある。よって、市場リスクは経営上の重大なリスクとなりうる。
\end{itemize}
\item[]  流動性リスクについては、特に新商品の解約返戻金水準を削減しない場合に、解約の集中リスクが新たに生じる。
\begin{itemize}
\item[] 競合他社にて競争力が高い商品が発売された場合や、当社の風評リスク発現の際に、解約が集中する。
\item[] また、市中金利との関係では、新商品は医療保障が主たる保障であり、資産運用を主眼とした商品に比べれば直接的な相関は高くないと考えられるものの、金利上昇局面で(当社もしくは他社が)料率の引き下げ等を行った場合には、解約リスクもあろう。このとき、手元資金がショートするリスクや、流動性の低い資産を安値で売却する必要に迫られるリスクがある。
\item[] 一方で、新商品が入院日額・月額に応じて支払う設計の場合、1回あたりの給付金額は大きくない傾向にあるため、定期保険ほど給付金支払の集中による流動性リスクは高くない可能性がある。
\end{itemize}
\item[]  オペレーショナルリスクについては、医療給付や資産運用に伴い追加業務が発生。
\begin{itemize}
\item[] 新商品の発売に際して、契約管理や資産運用に伴うシステム開発が必要となり、システム改定のミスやスケジュール遅延などのリスクが発生。\
\item[]また、これまで医療保険に対する引受・支払査定等を行っていないため、査定基準の設定ミスや査定業務・支払業務での事務ミスのリスクが発生。
\item[ ]期中の減額や払済保険への移行、前納などの契約変更、保全手続きの事務が新規に発生する場合、これらへの事務リスクも新しく発生。
\item[ ]資産運用については、資産管理や誤発注等の事務ミス、管理システムの不備など、新しいリスクが発生。
\end{itemize}
\item[]  その他、以下のリスクも考えられる。
\begin{itemize}
\item[] 新商品の発売に伴い、当初企図していた程販売できず、初期コストが回収できない戦略リスクが発生。
\item[] 新商品の発売に伴い、資産運用先として社債やデリバティブ等を活用する場合には、信用リスクが発生。また、新規の再保険活用等新しいカウンターパーティと取引する際にも発生。
\item[] 新商品の発売に伴い、販売商品が複雑化するため、風評リスクや法的リスク、規制リスク、コンダクトリスクの管理が煩雑化する。特に、人的資源が限られる場合に、重大な影響となりうる。
\item[] 新商品の発売に伴い、パンデミックによる入院の急増といった、エマージングリスクも増大する。
\end{itemize}
\end{itemize}

B.A.に対するリスク対応

\begin{itemize}
\item[] リスク対応では、回避・受容・軽減・移転(共有)といった方針を検討する。
これらの方針は経営レベルで決定され、リスクアペタイト、リスク許容度、リスクリミットに反映される。決定に際しては、当該リスクや企業全体のリスク・リターンプロファイルへの影響を考慮する必要がある。企業の資本のポジションにそのリスクが与える影響は、重要な検討事項であり、統合に関しての手法が関連してくる。
\item[] 保険リスクについては、医療保険の収益の源泉であるため、完全な回避や移転は難しい。
一方で、既存商品にない新しいリスクであり、販売量によっては経営に与える影響が過大となりうる。ストレステスト等を入念に行ったうえで、自己資本に合わせた受容を検討する。自己資本が不十分な場合は、(初期の)販売量の制限や販売対象年齢の限定、引受査定基準の厳格化、一部支払事由や高額契約に対する再保険の活用といった、受容するリスクの限定や移転の検討が必要となる。このリスク対応方針は、自己資本水準に応じたリスクリミット等の設定と整合する必要があろう。
なお、これらの対応は収益機会とトレードオフなので、ノウハウやデータが蓄積するに応じて、順次基準を緩和するなど、リスクリターン効率の最大化に努めることも検討できよう。
\item[] 市場リスクについては、自社に十分なノウハウやリスク管理体制がある場合には、自社での受容を選択できる。
投資対象についても、自己資本の水準と合わせて、すべてのアセットクラスを対象に、リスクリミットを設定しながら、運用の最適化を図る等の方針が考えられる。
一方で、当社は、現在の運用資産が限定的である。運用の実績やリスク管理体制が乏しい場合や自己資本が十分でない場合は、投資対象の国債・高格付債への限定や、ヘッジ手段を多用するなど、できるだけ市場リスクを抑えるほうが現実的であろう。運用戦略の決定に際しては、デュレーションマッチングやキャッシュフローテスト等を実施して、適切な資産デュレーションを保つ必要がある。また、自己資本水準を踏まえてリスクリミットを適切に設定したうえで、市場リスクモニタリングを強化する必要があろう。
価格面での競争の激しいマーケットでは、予定利率も一定水準が求められる可能性があり、この場合、逆ざやリスクがより重要な要素となる。自社の運用能力を考慮したうえで、財務再保険のように、資産運用ごと他社に出再することで、運用リスクの移転も検討できる。ただし、カウンターパ
ーティリスクの配慮が必要である。
\item[] 流動性リスクについては、十分な流動性を確保したうえで、受容も検討できる。ただし、流動性の高い資産は一般的にリターンが低いので、流動性確保には逆ざやリスクとバランスをとる必要がある。また、外部資金の調達や、銀行との間でコミットメントラインを設定する等、緊急時の資金調達手段を確保することも、併せて検討できる。
また、自社商品の解約率動向のモニタリング強化が必要となろう。特に他社の医療保険の料率改定や市中金利との関係など、解約率の決定要因を日々分析し、ノウハウを蓄積することが必要である。モニタリング結果次第では、解約返戻金水準の抑制や、既契約を含めた保険料率の機動的な変更など、リスクの軽減策も検討できる。ただし、予定解約率に関するリスクや収益性の低下等、新しい検討項目が発生することに留意が必要となろう。
\item[] オペレーショナルリスクは、完全な回避や移転が難しい場合、軽減が基本的な対応となろう。人的資源を確保しながら、システム開発の進捗管理や事務基準の策定にルールを策定し、適切にモニタリングすることが必要となる。また、他社ヒアリングや外部からの有識者の招致等を通じた実務に関するノウハウの収集や、事前の実務者向けの研修等を実施することで、リスク軽減策を図るべきであろう。
また、ミスが発生した場合の報告プロセスを明確化し、ミスの深刻度に応じたレポーティングをするとともに、再発防止策の策定と徹底を社内にて図るべきである。顧客影響の大きい情報漏洩や支払い遅延は、社外公表や当局報告なども併せて検討し、外部の目のある状態で、改善・管理体制の整備を進めることも検討できる。
一方で、契約管理システムの管理や支払事務の一部などを他社に業務委託することによってリスクを移転することも考えられる。ただし、この場合は、追加のオペレーショナルリスクやカウンターパーティリスクなどの別のリスクに変容してしまうことに留意が必要となろう。
\item[] その他のリスクについても、基本的には発生頻度や発生時の深刻度、それぞれのリスクリターン効率に応じ、リスク対応を検討する。
\end{itemize}

C.リスク管理の高度化に向けて検討すべき点

\begin{itemize}
\item[] リスク特定・リスクアセスメントの高度化
新商品の発売によって、業務プロセスの中に新規のリスクが発生している。その一部は、リスク管理部門では予期できないものである可能性がある。よって、リスク特定を疎漏なく実施するため、特に業務変更があった部署や上級経営層へのヒアリングにて、新商品発売にフォーカスする項目を
設けるなど、留意が必要であろう。
新しいリスクに対しては、ノウハウやデータが不足する中、アセスメントが正しくなされない可能性に留意が必要である。公的データや他社の先行事例を収集し、毎年重点的に検証するなどのプロセスが求められよう。
\item[] リスク計測の高度化
リスク計測において、リスク尺度の選択やモデルの種類・洗練度、リスクの統合手法等、技術的な論点が多数存在する。新規のリスクをとった場合、データ・ノウハウのない中で、一定保守的な手法や業界共通ルール等で代替しながら始める必要があろう。一方で、モニタリング等の結果、データが蓄積してきた場合は、現行モデルを検証し、高度化を図っていくことが必要である。なお、モデルの高度化に際しては、適切な承認プロセス、文書化や第三者による検証・確認等、モデルガバナンス上の要件とセットで実施するように留意する。
リスク計測の高度化に合わせて、販売制限の緩和や投資可能なアセットクラスを広げるなど、リスクリターン最適化が検討できよう。
\item[] モニタリング高度化、結果の蓄積・適切なフィードバック
B.で記載した新規のリスク対応については、リスクリミットの設定やモニタリングの強化とセットで実施する必要がある。モニタリングした結果、得られたノウハウについては、経営層をはじめ、必要な関係者に対して、適宜適切にレポーティングしたうえで、更にリスク管理手法の高度化にフィードバックする体制が必要となろう。将来収支分析やストレステストにおけるシナリオ設定、トレンド分析等の際には、事業部門と密接に情報交換を行いながら、実施することが必要になる。
また、特に新商品の発売初期においては、新たなリスクの特定やリスクアセスメント等の高度化が可能となることもある。リスク管理プロセスを継続的に続けることで、データやノウハウをタイムリーにリスク管理へ反映する必要があろう。
\item[] 社内外への報告・情報開示
新しいリスクを保有する場合、社内外で使用しているリスク指標への反映や影響度についての公表も必要となろう。例えば、ESRなどの健全性指標であれば、新商品の発売に伴い、リスクの上昇と自己資本への貢献がどの程度なのか、開示することも検討できる。また、当初企図していたリス
ク調整後のリターンと、実績値を比較・検討することも考えられる。
これらは、社外のステークホルダーの監視を受けるとともに、社内における販売戦略の妥当性検証の際の材料となる。ただし、その場合は、情報の受け手の理解レベルに応じ、誤導的な表記とならないよう、理解度を促進するよう説明を補足する等、工夫が求められよう。
\item[] リスク文化(カルチャー)の醸成
リスク管理において、企業が適切なリスク文化を有しているかどうかを検討することは重要である。
ここには、リスク管理が社内の上級経営陣によって適切に支持されているかどうかも含まれる。
特に、新商品の発売に際して、当社においてこれまでリスク管理の対象としてこなかったリスクについても大幅に引き受けることとなる。社内のリスク管理体制が十分でない状況で発売に踏み切っている場合、経営陣の意識が営業成績に偏重するなど、リスク管理の重要性が十分に認識されていない可能性が懸念される。経営企画や営業部を含め、全社的な意思決定の際のリスクに対する認識を強化し、リスク文化を醸成する必要があろう。この場合、経営トップのコミットが重要となろう。
意識醸成のため、例えば新しいリスクも含めたヒートマップ等を作成し、経営層との定期的な対話・認識の共有化を図ることで、自社の網羅的なリスク認識を確立することが考えられる。適切なタイミングでリスク管理状況のレビューを実施することで、正しいリスク認識をタイムリーに共有することが必要であろう。
\end{itemize}
%Anki Complete
\end{document}